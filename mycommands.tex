\newref{tab}{name=table~,names=tables~,Name=Table~,Names=Tables~}
\newref{abb}{name=figure~,Name=Figure~}
\newref{abschnitt}{name=section~,Name=Section~}
\newref{sec}{name=section~,Name=Section~}
\newref{kap}{name=chapter~,Name=Chapter~}
\newref{zei}{name=line~,names=lines~,Name=Line~,Names=Lines~}
\newref{src}{name=source code~,names=source codes~,Name=Source Code~,Names=Source Codes~}
\newref{req}{name=requirement~,Name=Requirement~}
\newref{glei}{name=equation~,Name=Equation~}

% Farbdefinitionen
\definecolor{stateblue}{RGB}{173, 216, 230}      % Hellblau für normale Zustände
\definecolor{initialgreen}{RGB}{144, 238, 144}   % Hellgrün für Startzustand
\definecolor{gameoverred}{RGB}{240, 128, 128}    % Hellrot für Spielende
\definecolor{arrowblack}{RGB}{0, 0, 0}           % Schwarz für Pfeile
\definecolor{textblack}{RGB}{0, 0, 0}            % Schwarz für Text
\definecolor{white}{RGB}{255, 255, 255}          % Weiß für Label-Hintergrund


%Abkürzungen
\newcommand{\true}{\texttt{True}}
\newcommand{\false}{\texttt{False}}
\newcommand{\forSchleife}{\texttt{for}-loop}
\newcommand{\whileSchleife}{\texttt{while}-loop}
\newcommand{\NoLinkColor}{\hypersetup{allcolors=.}}

\newcommand{\myindex}[2]{\setindex{#1}\index{#2}\setindex{main}}

\definecolor{randyellowback}{RGB}{255,249,230}  % sehr helles Pastellgelb
\definecolor{randyellowtext}{RGB}{110,90,40}    % warmes Dunkelgelb/Braun

\tcbset{
	randnote/.style={
		enhanced,
		frame hidden,
		colback=randyellowback,
		%coltext=randyellowtext,
		boxrule=0pt,
		arc=1.5mm,
		left=1.0mm,
		right=1.0mm,
		top=1mm,
		bottom=1mm,
		boxsep=0pt,
		width=1.5cm,      % passend zu deiner bisherigen Parbox
		fontupper=\tiny,  % Randnotiz-Schrift
	},
}



\newcommand{\randnotiz}[1]{%
	\marginline{%
		\begin{tcolorbox}[randnote]%
			#1%
		\end{tcolorbox}%
	}%
}


%Requirement
\floatstyle{ruled}
\newfloat{Requirement}{H}{lor}
\newcommand{\br}[2]%
        {\begin{Requirement}%
                        \caption[#1]{\parbox[c][1.5ex][c]{7cm}{#1\color{White}gD}}\label{#2}\index{#1}%
                        \begin{quote}%
                                \begin{itshape}%
                                \vspace{-0.7em}
        }
\newcommand{\er}%
        {%
                                \end{itshape}%
                        \end{quote}%
                        \vspace{-1.5ex}
        \end{Requirement}%
        \vspace{-1.5ex}
        }



%Definition
\floatstyle{ruled}
\newfloat{Definition}{H}{lod}
\newcommand{\bd}[2]%
        {\begin{Definition}%
                        \caption[#1]{\parbox[c][1.5ex][c]{7cm}{#1\color{White}gD}}\label{#2}\index{#1}%
                        \begin{quote}%
                                \begin{itshape}%
                                \vspace{0.4em}
        }
\newcommand{\ed}%
        {%
                                \end{itshape}%
                        \end{quote}%
                        \vspace{-1.5ex}
        \end{Definition}%
        \vspace{-1.5ex}
        }
%Source Code Listen
\floatstyle{ruled}
\newfloat{Source Code}{H}{losc}
%% lstSource ist die neue Version, sollte immer verwendet werden!
%#1 Dateiname
%#2 Startzeile
%#3 Endezeile
%#4 Sprache
%#5 Überschrift
%#6 Label
\newcounter{StartZeilenNr}%
\newcommand{\lstsource}[6]%
{%
  \setcounter{StartZeilenNr}{#2}%
  \addtocounter{StartZeilenNr}{0}%
  \lstset%
  {%
    language={#4},%
    numbers={left},%
    numberstyle=\color[gray]{0.5},%
    escapechar={§},%
    basicstyle={\ttfamily\scriptsize},%
    firstnumber={#2},%
    breaklines=true,%
    breakatwhitespace=true,%
    frame={tb},%
    framextopmargin={1mm},%
    xleftmargin={0mm},%
    framexleftmargin={0mm},%
    belowcaptionskip={5pt},%
    abovecaptionskip={4mm},%
    keywordstyle={\color{blue}},%
    commentstyle={\color{ForestGreen}},%
    stringstyle={\color{BrickRed}},%
    showspaces=false,%
    showstringspaces=false,%
    literate=%
    {Ö}{{\"O}}1%
    {Ä}{{\"A}}1%
    {Ü}{{\"U}}1%
    {ß}{{\ss}}1%
    {ü}{{\"u}}1%
    {ä}{{\"a}}1%
    {ö}{{\"o}}1%
  }%
  \lstinputlisting[caption={#5},label={#6},firstline={#2},lastline={#3}]{#1}%
}%

%Aufgaben
\newtheorem{aufgabe}{Exercise}[chapter]


\newcommand{\myfigure}[4]{%
\renewcommand{\figurename}{Fig.}
 \includegraphics[scale=#2,clip=true]{#1}%
 \caption{#3}\label{#4}%
\renewcommand{\figurename}{Figure}
}

%Bilder
\newenvironment{ebild}[4]
{
	\begin{figure}[hbtp]
		\begin{center}%
			\includegraphics[scale=#2,clip=true]{#1}%
			\caption{#3}\label{#4}%
		\end{center}%
	}
	{
	\end{figure}
}
% #1 = file
% #2 = scale
% #3 = caption
% #4 = label
\newcommand{\myebild}[4]{\begin{ebild}{#1}{#2}{#3}{#4} \end{ebild}}


\newenvironment{ezweihbild}[8]
        {%
                \begin{figure}[hbtp]%
                        \centering%
                        \begin{minipage}[b]{6.5cm}%
                                \centering%
                                \includegraphics[scale=#2]{#1}%
                                \caption{#3}\label{#4}%
                        \end{minipage}%
                \hfil%
                        \begin{minipage}[b]{6.5cm}%
                                \centering%
                                \includegraphics[scale=#6]{#5}%
                                \caption{#7}\label{#8}%
                        \end{minipage}%
        }
        {
                \end{figure}%
        }
\newcommand{\myezweihbild}[8]{\begin{ezweihbild}{#1}{#2}{#3}{#4}{#5}{#6}{#7}{#8} \end{ezweihbild}}

\newenvironment{ezweivbild}[8]
        {%
                \begin{figure}[hbtp]%
                        \centering%
                        \begin{minipage}[b]{13cm}%
                                \centering%
                                \includegraphics[scale=#2]{#1}%
                                \caption{#3}\label{#4}%
                        \end{minipage}%
                \vfil%
                        \begin{minipage}[b]{13cm}%
                                \centering%
                                \includegraphics[scale=#6]{#5}%
                                \caption{#7}\label{#8}%
                        \end{minipage}%
        }
        {
                \end{figure}%
        }
\newcommand{\myezweivbild}[8]{\begin{ezweivbild}{#1}{#2}{#3}{#4}{#5}{#6}{#7}{#8} \end{ezweivbild}}

\DeclareDocumentCommand{\newdualentry}{ O{} O{} m m m m } {
  \newglossaryentry{gls-#3}{name={#5},text={#5\glsadd{#3}},
    description={#6},#1
  }
  \makeglossaries
  \newacronym[see={[Glossar:]{gls-#3}},#2]{#3}{#4}{#5\glsadd{gls-#3}}
}


% Minted
%\setminted{
%	fontsize=\small,
%	breaklines=true,
%	autogobble=true,
%	tabsize=4,
%	frame=lines,
%	framesep=2mm
%}