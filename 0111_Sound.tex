% !TeX spellcheck = en_US
\newpage
%%%%%%%%%%%%%%%%%%%%%%%%%%%%%%%%%%%%%%%%%%%%%%%%%%%%%%%%%%%%%%%%%%%%%%%%%%%
\section{Sound}\index{Sound}
Without background sounds and/or music, many games would simply be boring.  
Therefore, I~would like to present three different topics here: background music or ambient sounds, sound events, and stereo effects.

%%%%%%%%%%%%%%%%%%%%%%%%%%%%%%%%%%%%%%%%%%%%%%%%%%%%%%%%%%%%%%%%%%%%%%%%%%%
\subsection{Introduction}

%%%%%%%%%%%%%%%%%%%%%%%%%%%%%%%%%%%%%%%%%%%%%%%%%%%%%%%%%%%%%%%%%%%%%%%%%%%
\subsubsection{Sound: Music}\index{sound!music}
\begin{diskbox}
	\url{https://github.com/adamsralf/pygame_book/tree/main/src/00%20Introduction/11%20Sound/music}
\end{diskbox}

The first example covers the following features:
\begin{itemize}
	\item Background music is loaded and played in an endless loop.
	\item The volume can be adjusted using the mouse wheel.
	\item Pressing \keys{p} pauses the background music or resumes playback.
	\item Pressing \keys{j} fades out the background music.
\end{itemize}

I~will not explain the imports, \texttt{config.py}, and the other familiar building blocks in detail anymore, as they have already appeared many times before.

\lstsource{SRC/00 Introduction/11 Sound/music/config.py}{1}{28}{python}{Sound -- \texttt{config.py}}{srcSound00a} 

Before sound can be used, the corresponding subsystem must be initialized. This can be done explicitly using \texttt{pygame.mixer.init()}\randnotiz{init()}\myindex{pyg}{\texttt{mixer}!\texttt{init()}|underline}, or implicitly -- as in the source code at \zeiref{srcSound0002} -- by calling \texttt{pygame.init()}\myindex{pyg}{\texttt{init()}}.  

In the \texttt{sounds()} method, the preparatory steps for sound output are encapsulated. Background music\randnotiz{background music}\index{background music} is loaded into the mixer's internal memory using \texttt{pygame\-.mixer\-.music\-.load()}\myindex{pyg}{\texttt{mixer}!\texttt{music}!\texttt{load()}|underline}. However, loading the music does not start playback yet.  

Playback begins after the volume has been set in \zeiref{srcSound0004} using\randnotiz{set\_volume()} \texttt{pygame\-.mixer\-.music\-.set\-\_volume()}\myindex{pyg}{\texttt{mixer}!\texttt{music}!\texttt{set\_volume()}|underline}, by calling the method in \zeiref{srcSound0005}. The method \texttt{pygame\-.mixer\-.music\-.play()}\myindex{pyg}{\texttt{mixer}!\texttt{music}!\texttt{play()}|underline}\randnotiz{play()} accepts three parameters: 
\begin{itemize}
	\item The first parameter, \texttt{loops}, controls the number of repetitions; a value of~$-1$ means that the music is repeated indefinitely. 
	\item The second parameter, \texttt{start}, specifies the position at which playback should begin; the default is~$0.0$. 
	\item If the music should start quietly and then become louder (\gls{fade}\index{fade}\randnotiz{fade}), this can be achieved using the third parameter \texttt{fade}. Here, you can specify how many milliseconds are available for the fade-in; if nothing is specified, playback starts immediately at the target volume.
\end{itemize}

\lstsource{SRC/00 Introduction/11 Sound/music/sound.py}{9}{23}{python}{Sound -- Constructor and \texttt{sounds()} of \texttt{Game}}{srcSound00b} 

 The \texttt{watch\_for\_events()} method acts purely as a dispatcher. Depending on which key is pressed or which mouse element is used, the corresponding helper methods are called.

\lstsource{SRC/00 Introduction/11 Sound/music/sound.py}{25}{43}{python}{Sound -- \texttt{Game.watch\_for\_events()}}{srcSound00c} 

I~want to start the background music at some times and fade it out at others. This is handled by the helper method \texttt{music\_start\_stop()}. The background music is stopped using \texttt{pygame.mixer.music.fadeout()}\myindex{pyg}{\texttt{mixer}!\texttt{music}!\texttt{fadeout()}|underline}\randnotiz{fadeout()}. Here, you have to specify over how many milliseconds the music should gradually become quieter until it stops -- in our example, this is~\SI{5000}{ms}. The method \texttt{pygame.mixer.music.play()}\myindex{pyg}{\texttt{mixer}!\texttt{music}!\texttt{play()}} used to start the background music has already been explained above.

\lstsource{SRC/00 Introduction/11 Sound/music/sound.py}{46}{50}{python}{Sound -- \texttt{Game.music\_start\_stop()}}{srcSound00e} 

Pressing \keys{p} pauses the background music or resumes playback. The current state is stored in the attribute \texttt{pause}. This attribute then determines which of the two \texttt{music} methods is called in the \texttt{pause\_alter()} method -- either  \texttt{pygame.mixer.music.pause()}\myindex{pyg}{\texttt{mixer}!\texttt{music}!\texttt{pause()}|underline}\randnotiz{pause()} or \texttt{pygame.mixer.music.unpause()}\myindex{pyg}{\texttt{mixer}!\texttt{music}!\texttt{unpause()}|underline}\randnotiz{unpause()}. Finally, in \zeiref{srcSound0007}, the \texttt{pause} flag is toggled.

\lstsource{SRC/00 Introduction/11 Sound/music/sound.py}{52}{57}{python}{Sound -- \texttt{Game.pause\_alter()}}{srcSound00f} 

As the final feature, volume control is introduced. It is encapsulated in the \texttt{volume\-\_alter()} method. Instead of passing an absolute volume value to this method, a delta value is provided.

First, this value is added to the \texttt{volume} variable. Afterwards, the value is clamped to the interval $[0, 1]$, and finally the new volume is set using \texttt{pygame.mixer.music.set\-\_volume()}\myindex{pyg}{\texttt{mixer}!\texttt{music}!\texttt{set\_volume()}|underline}\randnotiz{set\_volume()}.

\lstsource{SRC/00 Introduction/11 Sound/music/sound.py}{59}{63}{python}{Sound -- \texttt{volume\_alter()} von \texttt{Game}}{srcSound00g} 

And finally, we deal with the remaining bits.

\lstsource{SRC/00 Introduction/11 Sound/music/sound.py}{65}{85}{python}{Sound -- \texttt{draw()}, \texttt{update()}, \texttt{run()} of \texttt{Game}}{srcSound00h} 

%%%%%%%%%%%%%%%%%%%%%%%%%%%%%%%%%%%%%%%%%%%%%%%%%%%%%%%%%%%%%%%%%%%%%%%%%%%
\subsubsection{Sound: Events}\index{sound!event}
\begin{diskbox}
	\url{https://github.com/adamsralf/pygame_book/tree/main/src/00%20Introduction/11%20Sound/effects}
\end{diskbox}

For sound effects\index{sound effects}\randnotiz{sound effects}, a separate \texttt{Sound} object is created in each case (\zeiref{srcSound0101}ff.). The constructor of \texttt{pygame.mixer.Sound}\myindex{pyg}{\texttt{mixer}!\texttt{Sound}|underline} is given the file name including the path. If you already have an open file object, you can pass that instead; in this case, however, you should provide a second parameter specifying the sound encoding, for example \texttt{.\gls{ogg}}\index{ogg} or \texttt{.\gls{mp3}}\index{mp3}. As with background music, loading a sound is not the same as playing it.

\lstsource{SRC/00 Introduction/11 Sound/effects/sound.py}{19}{21}{python}{Sound -- \texttt{Game.sound()}}{srcSound01a} 

In \zeiref{srcSound0106}, the current volume is first stored in a variable using
\texttt{pygame\-.mixer\-.Sound\-.get\_\-volume()}\randnotiz{get\_volume()}\myindex{pyg}{\texttt{mixer}!\texttt{Sound}!\texttt{get\_volume()}}. To ensure that both sounds are played at the same volume, this value is modified and then applied to both sounds using
\texttt{pygame.mixer.Sound.set\_volume()}\randnotiz{set\_volume()}\myindex{pyg}{\texttt{mixer}!\texttt{Sound}!\texttt{set\_volume()}}.

\newpage
\lstsource{SRC/00 Introduction/11 Sound/effects/sound.py}{40}{45}{python}{Sound -- \texttt{Game.volume\_alter()}}{srcSound01b} 

For simplicity, the sounds are played directly in \texttt{watch\_for\_events()} (see \zeiref{srcSound0103} and \zeiref{srcSound0104}). The actual playback is done using \texttt{pygame.mixer.Sound.play()}\myindex{pyg}{\texttt{mixer}!\texttt{Sound}!\texttt{play()}|underline}\randnotiz{play()}. You can see that the \texttt{play()} method is called on the corresponding \texttt{Sound} object.

The \texttt{play()} method provides three optional arguments:
\begin{itemize}
	\item \texttt{loops}: number of repetitions ($-1$ means infinite playback, default)
	\item \texttt{maxtime}: maximum playback time in milliseconds ($0$ means unlimited, default)
	\item \texttt{fade\_ms}: duration of the fade-in in milliseconds (default: $0$)
\end{itemize}

If -- as in this case --no arguments are provided, the sound starts playing immediately and stops automatically after it has finished playing. Any other sounds that are currently being played by other \texttt{Sound} objects are not interrupted. This means that multiple sounds can be played at the same time.

\lstsource{SRC/00 Introduction/11 Sound/effects/sound.py}{23}{38}{python}{Sound -- \texttt{Game.watch\_for\_events()()}}{srcSound01c} 


%%%%%%%%%%%%%%%%%%%%%%%%%%%%%%%%%%%%%%%%%%%%%%%%%%%%%%%%%%%%%%%%
\newpage
\subsection{More Input}

%%%%%%%%%%%%%%%%%%%%%%%%%%%%%%%%%%%%%%%%%%%%%%%%%%%%%%%%%%%%%%%%
\subsubsection{Stereo}\label{secStereo}
\begin{diskbox}
	\url{https://github.com/adamsralf/pygame_book/tree/main/src/00%20Introduction/11%20Sound/stereo}
\end{diskbox}

A small example is intended to illustrate the use of channels and \gls{stereo} effects. The topic is too extensive to be presented in full detail, but I~hope that this chapter provides a helpful introduction.

In \abbref[vref]{picStereo00}, you can see a tank driving from left to right or from right to left. While driving, it can fire up to 5~shots. It would be nice if the driving sound indicated acoustically where the tank is currently located. That is, if the tank is more to the right, the driving sound or the shot should be louder on the right speaker than on the left speaker (\gls{stereopanning}\randnotiz{stereo panning}). When driving from right to left, the driving sound would therefore move along with the tank.

\myebild{stereo00.png}{0.7}{Example Stereo Sound}{picStereo00}

First, the necessary boilerplate, which should not require any further explanation:

\lstsource{SRC/00 Introduction/11 Sound/stereo/config.py}{1}{43}{python}{Sound-Stereo -- \texttt{config.py}}{srcSound01aa} 

\lstsource{SRC/00 Introduction/11 Sound/stereo/sound.py}{12}{23}{python}{Sound-Stereo -- Class \texttt{Ground}}{srcSound01ab} 

In \zeiref{srcSound0101}, a \texttt{Sound} object\myindex{pyg}{\texttt{mixer}!\texttt{Sound}}\randnotiz{Sound object} is created. This object is played to emphasize the movement of the tank with appropriate sounds. In the following line (\zeiref{srcSound0201}), the helper method \texttt{stereo()} is called (see below), and then the playback of the driving sound starts in an infinite loop (\zeiref{srcSound0203}). 

It is noticeable that the output is not started using \texttt{pygame.mixer.Sound.play()}\myindex{pyg}{\texttt{mixer}!\texttt{Sound}!\texttt{play()}}. Normally, this would be a good choice, since this command automatically selects one of the eight available Pygame sound channels\index{channel}\randnotiz{channel}.

\begin{hintbox}[Selecting channels manually]
	However, it is also possible to address a Pygame channel directly and thus gain more control over the sound behavior. In \zeiref{srcSound0204}, a free \texttt{pygame.mixer.Channel} object\myindex{pyg}{\texttt{mixer}!\texttt{Channel}|underline} is determined for this purpose. The method \texttt{pygame.mixer.find\-\_chan\-nel()} \myindex{pyg}{\texttt{mixer}!\texttt{find\_channel()}|underline} returns the first pygame channel and stores it in the attribute \texttt{channel}.
\end{hintbox} 

Playback in \zeiref{srcSound0203} is then no longer started via a method of the \texttt{Sound} object, but by using \texttt{pygame.mixer.Channel.play()}\myindex{pyg}{\texttt{mixer}!\texttt{Channel}!\texttt{play()}|underline}\randnotiz{play()}.

This makes it possible to adjust volume and stereo panning dynamically while the sound is playing.

\newpage
\lstsource{SRC/00 Introduction/11 Sound/stereo/sound.py}{26}{50}{python}{Sound-Stereo -- Constructor of \texttt{Tank}}{srcSound02b} 

The \texttt{update()} method is shown here only for completeness. It does not contain any code related to sound playback.

\lstsource{SRC/00 Introduction/11 Sound/stereo/sound.py}{52}{75}{python}{Sound-Stereo -- \texttt{Tank.update()}}{srcSound02c} 

The \texttt{stereo()} method is surprisingly simple. The method \texttt{pygame.mixer.Channel\-.set-\_vol\-ume()}\myindex{pyg}{\texttt{mixer}!\texttt{Channel}!\texttt{set\_volume()}|underline}\randnotiz{set\_volume()} provides two parameters: \emph{left} and \emph{right}. Both parameters have a value range of~$[0, 1]$.

As discussed before, we want the right speaker to play the engine sound louder the further to the right the tank is positioned, and vice versa (\gls{stereopanning}) \randnotiz{Stereo Panning}. To achieve this, I~calculate the relative horizontal position of the tank’s center with respect to the window width in \zeiref{srcSound0203}. This calculation also yields a value in the interval~$[0, 1]$.

Once this value is known, the relative value for the left speaker can be determined in the following line by $left = 1 - right $ . After that, both values are passed to the \texttt{set\_volume()} method.

\begin{hintbox}[Hint]
	The method \texttt{pygame.mixer.Channel\-.set\-\_vol\-ume()} allows different volume levels to be specified for the left and right Pygame channels, whereas the methods \texttt{pygame\-.mix\-er\-.Sound\-.set\-\_vol\-ume()}\myindex{pyg}{\texttt{mixer}!\texttt{Sound}!\texttt{set\_volume()}} and \texttt{pygame.mixer.music\-.set\-\_vol\-ume()}\myindex{pyg}{\texttt{mixer}!\texttt{music}!\texttt{set\_volume()}} do not. 
\end{hintbox}
 
 \lstsource{SRC/00 Introduction/11 Sound/stereo/sound.py}{77}{80}{python}{Sound-Stereo -- \texttt{Tank.stereo()}}{srcSound02d} 

What else could this effect be used for? For example, think of two people talking to each other, sound sources in a room, and so on. Whenever audio is meant to make localization easier, or when individual sounds should stand out or be easier to distinguish, different volume levels -- i.e., stereo -- are a good option.

Nothing related to sound output happens in \texttt{turn()} and \texttt{update\_imageindex()}.

\lstsource{SRC/00 Introduction/11 Sound/stereo/sound.py}{82}{89}{python}{Sound-Stereo -- \texttt{Tank.turn()} and \texttt{Tank.update\_imageindex()}}{srcSound02e} 

The sound output of the \texttt{Bullet} could also have been implemented in the \texttt{Tank} class. However, I~find it more natural to place it in \texttt{Bullet}. After all, it might later be extended to include an impact sound or an explosion.

Before the constructor, the static variable \texttt{sound\_fire} is defined in \zeiref{srcSound0206}. Although there are many bullets, they all use the same firing sound. Reading this sound file repeatedly and creating a new object each time would therefore waste memory and reduce performance. Instead, starting at \zeiref{srcSound0207}, a kind of \gls{singleton} check is performed. This ensures that the sound file is read and the corresponding object is created exactly once.

After that, a free channel is searched for, just as with the tank, and the volume of the left and right speakers is determined based on the position.
Finally, the sound is played based on the horizontal position of the bullet.

\lstsource{SRC/00 Introduction/11 Sound/stereo/sound.py}{92}{126}{python}{Sound-Stereo -- Class \texttt{Bullet}}{srcSound02g} 

The remaining source code is shown here only for the sake of completeness.

\lstsource{SRC/00 Introduction/11 Sound/stereo/sound.py}{129}{186}{python}{Sound-Stereo -- Rest}{srcSound02h} 


%%%%%%%%%%%%%%%%%%%%%%%%%%%%%%%%%%%%%%%%%%%%%%%%%%%%%%%%%%%%%%%%
\subsubsection{Sound Formats and Technical Basics}

Pygame does not support all audio formats equally well. The following formats have proven to be particularly reliable:

\begin{itemize}
	\item[\texttt{.wav}] -- uncompressed, fast to load, ideal for sound effects
	\item[\texttt{.ogg}] -- compressed, well suited for music
	\item[\texttt{.mp3}] -- limited support, not recommended
\end{itemize}

\newpage
\begin{warningbox}[Don'ts]
	\begin{itemize}
		\item Mono sounds are often used for sound effects because they require less memory and can be positioned spatially more effectively.
		\item Large files and high sample rates can negatively affect loading times and performance.
	\end{itemize}
\end{warningbox}

%%%%%%%%%%%%%%%%%%%%%%%%%%%%%%%%%%%%%%%%%%%%%%%%%%%%%%%%%%%%%%%%
\subsubsection{Volume Hierarchies and Sound Mixing}

Games often use multiple volume levels:
\begin{enumerate}
	\item Master volume (everything)
	\item Music volume
	\item Effect volume
\end{enumerate}

These can be combined:
\lstset{firstnumber=1}
\begin{lstlisting}
	master_volume = 0.8
	effects_volume = 0.5
	
	sound.set_volume(master_volume * effects_volume)
\end{lstlisting}

This makes it easy to implement audio settings for game menus later on.

%%%%%%%%%%%%%%%%%%%%%%%%%%%%%%%%%%%%%%%%%%%%%%%%%%%%%%%%%%%%%%%%
\subsubsection{Mono Sounds and Stereo Panning}

Mono sounds are particularly suitable for position-dependent audio. Only mono sounds can be cleanly distributed between the left and right speakers.

\lstset{firstnumber=1}
\begin{lstlisting}
	channel.set_volume(left, right)
\end{lstlisting}

\begin{warningbox}[Don'ts]
	Stereo sounds already contain spatial information and may produce unexpected results when additional panning is applied.
\end{warningbox}

%%%%%%%%%%%%%%%%%%%%%%%%%%%%%%%%%%%%%%%%%%%%%%%%%%%%%%%%%%%%%%%%
\newpage
\subsubsection{Sound Lifetime and Resource Management}

\begin{hintbox}
	Sounds should not be reloaded for every event. Instead, they should be loaded once and reused.
\end{hintbox}

\lstset{firstnumber=1}
\begin{lstlisting}
	class Bullet:
		sound_fire = None
	
		def __init__(self):
		if sound_fire is None:
			sound_fire = pygame.mixer.Sound("fire.wav")
\end{lstlisting}


This saves memory and avoids unnecessary loading times.

%%%%%%%%%%%%%%%%%%%%%%%%%%%%%%%%%%%%%%%%%%%%%%%%%%%%%%%%%%%%%%%%
\subsubsection{Event-driven Sound Output}
\begin{hintbox}
	Sounds should be played in an event-driven manner, not frame-based.
\end{hintbox}

It is incorrect to call \texttt{sound.play()} directly or indirectly inside the update loop. Instead, sounds should be triggered by events:

\lstset{firstnumber=1}
\begin{lstlisting}
	elif event.key == K_SPACE:
		sound.play()
\end{lstlisting}


%%%%%%%%%%%%%%%%%%%%%%%%%%%%%%%%%%%%%%%%%%%%%%%%%%%%%%%%%%%%%%%%
\subsubsection{Looping and Transitions}

Loops are used for continuous sounds (engines, wind, music):\index{sound!endless loops} 

\lstset{firstnumber=1}
\begin{lstlisting}
	channel.play(sound, loops=-1)
\end{lstlisting}

Smooth transitions can be achieved using fade-in and fade-out effects:\index{sound!fade-in}\index{sound!fade-out}

\lstset{firstnumber=1}
\begin{lstlisting}
	sound.fadeout(1000)   # 1 second
\end{lstlisting}


%%%%%%%%%%%%%%%%%%%%%%%%%%%%%%%%%%%%%%%%%%%%%%%%%%%%%%%%%%%%%%%%
\subsubsection{Muting and Pausing}

Many games offer an option to mute or pause sound globally.

\lstset{firstnumber=1}
\begin{lstlisting}
	pygame.mixer.pause()
	...
	pygame.mixer.unpause()
\end{lstlisting}


Alternatively, this can be done via volume control:

\lstset{firstnumber=1}
\begin{lstlisting}
	pygame.mixer.music.set_volume(0)
\end{lstlisting}


This is especially important for pause menus or when the game window loses focus.

%%%%%%%%%%%%%%%%%%%%%%%%%%%%%%%%%%%%%%%%%%%%%%%%%%%%%%%%%%%%%%%%
\subsubsection{Typical Errors and Debugging}

\begin{warningbox}[Common problems with sound in Pygame include:]
\begin{itemize}
	\item Mixer not initialized
	\item Sound played too frequently
	\item No free channels available
	\item Distorted sound (incorrect format)
	\item Meaningless increase of the number of sound channels, such as \texttt{pygame\-.mixer\-.set\-\_num\-\_channels(16)}
\end{itemize}
\end{warningbox}



Checking the current state often helps:

\lstset{firstnumber=1}
\begin{lstlisting}
	print(pygame.mixer.music.get_busy())
\end{lstlisting}
%%%%%%%%%%%%%%%%%%%%%%%%%%%%%%%%%%%%%%%%%%%%%%%%%%%%%%%%%%%%%%%%
\subsection{What was new?}
\begin{hintbox}
	Two options are available for sound support. One option is background music, while the other uses individual sounds played on different channels and, if possible, distributed across the left and right speakers.
\end{hintbox}

\begin{pygbox}
\begin{itemize}
	\item\texttt{pygame.mixer.Channel} :
	\myindex{pyg}{\texttt{mixer}!\texttt{Channel}}\\ \url{https://pyga.me/docs/ref/music.html#pygame.mixer.Channel}

	\item \texttt{pygame.mixer.Channel.play()}:
	\myindex{pyg}{\texttt{mixer}!\texttt{Channel}!\texttt{play()}}\\ \url{https://pyga.me/docs/ref/mixer.html#pygame.mixer.Channel.play}

	\item \texttt{pygame.mixer.Channel.set\_volume()}:
	\myindex{pyg}{\texttt{mixer}!\texttt{Channel}!\texttt{set\_volume()}}\\ \url{https://pyga.me/docs/ref/mixer.html#pygame.mixer.Channel.set_volume}

	\item\texttt{pygame.mixer.find\_channel()} :
	\myindex{pyg}{\texttt{mixer}!\texttt{find\_channel()}}\\ \url{https://pyga.me/docs/ref/music.html#pygame.mixer.find_channel}

	\item \texttt{pygame.mixer.init()}:
	\myindex{pyg}{\texttt{mixer}!\texttt{init()}}\\ \url{https://pyga.me/docs/ref/mixer.html#pygame.mixer.init}

	\item \texttt{pygame.mixer.set\_num\_channels()}:
	\myindex{pyg}{\texttt{mixer}!\texttt{set\_num\_channels()}}\\ \url{https://pyga.me/docs/ref/mixer.html#pygame.mixer.set_num_channels}

	\item\texttt{pygame.mixer.music.fadeout()}:
	\myindex{pyg}{\texttt{mixer}!\texttt{music}!\texttt{fadeout()}}\\ \url{https://pyga.me/docs/ref/music.html#pygame.mixer.music.fadeout}

	\item\texttt{pygame.mixer.music.get\_busy()} :
	\myindex{pyg}{\texttt{mixer}!\texttt{music}!\texttt{get\_busy()}}\\ \url{https://pyga.me/docs/ref/music.html#pygame.mixer.music.get_busy}

	\item \texttt{pygame.mixer.music.get\_volume()}:
	\myindex{pyg}{\texttt{mixer}!\texttt{music}!\texttt{get\_volume()}}\\ \url{https://pyga.me/docs/ref/music.html#pygame.mixer.music.get_volume}

    \item \texttt{pygame.mixer.music.load()}:
	\myindex{pyg}{\texttt{mixer}!\texttt{music}!\texttt{load()}}\\ \url{https://pyga.me/docs/ref/music.html#pygame.mixer.music.load}

	\item \texttt{pygame.mixer.music.pause()}:
	\myindex{pyg}{\texttt{mixer}!\texttt{music}!\texttt{pause()}}\\ \url{https://pyga.me/docs/ref/music.html#pygame.mixer.music.pause}

	\item \texttt{pygame.mixer.music.play()}:
	\myindex{pyg}{\texttt{mixer}!\texttt{music}!\texttt{play()}}\\ \url{https://pyga.me/docs/ref/music.html#pygame.mixer.music.play}

	\item \texttt{pygame.mixer.music.set\_volume()}:
	\myindex{pyg}{\texttt{mixer}!\texttt{music}!\texttt{set\_volume()}}\\ \url{https://pyga.me/docs/ref/music.html#pygame.mixer.music.set_volume}

	\item \texttt{pygame.mixer.music.unpause()}:
	\myindex{pyg}{\texttt{mixer}!\texttt{music}!\texttt{unpause()}}\\ \url{https://pyga.me/docs/ref/music.html#pygame.mixer.music.unpause}

	\item \texttt{pygame.mixer.Sound}:
	\myindex{pyg}{\texttt{mixer}!\texttt{Sound}}\\ \url{https://pyga.me/docs/ref/mixer.html#pygame.mixer.Sound}

	\item \texttt{pygame.mixer.Sound.get\_volume()}:
	\myindex{pyg}{\texttt{mixer}!\texttt{Sound}!\texttt{get\_volume()}}\\ \url{https://pyga.me/docs/ref/mixer.html#pygame.mixer.Sound.get_volume}

	\item \texttt{pygame.mixer.Sound.play()}:
	\myindex{pyg}{\texttt{mixer}!\texttt{Sound}!\texttt{play()}}\\ \url{https://pyga.me/docs/ref/mixer.html#pygame.mixer.Sound.play}

	\item \texttt{pygame.mixer.Sound.set\_volume()}:
	\myindex{pyg}{\texttt{mixer}!\texttt{Sound}!\texttt{set\_volume()}}\\ \url{https://pyga.me/docs/ref/mixer.html#pygame.mixer.Sound.set_volume}

\end{itemize}
\end{pygbox}
