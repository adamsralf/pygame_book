% !TeX spellcheck = en_US
\documentclass[12pt, twoside, openright]{scrbook}
\usepackage{showframe}

% -------------------------------------------------
% Language & Localization
% -------------------------------------------------
\usepackage[english]{babel}
\usepackage[english]{varioref}
\usepackage[english,nokeyprefix]{refstyle}
% -------------------------------------------------
% Typography & Fonts (LuaLaTeX)
% -------------------------------------------------
\usepackage{fontspec}
% -------------------------------------------------
% Colors & Graphics
% -------------------------------------------------
\usepackage[dvipsnames]{xcolor}
\usepackage{graphicx}
\usepackage{wrapfig}
\usepackage{flafter}
\usepackage{float}
% -------------------------------------------------
% Math & Units
% -------------------------------------------------
\usepackage{amsmath}
\usepackage{siunitx}
\sisetup{
	locale = US,
	detect-all,
	evaluate-expression = true
}
% -------------------------------------------------
% Code Listings
% -------------------------------------------------
%\usepackage[newfloat]{minted}
\usepackage{listings}
\usepackage{verbatim}
% -------------------------------------------------
% Enumerations
% -------------------------------------------------
\usepackage{enumerate}
% -------------------------------------------------
% Hyperlinks & PDF Metadata
% -------------------------------------------------
\usepackage[
	luatex,
	unicode=true,
	breaklinks=true,
	bookmarksnumbered=true,
	colorlinks,
	linkcolor=blue,
	linktocpage,
	pdfauthor={Ralf Adams},
	pdftitle={Introduction to Pygame-ce},
	pdfsubject={Game Development with Python and Pygame-ce},
	pdfkeywords={Python, Pygame, Game Development, Programming},
	pdfcreator={LuaLaTeX},
	pdfproducer={LaTeX}
]{hyperref}
% -------------------------------------------------
% Tables
% -------------------------------------------------
\usepackage{longtable}
\usepackage{booktabs}
% -------------------------------------------------
% Filebox
% -------------------------------------------------
\usepackage[most]{tcolorbox}
\usepackage{fontawesome5} % liefert \faSave (Diskette)

\usepackage[footsepline,plainfootsepline,footbotline,plainfootbotline,draft=false]{scrlayer-scrpage}
\usepackage{manyind}
\usepackage{colortbl}
\usepackage[font={small}]{caption}
\usepackage{lastpage}
\usepackage[acronym]{glossaries}
\usepackage{xparse}
\usepackage{tikz}
\usepackage{tkz-euclide}
\usetikzlibrary{arrows.meta}
\usetikzlibrary{shapes.geometric}
\usetikzlibrary{positioning}
\usetikzlibrary{calc}
\usetikzlibrary{angles}
\usetikzlibrary{quotes}
\usetikzlibrary{fit}
\usetikzlibrary{automata}
\usepackage{pgf-umlcd}
\usepackage{multirow}
\usepackage{colortbl}
\usepackage{chngcntr}
\usepackage[os=win]{menukeys}

\graphicspath{{./pics/}}
\selectlanguage{english}


\setcounter{secnumdepth}{3}
\setcounter{tocdepth}{3}
\deffootnote[1.0em]{1.1em}{1.1em}{\textsuperscript{\thefootnotemark\ }}

\newref{tab}{name=table~,names=tables~,Name=Table~,Names=Tables~}
\newref{abb}{name=figure~,Name=Figure~}
\newref{abschnitt}{name=section~,Name=Section~}
\newref{sec}{name=section~,Name=Section~}
\newref{kap}{name=chapter~,Name=Chapter~}
\newref{zei}{name=line~,names=lines~,Name=Line~,Names=Lines~}
\newref{src}{name=source code~,names=source codes~,Name=Source Code~,Names=Source Codes~}
\newref{req}{name=requirement~,Name=Requirement~}
\newref{glei}{name=equation~,Name=Equation~}

% Farbdefinitionen
\definecolor{stateblue}{RGB}{173, 216, 230}      % Hellblau für normale Zustände
\definecolor{initialgreen}{RGB}{144, 238, 144}   % Hellgrün für Startzustand
\definecolor{gameoverred}{RGB}{240, 128, 128}    % Hellrot für Spielende
\definecolor{arrowblack}{RGB}{0, 0, 0}           % Schwarz für Pfeile
\definecolor{textblack}{RGB}{0, 0, 0}            % Schwarz für Text
\definecolor{white}{RGB}{255, 255, 255}          % Weiß für Label-Hintergrund


%Abkürzungen
\newcommand{\true}{\texttt{True}}
\newcommand{\false}{\texttt{False}}
\newcommand{\forSchleife}{\texttt{for}-loop}
\newcommand{\whileSchleife}{\texttt{while}-loop}
\newcommand{\NoLinkColor}{\hypersetup{allcolors=.}}

\newcommand{\myindex}[2]{\setindex{#1}\index{#2}\setindex{main}}

\definecolor{randyellowback}{RGB}{255,249,230}  % sehr helles Pastellgelb
\definecolor{randyellowtext}{RGB}{110,90,40}    % warmes Dunkelgelb/Braun

\tcbset{
	randnote/.style={
		enhanced,
		frame hidden,
		colback=randyellowback,
		%coltext=randyellowtext,
		boxrule=0pt,
		arc=1.5mm,
		left=1.0mm,
		right=1.0mm,
		top=1mm,
		bottom=1mm,
		boxsep=0pt,
		width=1.5cm,      % passend zu deiner bisherigen Parbox
		fontupper=\tiny,  % Randnotiz-Schrift
	},
}



\newcommand{\randnotiz}[1]{%
	\marginline{%
		\begin{tcolorbox}[randnote]%
			#1%
		\end{tcolorbox}%
	}%
}


%Requirement
\floatstyle{ruled}
\newfloat{Requirement}{H}{lor}
\newcommand{\br}[2]%
        {\begin{Requirement}%
                        \caption[#1]{\parbox[c][1.5ex][c]{7cm}{#1\color{White}gD}}\label{#2}\index{#1}%
                        \begin{quote}%
                                \begin{itshape}%
                                \vspace{-0.7em}
        }
\newcommand{\er}%
        {%
                                \end{itshape}%
                        \end{quote}%
                        \vspace{-1.5ex}
        \end{Requirement}%
        \vspace{-1.5ex}
        }



%Definition
\floatstyle{ruled}
\newfloat{Definition}{H}{lod}
\newcommand{\bd}[2]%
        {\begin{Definition}%
                        \caption[#1]{\parbox[c][1.5ex][c]{7cm}{#1\color{White}gD}}\label{#2}\index{#1}%
                        \begin{quote}%
                                \begin{itshape}%
                                \vspace{0.4em}
        }
\newcommand{\ed}%
        {%
                                \end{itshape}%
                        \end{quote}%
                        \vspace{-1.5ex}
        \end{Definition}%
        \vspace{-1.5ex}
        }
%Source Code Listen
\floatstyle{ruled}
\newfloat{Source Code}{H}{losc}
%% lstSource ist die neue Version, sollte immer verwendet werden!
%#1 Dateiname
%#2 Startzeile
%#3 Endezeile
%#4 Sprache
%#5 Überschrift
%#6 Label
\newcounter{StartZeilenNr}%
\newcommand{\lstsource}[6]%
{%
  \setcounter{StartZeilenNr}{#2}%
  \addtocounter{StartZeilenNr}{0}%
  \lstset%
  {%
    language={#4},%
    numbers={left},%
    numberstyle=\color[gray]{0.5},%
    escapechar={§},%
    basicstyle={\ttfamily\scriptsize},%
    firstnumber={#2},%
    breaklines=true,%
    breakatwhitespace=true,%
    frame={tb},%
    framextopmargin={1mm},%
    xleftmargin={0mm},%
    framexleftmargin={0mm},%
    belowcaptionskip={5pt},%
    abovecaptionskip={4mm},%
    keywordstyle={\color{blue}},%
    commentstyle={\color{ForestGreen}},%
    stringstyle={\color{BrickRed}},%
    showspaces=false,%
    showstringspaces=false,%
    literate=%
    {Ö}{{\"O}}1%
    {Ä}{{\"A}}1%
    {Ü}{{\"U}}1%
    {ß}{{\ss}}1%
    {ü}{{\"u}}1%
    {ä}{{\"a}}1%
    {ö}{{\"o}}1%
  }%
  \lstinputlisting[caption={#5},label={#6},firstline={#2},lastline={#3}]{#1}%
}%

%Aufgaben
\newtheorem{aufgabe}{Exercise}[chapter]


\newcommand{\myfigure}[4]{%
\renewcommand{\figurename}{Fig.}
 \includegraphics[scale=#2,clip=true]{#1}%
 \caption{#3}\label{#4}%
\renewcommand{\figurename}{Figure}
}

%Bilder
\newenvironment{ebild}[4]
{
	\begin{figure}[hbtp]
		\begin{center}%
			\includegraphics[scale=#2,clip=true]{#1}%
			\caption{#3}\label{#4}%
		\end{center}%
	}
	{
	\end{figure}
}
% #1 = file
% #2 = scale
% #3 = caption
% #4 = label
\newcommand{\myebild}[4]{\begin{ebild}{#1}{#2}{#3}{#4} \end{ebild}}


\newenvironment{ezweihbild}[8]
        {%
                \begin{figure}[hbtp]%
                        \centering%
                        \begin{minipage}[b]{6.5cm}%
                                \centering%
                                \includegraphics[scale=#2]{#1}%
                                \caption{#3}\label{#4}%
                        \end{minipage}%
                \hfil%
                        \begin{minipage}[b]{6.5cm}%
                                \centering%
                                \includegraphics[scale=#6]{#5}%
                                \caption{#7}\label{#8}%
                        \end{minipage}%
        }
        {
                \end{figure}%
        }
\newcommand{\myezweihbild}[8]{\begin{ezweihbild}{#1}{#2}{#3}{#4}{#5}{#6}{#7}{#8} \end{ezweihbild}}

\newenvironment{ezweivbild}[8]
        {%
                \begin{figure}[hbtp]%
                        \centering%
                        \begin{minipage}[b]{13cm}%
                                \centering%
                                \includegraphics[scale=#2]{#1}%
                                \caption{#3}\label{#4}%
                        \end{minipage}%
                \vfil%
                        \begin{minipage}[b]{13cm}%
                                \centering%
                                \includegraphics[scale=#6]{#5}%
                                \caption{#7}\label{#8}%
                        \end{minipage}%
        }
        {
                \end{figure}%
        }
\newcommand{\myezweivbild}[8]{\begin{ezweivbild}{#1}{#2}{#3}{#4}{#5}{#6}{#7}{#8} \end{ezweivbild}}

\DeclareDocumentCommand{\newdualentry}{ O{} O{} m m m m } {
  \newglossaryentry{gls-#3}{name={#5},text={#5\glsadd{#3}},
    description={#6},#1
  }
  \makeglossaries
  \newacronym[see={[Glossar:]{gls-#3}},#2]{#3}{#4}{#5\glsadd{gls-#3}}
}


% Minted
%\setminted{
%	fontsize=\small,
%	breaklines=true,
%	autogobble=true,
%	tabsize=4,
%	frame=lines,
%	framesep=2mm
%}
%\lstloadlanguages{C}
%\renewcommand{\lstlistingname}{Source Code}
%\renewcommand{\lstlistlistingname}{Quelltexte}

\clearpairofpagestyles
\clearplainofpairofpagestyles
\automark[section]{chapter}
\setkomafont{pagehead}{\small\sffamily\bfseries}
\setkomafont{pagefoot}{\small\sffamily}
\renewcommand*{\chapterpagestyle}{scrheadings}


\tcbset{
	page/header/.style={
		enhanced,
		frame hidden,
		colback=headblue,
		coltext=white,
		boxrule=0pt,
		arc=0mm,
		left=6mm,
		right=6mm,
		top=1.5mm,
		bottom=1.5mm,
%		width=\paperwidth,
		enlarge left by=-\hoffset,
		enlarge right by=\hoffset,
	},
	page/footer/.style={
		enhanced,
		frame hidden,
		colback=headblue,
		coltext=white,
		boxrule=0pt,
		arc=0mm,
		left=6mm,
		right=6mm,
		top=1.2mm,
		bottom=1.2mm,
%		width=\paperwidth,
		enlarge left by=-\hoffset,
		enlarge right by=\hoffset,
	},
}



% ---------------- HEADER ----------------
\newcommand{\tcbheader}[1]{%
	\begin{tcolorbox}[page/header]%
		\hspace{0.1em}#1\hspace{0.1em}%
	\end{tcolorbox}%
}
\ihead[\tcbheader{\rightmark}]{\tcbheader{\rightmark}}
\ohead[\tcbheader{\leftmark}]{\tcbheader{\leftmark}}


\newcommand{\tcbfooter}[2]{%
	\begin{tcolorbox}[page/footer]%
		\NoLinkColor
		#1\hfill #2%
	\end{tcolorbox}%
}
% ---------------- FOOTER ----------------
\ifoot{\tcbfooter{2D Game Programming with Pygame-ce}{Page \pagemark\ of \pageref{LastPage}\hspace{0.1em}}}
\ofoot{\tcbfooter{Page \pagemark\ of \pageref{LastPage}}{2D Game Programming with Pygame-ce\hspace{0.1em}}}

\pagestyle{scrheadings}

% -------------------------------------------------
% Header
% -------------------------------------------------
% Chapter title on even pages (left pages)
%\ihead{\leftmark\hspace{0.1em}}
% Section title on odd pages (right pages)
%\ohead{\rightmark\hspace{0.1em}}

% -------------------------------------------------
% Footer
% -------------------------------------------------
%\ifoot[2D Game Programming with Pygame-ce\hspace{0.1em}]{2D Game Programming with Pygame-ce\hspace{0.1em}}
%\ofoot[Page \pagemark\ of \pageref{LastPage}\hspace{0.1em}]{Page \pagemark\ of \pageref{LastPage}\hspace{0.1em}}


\definecolor{headblue}{RGB}{90,135,200}
\definecolor{chapblue}{RGB}{120,165,220}
\definecolor{secblue}{RGB}{232,242,255}

% Styles für Kapitel/Abschnitt
\tcbset{
	heading/chapter/.style={
		enhanced,
		colback=chapblue,
		coltext=white,
		boxrule=0.6pt,
		arc=2mm,
		left=2.5mm, right=2.5mm, top=2.3mm, bottom=2.0mm, % kleine Innenabstände
		width=\textwidth,
		boxsep=0pt,
		halign title=center,
		frame hidden,        % <-- Rahmen AUS
	},
	heading/section/.style={
		enhanced,
		colback=secblue,
%		coltext=secorangetext,
		boxrule=0.5pt,
		arc=1.6mm,
		left=2.2mm, right=2.2mm, top=1.8mm, bottom=1.5mm, % noch etwas kompakter
		width=\textwidth,
		boxsep=0pt,
		frame hidden,        % <-- Rahmen AUS
	},
}
% Kapitel/Section-Abstände (kompakter als Default)
\renewcommand*{\chapterpagestyle}{scrheadings}
\renewcommand*\chapterheadstartvskip{\vspace*{-\topskip}} % bündig oben
\renewcommand*\chapterheadendvskip{\vspace*{0pt}}         % kein Extra-Abstand danach
\RedeclareSectionCommand[
beforeskip= 0ex,%-1.2ex plus -0.4ex minus -0.2ex,
afterskip =  1.0ex plus  0.2ex
]{chapter}

\RedeclareSectionCommand[
beforeskip=  0.9ex plus 0.2ex,
afterskip =  0.6ex plus 0.1ex
]{section}

% Optional auch für subsections:
% \RedeclareSectionCommand[beforeskip=0.7ex, afterskip=0.4ex]{subsection}

\makeatletter

% Kapitel: #1 = "Chapter", #2 = Nummer (kann leer sein), #3 = Titel
\renewcommand*\chapterlinesformat[3]{%
	\begin{tcolorbox}[heading/chapter]%
		%\raggedright%
		{\Large\bfseries%
			\ifstrempty{#2}{#3}{#2\enspace #3}%
		}%
	\end{tcolorbox}%
}

% Section: #1 = Name, #3 = Nummer (kann leer), #2 = Einzug, #4 = Titel
\renewcommand*\sectionlinesformat[4]{%
	\begin{tcolorbox}[heading/section]%
		\raggedright%
		{\large\bfseries%
			\ifstrempty{#3}{#4}{#3\enspace #4}%
		}%
	\end{tcolorbox}%
}

\makeatother


\counterwithin*{Requirement}{section}


%%%%%%%%%%%%%%%%%%%%%%% START DOKUMENT %%%%%%%%%%%%%%%%%%%%%%%%%%%
\loadglsentries[main]{gloss.tex}

\makeindex
\makeglossaries


\begin{document}

  \title{Introduction to \\ Pygame-ce\\2D Game Programming}
  \author{Ralf Adams}
  \date{Version 1.0 (\today)}
%  \maketitle
  
\begin{titlepage}
	\centering
	\includegraphics[clip=true, width=0.7\textwidth]{pygame_ce_lofi.png}\par\vspace{1cm}
	{\Huge\bfseries Introduction to Pygame-ce\\\par}
	\vspace{1cm}
	{\Large 2D Game Programming\par}
	\vfill
	{\large Ralf Adams\par}
	{\large Version 1.0 (\today)\par}
\end{titlepage}
  
  
  \tableofcontents

%%%%%%%% Einige eigene Einstellungen
\setlength{\parindent}{0.0em}
\setlength{\parskip}{1.0ex plus0.5ex minus0.5ex}
\setlength{\itemsep}{-0.3ex plus0.2ex}

%%%%%%%%%%%%%%%%%%%%%%% START TEXT

\chapter{Goals}\label{secGoals}
	% !TeX spellcheck = en_US
%%%%%%%%%%%%%%%%%%%%%%%%%%%%%%%%%%%%%%%%%%%%%%%%%%%%%%%%%%%%%%%%%%%%%
%%%%%%%%%%%%%%%%%%%%%%%%%%%%%%%%%%%%%%%%%%%%%%%%%%%%%%%%%%%%%%%
%
% Goals
%
%%%%%%%%%%%%%%%%%%%%%%%%%%%%%%%%%%%%%%%%%%%%%%%%%%%%%%%%%%%%%%%

In this script, you will learn how to program simple 2D games using the programming language \Gls{python} and the game library \Gls{pygame}.

The main goal is not to create a perfect or finished game. Instead, this script focuses on helping you understand the basic ideas and principles behind game programming.

You will learn, step by step,

\begin{itemize}
	\item how a simple game is structured,
	\item how graphics are drawn on the screen,
	\item how bitmaps are drawn on the screen,
	\item how to move game elements,
	\item how to use the classes \texttt{Sprite} and \texttt{Group},
	\item how keyboard and mouse input work,
	\item how to produce text outputs by fonts and bitmaps,
	\item how sounds and music can be used,
	\item how system events and user defined events work,
	\item how game objects like figures or obstacles interact i.e. collision detection,
	\item how to implement time based logic,
	\item and many small details about pygame-ce.
\end{itemize}

I will also present some programming techniques in the chapter \emph{Techniques} that you may find useful. This chapter is still fairly thin, but it already contains an introduction to the topics: animation, tile-based graphics, and how to handle very large game worlds. Additional techniques -- such as a 3D-style visual effect for passing landscapes -- are currently being developed.

In the final chapter, I introduce a few smaller games in order to demonstrate the concrete application of these techniques: the classic example \emph{Pong}, a bubble sticking one, and the \emph{Moonlander}.

You can find all source code and resources on GitHub (\href{https://https://github.com//}{\nolinkurl{github.com}}) and will be updated regularly

What is \emph{not} part of this script:
\begin{itemize}
	\item camera, controller, touch pad, joystick as input devices
	\item clipboard support
	\item test module
	\item freetype font
	\item interacting with other languages like C/C++
	\item other platforms like phone, web browser, etc.
	\item client-server communication
	\item midi sound
	\item direct usage of SDL
\end{itemize}

One thing I’m really not good at is creating visually appealing game worlds. And if I’m being honest, I’ve always cared far more about programming than about game design. So if you’re looking for a deep dive into everything from sketchbooks and graphics tools to a polished final game, you’ll be better off turning to other authors. 

This script is especially designed for beginners. You do not need any previous experience with game programming. Basic knowledge of Python is required.

Many examples are kept short and simple. You are encouraged to \textbf{try things out, experiment, and change the code}. Making mistakes is part of learning — and often the best way to understand how things work. At the end of this script, you should be able to create your own small 2D games and continue learning on your own.

%It is up to you which development environment you use; in this script, I use \Gls{vscode}.

This script is based on the Pygame fork \emph{Pygame Community Edition} (\href{https://pyga.me/}{\nolinkurl{Pygame-ce}}). The source code examples are \textbf{not} checked for compatibility with the original Pygame. To keep things simple and easier to read, I will usually just say \emph{Pygame} and will not make a distinction between the two versions.


If you enjoyed this book and found it helpful, you’re welcome to support my work with a small voluntary contribution. Writing, testing, and explaining things takes time -- and occasionally coffee.

\textbf{If you feel like buying me one (or helping fund the next version of this script), you can do so via PayPal: \emph{adamsralf@outlook.de}.}

Of course, this is entirely optional -- but very much appreciated. Thank you for reading.

\vspace{1em}

If you have any suggestions or feedback, feel free to get in touch: \href{mailto:adamsralf@outlook.de}{\nolinkurl{adamsralf@outlook.de}}

\vspace{2em}

\noindent
Have fun programming and creating your first games!\\
\textit{Ralf Adams} 


\chapter{Basics}\label{secBasics}
	% !TeX spellcheck = en_US
%%%%%%%%%%%%%%%%%%%%%%%%%%%%%%%%%%%%%%%%%%%%%%%%%%%%%%%%%%%%%%%%%%%%%%%%%%%%%
\section{Kind of \emph{Hello World!}}
\begin{diskbox}
	\begin{itemize}
		\item  \url{https://github.com/adamsralf/pygame_book/blob/main/src/00%20Introduction/01%20TheBeginning/v01}
		\item  \url{https://github.com/adamsralf/pygame_book/blob/main/src/00%20Introduction/01%20TheBeginning/v02}
	\end{itemize}
\end{diskbox}
\subsection{The Very First Steps}
\lstsource{src/00 Introduction/01 TheBeginning/v01/start.py}{1}{24}{python}{My first \emph{Game}, Version 1.0}{srcStart00}

\begin{wrapfigure}[7]{r}{5.0cm}
	\vspace{-2.0em}
	\myfigure{greene_plane.png}{0.3}{Playground}{picGreenPlane}
\end{wrapfigure}%
When you start the application now, you will see a nicely designed window with a green background (\abbref[vref]{picGreenPlane}). At the moment, however, not much is happening. The only thing you can do is close the window by clicking on the \emph{X} button in the upper right corner of the window frame.

In order to use Pygame, the module \texttt{pygame} must be imported into the program (\zeiref{srcStart0001}). This makes the \glspl{constant}, \glspl{function}, \glspl{event}, and \glspl{class} of the \gls{namespace} available.


Pygame is not just about calling functions or creating objects; a hole subsystems must be initialized explicitly. In this example, this is done using the static function
\texttt{pygame.init()}. Pygame is now connected to parts of the \Gls{os} system, to deliver and receive required information and actions. In \zeiref{srcStart0003} the Pygame engine is started by calling \texttt{init()}\myindex{pyg}{\texttt{init()}|underline}\randnotiz{init()}. It ist also possible to start only parts of the engine e.g. the sound subsystem by \texttt{pygame.mixer.init()}\myindex{pyg}{\texttt{mixer}!\texttt{init()}}.

For our games, we need a \emph{playfield}/a window in which everything takes place. The class \texttt{pygame.Window}\myindex{pyg}{\texttt{Window}}\randnotiz{Window}
represents such a playfield. In \zeiref{srcStart0005}, the constructor receives one argument -- namely the width and the height of the window as the 2-tuple
\texttt{size}. Our window is therefore \SI{600}{px} wide and \SI{400}{px} high
(see \Gls{PX}). The method \texttt{get\_surface()}
\myindex{pyg}{\texttt{Window}!\texttt{get\_surface()}}\randnotiz{get\_surface()}
in \zeiref{srcStart0012} returns a \texttt{pygame.Surface} object \myindex{pyg}{\texttt{Surface}}, which is roughly something like a \gls{bitmap}.

In \zeiref{srcStart0012}, I~store this return value in the
\gls{variable} named \texttt{screen}\index{screen}\randnotiz{screen}. I~can then assign a title to the window using the attribute \texttt{Window.title}\myindex{pyg}{\texttt{Window}!\texttt{title}}\randnotiz{title} (see \zeiref{srcStart0004}) and set the position of the window relative to the desktop using the attribute \texttt{Window.po\-si\-tion}\myindex{pyg}{\texttt{Window}!\texttt{position}} (see \zeiref{srcStart0002}).

The game itself -- just like all future games -- runs inside a \gls{mainloop}\index{main loop}\index{main loop}\randnotiz{main loop}. The loop starts in \zeiref{srcStart0006} and ends in \zeiref{srcStart0011}. 

\begin{hintbox}[Inside this loop, three things will always happen in the future:]
\begin{itemize}
	\item Reading and processing events: As shown in \zeiref{srcStart0007}f., mouse, keyboard, or game events are detected and passed on to the game elements. In our case, only clicking the X in the upper right corner of the window is registered.

	\item Updating the state of the game elements: Based on the events detected above and the current states of the game elements, the new states are determined (the player moves, a projectile bounces off, points increase, etc.). In our case, only the \gls{flag}\index{Flag} \texttt{running} of the main program loop is set to \false.
	
	\item Drawing the bitmaps of the game elements:	The game elements have a new position or a new appearance and must therefore be redrawn. In this minimal example, only the background of the playfield is colored in \zeiref{srcStart0009}, and afterwards the \gls{doublebuffer}\index{Doublebuffer} is swapped using \texttt{Window.flip()}\myindex{pyg}{\texttt{Window}!\texttt{flip()}} in \zeiref{srcStart0010}.
\end{itemize}
\end{hintbox}

By calling \texttt{py\-game.\-init()}, Pygame places a kind of listener inside the operating system. More precisely, Pygame listens to the \emph{\gls{messagequeue}}. This is where the operating system collects all messages that are triggered by events. These can include \glslink{usb}{USB} connection messages, \glslink{ssd}{SSD} error messages, mouse actions, program starts or crashes, and many others.

Pygame now retrieves from the message queue, using \texttt{pygame.event.get()}\myindex{pyg}{\texttt{event}!\texttt{get()}}\randnotiz{event.get()}, all events that could be relevant to the game. Using a \forSchleife, I~then iterate over these events starting at \zeiref{srcStart0007} and pick out the ones that are relevant to me.

First, I~check which type of event (\texttt{pygame.event.type}) \myindex{pyg}{\texttt{event}!\texttt{type}}\randnotiz{event.type} is being offered.
At the moment, only the type \texttt{pygame.QUIT} \randnotiz{QUIT}\myindex{pyg}{\texttt{QUIT}} is important to me. This type is triggered when the operating system sends a \emph{quit} message to the application. If I~receive such a message, I~set the flag \texttt{running} to \false{}, so that the main program loop is terminated.

If I~do not receive this signal, the main program loop continues to run happily and fills the entire playfield with a color in \zeiref{srcStart0009} using \texttt{screen.fill()}\myindex{pyg}{\texttt{Surface}!\texttt{fill()}} -- in this case, green. Please note that, similar to \zeiref{srcStart0005}, the function expects
one argument -- namely a 3-tuple. This 3-tuple encodes the color using \glslink{rgb}{RGB}\randnotiz{RGB} values between~0 and~255. Predefined color names \randnotiz{color names}such as \emph{green} can also be used here.


What remains is \zeiref{srcStart0010}: Here, the function \texttt{pygame.quit()}
\myindex{pyg}{\texttt{quit()}}\randnotiz{quit()} is called. This function is essentially the opposite of \texttt{pygame.init()} in \zeiref{srcStart0003}. All reserved resources are released again, and the Pygame listeners are removed from the system. You should always make sure to call this function at the end of your application; do not simply terminate the game. The difference is similar to just running out of your apartment versus properly turning off the lights and locking the door when leaving.

If we take a look at the game in the task manager (see~\abbref[vref]{picTaskManager00}), we might be a bit surprised: around \SI{30}{\%} of the CPU time is being used by this \emph{IAmActuallyDoingNothing} game.


\myebild{TaskManager00.png}{0.7}{Resource usage without timing control}{picTaskManager00}

However, if we take a closer look at the main program loop, this should not really be surprising. A bitmap is being drawn onto the screen without any limitation and without interruption. It would be better to allow enough time in each loop iteration to collect events, calculate the new states, and only then generate the screen output.
The screen output itself should also not happen arbitrarily fast or too often; in general, about \SI{60}{\gls{fps}}\randnotiz{fps} are sufficient for motion to be perceived as smooth.

\lstsource{src/00 Introduction/01 TheBeginning/v02/start.py}{1}{25}{python}{My first \emph{Game}, Version 1.1}{srcStart01}

In \zeiref{srcStart0101}, a \texttt{pygame.time.Clock}\randnotiz{Clock}\myindex{pyg}{\texttt{time}!\texttt{Clock}|underline} object is created for timing control. With the help of this object, various time-related tasks can be handled; for the moment, however, we only need it for timing in \zeiref{srcStart0102}. There, \texttt{pygame.time.Clock.\-tick()}\randnotiz{tick()}\myindex{pyg}{\texttt{time}!\texttt{Clock}!\texttt{tick()}} is called with a frame rate measured in $fps$. This function ensures that the application now runs at a maximum of \SI{60}{fps}. This can be seen in the significantly reduced CPU usage shown in
\abbref[vref]{picTaskManager01}.

\begin{hintbox}[Hint]
	The Pygame documentation points out that the function \texttt{tick()} is very resource-efficient, but somewhat imprecise. If accuracy is important for timing, the function \texttt{tick\_\_loop()}\myindex{pyg}{\texttt{time}!\texttt{Clock}!\texttt{tick\_busy\_loop()}} is recommended instead. Its disadvantage, however, is that it consumes significantly more processing time than \texttt{tick()}.
\end{hintbox}

\myebild{TaskManager01.png}{0.7}{Resource usage with timing control}{picTaskManager01}

\newpage
\subsection{More Input}
\subsubsection{Multiple Windows}\label{secMultipleWindows}
\begin{diskbox}
	\url{https://github.com/adamsralf/pygame_book/blob/main/src/00%20Introduction/01%20TheBeginning/v03/start.py}
\end{diskbox}

You can also create multiple windows for a game (see \url{https://pyga.me/docs/ref/window.html}).

\lstsource{src/00 Introduction/01 TheBeginning/v03/start.py}{4}{32}{python}{Multiple Windows}{srcStart02}

\begin{figure}[hbtp] 
	\begin{center}
		\begin{tikzpicture}
			\node (main) at (0.0, 0) {\fbox{\includegraphics[scale=0.7]{MainWindow.png}}};
			\node (side) at (6.0, 0) {\fbox{\includegraphics[scale=0.7]{SideWindow.png}}};
		\end{tikzpicture}
		\caption{Multiple Windows}\label{picMultipleWindows}
	\end{center}
\end{figure}

\newpage
\subsubsection{Information About the Graphics Environment}
\begin{diskbox}
	\url{https://github.com/adamsralf/pygame_book/blob/main/src/00%20Introduction/01%20TheBeginning/v04/start.py}
\end{diskbox}

Sometimes it is necessary to know information about the graphics environment: perhaps to identify performance problems, or perhaps to find out which display features are available. Using the function \texttt{pygame.display.Info()}\myindex{pyg}{\texttt{display}!\texttt{Info()}}\randnotiz{Info()}, various parameters can be queried. On my system, calling this function produced the output shown in \abbref[vref]{picInfo00}.

Please refer to \tabref[vref]{tabDisplayInfo} for the meaning of the values (source: \url{https://pyga.me/docs/ref/display.html#pygame.display.Info}).


\begin{longtable}{p{4cm} p{10cm}}
	\caption{Fields of \texttt{pygame.display.Info()}}\label{tabDisplayInfo}\\
	\toprule
	\textbf{Field} & \textbf{Description} \\
	\midrule
	\endfirsthead
	
	\toprule
	\textbf{Field} & \textbf{Description} \\
	\midrule
	\endhead
	
	\midrule
	\multicolumn{2}{r}{\emph{continued on next page}} \\
	\endfoot
	
	\bottomrule
	\endlastfoot
	
	\texttt{hw} &
	\texttt{1} if the display is hardware accelerated. \\
	
	\texttt{wm} &
	\texttt{1} if windowed display modes can be used. \\
	
	\texttt{video\_mem} &
	The amount of video memory on the display in megabytes.
	This value is \texttt{0} if the amount is unknown. \\
	
	\texttt{bitsize} &
	Number of bits used to store each pixel. \\
	
	\texttt{bytesize} &
	Number of bytes used to store each pixel. \\
	
	\texttt{masks} &
	Four values used to pack RGBA values into pixels. \\
	
	\texttt{shifts} &
	Four values used to pack RGBA values into pixels. \\
	
	\texttt{losses} &
	Four values used to pack RGBA values into pixels. \\
	
	\texttt{blit\_hw} &
	\texttt{1} if hardware Surface blitting is accelerated. \\
	
	\texttt{blit\_hw\_CC} &
	\texttt{1} if hardware Surface colorkey blitting is accelerated. \\
	
	\texttt{blit\_hw\_A} &
	\texttt{1} if hardware Surface pixel alpha blitting is accelerated. \\
	
	\texttt{blit\_sw} &
	\texttt{1} if software Surface blitting is accelerated. \\
	
	\texttt{blit\_sw\_CC} &
	\texttt{1} if software Surface colorkey blitting is accelerated. \\
	
	\texttt{blit\_sw\_A} &
	\texttt{1} if software Surface pixel alpha blitting is accelerated. \\
	
	\texttt{current\_w}, \texttt{current\_h} &
	Width and height of the current video mode, or of the desktop mode if
	called before \texttt{display.set\_mode()}.
	The values are \texttt{-1} on error. \\
	
	\texttt{pixel\_format} &
	The pixel format of the display surface as a string,
	for example \texttt{PIXELFORMAT\_RGB888}. \\
	
\end{longtable}


\lstsource{SRC/00 Introduction/03 Bitmaps/ginfo.py}{5}{9}{python}{\texttt{pygame.display.Info()}}{Info01}

\myebild{ginfo.png}{1.0}{Infos about the graphical environment}{picInfo00}

\newpage
%%%%%%%%%%%%%%%%%%%%%%%%%%%%%%%%%%%%%%%%%%%%%%%%%%%%%%%%%%%%%%%%%%%%%%%%%%%
\subsection{What was new?}

\begin{hintbox}[To start a minimal Pygame application]

\begin{itemize}
	\item Import the Pygame library.
	\item Initialize the Pygame system.
	\item Create a window / a playfield.
	\item Set up a main program loop:
	\begin{enumerate}
		\item Poll events.
		\item Update game objects.
		\item Render the screen contents.
		\item Control the timing of the loop iterations.
	\end{enumerate}
	\item Shut down the Pygame system when exiting.
\end{itemize}
\end{hintbox}

\begin{pygbox}[The following Pygame elements were introduced]
\begin{itemize}
	\item \texttt{import pygame}:\\ \url{https://pyga.me/docs/tutorials/en/import-init.html}
	
	\item \texttt{pygame.init()}:
	\myindex{pyg}{\texttt{init()}}\\
	\url{https://pyga.me/docs/ref/pygame.html#pygame.init}
	
	\item \texttt{pygame.quit()}:
	\myindex{pyg}{\texttt{quit()}}\\
	\url{https://pyga.me/docs/ref/pygame.html#pygame.quit}

	\item \texttt{pygame.QUIT}:
	\myindex{pyg}{\texttt{QUIT}}\\
    \url{https://pyga.me/docs/ref/event.html#pygame.event.EventType.type}
	\item \texttt{pygame.WINDOWCLOSE}:
	\myindex{pyg}{\texttt{WINDOWCLOSE}}\\
	\url{https://pyga.me/docs/ref/event.html#pygame.event.EventType.type}

	\item \texttt{pygame.event.get()}:
	\myindex{pyg}{\texttt{event}!\texttt{get()}}\\
	\url{https://pyga.me/docs/ref/event.html#pygame.event.get}
	
	\item \texttt{pygame.event.type}:
	\myindex{pyg}{\texttt{event}!\texttt{type}}\\
	\url{https://pyga.me/docs/ref/event.html#pygame.event.EventType.type}

	\item \texttt{pygame.time.Clock}:
	\myindex{pyg}{\texttt{time}!\texttt{Clock}}\\
	\url{https://pyga.me/docs/ref/time.html#pygame.time.Clock}

	\item \texttt{pygame.time.Clock.tick()}:
	\myindex{pyg}{\texttt{time}!\texttt{Clock}!\texttt{tick()}}\\
	\url{https://pyga.me/docs/ref/time.html#pygame.time.Clock.tick}
	
	\item \texttt{pygame.time.Clock.tick\_busy\_loop()}:
	\myindex{pyg}{\texttt{time}!\texttt{Clock}!\texttt{tick\_busy\_loop()}}\\
	\url{https://pyga.me/docs/ref/time.html#pygame.time.Clock.tick_busy_loop}
	
	\item \texttt{pygame.Surface.fill()}:
	\myindex{pyg}{\texttt{Surface}!\texttt{fill()}}\\
	\url{https://pyga.me/docs/ref/surface.html#pygame.Surface.fill}

	\item \texttt{pygame.Window}:
	\myindex{pyg}{\texttt{Window}}\\
	\url{https://pyga.me/docs/ref/window.html}
	
	\item \texttt{pygame.Window.destroy()}:
	\myindex{pyg}{\texttt{Window}!\texttt{destroy()}}\\
	\url{https://pyga.me/docs/ref/window.html#pygame.Window.destroy}

	\item \texttt{pygame.Window.flip()}:
	\myindex{pyg}{\texttt{Window}!\texttt{flip()}}\\
	\url{https://pyga.me/docs/ref/window.html#pygame.Window.flip}
	
	\item \texttt{pygame.Window.get\_surface()}:
	\myindex{pyg}{\texttt{Window}!\texttt{get\_surface()}}\\
	\url{https://pyga.me/docs/ref/window.html#pygame.Window.get_surface}
	
	\item \texttt{pygame.Window.title}:
	\myindex{pyg}{\texttt{Window}!\texttt{title}}\\
	\url{https://pyga.me/docs/ref/window.html#pygame.Window.title}
	
	\item \texttt{pygame.Window.position}:
	\myindex{pyg}{\texttt{Window}!\texttt{position}}\\
	\url{https://pyga.me/docs/ref/window.html#pygame.Window.position}
\end{itemize}	
\end{pygbox}

%\begin{pygbox}[The following Pygame elements were introduced]
%\tcbsubtitle{import pygame}
%\url{https://pyga.me/docs/tutorials/en/import-init.html}
%
%\tcbsubtitle{pygame.init()}
%\myindex{pyg}{\texttt{init()}}
%\url{https://pyga.me/docs/ref/pygame.html#pygame.init}
%	
%\tcbsubtitle{pygame.quit()}
%\myindex{pyg}{\texttt{quit()}}
%\url{https://pyga.me/docs/ref/pygame.html#pygame.quit}
%
%\tcbsubtitle{pygame.QUIT}
%\myindex{pyg}{\texttt{QUIT}}
%\url{https://pyga.me/docs/ref/event.html#pygame.event.EventType.type}
%
%\tcbsubtitle{pygame.WINDOWCLOSE}
%\myindex{pyg}{\texttt{WINDOWCLOSE}}
%\url{https://pyga.me/docs/ref/event.html#pygame.event.EventType.type}
%
%\tcbsubtitle{pygame.event.get()}
%\myindex{pyg}{\texttt{event}!\texttt{get()}}
%\url{https://pyga.me/docs/ref/event.html#pygame.event.get}
%
%\tcbsubtitle{pygame.event.type}
%\myindex{pyg}{\texttt{event}!\texttt{type}}
%\url{https://pyga.me/docs/ref/event.html#pygame.event.EventType.type}
%
%\tcbsubtitle{pygame.time.Clock}
%\myindex{pyg}{\texttt{time}!\texttt{Clock}}
%\url{https://pyga.me/docs/ref/time.html#pygame.time.Clock}
%
%\tcbsubtitle{pygame.time.Clock.tick()}
%\myindex{pyg}{\texttt{time}!\texttt{Clock}!\texttt{tick()}}
%\url{https://pyga.me/docs/ref/time.html#pygame.time.Clock.tick}
%
%\tcbsubtitle{pygame.time.Clock.tick\_busy\_loop()}
%\myindex{pyg}{\texttt{time}!\texttt{Clock}!\texttt{tick\_busy\_loop()}}
%\url{https://pyga.me/docs/ref/time.html#pygame.time.Clock.tick_busy_loop}
%
%\tcbsubtitle{pygame.Surface.fill()}
%\myindex{pyg}{\texttt{Surface}!\texttt{fill()}}
%\url{https://pyga.me/docs/ref/surface.html#pygame.Surface.fill}
%
%\tcbsubtitle{pygame.Window}
%\myindex{pyg}{\texttt{Window}}
%\url{https://pyga.me/docs/ref/window.html}
%
%\tcbsubtitle{pygame.Window.destroy()}
%\myindex{pyg}{\texttt{Window}!\texttt{destroy()}}
%\url{https://pyga.me/docs/ref/window.html#pygame.Window.destroy}
%
%\tcbsubtitle{pygame.Window.flip()}
%\myindex{pyg}{\texttt{Window}!\texttt{flip()}}
%\url{https://pyga.me/docs/ref/window.html#pygame.Window.flip}
%
%\tcbsubtitle{pygame.Window.get\_surface()}
%\myindex{pyg}{\texttt{Window}!\texttt{get\_surface()}}
%\url{https://pyga.me/docs/ref/window.html#pygame.Window.get_surface}
%
%\tcbsubtitle{pygame.Window.title}
%\myindex{pyg}{\texttt{Window}!\texttt{title}}
%\url{https://pyga.me/docs/ref/window.html#pygame.Window.title}
%
%\tcbsubtitle{pygame.Window.position}
%\myindex{pyg}{\texttt{Window}!\texttt{position}}
%\url{https://pyga.me/docs/ref/window.html#pygame.Window.position}
%	
%\end{pygbox}


\subsection{Homework}
Please have a look at \url{https://pyga.me/docs/ref/window.html} and then try to solve the following small exercises: 

\begin{enumerate}
	\item Set up a working environment for your game programming. Install Python, your preferred editor, and the latest Pygame-ce version. Open the directory containing your Pygame source code and try to run \texttt{start01.py}.
	
	\item Change the background color of the window. Use RGB values as well as named colors.
	
	\item Change the size of the window.
	
	\item Change the position of the window. Use position values and also try
	\texttt{WINDOWPOS\-\_CENTERED}
	\randnotiz{WINDOWPOS\_CENTERED}\myindex{pyg}{\texttt{WINDOWPOS\_CENTERED}}
	and
	\texttt{WINDOWPOS\-\_UNDEFINED}
	\randnotiz{WINDOWPOS\_UNDEFINED}\myindex{pyg}{\texttt{WINDOWPOS\_UNDEFINED}}.
	
	\item Create the window as a resizable window and try to resize it.
	
	\item Define a minimum an maximum window size, show the actual size in the title bar, and try to resize the window to its limits.
	
	\item Show the actual window position in the title bar.
	
	\item Play a little bit with the \texttt{opacity} attribute of the window. 
	
	\item Change the title of the window during runtime according to a counter. Shut down the program if counter is greater~600.
	
	\item Create a borderless window.
	
	\item Try a fullscreen window.
	
	\item Try to arrange three windows in a row. Compute the x-position of the second and third window based on the window size and a useful gap between them. 
	
	
\end{enumerate}

	% !TeX spellcheck = en_US
\newpage
%%%%%%%%%%%%%%%%%%%%%%%%%%%%%%%%%%%%%%%%%%%%%%%%%%%%%%%%%%%%%%%%%%%%%%%%%%%
\section{Graphic Primitives}\index{Graphic primitives}\label{secGrafikprimitive}
\subsection{Introduction}
Graphic primitives are simple graphical shapes that are drawn, such as lines\index{Line}\index{Graphic primitives!Line}, points\index{Point}\index{Graphic primitives!Point}, circles\index{Circle}\index{Graphic primitives!Circle}, and so on.
They do not play a very important role in game programming, but they can be quite useful. For this reason, I will only introduce a few of them here.

\lstsource{src/00 Introduction/02 Primitives/primitives00.py}{1}{44}{python}{Graphic Primitives}{srcPrimitives00}

The basic structure is the same as in \srcref[vref]{srcStart01}. The differences begin in \zeiref{srcPrimitives01}. The class \texttt{pygame.Color}\randnotiz{Color}\myindex{pyg}{\texttt{Color}|underline} can encode color information\index{Color!Information} in various formats, including an \glslink{alpha}{alpha channel}\index{Color!Alpha channel}\index{Alpha channel} (\glslink{alphablending}{alpha blending}\index{Color!Alpha blending}\index{Alpha blending}, transparency)\index{Transparency}; more about this will follow later in
\abschnittref[vref]{secBitmapLaden}. Here, I use RGB encoding with color channel values between~0 and~255.

In most cases, however, I do not need to define my own colors. Pygame provides a really extensive list of 664 predefined color names \randnotiz{Color names}\index{Color!Names}. Wherever color values are expected, I can pass either a
\texttt{Color} object, a numeric color code, or a color name as a string.


\begin{wrapfigure}[18]{r}{7.8cm}%
	\begin{center}%
		\vspace{-1cm}%
		\myfigure{primitives.png}{0.55}{Some graphic primitives}{picPrimitive}%
	\end{center}%
\end{wrapfigure}%
Let us go through the individual shapes one by one and start with the
rectangle. There are several ways to define a rectangle in Pygame. Since we will need it very often later on, I would like to introduce the class \texttt{pygame.rect.Rect}
\myindex{pyg}{\texttt{rect}!\texttt{Rect}}\randnotiz{Rect} here. It is defined by four parameters: the upper-left corner, its width, and its height. In \zeiref{srcPrimitives02}, a rectangle is therefore defined at the position $(10,10)$ with a width of \SI{20}{px} and a height of \SI{30}{px}.

\textbf{Note:}
The class \texttt{Rect} is not a drawn rectangle, but merely a container
for information that is relevant for a rectangle.

In \zeiref{srcPrimitives03}, \texttt{pygame.draw.rect()}\myindex{pyg}{\texttt{draw}!\texttt{rect()}}\randnotiz{rect()}\index{Rectangle|underline}\index{Graphic primitives!Rectangle|underline} draws a filled rectangle. The \Gls{semantik} of the parameters should be self-explanatory.
The call in \zeiref{srcPrimitives04}, however, is different. The first parameter after the rectangle -- here~\texttt{3} -- specifies the thickness of the line. If this parameter is given and greater than~0, the rectangle is no longer filled. The value~\texttt{10} specifies the rounding of the corners. Here, a value between~0 and $min(width, height)/2$ can be used, as this value corresponds to the radius of the corner rounding.

More general than a rectangle is a \Gls{polygon}. A polygon\index{Graphic primitives!Polygon|underline}\index{Polygon|underline} is a closed chain of lines that is defined in Pygame by its points (vertices). Similar to rectangles, there are filled (\zeiref{srcPrimitives06}) and unfilled (\zeiref{srcPrimitives07}) variants. Both are drawn using \texttt{pygame.draw.polygon()}\randnotiz{polygon()}\myindex{pyg}{\texttt{draw}!\texttt{polygon()}}. Be careful with the line thickness: the lines grow outward, which can quickly lead to ugly offsets at the corners. Try it out by changing the value~\texttt{2} to~\texttt{5}.

For individual lines\index{Line|underline}\index{Graphic primitives!Line|underline},
there is \texttt{pygame.draw.line()}\myindex{pyg}{\texttt{draw}!\texttt{line()}}\randnotiz{line()}, and for a \gls{linienzug} -- without an example here -- there is \texttt{pygame.draw.lines()}\myindex{pyg}{\texttt{draw}!\texttt{lines()}}\randnotiz{lines()}. An example can be found in \zeiref{srcPrimitives08}.

A circle\index{Graphic primitives!Circle|underline}\index{Circle|underline} is defined by two values: its center point and its radius. In \zeiref{srcPrimitives09}, a filled circle with the center at $(40, 150)$ and a radius of \SI{30}{px} is drawn using \texttt{pygame.draw.circle()}\myindex{pyg}{\texttt{draw}!\texttt{circle()}}\randnotiz{circle()}. As with rectangles and polygons, there are also unfilled variants (\zeiref{srcPrimitives10}). Of particular interest is the circular arc segment in
\zeiref{srcPrimitives11}. Here, Boolean variables are used to control which section of the circular arc is drawn (for more details, see the Pygame reference).

Finally, one small color experiment. Strangely enough, Pygame does not provide a dedicated function for drawing a single point or pixel\index{Graphic primitives!Point|underline}\index{Point|underline}. Here, I have implemented three workarounds that I found. One could think of additional ones as well: a line with $start = end$, a circle with a radius of \SI{1}{px}, and so on.

In \zeiref{srcPrimitives12}, a point is drawn by setting a single color value at a specific position using \texttt{pygame.Surface.set\_at()}\myindex{pyg}{\texttt{Surface}!\texttt{set\_at()}}\randnotiz{set\_at()}. Alternatively, the \texttt{fill()} surface function used earlier can be applied with an area of only one pixel in width and height (\zeiref{srcPrimitives13}). Another way to set a pixel using a graphics library is the experimental \texttt{gfxdraw} module. In \zeiref{srcPrimitives14}, a single pixel is set using \texttt{pygame.gfxdraw.pixel()}\myindex{pyg}{\texttt{gfxdraw}!\texttt{pixel()}}\randnotiz{pixel()}. The \texttt{gfxdraw} module is not imported automatically by
\texttt{import pygame} (see \zeiref{srcPrimitives15}).



\subsection{More Input}

\subsubsection{Example: Particle Swarm}

\begin{wrapfigure}[20]{r}{5.8cm}%
	\begin{center}%
		\vspace{-1cm}%
		\myfigure{circles01.png}{0.6}{Not a particle swarm}{picCircles01}%
	\end{center}%
\end{wrapfigure}%
Using graphic primitives, it is possible to create dynamic effects, such as particle swarms. Here, I would like to present a very simple example of a mouse-controlled fountain made of circles.

Let us first build a small program that draws a circle at the mouse position. The class \texttt{Circle} (see \zeiref{srcCircles0101}) contains all the information I need to draw circles: position, radius, and color. The position is defined via a constructor argument. In the method \texttt{draw()}, the screen output is encapsulated.

The function \texttt{main()} now contains a lot of familiar elements, but also a few new ones. In \zeiref{srcCircles0101}, the screen size is stored in a list, because we still need this information at another place, namely in \zeiref{srcCircles0103}. Below that, in \zeiref{srcCircles0104}, a list for storing the circles is defined.

Inside the main program loop, \zeiref{srcCircles0105} checks whether the left mouse button\randnotiz{\texttt{get\_pressed()}}\myindex{pyg}{\texttt{mouse}!\texttt{get\_pressed()}|underline} has been pressed. If so, a circle is drawn at the mouse position \randnotiz{\texttt{get\_pos()}}\myindex{pyg}{\texttt{mouse}!\texttt{get\_pos()}}. After that, the screen is filled with white color and the circles stored in the container are drawn.

The result is not very impressive yet (see \abbref[vref]{picCircles01}) and is more reminiscent of a drawing program like Paint.

\lstsource{src/00 Introduction/02 Primitives/circles/v01/circles.py}{1}{999}{python}{Particle swarm, Version 1.0}{srcCircles01}

\begin{wrapfigure}[10]{r}{6.0cm}%
	\begin{center}%
		\vspace{-1cm}%
		\myfigure{circles02a.png}{0.6}{Particle swarm\\Version 2}{picCircles02}%
	\end{center}%
\end{wrapfigure}%
In the next step, we want to turn the bulky circles into colorful particles. These particles should also no longer appear exactly at the mouse position, but be scattered around it. To achieve this, only minimal changes need to be made to the
\texttt{Circle} class. 

The two position values are now extended by a random value between~$-2$ and~$+2$.
The radius is also reduced to \SI{2}{px}. The color is likewise varied using random values. I tried out several combinations here, and I quite like this color variation.
Feel free to experiment with the color channels and the random values yourself. The result shown in \abbref[vref]{picCircles02} already looks much better.


\lstsource{src/00 Introduction/02 Primitives/circles/v02/circles.py}{7}{12}{python}{Particle swarm, Version 2.0}{srcCircles02}

Now we want to add a bit of dynamics to the game. The particles should first rise upward and then fall down again. To achieve this, I added the vertical velocity \texttt{speedy} to the \texttt{Circle} class and assigned it a random initial value
(\zeiref{srcCircles0301}). The division by~$10.1$ ensures that no smooth, rounded values are created. Here as well, feel free to experiment with the values to see the
different effects.

The class also needs to be extended by the method \texttt{update()}. In this method, the new vertical position \texttt{posy} is calculated based on the vertical velocity \texttt{speedy}, and the velocity is in turn modified with respect to \gls{gravity}\randnotiz{Gravity}\index{Gravity}. In order for all particles to be subject to the same gravitational force, I defined \texttt{GRAVITY} as a static attribute (\zeiref{srcCircles0302}).

\lstsource{src/00 Introduction/02 Primitives/circles/v03/circles.py}{6}{18}{python}{Particle swarm, Version 3.0, Class \texttt{Circle}}{srcCircles03a}

All that remains is the call of \texttt{update()} inside the main program loop.

\lstsource{src/00 Introduction/02 Primitives/circles/v03/circles.py}{38}{42}{python}{Particle swarm, Version 3.0, Call of \texttt{update()}}{srcCircles03b}

The fountain is still not really lively yet. So let us also scatter the particles horizontally. For this purpose, the attribute \texttt{speedx} is added in the
constructor. The upper and lower bounds of the random number generator determine
the width of the particle fountain. Try out values here that match your own sense of aesthetics. In \texttt{update()}, the new horizontal position \texttt{posx} then
has to be calculated.

The horizontal velocity does not need to be adjusted, since \texttt{GRAVITY} is only supposed to act downward.

\lstsource{src/00 Introduction/02 Primitives/circles/v04/circles.py}{9}{20}{python}{Particle swarm, Version 4.0, \texttt{Circle.update()}}{srcCircles04}

\begin{wrapfigure}[17]{r}{6.0cm}%
	\begin{center}%
		\vspace{-1cm}%
		\myfigure{circles05.png}{0.6}{Particle swarm, \\Version 5: nearly finished}{picCircles05}%
	\end{center}%
\end{wrapfigure}%
After some time, the list \texttt{circles} contains many particles that are no longer displayed at all. We want to remove these particles. To do this, the \texttt{Circle} class needs to determine whether the \texttt{object} can be deleted. 

As a first step, we add the deletion flag \texttt{todelete} to the class (see \zeiref{srcCircles0500}), which is initialized to \false; a new particle should of course not be deleted immediately.

In \zeiref{srcCircles0501}, it is checked whether the right edge of the particle (center point plus radius) lies outside the screen on the left. If this is the case, the deletion flag must be set to \true. Analogously, the right and the bottom edges of the screen are checked in \zeiref{srcCircles0502} and \zeiref{srcCircles0503}.

For this purpose, the attribute \texttt{pygame.Window.size}\randnotiz{size}\myindex{pyg}{\texttt{Window}!\texttt{size}} is used to determine the width and height of the screen. This attribute returns the screen size as a 2-tuple.
The zeroth value represents the width, and the first value represents the height. A~check to see whether the particle has disappeared upward is not necessary, since it will eventually fall down again and thus become visible once more.

\lstsource{src/00 Introduction/02 Primitives/circles/v05/circles.py}{6}{30}{python}{Particle swarm, Version 5.0, Class \texttt{Circle}}{srcCircles05a}

In the main program, I now need to implement suitable deletion logic. But first, I want my fountain to have a bit more \emph{oomph}: In \zeiref{srcCircles0504}, not just one particle is created, but always five at once.

In \zeiref{srcCircles0505}, an empty list is created that will contain the particles to be deleted. Inside the update loop, it is now additionally checked whether a particle should be deleted (\zeiref{srcCircles0506}). If so, this particle is added to the list \texttt{todelete}. After the update loop has finished, the particles to be deleted are removed from the list \texttt{circles} starting at \zeiref{srcCircles0507}.

In \abbref[vref]{picCircles05}, you can see a fountain. It really starts to look cool only when you move the mouse while it is running.

\lstsource{src/00 Introduction/02 Primitives/circles/v05/circles.py}{41}{64}{python}{Particle swarm, Version 5.0, Main loop}{srcCircles05b}

Why do I not call \texttt{remove()} already inside the update loop? Because: \emph{Never increase or decrease the length of a list while you are iterating over it.} Very strange effects can occur. Try to guess the number of loop iterations of the following program:

\begin{lstlisting}[firstnumber=1]
	values = [1, 2, 3]
	for a in values:
		values.append(a*10)
	print(vlues)	
\end{lstlisting}

%\begin{wrapfigure}[17]{r}{6.0cm}%
%	\begin{center}%
%			\vspace{-1cm}%
%			\myfigure{circles06.png}{0.6}{Partikelfontaine, \\Version 6: fertig}{picCircles06}%
%		\end{center}%
%\end{wrapfigure}%
Even small changes to the parameters can already produce interesting visual effects.
Unfortunately, these cannot be shown very well using images here, so: program it yourself and try it out.

\lstsource{src/00 Introduction/02 Primitives/circles/v06/circles.py}{1}{999}{python}{Particle swarm, Version 6.0}{srcCircles06}

\subsubsection{Example: Landscape}\label{secLandscape}

In this example, we combine graphic primitives, object-oriented design, and simple mathematics to create a small animated scene.

\myebild{landscape_sun.png}{0.5}{Example: Drawing a Landscape}{picLandscape00}

I would like to create a small animated landscape that should look like in \abbref[vref]{picLandscape00}. The sun should rise on the left, move across the sky, and set on the right. The blue color of the sky should also change depending on the time of
day.

Let us start with a basic framework that I want to expand step by step. The elements of the \srcref[vref]{srcLandscape01} should be self-explanatory. The variable \texttt{horizon} is meant to control the boundary between the sky and the meadow -- in other words, it forms the horizon. The sky, sun, tree, house, and meadow should all align with this
boundary.

\lstsource{src/00 Introduction/02 Primitives/landscape/v01/landscape.py}{1}{999}{python}{Landscape, Version 1.0}{srcLandscape01}

I extend the program with the very simple class \texttt{Meadow}. In the constructor, a reference to the window and the horizon are stored, and the color is defined — in this case, a custom shade of green. After that, the upper-left corner and the size of the meadow are calculated. Both values take the horizon into account. The method \texttt{draw()} then draws the meadow as a green rectangle into the window.

\lstsource{src/00 Introduction/02 Primitives/landscape/v02/landscape.py}{4}{14}{python}{Landscape, Version 2.0, Class \texttt{Meadow}}{srcLandscape02a}

In \texttt{main()}, an object of the class \texttt{Meadow} is now created in \zeiref{srcLandscape0201}, and in \zeiref{srcLandscape0202} the meadow is drawn using the \texttt{draw()} method. The result looks like the one shown in \abbref[vref]{picLandscape02}.


\lstsource{src/00 Introduction/02 Primitives/landscape/v02/landscape.py}{22}{37}{python}{Landscape, Version 2.0, \texttt{main()}}{srcLandscape02b}

The class \texttt{Sky} is similarly simple. Here as well, reference data is stored in the constructor, and in \texttt{draw()} a blue rectangle is drawn whose size depends on \texttt{horizon}.


\lstsource{src/00 Introduction/02 Primitives/landscape/v03/landscape.py}{16}{24}{python}{Landscape, Version 3.0, Class \texttt{Sky}}{srcLandscape03a}

All that remains is to integrate it into \texttt{main()} in the same way as \texttt{Meadow} (see \abbref[vref]{picLandscape03}). Play around a bit with the variable \texttt{horizon} to see the effect.

\lstsource{src/00 Introduction/02 Primitives/landscape/v03/landscape.py}{31}{48}{python}{Landscape, Version 3.0, \texttt{main()}}{srcLandscape03b}

\myezweihbild{landscape_sun02.png}{0.45}{Drawing a Landscape (2)}{picLandscape02}%
{landscape_sun03.png}{0.45}{Drawing a Landscape (3)}{picLandscape03}

The class \texttt{Tree} consists of two parts: a tree trunk and a leafy crown. The tree trunk is created in \texttt{draw()} using a rectangle, and the leafy crown is created using a circle. I will not show the integration into \texttt{main()} here, since it is completely analogous to the integration of \texttt{Meadow} and \texttt{Sky}. Only the order needs to be considered, because the tree is supposed to
appear in the foreground. The tree should look like the one shown in \abbref[vref]{picLandscape04}.


\lstsource{src/00 Introduction/02 Primitives/landscape/v04/landscape.py}{28}{37}{python}{Landscape, Version 4.0, Class \texttt{Tree}}{srcLandscape04a}

The basic principle of the class \texttt{House} is the same as for the other classes. It is just a bit more complex, since it consists of two rectangles and a triangle. Here as well, the integration into \texttt{main()} is trivial and is left to you. In \abbref[vref]{picLandscape05}, only the sun is missing now.

\lstsource{src/00 Introduction/02 Primitives/landscape/v05/landscape.py}{40}{54}{python}{Landscape, Version 5.0, Class \texttt{House}}{srcLandscape05a}

\myezweihbild{landscape_sun04.png}{0.45}{Drawing a Landscape (4)}{picLandscape04}{landscape_sun05.png}{0.45}{Drawing a Landscape (5)}{picLandscape05}

In its basic shape, the sun is a simple yellow filled circle. However, we want it to move across the sky. Therefore, we need a start position below the horizon (see \zeiref{srcLandscape0601}) and a method \texttt{update()} that calculates the new position of the sun in each frame.

In \texttt{update()}, the new horizontal position is first calculated based on \texttt{speed}. After that, I calculate how far the sun has already progressed along its path. This value is relative and has a range of~$[0, 1]$. If the sun has covered a quarter of the distance, the value of \texttt{progress} is~$0.25$, at halfway it is~$0.5$, and so on.

How do I calculate the height now? For the sake of simplicity, I let the sun follow the first half of the sine function. For this, the domain must be~$[0, \pi]$; this is the hump of the sine function that lies above the x-axis. The range of the sine function from~$0$ to~$\pi$ is $[0, 1]$. If I multiply this value by the horizon, I obtain values from~$0$ to \texttt{horizon}. Finally, I add the radius so that the sun just touches the upper edge at its highest point.

The function \texttt{update()} returns the value of \texttt{progress} so that I can reuse this value to calculate the color of the sky, which still needs to be implemented. Everything clear? By the way, do not forget to add \texttt{import math} at the beginning because of the sine function!

\lstsource{src/00 Introduction/02 Primitives/landscape/v06/landscape.py}{59}{76}{python}{Landscape, Version 6.0, Class \texttt{Sun}}{srcLandscape06a}

Like the other classes, the sun is integrated into \texttt{main()}: the object is created in \zeiref{srcLandscape0602} and drawn using \texttt{draw()} in \zeiref{srcLandscape0604}. Only the call to \texttt{update()} in \zeiref{srcLandscape0603} is new. Important: The order of the \texttt{draw()} calls must be observed! The sun should be drawn after the sky, but before the meadow and the
tree.

\lstsource{src/00 Introduction/02 Primitives/landscape/v06/landscape.py}{85}{108}{python}{Landscape, Version 6.0, \texttt{main()}}{srcLandscape06b}

The final stage of the extension concerns the color of the sky. Depending on the position of the sun -- more precisely, on the progress of the sun -- the blue color of the sky should change. This is done in the new method \texttt{update()} of the class \texttt{Sky} (see \srcref{srcLandscape07a}). Here as well, the green component of the color is calculated using the sine function; a linear approach would also have been possible, but the sine function produces smoother transitions near sunrise and sunset.

\lstsource{src/00 Introduction/02 Primitives/landscape/v07/landscape.py}{25}{28}{python}{Landscape, Version 7.0, \texttt{Sky.update()}}{srcLandscape07a}

In \texttt{main()}, only the relative progress of the sun is now taken and passed to the \texttt{update()} method of \texttt{Sky}. Done :-)

\lstsource{src/00 Introduction/02 Primitives/landscape/v07/landscape.py}{104}{106}{python}{Landscape, Version 7.0, \texttt{main()}}{srcLandscape07b}

\newpage
\subsection{What was new?}

Using graphic primitives, you can create and use your own drawings. They are usually available in both filled and unfilled variants. Colors can either be defined manually or selected from a list of predefined colors.

\textbf{Rule of thumb}: Objects drawn later appear in front of earlier ones.

The following Pygame elements were introduced:

\begin{itemize}
	\item Named colors:\index{Colors named|underline}\\ 
	\url{https://pyga.me/docs/ref/color_list.html}
	
	\item \texttt{import pygame.gfxdraw}:\\ \url{https://pyga.me/docs/ref/gfxdraw.html}
	
	\item \texttt{pygame.Color}:
	\myindex{pyg}{\texttt{Color}}\\
	\url{https://pyga.me/docs/ref/color.html}
	
	\item \texttt{pygame.draw.circle()}:
	\myindex{pyg}{\texttt{draw}!\texttt{circle()}}\\
	\url{https://pyga.me/docs/ref/draw.html#pygame.draw.circle}
	
	\item \texttt{pygame.draw.line()}:
	\myindex{pyg}{\texttt{draw}!\texttt{line()}}\\
	\url{https://pyga.me/docs/ref/draw.html#pygame.draw.line}
	
	\item \texttt{pygame.draw.lines()}:
	\myindex{pyg}{\texttt{draw}!\texttt{lines()}}\\
	\url{https://pyga.me/docs/ref/draw.html#pygame.draw.lines}
	
	\item \texttt{pygame.draw.polygon()}:
	\myindex{pyg}{\texttt{draw}!\texttt{polygon()}}\\
	\url{https://pyga.me/docs/ref/draw.html#pygame.draw.polygon}
	
	\item \texttt{pygame.draw.rect()}:
	\myindex{pyg}{\texttt{draw}!\texttt{rect()}}\\
	\url{https://pyga.me/docs/ref/draw.html#pygame.draw.rect}
	
	\item \texttt{pygame.gfxdraw.pixel()}:
	\myindex{pyg}{\texttt{gfxdraw}!\texttt{pixel()}}\\ \url{https://pyga.me/docs/ref/gfxdraw.html#pygame.gfxdraw.pixel}
	
	\item \texttt{pygame.mouse.get\_pos()}:
	\myindex{pyg}{\texttt{mouse}!\texttt{get\_pos()}}\\
	\url{https://pyga.me/docs/ref/mouse.html#pygame.mouse.get_pos}
	
	\item \texttt{pygame.mouse.get\_pressed()}:
	\myindex{pyg}{\texttt{mouse}!\texttt{get\_pressed()}}\\
	\url{https://pyga.me/docs/ref/mouse.html#pygame.mouse.get_pressed}
	
	\item \texttt{pygame.rect.Rect}:
	\myindex{pyg}{\texttt{rect}!\texttt{Rect}}\\
	\url{https://pyga.me/docs/ref/rect.html}
	
	\item \texttt{pygame.Surface.set\_at()}:
	\myindex{pyg}{\texttt{Surface}!\texttt{set\_at()}}\\
	\url{https://pyga.me/docs/ref/surface.html#pygame.Surface.set_at}
	
	\item \texttt{pygame.Window.size}:
	\myindex{pyg}{\texttt{Window}!\texttt{size}}\\
	\url{https://pyga.me/docs/ref/window.html#pygame.Window.size}
	
\end{itemize}

\subsection{Homework}

Please have a look at \url{https://pyga.me/docs/ref/draw.html} and then try to solve the following small exercises: 

\begin{enumerate}
	\item Program the following: Randomly choose a point and a radius. Using these values, draw a circle with a random, semi-transparent color. As an additional challenge, make sure that the circle may touch the edge of the window at most, but must not go beyond it.

	\item Create a window with a color gradient from blue in the upper-left corner 	to red in the lower-left corner. 	Then draw two white filled circles with the same radius. One circle should be created using texttt{draw.circle()}\myindex{pyg}{\texttt{draw}!\texttt{circle()}} and the other using \texttt{draw.aacircle()}\myindex{pyg}{\texttt{draw}!\texttt{aacircle()}}\randnotiz{aacircle()}. 	Compare the results.
	
	\item Draw 10~random lines in a window using \texttt{draw.aaline()}\myindex{pyg}{\texttt{draw}!\texttt{aaline()}}\randnotiz{aaline()}. Each line should start at the edge of the window and end at the edge.	Then apply \texttt{draw.flood\_fill()}\myindex{pyg}{\texttt{draw}!\texttt{flood\_fill()}}\randnotiz{flood\_fill()}	to the center of the window and observe the effect.
	
	\item Try to draw the Moonlander like in \abbref[vref]{picMoonlander00} using only the functions in \texttt{pygame.\-draw}. Of course, you may also choose any other non-trivial object instead.
\end{enumerate}

\myebild{moonlander00.png}{1.0}{Example: Drawing a Moonlander}{picMoonlander00}
	% !TeX spellcheck = en_US
\newpage
%%%%%%%%%%%%%%%%%%%%%%%%%%%%%%%%%%%%%%%%%%%%%%%%%%%%%%%%%%%%%%%%%%%%%%%%%%%
\section{Load and Blit Bitmaps}\label{secBitmapLaden}\index{Bitmap}\index{Bitmap!load}\index{Bitmap!blit}
\begin{diskbox}
	\url{https://github.com/adamsralf/pygame_book/tree/main/src/00%20Introduction/03%20Bitmaps}
\end{diskbox}
%%%%%%%%%%%%%%%%%%%%%%%%%%%%%%%%%%%%%%%%%%%%%%%%%%%%%%%%%%%%%%%%%%%%%%%%%%%
\subsection{Introduction}

\begin{hintbox}[config.py]
	In Python, it is common practice to move program settings, global variables, and similar configuration data into a file named \texttt{config.py}.
\end{hintbox}

\lstsource{SRC/00 Introduction/03 Bitmaps/config.py}{1}{3}{python}{Load and blit bitmaps: config.py (1)}{src03Config01}

The file is then imported and usually given a shorter name, so its contents can be used across the whole project.

\lstsource{SRC/00 Introduction/03 Bitmaps/invader01.py}{3}{30}{python}{Load and blit bitmaps, Version 1.0}{srcInvader01}

In \srcref{srcInvader01}, two bitmaps -- in this case two \Gls{png} files -- are loaded and displayed on the screen.

Loading is done using the function \texttt{pygame.image.load()}\myindex{pyg}{\texttt{image}!\texttt{load()}}\randnotiz{load()}. In \zeiref{srcInvader0101}f., the bitmaps -- also called \glspl{sprite} -- are loaded and converted into a \texttt{Surface} object. In \zeiref{srcInvader0102} the two bitmaps are then printed onto the \texttt{screen} surface without any further processing using \texttt{pygame.Sur\-face.\-blit()}\myindex{pyg}{\texttt{Surface}!\texttt{blit()}}\randnotiz{blit()}. The first parameter of \texttt{blit()} is the \texttt{Surface} object that is to be drawn, followed by the position. Here, the horizontal (x) coordinate is specified first, and then the vertical (y) coordinate. You can \emph{admire} the result in \abbref[vref]{picInvader01}.

\begin{hintbox}[The coordinate system’s origin]
	Unlike in school mathematics, the origin is not at the lower left, but at the upper left. 
\end{hintbox}

\myebild{invader01.png}{0.8}{Load and blit bitmaps, Version 1.0}{picInvader01}

We now want to adapt the bitmaps a bit to better suit our needs. First, the documentation recommends converting the bitmap into a format that is easier for Pygame to process after loading.
In addition, I want to adjust the size ratios of the two bitmaps, because the enemy appears too large compared to the defender.

\lstsource{SRC/00 Introduction/03 Bitmaps/invader02.py}{15}{19}{python}{Load and blit bitmaps, Version 1.1}{srcInvader02}

\begin{wrapfigure}[8]{r}{4cm}%
    \vspace{-1em}%
	\myfigure{invader02.png}{0.8}{Sizes OK}{picInvader02}%
\end{wrapfigure}%
The function \texttt{pygame.Surface.load()} returned a \texttt{Surface} object. The \texttt{Surface} class now provides a method that performs the desired conversion: \texttt{pygame.Surface\-.convert()}\myindex{pyg}{\texttt{Surface}!\texttt{convert()}}\randnotiz{convert()}. As an example, please refer to \zeiref{srcInvader0201}.\label{pageTransparenz}
 
Resizing is done using \texttt{pygame\-.trans\-form\-.scale()}\myindex{pyg}{\texttt{transform}!\texttt{scale()}|underline}\randnotiz{scale()}. In \zeiref{srcInvader0202}, the image is scaled to the specified $(width, height)$ in the unit of pixels. The result shown in \abbref{picInvader02} does not quite meet my
expectations.

I do like the size ratios now, but why does a black background suddenly appear? The reason is that the conversion using \texttt{convert()} caused the transparency information\index{Transparency|underline} to be lost. Transparency controls how \emph{see-through} a pixel is. This is achieved by storing not only the three RGB values for each
pixel, but also an opacity value. This additional piece of information is called the
\emph{alpha channel}\index{Alpha channel}\randnotiz{Alpha channel}.

I now have two options to make this transparency available again:
\begin{hintbox}[Transparency of loaded bitmaps]
\begin{itemize}
	\item \texttt{pygame.Surface.convert\_alpha()}\myindex{pyg}{\texttt{Surface}!\texttt{convert\_alpha()}}: Put very simply, the alpha channel is preserved during the
	conversion.	If possible, this should be your method of choice.
	
	\item \texttt{pygame.Surface.set\_colorkey()}\myindex{pyg}{\texttt{Surface}!\texttt{set\_colorkey()}}: Here, you pass the color that Pygame should skip when drawing onto the target surface. This can lead to two disadvantages. First, transparency levels between fully visible and fully invisible cannot be represented. It would therefore not be possible to make a pixel \emph{semi-transparent}. Second, parts of the figure that have the same color as the background will also appear transparent. If our alien had a black eye in the middle, it would disappear and the alien would have a hole in the center.
\end{itemize}
\end{hintbox}

\lstsource{SRC/00 Introduction/03 Bitmaps/invader03.py}{15}{20}{python}{Load and blit bitmaps, Version 1.2}{srcInvader03}


\begin{wrapfigure}[6]{r}{4cm}
%    \vspace{-1.5em}
	\myfigure{invader03.png}{0.8}{$\alpha$ OK}{picInvader03}
\end{wrapfigure}In \srcref[vref]{srcInvader03}, I tried out both variants, and you can see the result in \abbref[vref]{picInvader03}. Now both bitmaps are visible without a black background; the white background shows through again.

What I still do not like is the position and the number of attackers. I want to place the defender centered at the bottom and the attackers along the top edge of the screen, arranged so that they are horizontally \gls{aequidistant}\index{equidistant}\randnotiz{equidistant}. There are two ways to do this: I can specify a minimum spacing and compute the number of attackers, or I can specify the maximum number of attackers and compute the spacing. Which approach I choose depends on my game logic; in most cases, the number is fixed in advance.

\lstsource{SRC/00 Introduction/03 Bitmaps/config.py}{1}{4}{python}{Load and blit bitmaps: config.py (2)}{src03Config02}

\lstsource{SRC/00 Introduction/03 Bitmaps/invader04.py}{15}{42}{python}{Bitmap: positioning, Version 1.4}{srcInvader04}

In \srcref[vref]{srcInvader04}, the requirements above have been implemented. Let us take a closer look at the individual aspects. 

The defender should be positioned centered at the bottom. We remember that the function \texttt{blit()} also expects the coordinates of the upper-left corner. So this position has to be calculated first. For the sake of clarity -- in a normal source code I would not write the calculation in such a fine-grained way -- I calculate the coordinates separately here.

The top edge is fairly easy to determine. If we set \texttt{defender\_top} to the full height of the screen, \texttt{cfg.WINDOWS\_HEIGHT}, we would not see the defender because it would stick out below the screen completely. So by how many pixels do we need to move the top edge upward? Exactly by the height of the spaceship, \SI{30}{px}:


\lstset{firstnumber=18}
\begin{lstlisting}
	defender_pos_top = cfg.WINDOWS_HEIGHT - 30
\end{lstlisting}

However, I do not like how the defender looks glued to the edge this way. So I give it an additional \SI{5}{px} of space, making it look more as if it were floating in space:

\lstset{firstnumber=18}
\begin{lstlisting}
	defender_pos_top = cfg.WINDOWS_HEIGHT - 30 - 5
\end{lstlisting}

In \zeiref{srcInvader0401}, the distance of the left edge of the bitmap from the edge of the playfield is calculated. Using


\lstset{firstnumber=17}
\begin{lstlisting}
	defender_pos_left = cfg.WINDOWS_WIDTH // 2
\end{lstlisting}

we would calculate the horizontal center of the screen. However, we cannot use this value, because it would place the left edge of the defender at the horizontal center -- that is, too far to the right (see \abbref[vref]{picInvader04a}).

\myebild{invader04a.png}{0.7}{Bitmaps positioning defender}{picInvader04a}

However, we can determine exactly how many pixels we have shifted too far to the right and then subtract this value: it is exactly half of the width of the defender (here \SI{30}{px}):

\begin{lstlisting}{firstnumber=17}
	defender_pos_left = cfg.WINDOWS_WIDTH // 2 - 30 // 2
\end{lstlisting}

With the help of a little fraction arithmetic, the expression can be simplified:

\begin{lstlisting}{firstnumber=17}
	defender_pos_left = (cfg.WINDOWS_WIDTH - 30) // 2
\end{lstlisting}

Now we move on to the aliens. In the first approach, we want to display them one after another at the top without any overlap. The top edge \texttt{alien\_top} can be set to a constant value with a pleasant distance of \SI{10}{px} from the upper edge:

\lstset{firstnumber=35}
\begin{lstlisting}
	alien_top = 10 
\end{lstlisting}

The left position \texttt{alien\_left} has to be determined individually for each alien. Since they are placed directly next to each other at first, the left edge of one alien is exactly one alien width away from the left edge of the next one. So if I am at the $0$th alien, its horizontal coordinate is directly at the left edge of the screen. For the $1$st alien it is exactly $1 \times \SI{50}{px}$, for the $2$nd exactly $2 \times \SI{50}{px}$, and so on, since the alien is
\SI{50}{px} wide. Written as a \forSchleife, it looks like this:

\lstset{firstnumber=36}
\begin{lstlisting}
	for i in range(cfg.ALIENS_NOF):
	   alien_left = i * 50
	   alien_pos = (alien_left, alien_top)
	   screen.blit(alien_image, alien_pos)
\end{lstlisting}


\myebild{invader04b.png}{0.8}{Bitmaps positioning alien, Version 1}{picInvader04b}

The entire remaining space after the last alien can now be distributed before, between, and after the aliens in such a way that the spacing is the same between the aliens, between the leftmost alien and the left edge of the screen, and between the rightmost alien and the right edge of the screen. So how many gaps are there? First of all, the two outer gaps on the far left and far right -- that makes~2:

\lstset{firstnumber=25}
\begin{lstlisting}
	space_nof = 2  
\end{lstlisting}

Then there are the gaps between the aliens. This is always one less than the number of aliens (count it to check!):

\lstset{firstnumber=25}
\begin{lstlisting}
	space_nof = cfg.ALIENS_NOF - 1 + 2
\end{lstlisting}

thus:

\lstset{firstnumber=25}
\begin{lstlisting}
	space_nof = cfg.ALIENS_NOF + 1     
\end{lstlisting}

Now the available space \texttt{space\_available} behind the aliens still has to be calculated.
I do this by first calculating the space occupied by the aliens, \texttt{space\_\-for\_\-aliens}

\lstset{firstnumber=28}
\begin{lstlisting}
	space_for_aliens = cfg.ALIENS_NOF * 50     
\end{lstlisting}

and subtract this value from the screen width.

\lstset{firstnumber=24}
\begin{lstlisting}
	space_availible = cfg.WINDOWS_WIDTH - space_for_aliens
\end{lstlisting}

So I now have the available space stored in \texttt{space\_available} and the number of gaps that need to be filled stored in \texttt{space\_nof}. If I now want to determine the width of the gaps,
\texttt{space\_between\_\-aliens}, I simply have to divide these two values:

\lstset{firstnumber=26}
\begin{lstlisting}
	space_between_aliens = space_availible // space_nof
\end{lstlisting}

Now we only need to adjust the calculation of \texttt{alien\_left}. First, we shift the starting position by one such gap (see \abbref[vref]{picInvader04c}):

\lstset{firstnumber=36}
\begin{lstlisting}
	for i in range(cfg.ALIENS_NOF):
	   alien_left = space_between_aliens + i * 50
	   alien_pos = (alien_left, alien_top)
	   screen.blit(alien_image, alien_pos)
\end{lstlisting}

\myebild{invader04c.png}{0.8}{Bitmaps positioning alien, Version 2}{picInvader04c}

Now the distance from one left edge to the next, which previously consisted only of the width of the alien, must be extended by the spacing \texttt{space\_between\_aliens}:

\lstset{firstnumber=36}
\begin{lstlisting}
	for i in range(cfg.ALIENS_NOF):
	   alien_left = (i + 1) * space_between_aliens + i * 50
	   alien_pos = (alien_left, alien_top)
	   screen.blit(alien_image, alien_pos)
\end{lstlisting} 

And just like that, everything fits (see \abbref[vref]{picInvader04d}).

\myebild{invader04d.png}{0.8}{Bitmaps positioning alien, Version 3}{picInvader04d}

\begin{hintbox}[Why am I explaining this in so much detail?]
	Not because it’s particularly difficult, but because when you’re just starting out with programming, it’s easy to feel overwhelmed when trying to understand other people’s source code. Where do numbers like {–1} even come from? This step-by-step approach is meant to reassure you. Complex solutions are often built from many small, simple pieces put together.
\end{hintbox}


%%%%%%%%%%%%%%%%%%%%%%%%%%%%%%%%%%%%%%%%%%%%%%%%%%%%%%%%%%%%%%%%%%%%%%%%%%%%%%%%
\subsection{More Input}
%%%%%%%%%%%%%%%%%%%%%%%%%%%%%%%%%%%%%%%%%%%%%%%%%%%%%%%%%%%%%%%%%%%%%%%%%%%
\subsubsection{Blitting Parts of a Bitmap}

Very often, only parts of a bitmap need to be blitted. For this purpose, the function \texttt{Surface.blit()}\myindex{pyg}{\texttt{Surface}!\texttt{blit()!\texttt{area}}} provides the parameter \texttt{area}\randnotiz{area}.

\begin{wrapfigure}[16]{l}{7cm}
	\vspace{-1.0em}
	\myfigure{part\_of\_a\_bitmap.png}{0.5}{Tiles to build a forest}{picPartofabitmap00}
\end{wrapfigure}As an example, I use a bitmap that consists of tiles of size $\SI{32}{px}\times\SI{32}{px}$. From these tiles, I could build a forest and lake landscape for a game. The logic of this small application is that, using the arrow keys, I jump \SI{32}{px} to the right, left, up, or down, and in this way move from tile to tile. The tile currently selected is marked with a red rectangle (\zeiref{srcBlitofparts02}) and drawn as a sub-image into the lower right corner of the window -- in this example, a tent or a small hut.

Using \texttt{clamp()}\myindex{pyg}{\texttt{math}!\texttt{clamp()}}\randnotiz{clamp()} I make sure that I cannot wander outside the image.

 In \zeiref{srcBlitofparts01}, this approach is applied as shown in \abbref[vref]{picPartofabitmap00}. The first parameter of \texttt{blit()} is the bitmap -- here referenced as \texttt{image} -- that is to be drawn. The second parameter \texttt{(512-32, 512-32)} specifies the position within the window where the image should be drawn. How do we arrive at these values? The entire image has a width of \SI{512}{px} and a height of \SI{512}{px}. Each tile has a size of $\SI{32}{px}\times\SI{32}{px}$, and therefore the upper-left corner of the tile must be positioned \SI{32}{px} away from the edges. The third parameter -- the \texttt{area} -- is a 4-tuple. Its values represent \emph{left}, \emph{top}, \emph{width}, and \emph{height}. The variables \verb+x+ and \verb+y+ are determined by movement using the arrow keys, which is a preview of later sections of this book (see \secref[vref]{secMoving}). The width and height of the tiles are fixed at \SI{32}{px}.

\lstsource{SRC/00 Introduction/03 Bitmaps/blitofparts.py}{4}{39}{python}{Blit a part of a bitmap}{blitofparts01}

%%%%%%%%%%%%%%%%%%%%%%%%%%%%%%%%%%%%%%%%%%%%%%%%%%%%%%%%%%%%%%%%%%%%%%%%%%%
\subsubsection{Message Box}\index{Messagebox}

A message box is a simple way to communicate an \gls{information}, a \gls{warning}, or an \gls{error} to the player. Its appearance can only be customized to a very limited extent and it usually does not fit into the visual design concept of a game. For this reason, message boxes are almost never used for in-game interactions. However, they are well suited for use during installation or configuration, or when real errors occur that require quick and clear interaction.


\begin{wrapfigure}[14]{l}{5cm}
	\vspace{-1.5em}
\renewcommand{\figurename}{Fig.}%
	\begin{tikzpicture}%
		\node (info) at (0, 4.0) {\fbox{\includegraphics[scale=0.5]{message_info.png}}};
		\node (warn) at (0, 2.0) {\fbox{\includegraphics[scale=0.5]{message_warning.png}}};
		\node (error) at (0, 0.0) {\fbox{\includegraphics[scale=0.5]{message_error.png}}};
	\end{tikzpicture}%
	\caption{Messageboxes}\label{picMessages}%
\renewcommand{\figurename}{Figure}%
\end{wrapfigure}Here, I have only shown the crucial part of the program in \srcref[vref]{messageboxes01}; the rest is not important. The call is made using \texttt{pygame.display.message\_box()}\myindex{pyg}{\texttt{display}!\texttt{message\_box()}|underline}\randnotiz{message\-\_box()}.

In \zeiref{srcMessageboxes01}, an information message is displayed. The first parameter is the text of the window’s title bar. The second parameter is the message text, which can also be much longer and formatted. After that, the type of the message box is specified. There are three -- self-explanatory -- types available: \texttt{info}, \texttt{warn}, and \texttt{error}. This setting determines the icon that is shown (see \abbref[vref]{picMessages}). The call for an error message works in exactly the same way, as shown in \zeiref{srcMessageboxes03}.

A bit more is demonstrated with the warning call starting at \zeiref{srcMessageboxes02}. First of all, you can see that the named parameter \texttt{buttons} \randnotiz{Button}\index{Button} is passed. It contains a list of strings. Each string is the label text of one displayed button (see \abbref[vref]{picMessages}). So how do we find out which button was pressed? By using the return value. Internally, the list of button labels is numbered, and each button is assigned an index\randnotiz{Index}\index{Index (Button)}. The index value of the pressed button is then returned and -- as in this example with \texttt{a} -- stored in a variable. Which button has which index, and how to proceed afterwards, is something you need to keep in mind. In this example, the value is simply printed to demonstrate the effect.

\lstsource{SRC/00 Introduction/03 Bitmaps/messageboxes.py}{19}{32}{python}{Types of Messageboxes}{messageboxes01}

%%%%%%%%%%%%%%%%%%%%%%%%%%%%%%%%%%%%%%%%%%%%%%%%%%%%%%%%%%%%%%%%%%%%%%%%%%%
\subsubsection{Creating Bitmaps}\label{secCreateBitmaps}
Bitmaps do not necessarily have to be loaded from disk using \texttt{pygame.image.load()}. It is also possible to create a bitmap at runtime using \texttt{pygame.Surface()}\myindex{pyg}{\texttt{Surface()}|underline}\randnotiz{Surface()}. Let us take another look at \secref[vref]{secLandscape}. In that section, an animated landscape was created using drawing primitives. These primitives are drawn in every frame -- for example 60 times per second. In the end, this is an enormous waste of computing time.

\begin{hintbox}[Creating bitmaps during runtime]
	A much more efficient approach is to draw the graphics once onto a \texttt{Surface} object and then only blit this bitmap to the correct position. Blitting \texttt{Surface} objects is much faster than drawing the shapes again and again.
\end{hintbox}

Let us look at this starting with the \texttt{Meadow} class. In the constructor, a \texttt{Surface} object is created in \zeiref{srcLandscapeBit00}. For that, we need a width and a height. Both values are computed beforehand and stored in the local variable \texttt{widthheight}; this does not need to be a class attribute anymore, because the information is only required to create the bitmap and is not needed later. 

After that, the surface is filled completely with a green color. Now we have a finished meadow graphic, and in \texttt{draw()} it only needs to be blitted.

\newpage
\lstsource{SRC/00 Introduction/03 Bitmaps/landscape.py}{6}{16}{python}{Creating Bitmaps: Class \texttt{Meadow}}{landscapebitmaps01}

The \texttt{Sky} class follows a similar approach. In the constructor, a \texttt{Surface} object is created and filled with a shade of blue, and in \texttt{draw()} the bitmap is simply blitted. Only \texttt{update()} remains computationally more expensive, since the color of the sky changes depending on the position of the sun.

\lstsource{SRC/00 Introduction/03 Bitmaps/landscape.py}{18}{34}{python}{Creating Bitmaps: Class \texttt{Sky}}{landscapebitmaps02}

The \texttt{Tree} class is more similar to \texttt{Meadow}. In \zeiref{srcLandscapeBit01}, the bitmap is created and its contents are drawn using drawing primitives. However, the \texttt{Surface} object is created with the additional parameter \texttt{pygame.SRCALPHA}\myindex{pyg}{\texttt{SRCALPHA}|underline}\randnotiz{SRCALPHA}. This parameter ensures that the unpainted background of the \texttt{Surface} object remains transparent. Otherwise, a black background with the size of the \texttt{Surface} object would appear around the tree; with transparency enabled, the sky, sun, and meadow can be seen through it.

\lstsource{SRC/00 Introduction/03 Bitmaps/landscape.py}{36}{46}{python}{Creating Bitmaps: Class \texttt{Tree}}{landscapebitmaps03}

The same approach is used in the \texttt{House} class in \zeiref{srcLandscapeBit03}.


\lstsource{SRC/00 Introduction/03 Bitmaps/landscape.py}{49}{64}{python}{Creating Bitmaps: Class \texttt{House}}{landscapebitmaps04}

And, for the sake of completeness, the \texttt{Sun} class as well.

\lstsource{SRC/00 Introduction/03 Bitmaps/landscape.py}{67}{85}{python}{Creating Bitmaps: Class \texttt{Sun}}{landscapebitmaps05}

%%%%%%%%%%%%%%%%%%%%%%%%%%%%%%%%%%%%%%%%%%%%%%%%%%%%%%%%%%%%%%%%%%%%%%%%%%%%%%%%
\newpage
\subsection{What was new?}

\begin{hintbox}[]
	\begin{itemize}
		\item 	The position values are needed when drawing on the screen. Later, we will see that we also need these position values for other questions, such as \gls{kollisionserkennung}. The position always refers to the upper-left corner of the bitmap, or in other words: \emph{The coordinate system has its origin in the upper left, not in the lower left.}
		
		\item We often have to perform basic geometry calculations, and it is best to do them step by step.
		
		\item For such geometry calculations, the following information is needed: the position of the bitmap, its width, and its height. So far, we have treated width and height as constants, but that is not a good long-term solution.
	\end{itemize}
\end{hintbox}

\begin{pygbox}
	\begin{itemize}
	\item \texttt{pygame.display.Info()}	
	\myindex{pyg}{\texttt{display}!\texttt{Info()}}:\\	\url{https://pyga.me/docs/ref/display.html#pygame.display.Info}
	
	\item \texttt{pygame.display.message\_box()}    \myindex{pyg}{\texttt{display}!\texttt{message\_box()}}:\\    \url{https://pyga.me/docs/ref/display.html#pygame.display.message_box}

	\item \texttt{pygame.image}
	\myindex{pyg}{\texttt{image}}:\\
	\url{https://pyga.me/docs/ref/image.html}

	\item \texttt{pygame.image.load()}
	\myindex{pyg}{\texttt{image}!\texttt{load()}}:\\
	\url{https://pyga.me/docs/ref/image.html#pygame.image.load}
	
	\item \texttt{pygame.Surface()}:
	\myindex{pyg}{\texttt{Surface()}}\\
	\url{https://pyga.me/docs/ref/surface.html}
	
	\item \texttt{pygame.Surface.blit()}:
	\myindex{pyg}{\texttt{Surface}!\texttt{blit()}}\\
	\url{https://pyga.me/docs/ref/surface.html#pygame.Surface.blit}

	\item \texttt{pygame.Surface.convert()}:
	\myindex{pyg}{\texttt{Surface}!\texttt{convert()}}\\
	\url{https://pyga.me/docs/ref/surface.html#pygame.Surface.convert}
	
	\item \texttt{pygame.Surface.convert\_alpha()}:
	\myindex{pyg}{\texttt{Surface}!\texttt{convert\_alpha()}}\\
	\url{https://pyga.me/docs/ref/surface.html#pygame.Surface.convert\_alpha}
	
	\item \texttt{pygame.Surface.set\_colorkey()}:
	\myindex{pyg}{\texttt{Surface}!\texttt{set\_colorkey()}}\\
	\url{https://pyga.me/docs/ref/surface.html#pygame.Surface.set\_colorkey}
	
	\item \texttt{pygame.SRCALPHA}:
	\myindex{pyg}{\texttt{SRCALPHA}}\\
	\url{https://pyga.me/docs/ref/surface.html}

	\item \texttt{pygame.transform.scale()}:
	\myindex{pyg}{\texttt{transform}!\texttt{scale()}}\\
	\url{https://pyga.me/docs/ref/transform.html#pygame.transform.scale}
	
\end{itemize}
\end{pygbox}
%%%%%%%%%%%%%%%%%%%%%%%%%%%%%%%%%%%%%%%%%%%%%%%%%%%%%%%%%%%%%%%%%%%%%%%%%%%%%%%%
\subsection{Homework}

\begin{enumerate}
	\item Look for freely available sources of game graphics (sprites). If you know that you want to work with this more intensively, also look for sources that are behind a paywall.

	\item Blit your own graphics to sensible positions within your window. 
	
	\item Try to build a realistic background for a simple game using graphics. If needed, make use of the option to blit sub-images from a larger bitmap.
\end{enumerate}
	% !TeX spellcheck = en_US
\newpage
\section{Moving Bitmaps}\index{Bitmap!moving}
\subsection{Class \texttt{Rect}/\texttt{FRect}}
 In the summary of the previous chapter, we noted that when displaying bitmaps we need the \emph{upper-left corner} as the position value, and we need the \emph{height} and \emph{width}, for example for distance calculations. These values can be conveniently encoded in a rectangle. For this purpose, Pygame provides the classes \texttt{pygame.rect.Rect}\myindex{pyg}{\texttt{rect}!\texttt{Rect}|underline}\randnotiz{Rect} (integers only) and \texttt{pygame.rect.FRect}\myindex{pyg}{\texttt{rect}!\texttt{FRect}|underline}\randnotiz{FRect} (floating-point numbers). In \abbref[vref]{picRect01}, you can find what I consider to be the most important attributes of this class.

\begin{figure}[H]
\begin{tikzpicture}
%Bildschirm Koordinatensystem
\draw
 (0,10) node (o) {(0,0)}
 (0,0) node (y) {(0,400)}
 (14,10) node (x) {(600,0)}
;
\draw[->] (o) -- (x);
\draw[->] (o) -- (y);

%Rechteck
\draw
 (3,7) node (topleft) {}
 (13,3) node (bottomright) {}
 (8.0,5) node (centerxy) {}
;
\filldraw[fill=green!20] (topleft) rectangle (bottomright);
\draw (topleft) node[above, rotate=45] {\emph{topleft}};
\draw (bottomright) node[below, rotate=45] {\emph{bottomright}};
\filldraw[fill=black] (centerxy) circle (2pt) node[below, rotate=45] {\emph{center}};
\filldraw[fill=black] (topleft) circle (2pt);
\filldraw[fill=black] (bottomright) circle (2pt);

%Left
\draw
 (-0.15,3.4) node (left1) {}
 (3.15,3.4) node (left2) {}
;
\draw[<->, very thick, red] (left1) -- (left2) node[above, black, xshift=-2.6cm] {left};


%Right
\draw
 (-0.15,4.2) node (right1) {}
 (13.15,4.2) node (right2) {}
;
\draw[<->, very thick, red] (right1) -- (right2) node[above, black, xshift=-12.45cm] {right};

%Centerx
\draw
 (-0.15,5.0) node (cx1) {}
 (8.05,5.0) node (cx2) {}
;
\draw[<->, very thick, red] (cx1) -- (cx2) node[above, black, xshift=-7.1cm] {centerx};

%top
\draw
 (9.6,10.15) node (top1) {}
 (9.6,6.85) node (top2) {}
;
\draw[<->, very thick, blue] (top1) -- (top2) node[above, black, yshift=2.4cm, rotate=90] {top};

%bottom
\draw
 (8.8,10.15) node (bottom1) {}
 (8.8,2.85) node (bottom2) {}
;
\draw[<->, very thick, blue] (bottom1) -- (bottom2) node[above, black, yshift=6.1cm, rotate=90] {bottom};

%Centery
\draw
 (8.0,10.15) node (cy1) {}
 (8.0,4.95) node (cy2) {}
;
\draw[<->, very thick, blue] (cy1) -- (cy2) node[above, black, yshift=4cm, rotate=90] {centery};

%Width
\draw
 (2.85,6.2) node (w1) {}
 (13.15,6.2) node (w2) {}
;
\draw[<->, very thick, purple] (w1) -- (w2) node[above, black, xshift=-1.5cm] {width};

%height
\draw
 (12.3,7.15) node (h1) {}
 (12.3,2.85) node (h2) {}
;
\draw[<->, very thick, purple] (h1) -- (h2) node[above, black, yshift=2.4cm, rotate=90] {height};

\end{tikzpicture}
\caption{Elements of a \texttt{Rect}-Object}\label{picRect01}
\end{figure}

\myindex{pyg}{\texttt{rect}!\texttt{Rect}!\texttt{centerx}}%
\myindex{pyg}{\texttt{rect}!\texttt{Rect}!\texttt{right}}%
\myindex{pyg}{\texttt{rect}!\texttt{Rect}!\texttt{left}|underline}%
\myindex{pyg}{\texttt{rect}!\texttt{Rect}!\texttt{centery}}%
\myindex{pyg}{\texttt{rect}!\texttt{Rect}!\texttt{bottom}}%
\myindex{pyg}{\texttt{rect}!\texttt{Rect}!\texttt{top}|underline}%
\myindex{pyg}{\texttt{rect}!\texttt{Rect}!\texttt{topleft}}%
\myindex{pyg}{\texttt{rect}!\texttt{Rect}!\texttt{bottomright}}%
\myindex{pyg}{\texttt{rect}!\texttt{Rect}!\texttt{center}|underline}%
\myindex{pyg}{\texttt{rect}!\texttt{Rect}!\texttt{width}|underline}%
\myindex{pyg}{\texttt{rect}!\texttt{Rect}!\texttt{height}|underline}%
\myindex{pyg}{\texttt{rect}!\texttt{Rect}!\texttt{size}|underline}%

\myindex{pyg}{\texttt{rect}!\texttt{FRect}!\texttt{centerx}}%
\myindex{pyg}{\texttt{rect}!\texttt{FRect}!\texttt{right}}%
\myindex{pyg}{\texttt{rect}!\texttt{FRect}!\texttt{left}|underline}%
\myindex{pyg}{\texttt{rect}!\texttt{FRect}!\texttt{centery}}%
\myindex{pyg}{\texttt{rect}!\texttt{FRect}!\texttt{bottom}}%
\myindex{pyg}{\texttt{rect}!\texttt{FRect}!\texttt{top}|underline}%
\myindex{pyg}{\texttt{rect}!\texttt{FRect}!\texttt{topleft}}%
\myindex{pyg}{\texttt{rect}!\texttt{FRect}!\texttt{bottomright}}%
\myindex{pyg}{\texttt{rect}!\texttt{FRect}!\texttt{center}|underline}%
\myindex{pyg}{\texttt{rect}!\texttt{FRect}!\texttt{width}|underline}%
\myindex{pyg}{\texttt{rect}!\texttt{FRect}!\texttt{height}|underline}%
\myindex{pyg}{\texttt{rect}!\texttt{FRect}!\texttt{size}|underline}%

In the figure, line segments are shown in normal font, while points are shown in \textit{italic font}. The segments are one-dimensional, and the points are two-dimensional $(x, y)$. The coordinate~$x$ represents the horizontal distance from the origin of the coordinate system, and~$y$ represents the vertical distance. The meaning of the individual labels should be self-explanatory.

The nice advantage is that all these values are computed from each other. For example, if I set \texttt{topleft = (10, 10)} and \texttt{width, height = 30, 40}, all other values are calculated automatically. I no longer need to compute the right edge manually using \texttt{left + width}; instead, I can directly use \texttt{right}.

It is also often useful to work with the center position \texttt{center} or the corresponding coordinates \texttt{centerx} and \texttt{centery}. If I change the center to \texttt{center = (100, 10)}, all other values are updated accordingly and do not need to be recalculated by me -- very convenient.

Let us take a look at a reduced version of the last source code. In \srcref[vref]{srcInvader05}, the \texttt{Rect} class is already being used. For example, in \zeiref{srcInvader0504a} the window dimensions are stored in a \texttt{Rect} object. 

\lstsource{SRC/00 Introduction/04 Moving/config.py}{1}{999}{python}{Moving Bitmaps: \texttt{config.py}}{srcInvaderConfig00}

As a result, the screen information can be accessed conveniently and without performing manual calculations in \zeiref{srcInvader0504b}, \zeiref{srcInvader0502}, and \zeiref{srcInvader0503}.

\lstsource{SRC/00 Introduction/04 Moving/invader05.py}{6}{36}{python}{Moving Bitmaps, Version 1.0}{srcInvader05}

For \texttt{Surface} objects, we can conveniently create a \texttt{Rect} object using \texttt{pygame.Surface\-.get\-\_rect()}\myindex{pyg}{\texttt{Surface}!\texttt{get\_rect()}}\randnotiz{get\_rect()} (\zeiref{srcInvader0501}). Positioning can now be handled much more easily via the attributes. For example, the center no longer needs to be part of a calculation; instead, I can directly set the horizontal center to half the window width (\zeiref{srcInvader0502}). Likewise, the vertical coordinate no longer has to be considered from the top edge; instead, I can specify the distance of the bottom edge from the screen edge in a much more intuitive way (\zeiref{srcInvader0503}). And as a final bonus, the \texttt{Rect} object can even be passed directly as a parameter to the \texttt{blit()} function\randnotiz{blit()}\myindex{pyg}{\texttt{Surface}!\texttt{blit()}} (\zeiref{srcInvader0504}).

\myebild{invader05.png}{0.8}{Moving Bitmaps, Version 1.0}{picInvader05}

The result is unspectacular (see \abbref[vref]{picInvader05}) and has
nothing to do with movement yet.

\subsection{Introduction}

 Movement in games is animated by changing positions. If the spaceship is supposed to move to the right, the horizontal coordinate of the ship therefore has to increase. Which horizontal coordinate you use for this -- \texttt{left}, \texttt{right}, or \texttt{centerx} -- can be chosen depending on your game logic. In our example, this does not matter, so I will use \texttt{left}.

\lstset{firstnumber=26}
\begin{lstlisting}
defender_rect.left += 1
\end{lstlisting}

This small addition alone now causes our spaceship to move to the right. The \texttt{+1} encodes two pieces of information:

\begin{itemize}
	\item \textbf{Direction:}\randnotiz{Direction}\index{Direction} Here, the sign is \texttt{+}. This increases the value of \texttt{left} in each loop iteration; as a result, the left edge of the graphic moves to the right. If you wanted to move to the left, the sign would have to be	\texttt{-}.	In that case, the horizontal coordinate would become smaller and approach~0. Completely analogously, the sign also controls the direction in the vertical axis.	A \texttt{+} moves the graphic downward, and a \texttt{-} moves it upward.	Try it out!
	
	\item \textbf{Speed:}\randnotiz{Speed}\index{Speed}	The \texttt{1} specifies by how much the value of \texttt{left}	changes. The larger this value is, the larger the jumps between frames;	the movement appears faster.
\end{itemize}


\lstsource{SRC/00 Introduction/04 Moving/invader05b.py}{17}{28}{python}{Moving Bitmaps, Version 1.2}{srcInvader05b}

These two pieces of information are now used in \srcref[vref]{srcInvader05b} to make movement much more flexible. In \zeiref{srcInvader0505}, the speed is now represented by the variable \texttt{defender\_speed}. This would allow us to change the speed dynamically during the game, for example when accelerating by firing rocket thrusters.

The direction is also stored in a variable in \zeiref{srcInvader0506}: \texttt{defender\_direction}. At the moment it is positive, but we will soon see that we can also use it for changing direction.

Both values can now be used in \zeiref{srcInvader0507} to calculate the new horizontal position.

If you run the program, the defender will leave the screen after a while and disappear beyond the right edge, never to be seen again. Let us now use our rectangle for a first simple collision check. I want the spaceship to \emph{bounce} off the edges and reverse its direction.

\lstsource{SRC/00 Introduction/04 Moving/invader05c.py}{27}{32}{python}{Move Bitmaps, Version 1.3}{srcInvader05c}

I hope you can recognize the idea behind the code. After calculating the new horizontal position, \zeiref{srcInvader0508} checks whether the new right edge of the bitmap has reached or exceeded the right edge of the screen. If this is the case, the sign of the direction variable is simply reversed\index{Direction change}\randnotiz{Direction change}! The same logic works analogously when the left edge of the screen is reached.

Try combining this with vertical movement as well.

There is still one problem: In \zeiref{srcInvader0507}, the new position is assigned to the \texttt{Rect} object even though it may already extend beyond the edge. With a speed of \texttt{1} or \texttt{2}, this may not be very noticeable, but if we set the speed to the width of the spaceship, the problem becomes obvious (temporarily set \texttt{cfg.FPS = 5} so that you can see it clearly). The spaceship ends up leaving the screen halfway. 
 
So we should check the new position first and only then assign it to the \texttt{Rect} object \texttt{defender\_rect}. In this context, let us introduce a very useful method of the \texttt{Rect} class: \texttt{pygame.rect.Rect.move()}\myindex{pyg}{\texttt{rect}!\texttt{Rect}!\texttt{move()}|underline}\randnotiz{move()}. 
 
\lstsource{SRC/00 Introduction/04 Moving/invader05d.py}{27}{35}{python}{Move Bitmaps, Version 1.4}{srcInvader05d}

The new function appears for the first time in \zeiref{srcInvader0510}. It takes two parameters. The first one specifies the horizontal displacement, and the second one specifies the vertical displacement. Since we do not want to change the vertical position, this parameter is constant~0 in our example. The function returns a new \texttt{Rect} object containing the updated position values. We store this temporarily in \texttt{newpos}.

The subsequent collision checks are then performed using the \texttt{newpos} rectangle. If a collision occurs, the direction values are changed as before. Likewise, the left edge of the bitmap is aligned with the left edge of the screen, and the right edge is aligned with the right edge. After that, \texttt{newpos} becomes the new rectangle for the defender (\zeiref{srcInvader0511}).


\begin{figure}[H]
\begin{center}
\begin{tikzpicture}
\node (myfirstpic) at (0,0) {\includegraphics[scale=0.8]{invader06.png}};

\draw
 (-1.8cm,-1.3cm) node (a) {}
 (5.8cm,-1.3cm) node (b) {}
 (5.8cm,-0.57cm) node (c) {}
 (3.4cm,-0.57cm) node (d) {}
;
\draw[>->, very thick, red, densely dotted] (a) -- (b) ;
\draw[>->, very thick, red, densely dotted] (c) -- (d) ;

%node[above, black, xshift=-2.6cm] {move left}
%\draw[->] (a) -- (b);
%\draw[->] (o) -- (y);


%Left
%\draw
% (-0.15,3.4) node (left1) {}
% (3.15,3.4) node (left2) {}
%;
%
\end{tikzpicture}
\caption{The Defender moves and bounces}\label{picBewegung01}
\end{center}
\end{figure}

%%%%%%%%%%%%%%%%%%%%%%%%%%%%%%%%%%%%%%%%%%%%%%%%%%%%%%%%%%%%%%%%%%%%%%%%%%%%%%%%
\subsection{More Input}
\subsubsection{Normalizing Speeds (\emph{delta time})}\index{deltatime}\label{secDeltatime}
At the moment, the movement does not depend only on
\texttt{defender\_speed}, but also on the frame rate
\texttt{cfg.FPS}.
To illustrate this dependency, I modified the previous source code for a small experiment (see \srcref{srcInvaderConfig05e} and \srcref[vref]{srcInvader05e}).

\lstsource{SRC/00 Introduction/04 Moving/dt/e/config.py}{1}{99}{python}{Movement without normalization: \texttt{config.py}}{srcInvaderConfig05e}

In \zeiref{srcInvader05e01}, you can see the different frame rates with which the experiment was carried out. In the line above, the window dimensions are set so that the window is tall and narrow, and in the line below the absolute number of milliseconds is specified during which the spaceship moves upward.


\Zeiref{srcInvader05e02} stores the time at which the spaceship’s ascent began. For this purpose, the function \texttt{pygame.time.get\_ticks()}\myindex{pyg}{\texttt{time}!\texttt{get\_ticks()}}\randnotiz{get\_ticks()} returns the number of milliseconds since the call to \texttt{pygame.init()}\myindex{pyg}{\texttt{init()}}; for example, \SI{5}{ms}.

Inside the main program loop, the spaceship now moves upward by a certain number of pixels per frame. The new position is calculated by adding the product of direction and speed to the \texttt{top} coordinate of the old position (\zeiref{srcInvader05e03}) -- so there is nothing new at this point.

After a fixed time interval (\texttt{cfg.LIMIT}, here \SI{500}{ms}), the direction stored in \texttt{defender\-\_direction} is set to~0, causing the movement to stop. To do this, \zeiref{srcInvader05e04} checks whether the current number of milliseconds since the start of the program is greater than \texttt{start\_time} plus \texttt{cfg.LIMIT}. In numerical terms: during the first loop iteration (frame~1), the condition would be, for example, 
 
\emph{Is \SI{17}{ms} greater than \SI{5}{ms} + \SI{500}{ms}?}
 
The answer is \emph{No}, so the spaceship continues to move upward. At frame~61, the condition would be 

\emph{Is \SI{508}{ms} greater than \SI{5}{ms} + \SI{500}{ms}?}

Now the answer is \emph{Yes}, and the direction variable is therefore set to~0, stopping the movement.

\lstsource{SRC/00 Introduction/04 Moving/dt/e/invader05e.py}{6}{44}{python}{Movement without normalization}{srcInvader05e}

In \abbref[vref]{fpsbewegung00}, you can see screenshots of the distances the spaceship has traveled after half a second. In all experiments, the speed \texttt{defender\_speed} remained the same -- namely~2. Only the frame rate was increased.

How do these different heights come about? After all, only \texttt{defender\_speed} is supposed to define the speed. The relationship should become clear in \tabref[vref]{tabFpsBewegung01}. The first column shows the speed of an object; in our example, this is the variable \texttt{defender\_speed}. This value specifies how many pixels per frame the object is moved; this value does not change. The second column shows the frame rate, that is, the number of frames per second. In our example, this value is defined in \texttt{cfg.FPS}. The duration of the movement is shown in the third column. We have a duration of \SI{500}{ms}, that is \SI{0.5}{s}. In our example, this value is stored in \texttt{cfg.LIMIT} and is also the same for all experiments.
 
The last column shows the calculated distance in pixels that the moving object has traveled. Now the relationship becomes clear: because we repeat the main program loop a different number of times depending on the frame rate, different distances are covered within the same amount of time.


\begin{figure}[hbtp] 
\begin{center}
\begin{tikzpicture}
\node (fps010) at (0.0, 0) {\fbox{\includegraphics[scale=0.35]{fps_05e_010_00.png}}};
\node (fps030) at (2.0, 0) {\fbox{\includegraphics[scale=0.35]{fps_05e_030_00.png}}};
\node (fps060) at (4.0, 0) {\fbox{\includegraphics[scale=0.35]{fps_05e_060_00.png}}};
\node (fps120) at (6.0, 0) {\fbox{\includegraphics[scale=0.35]{fps_05e_120_00.png}}};
\node (fps240) at (8.0, 0) {\fbox{\includegraphics[scale=0.35]{fps_05e_240_00.png}}};
\node (fps300) at (10.0, 0) {\fbox{\includegraphics[scale=0.35]{fps_05e_300_00.png}}};
\node (fps600) at (12.0, 0) {\fbox{\includegraphics[scale=0.35]{fps_05e_600_00.png}}};
\end{tikzpicture}
	\caption[Movement without normalization]{Non-normalized movement with identical speed but different frame rates \\(from left to right: 10, 30, 60, 120, 240, 300, 600)}\label{fpsbewegung00}
\end{center}
\end{figure}

\begin{longtable}{r@{ * }r@{ * }r@{ = }r}
	\caption{Distance without normalized movement}\label{tabFpsBewegung01} \\
	% Definition des Tabellenkopfes auf der ersten Seite
	\toprule
    speed ($\frac{px}{f}$) & FPS ($\frac{f}{s}$) & time ($s$) & distance ($px$) \\
	\midrule
	\endfirsthead % Erster Kopf zu Ende
	% Definition des Tabellenkopfes auf den folgenden Seiten
	\caption{Distance without normalized movement (continued)}\\
	\toprule
	speed ($\frac{px}{f}$) & FPS ($\frac{f}{s}$) & time ($s$) & distance ($px$)\\
	\midrule
	\endhead % Zweiter Kopf ist zu Ende

	\midrule
	\multicolumn{4}{r}{\emph{continued on next page}} \\
	\endfoot
	
	\bottomrule
	\endlastfoot
	
	% Ab hier kommt der Inhalt der Tabelle
	2  &   10 & 0.5 &  10 \\ 
	2  &   30 & 0.5 &  30 \\ 
	2  &   60 & 0.5 &  60 \\ 
	2  &  120 & 0.5 & 120 \\ 
	2  &  240 & 0.5 & 240 \\ 
	2  &  300 & 0.5 & 300 \\ 
\end{longtable} 

What we therefore need is a mechanism that removes the influence of the frame rate again. This factor has to be constructed in such a way that, when multiplied by the frame rate, it always yields~1 as a result. In that case, the frame rate would effectively act like a multiplication by~1 in the overall product and would therefore no longer have any influence.

The obvious approach is to take the inverse of the frame rate, that is $\frac{1}{fps}$. This correction value is called \emph{delta time (dt)}\index{deltatime}\randnotiz{deltatime}. The calculation would then look, for example, as shown in \tabref[vref]{tabFpsBewegung02}. The second and third columns cancel each other out, so that the distance remains the same -- independent of the chosen frame rate.

That is exactly what we tried to achieve.

\begin{longtable}{r@{ * }r@{ * }r@{ * }r@{ = }r}
	\caption{Distance with normalized movement}\label{tabFpsBewegung02} \\
	\toprule
	speed ($\frac{px}{s}$) & FPS ($\frac{f}{s}$) & dt ($\frac{s}{f}$) & time ($s$) & distance ($px$) \\
	\midrule
	\endfirsthead % Erster Kopf zu Ende
	
	% Definition des Tabellenkopfes auf den folgenden Seiten
	\caption{Distance with normalized movement (continued)}\\
	\toprule
	speed ($\frac{px}{s}$) & FPS ($\frac{f}{s}$) & dt ($\frac{s}{f}$) & time ($s$) & distance ($px$) \\
	\midrule
	\endhead % Zweiter Kopf ist zu Ende
	
	\midrule
	\multicolumn{5}{r}{\emph{continued on next page}} \\
	\endfoot
	
	\bottomrule
	\endlastfoot
	
	
	% Ab hier kommt der Inhalt der Tabelle
	2  &   10 &  $\frac{1}{10}$   & 0.5 &  1 \\
	2  &   30 &  $\frac{1}{30}$   & 0.5 &  1 \\
	2  &   60 &  $\frac{1}{60}$   & 0.5 &  1 \\
	2  &  120 &  $\frac{1}{120}$  & 0.5 &  1 \\
	2  &  240 &  $\frac{1}{240}$  & 0.5 &  1 \\
	2  &  300 &  $\frac{1}{300}$  & 0.5 &  1 \\
\end{longtable} 


In this context, it becomes apparent that the distance is surprisingly short: only~\SI{1}{px} per second. Please note that the unit of the first column has changed as well. The speed no longer specifies the number of pixels per frame, but the number of pixels per second! We therefore need to choose a different speed value; based on our window size, I decided on~\SI{600}{px/s}. After one second, our spaceship will have reached the top.


In \tabref[vref]{tabFpsBewegung03}, I calculated the expected final position (\texttt{.top}) after half a second. In the left half of the table, the traveled distance is calculated. Surprisingly, it always amounts to \SI{300}{px}. From the window height (\texttt{WINDOW.height}), we have to subtract this distance. In addition, we subtract the height of our spaceship (\SI{30}{px}) and the small offset of \SI{5}{px}, since we did not want to start the spaceship directly at the bottom edge. We therefore expect our spaceship to reach the final position calculated in \tabref[vref]{tabFpsBewegung03} after half a second.


\begin{longtable}{r@{ * }r@{ * }r@{ * }r@{ = }r@{ $\rightarrow$ }c@{ - }r@{ - }r@{ = }r}%{r@{ * }r@{ * }r@{ * }r@{ + }r@{ + }r@{ = }r@{ $\rightarrow$ }r}
	\caption{Pixel coordinates with normalized speed}\label{tabFpsBewegung03} \\
	% Definition des Tabellenkopfes auf der ersten Seite
    \toprule
    speed & FPS & dt  & time  & distance & \texttt{WINDOW.height} & height & offset & \texttt{.top}\\
	\midrule
	\endfirsthead % Erster Kopf zu Ende

	% Definition des Tabellenkopfes auf den folgenden Seiten
	\caption{Pixel coordinates with normalized speed (continued)}\\
	\toprule
    speed & FPS & dt  & time  & distance & \texttt{WINDOW.height} & height & offset & \texttt{.top}\\
	\midrule
	\endhead % Zweiter Kopf ist zu Ende
	
	\midrule
	\multicolumn{9}{r}{\emph{continued on next page}} \\
	\endfoot

	\bottomrule
	\endlastfoot

	% Ab hier kommt der Inhalt der Tabelle
	600  &   10 &  $\frac{1}{10}$   & 0.5 & 300 & 650-300 & 30 & 5 & 315 \\ 
	600  &   30 &  $\frac{1}{30}$   & 0.5 & 300 & 650-300 & 30 & 5 & 315 \\ 
	600  &   60 &  $\frac{1}{60}$   & 0.5 & 300 & 650-300 & 30 & 5 & 315 \\ 
	600  &  120 &  $\frac{1}{120}$  & 0.5 & 300 & 650-300 & 30 & 5 & 315 \\ 
	600  &  240 &  $\frac{1}{240}$  & 0.5 & 300 & 650-300 & 30 & 5 & 315 \\ 
	600  &  300 &  $\frac{1}{300}$  & 0.5 & 300 & 650-300 & 30 & 5 & 315 \\ 
\end{longtable} 

Although \tabref{tabFpsBewegung03} may look complicated, the implementation is surprisingly simple. First, the adjustment in \texttt{config.py}. In \zeiref{srcInvader05f01}, the correction factor is defined -- as discussed above -- as the inverse of the frame rate. 

\lstsource{SRC/00 Introduction/04 Moving/dt/f/config.py}{1}{99}{python}{Movement with normalization and  $dt=1/fps$: \texttt{config.py}}{srcInvaderConfig05f}

The speed is adjusted from~2 to~600 in \zeiref{srcInvader05f02}, and in \zeiref{srcInvader05f03} the correction factor \texttt{DELTATIME} is included as a factor in the calculation. That’s it; in \abbref[vref]{fpsbewegung01} we can admire the \emph{perfect} result on one of my slower computers.

\lstsource{SRC/00 Introduction/04 Moving/dt/f/invader05f.py}{6}{44}{python}{Movement with normalization and  $dt=1/fps$}{srcInvader05f}

\subsubsection{Optimizing Normalized Speed}

Two issues cause the error shown in \abbref[vref]{fpsbewegung01}:

\begin{itemize}
	\item \textbf{\Glspl{roundingerror}:} In theory, multiplying the frame rate by the delta time should always yield~$1.0$. Unfortunately, this is not the case. When computing delta time, a value close to the exact value is stored due to the way a \gls{float} is represented\randnotiz{Rounding error};	for example, instead of the exact value $0.0\overline{3}$ for $\tfrac{1.0}{30.0}$, the stored value is $0.03333333333333330$. Over time, this rounding error accumulates to perceptible amounts.
	
	\item \textbf{Incorrect understanding of $fps$:} The frame rate does not define that the main program loop is executed \emph{exactly} 60 times per second, for example, but that it is executed \emph{at most} 60 times per second.	If the game logic or rendering takes more time than $\tfrac{1}{60}\,\si{s}$, at least one frame will be skipped. This can also happen if the computer loses performance due to other operations (for example, cloud synchronization).
\end{itemize}


We cannot solve the first problem without a significant loss of performance, so we will not consider it any further. The second problem, however, can be addressed. Instead of a fixed delta time, we need a value that is based on the actual duration of a frame. The method \texttt{pygame.clock.tick()}\myindex{pyg}{\texttt{clock}!\texttt{tick()}}\randnotiz{tick()} in \zeiref{srcInvader05g01} provides a good estimate of the frame time. Luckily, this feature is already built in and can therefore be used directly (see \srcref[vref]{srcInvader05g}). The result in \abbref[vref]{fpsbewegung02} is better, but still not satisfying~{:-(}. In \abbref[vref]{picFehlerInvers}, you can see that the red line more or less dances around the green one, and no clear trend is visible.

\myebild{error_invers.pdf}{1.0}{Position Error of $1/fps$ and \texttt{pygame.clock.tick()}}{picFehlerInvers}

\lstsource{SRC/00 Introduction/04 Moving/dt/g/invader05g.py}{46}{49}{python}{Normalized Movement with \texttt{pygame.clock.tick()}}{srcInvader05g}

The cause is a problem that we should have fixed immediately. In the assignment in \zeiref{srcInvader05f03} in \srcref[vref]{srcInvader05f}, the right-hand side is a floating-point value\index{float}, while the left-hand side is an \gls{int}\index{int}\randnotiz{Float for logic, Int for rendering}. As a result, the decimal places are truncated in every loop iteration. For example, if the spaceship were supposed to move by \SI{5.8}{px} in each frame, the following values would occur:

\begin{longtable}{lrrrrrrrrr}
	\caption{Error Propagation}\label{tabFpsBewegung04} \\
	% Definition des Tabellenkopfes auf der ersten Seite
	\toprule
    Frame & 1 & 2 & 3 & 4 & 5 & 6 & 7 & 8 & 9 \\
	\midrule
	\endfirsthead % Erster Kopf zu Ende
	
	% Definition des Tabellenkopfes auf den folgenden Seiten
	\caption{Error Propagation (continued)}\\
	\toprule
    Frame & 1 & 2 & 3 & 4 & 5 & 6 & 7 & 8 & 9 \\
	\midrule
	\endhead % Zweiter Kopf ist zu Ende

	\midrule
	\multicolumn{10}{r}{\emph{continued on next page}} \\
	\endfoot
	
	\bottomrule
	\endlastfoot
	% Ab hier kommt der Inhalt der Tabelle
	Actual Value  & 5.0 & 10.0 & 15.0 & 20.0 & 25.0 & 30.0 & 35.0 & 40.0 & 45.0 \\
	Correct Value & 5.8 & 11.6 & 14.4 & 23.2 & 29.0 & 34.8 & 40.6 & 46.4 & 52.2 \\
	Error         & 0.8 &  1.3 &  2.4 &  3.2 &  4.0 &  4.8 &  5.6 &  6.4 &  7.2 \\
\end{longtable} 

Recently, Pygame introduced a variant of \texttt{Rect}, namely \texttt{FRect}\myindex{pyg}{\texttt{rect}!\texttt{FRect}|underline}\randnotiz{FRect}. In this class, all values are stored as \texttt{float}s, so fractional parts are no longer truncated. Alternatively, we would have to	store the position values independently of the \texttt{Rect} object in an additional float variable in order to preserve the fractional parts, for example in a \texttt{pygame.math.Vector2}\myindex{pyg}{\texttt{math}!\texttt{Vector2}} object.
 
\newpage 
\lstsource{SRC/00 Introduction/04 Moving/dt/h/invader05h.py}{11}{49}{python}{Normalized Movement with Positions in Float}{srcInvader05h}

In \abbref[vref]{fpsbewegung03}, we can see that the result has already improved significantly. The deviation from the optimal value~\SI{315}{px} has also been reduced dramatically. The difference is visualized in \abbref[vref]{picFehlerFloat}.

\myebild{error_float.pdf}{1.0}{Position Error of Rect and FRect}{picFehlerFloat}

However, there is yet another source of error: \texttt{pygame.clock.tick()}\myindex{pyg}{\texttt{clock}!\texttt{tick()}} does not provide enough decimal precision. Over long runtimes, the missing fractional parts accumulate and again lead to noticeable errors. There are better Python functions for measuring elapsed time.

\zeiref{srcInvader05i01} of \srcref[vref]{srcInvader05i}, \texttt{time.time()}\randnotiz{time()}\index{time!time()|underline} is used to return the number of seconds since January~1,~1970 as a floating-point number. The fractional part represents fractions of a second. This measurement is more precise than the one provided by \texttt{pygame.clock.tick()} and, depending on the time-measurement capabilities of the computer architecture and the operating system, can provide more decimal places—up to the nanosecond range.

In \zeiref{srcInvader05i02}, the current time is measured after one frame has elapsed, and in the following line the elapsed time is computed. This value represents the actual \emph{delta time}\index{deltatime} of the frame, now with higher precision. Afterwards, in \zeiref{srcInvader05i04}, the new start time of the next frame is stored so that the elapsed time can be computed again after the next frame.

\abbref[vref]{fpsbewegung04} shows that the target positions are reached almost perfectly for all frame rates. However, comparing the position errors in \abbref[vref]{picFehlerFunktion} does not allow for a clear evaluation. I suspect that experiments with significantly longer runtimes would make a difference visible. We must -- and can -- live with the remaining error.

\myebild{error_function.pdf}{1.0}{Position Error with Different Time Functions}{picFehlerFunktion}

\newpage 
\begin{figure}[p] 
	\begin{center}
		\begin{tikzpicture}
			\node (fps010) at (0.0, 0) {\fbox{\includegraphics[scale=0.35]{fps_05f_010_00.png}}};
			\node (fps030) at (2.0, 0) {\fbox{\includegraphics[scale=0.35]{fps_05f_030_01.png}}};
			\node (fps060) at (4.0, 0) {\fbox{\includegraphics[scale=0.35]{fps_05f_060_02.png}}};
			\node (fps120) at (6.0, 0) {\fbox{\includegraphics[scale=0.35]{fps_05f_120_09.png}}};
			\node (fps240) at (8.0, 0) {\fbox{\includegraphics[scale=0.35]{fps_05f_240_04.png}}};
			\node (fps300) at (10.0, 0) {\fbox{\includegraphics[scale=0.35]{fps_05f_300_09.png}}};
			\node (fps600) at (12.0, 0) {\fbox{\includegraphics[scale=0.35]{fps_05f_600_09.png}}};
		\end{tikzpicture}
		\caption[Movement with normalization and $dt=1/fps$]{Movement with normalization and $dt=1/fps$ using constant speed but different fps (from left to right: 10, 30, 60, 120, 240, 300, 600)}\label{fpsbewegung01}
	\end{center}
\end{figure}


\begin{figure}[p] 
	\begin{center}
		\begin{tikzpicture}
			\node (fps010) at (0.0, 0) {\fbox{\includegraphics[scale=0.35]{fps_05g_010_08.png}}};
			\node (fps030) at (2.0, 0) {\fbox{\includegraphics[scale=0.35]{fps_05g_030_07.png}}};
			\node (fps060) at (4.0, 0) {\fbox{\includegraphics[scale=0.35]{fps_05g_060_05.png}}};
			\node (fps120) at (6.0, 0) {\fbox{\includegraphics[scale=0.35]{fps_05g_120_04.png}}};
			\node (fps240) at (8.0, 0) {\fbox{\includegraphics[scale=0.35]{fps_05g_240_08.png}}};
			\node (fps300) at (10.0, 0) {\fbox{\includegraphics[scale=0.35]{fps_05g_300_08.png}}};
			\node (fps600) at (12.0, 0) {\fbox{\includegraphics[scale=0.35]{fps_05g_600_06.png}}};
		\end{tikzpicture}
		\caption[Movement with normalization and \texttt{pygame.clock.tick()}]{Movement with normalization and \texttt{pygame.clock.tick()} using constant speed but different fps (from left to right: 10, 30, 60, 120, 240, 300, 600)}\label{fpsbewegung02}
	\end{center}
\end{figure}

\newpage
\begin{figure}[p] 
	\begin{center}
		\begin{tikzpicture}
			\node (fps010) at (0.0, 0)  {\fbox{\includegraphics[scale=0.35]{fps_05h_010_09.png}}};
			\node (fps030) at (2.0, 0)  {\fbox{\includegraphics[scale=0.35]{fps_05h_030_05.png}}};
			\node (fps060) at (4.0, 0)  {\fbox{\includegraphics[scale=0.35]{fps_05h_060_05.png}}};
			\node (fps120) at (6.0, 0)  {\fbox{\includegraphics[scale=0.35]{fps_05h_120_04.png}}};
			\node (fps240) at (8.0, 0)  {\fbox{\includegraphics[scale=0.35]{fps_05h_240_05.png}}};
			\node (fps300) at (10.0, 0)  {\fbox{\includegraphics[scale=0.35]{fps_05h_300_06.png}}};
			\node (fps600) at (12.0, 0)  {\fbox{\includegraphics[scale=0.35]{fps_05h_600_09.png}}};
		\end{tikzpicture}
		\caption[Movement with normalization and \texttt{pygame.clock.tick()} (float)]{Movement with normalization and \texttt{pygame.clock.tick()} (float) using constant speed but different fps (from left to right: 10, 30, 60, 120, 240, 300, 600)}\label{fpsbewegung03}%
	\end{center}
\end{figure}

\begin{figure}[p] 
	\begin{center}
		\begin{tikzpicture}
			\node (fps010) at (0.0, 0) {\fbox{\includegraphics[scale=0.35]{fps_05i_010_00.png}}};
			\node (fps030) at (2.0, 0) {\fbox{\includegraphics[scale=0.35]{fps_05i_030_05.png}}};
			\node (fps060) at (4.0, 0) {\fbox{\includegraphics[scale=0.35]{fps_05i_060_03.png}}};
			\node (fps120) at (6.0, 0) {\fbox{\includegraphics[scale=0.35]{fps_05i_120_07.png}}};
			\node (fps240) at (8.0, 0) {\fbox{\includegraphics[scale=0.35]{fps_05i_240_00.png}}};
			\node (fps300) at (10.0, 0) {\fbox{\includegraphics[scale=0.35]{fps_05i_300_00.png}}};
			\node (fps600) at (12.0, 0) {\fbox{\includegraphics[scale=0.35]{fps_05i_600_00.png}}};
		\end{tikzpicture}
		\caption[Movement with normalization and \texttt{time.time()}]{Movement with normalization and \texttt{time.time()} using constant speed but different fps (from left to right: 10, 30, 60, 120, 240, 300, 600), Version 3}\label{fpsbewegung04}%
	\end{center}
\end{figure}
\newpage

\lstsource{SRC/00 Introduction/04 Moving/dt/i/invader05i.py}{8}{49}{python}{Movement with normalization and \texttt{time.time()}}{srcInvader05i}


%%%%%%%%%%%%%%%%%%%%%%%%%%%%%%%%%%%%%%%%%%%%%%%%%%%%%%%%%%%%%%%%%%%%%%%%%%%%%%%%
\subsection{What was new?}
The position of an object is stored in a \texttt{Rect} or \texttt{FRect} object. In each frame, the position is checked and modified if necessary. When the screen is updated, this creates the impression of movement. The result of a movement is usually first stored temporarily in a variable and checked before it is applied as the new position.

The direction of movement is encoded by the sign, and the speed by the value of the speed variable. Horizontal and vertical movement are handled separately. 

To become independent of the actual frame rate, a correction factor (delta time) must be used when calculating the new position. This value can either be computed manually or obtained from a call to \texttt{pygame.time.Clock.tick()}.

The following Pygame elements were introduced:

\begin{itemize}

	\item \texttt{pygame.rect.FRect}:
	\myindex{pyg}{\texttt{rect}!\texttt{FRect}}\\
	\url{https://pyga.me/docs/ref/rect.html}
	
	\item \texttt{pygame.rect.FRect.move()}:
    \myindex{pyg}{\texttt{rect}!\texttt{FRect}!\texttt{move()}}\\
    \url{https://pyga.me/docs/ref/rect.html#pygame.Rect.move}

	\item \texttt{pygame.rect.Rect}:
    \myindex{pyg}{\texttt{rect}!\texttt{Rect}}\\
    \url{https://pyga.me/docs/ref/rect.html}

   \item \texttt{pygame.rect.Rect.move()}:
   \myindex{pyg}{\texttt{rect}!\texttt{Rect}!\texttt{move()}}\\
   \url{https://pyga.me/docs/ref/rect.html#pygame.Rect.move}

	\item \texttt{pygame.Surface.get\_rect()}:
	\myindex{pyg}{\texttt{Surface}!\texttt{get\_rect()}}\\
	\url{https://pyga.me/docs/ref/surface.html#pygame.Surface.get_rect}

	\item \texttt{pygame.time.get\_ticks()}:
    \myindex{pyg}{\texttt{time}!\texttt{get\_ticks()}}\\
    \url{https://pyga.me/docs/ref/surface.html#pygame.time.get_ticks}

    \item \texttt{pygame.math.Vector2}:
    \myindex{pyg}{\texttt{math}!\texttt{Vector2}}\\
    \url{https://pyga.me/docs/ref/math.html#pygame.math.Vector2}

    \item \texttt{pygame.math.Vector3}:
    \myindex{pyg}{\texttt{math}!\texttt{Vector3}}\\
    \url{https://pyga.me/docs/ref/math.html#pygame.math.Vector3}
\end{itemize}



%%%%%%%%%%%%%%%%%%%%%%%%%%%%%%%%%%%%%%%%%%%%%%%%%%%%%%%%%%%%%%%%%%%%%%%%%%%%%%%%
\subsection{Homework}

\begin{enumerate}
	\item Use \texttt{centerx}, \texttt{centery}, or \texttt{center} instead of \texttt{top} and \texttt{left}. Does it work? Do you notice any remarkable differences?
	
	\item Create an application in which two identical objects travel the same horizontal distance of \SI{800}{px}. One object uses position values stored as \texttt{int}, the other uses position values stored as \texttt{float}. Both should move at a speed of \SI{50}{px/s}. What can you observe?
	
	\item Create an application where the speed is not constant. Four objects should travel the same horizontal distance of \SI{800}{px} in parallel, just like before. The behavior should be as follows:
	
	\begin{enumerate}
		\item Object 1: It continuously accelerates and reaches its maximum speed at the right end.
		
		\item Object 2: It accelerates over time, reaches its maximum speed at the midpoint of the distance, and then slows down again. At the right end, its speed is \SI{0}{px/s}. The increase and decrease in speed are linear.
		
		\item Object 3: It accelerates over time, reaches its maximum speed at the midpoint of the distance, and then slows down again. At the right end, its speed is \SI{0}{px/s}. The increase and decrease in speed follow a sine curve on $[0, \pi]$.
		
		\item Object 4: Like object~3, but a variable controls how many intervals of $[0, \pi]$ are completed before reaching the right end. For example, the object speeds up and slows down again 5 times.
	\end{enumerate}
	
	\item Something for the ambitious among you: All four objects reach the right edge at the same time.
\end{enumerate}

	% !TeX spellcheck = en_US
\newpage
\section{Class \texttt{Sprite}}\myindex{pyg}{\texttt{sprite}!\texttt{Sprite}|underline}
\subsection{Introduction}
\begin{diskbox}
	\url{https://github.com/adamsralf/pygame_book/tree/main/src/00%20Introduction/05%20Sprite/invader}
\end{diskbox}

In \secref[vref]{secDeltatime}, it became apparent that many variables start with \texttt{defender\_}. In other words, they are attributes of a single entity and almost demand to be expressed as a class.

This class should contain all information related to updating and rendering the bitmap. Some elements, such as \texttt{defender\_image} and \texttt{defender\_rect}, seem to play a role in virtually every bitmap processing task. Furthermore, every bitmap will require some form of state update and screen output. In fact, 

\begin{hintbox}[Pygame already provides a class that offers exactly such a \gls{framework}]
	\texttt{pygame.sprite.Sprite}\myindex{pyg}{\texttt{sprite}!\texttt{Sprite}|underline}.
\end{hintbox}

Let us therefore define the \texttt{Defender} class as a subclass of \texttt{Sprite} (\zeiref{srcInvader06a01}).

\lstsource{SRC/00 Introduction/05 Sprite/invader/v01/invader.py}{7}{32}{python}{Sprites (1), Version 1.0}{srcInvader06a1}

The lines in the constructor (\zeiref{srcInvader06a02}ff.) correspond to those of the previous version. Only the prefix \texttt{defender\_} is replaced by \texttt{self.}, which turns the variables into attributes of the class. You should have no difficulty understanding these changes.

Every subclass of \texttt{Sprite} must provide two attributes: \texttt{rect}\index{Sprite!self.rect|underline}\randnotiz{self.rect} and \texttt{image}\index{Sprite!self.image|underline}\randnotiz{self.image}. These two attributes are accessed by the already predefined solutions for collision detection, rendering, and related tasks. We will see their usefulness later on.

In \zeiref{srcInvader06a03}ff., the boundary collisions and state changes are implemented. One detail that stands out is the computation of the new position using \texttt{move()}\myindex{pyg}{\texttt{FRect}!\texttt{move()}}.

New is the call to the method \texttt{change\_direction()}. This method (\zeiref{srcInvader06a08}) is more \emph{object-oriented} than the previous version. In object-oriented programming, algorithms are not implemented directly; instead, messages are sent to objects, which then handle the details internally -- in a way that is not visible from the outside. In this case, this means that instead of performing the direction change directly at the relevant point, I~send a message to the object telling it that the direction needs to be changed.

With the method \texttt{draw()} in \zeiref{srcInvader06a04}, the screen output is encapsulated.


\lstsource{SRC/00 Introduction/05 Sprite/invader/v01/invader.py}{35}{62}{python}{Sprites (2), Version 1.0}{srcInvader06a2}

Using the \texttt{Defender} class has now become straightforward. In \zeiref{srcInvader06a05}, an object of the class is created. In \zeiref{srcInvader06a06}, \texttt{update()} is called, and in \zeiref{srcInvader06a07}, \texttt{draw()} is executed.

One advantage of the new architecture is the improved clarity and readability of the main program. By following naming conventions (descriptive class and method names), the overall control flow becomes clearer and is no longer obscured by implementation details.

I~now want to make use of the capabilities of the \texttt{Sprite} class so that boundary collision checks no longer have to be implemented manually.

Let us get started: Since we want to organize collision detection differently, we first simplify \texttt{update()} again. We now only compute the new position. In doing so, the method \texttt{pygame.FRect.move\_ip()}\myindex{pyg}{\texttt{FRect}!\texttt{move\_ip()}|underline}\randnotiz{move\_ip()} is introduced in \zeiref{srcInvader06b01}. It works like \texttt{move()}, but in this case the modification is applied directly to the rectangle; \texttt{ip} stands for \emph{in place}. With \texttt{move()}, the original rectangle remains unchanged.

\lstsource{SRC/00 Introduction/05 Sprite/invader/v02/invader.py}{18}{19}{python}{Sprites (1), Version 1.1}{srcInvader06b1}

To make the boundaries visible and to better recognize collisions, the edges are now replaced by two stone walls on the left and right. These bitmaps are also implemented as subclasses of \texttt{pygame.sprite.Sprite}. Since the state of the two walls never changes, the implementation of \texttt{update()} can be omitted.

\lstsource{SRC/00 Introduction/05 Sprite/invader/v02/invader.py}{28}{39}{python}{Sprites (2), Version 1.1}{srcInvader06b2}

I~now create the two boundaries:

\lstsource{SRC/00 Introduction/05 Sprite/invader/v02/invader.py}{49}{50}{python}{Sprites (3), Version 1.1}{srcInvader06b3}

So far, everything has been easy.

\lstsource{SRC/00 Introduction/05 Sprite/invader/v02/invader.py}{60}{65}{python}{Sprites (4), Version 1.1}{srcInvader06b4}

What is happening here? The method \texttt{pygame.sprite.collide\_rect()}\myindex{pyg}{\texttt{sprite}!\texttt{collide\_rect()}}\randnotiz{collide\-\_rect()} checks whether the rectangles of two \texttt{Sprite} objects collide. This means that I~no longer have to manually check the left and right boundaries myself.

Here, the collision of a single object with both boundaries -- or more generally, with many \texttt{Sprite} objects -- is tested. 

\begin{hintbox}[Spritegroups]
	In practice, sprites rarely exist on their own; they are usually organized into groups. This concept is also built into Pygame and leads to further simplifications.
\end{hintbox}

\lstsource{SRC/00 Introduction/05 Sprite/invader/v03/invader.py}{36}{71}{python}{Sprites (1), Version 1.2}{srcInvader06c1}

The defender is no longer addressed directly, but is instead packed into a luxury box. I~will come back to this later. The two \texttt{Border} objects are no longer stored in two separate object variables either; instead, they are placed into a luxury box as well: a \texttt{pygame.sprite.Group}\myindex{pyg}{\texttt{sprite}!\texttt{Group}|underline}\randnotiz{Group}. Here, I~could also store additional boundaries or walls. From the point of view of the game logic, they could then all be processed together in one go. This becomes clear in this mini example in two places.


The first location is \zeiref{srcInvader06c01}, where a different variant of collision detection is used: \texttt{pygame.sprite.spritecollide()}\myindex{pyg}{\texttt{sprite}!\texttt{spritecollide()}|underline}\randnotiz{spritecollide()}. The first parameter is a \emph{single} \texttt{Sprite} object -- in our case, the defender. The second parameter is a sprite group containing all \texttt{Border} objects. Thus, the defender is checked for collisions with all members of the group. This only works if all sprites provide a \texttt{Rect} or \texttt{FRect} object named \texttt{rect} as an attribute. The third parameter -- \false\ in this example -- controls whether the colliding sprite should be removed from the group. This feature is quite useful in games, for example when rocks that are shot by a spaceship should be deleted.

The second location is \zeiref{srcInvader06c03}. Here, \texttt{draw()} is no longer called for each object individually, but once for the entire group. When using this service, the \texttt{draw()} method can be removed from your own classes (here \texttt{Border} and \texttt{Defender}), which simplifies things even further.

It therefore seems like a good idea to pack sprites into such luxury boxes. But what about the defender? To take advantage of sprite groups, you can also create groups that contain only a single element. To allow these groups to work more efficiently -- after all, it is known that they contain only one element -- Pygame provides the special case \texttt{pygame.sprite.GroupSingle}\myindex{pyg}{\texttt{sprite}!\texttt{GroupSingle}|underline}\randnotiz{GroupSingle}. Since there is often a need to access the single \texttt{Sprite} object of the \emph{group}, this group provides the additional attribute \texttt{sprite}\myindex{pyg}{\texttt{sprite}!\texttt{GroupSingle}!\texttt{sprite}} (see \zeiref{srcInvader06c01}f.).

%%%%%%%%%%%%%%%%%%%%%%%%%%%%%%%%%%%%%%%%%%%%%%%%%%%%%%%%%%%%%%%%%%%%%%%%%%%%%%%%
\subsection{More Input}
%%%%%%%%%%%%%%%%%%%%%%%%%%%%%%%%%%%%%%%%%%%%%%%%%%%%%%%%%%%%%%%%%%%%%%%%%%%%%%%%
\subsubsection{OO Issues}
\begin{diskbox}
	\begin{itemize}
		\item \url{https://github.com/adamsralf/pygame_book/tree/main/src/00%20Introduction/05%20Sprite/invader/v04}
		\item \url{https://github.com/adamsralf/pygame_book/tree/main/src/00%20Introduction/05%20Sprite/invader/v05}
	\end{itemize}
\end{diskbox}

I~want to pursue my object-oriented approach even further and also turn the main program into a \texttt{Game} class. What is important to me is to establish a sense of structural discipline right from the beginning. The longer you stay in software development, the more you will grow fond of terms like \emph{order} and \emph{structure}. They help you avoid losing the thread, even in more complex games. A particularly helpful concept here is the \Gls{srp}.

\begin{hintbox}[Single Responsibility Principle (SRP)]
	Each class or function should have exactly one responsibility. It should focus on that single task and do it well. A solution to a specific problem should be encapsulated in one class or one method, and changes to that responsibility should affect only that part of the code.
\end{hintbox}


\lstsource{SRC/00 Introduction/05 Sprite/invader/v04/invader.py}{37}{78}{python}{\texttt{Game}-Klasse}{srcInvader06d}

An example of this last point is the design of the \texttt{Game} class. Here, the source code is no longer simply placed in \texttt{\_\_main\_\_}\index{\texttt{\_\_main\_\_}}, but is instead encapsulated, structured, and thus made flexibly reusable. A clear example of the SRP can be seen in the methods \texttt{watch\_for\_events()}, \texttt{update()}, and \texttt{draw()}. It is simply not the responsibility of \texttt{run()} to organize everything. From the perspective of the main loop, it is irrelevant which events are queried or how they are processed. All that matters is that events are handled once per frame. Likewise, \texttt{run()} should not be concerned with the order in which sprites are drawn to the screen. That task belongs to the \texttt{draw()} method. The \texttt{run()} method merely ensures that sprites first update their state and that rendering happens afterwards.

One aspect remains that I~would still like to address here: the call to \texttt{change\_direction()} in \zeiref{srcInvader06d01} does not appeal to me. It violates object-oriented design rules, specifically the \Gls{lsp}.

\begin{hintbox}[Liskov Substitution Principle (LSP)]
	A principle of object-oriented programming stating that objects of a derived class must be usable anywhere an object of the base class is expected, without altering the correct behavior of the program. Formulated by Barbara Liskov in 1987. The LSP ensures that inheritance does not introduce unexpected side effects and that class hierarchies remain consistent.
\end{hintbox}

The sprite group is a collection of \texttt{Sprite} objects. However, the class \texttt{pygame\-.sprite\-.Sprite} does not define a method called \texttt{change\_direction()}. Calling such a method here is therefore not entirely clean. Python does not have a problem with this, but that should not be the benchmark.

A better approach is to adapt the \texttt{update()} method instead. If you take a closer look at the \gls{signature} of \texttt{pygame.sprite.Sprite.update()}\myindex{pyg}{\texttt{sprite}!\texttt{Sprite}!\texttt{update()}}\randnotiz{update()}, you will see that it is designed to accept freely definable parameters. I~have developed the habit of using a parameter named \texttt{action} to trigger specific behavior in subclasses. With this approach, \texttt{change\_direction()} is called internally from \texttt{update()} (see \zeiref{srcInvader06e02}) rather than being invoked from the outside.

\lstsource{SRC/00 Introduction/05 Sprite/invader/v05/invader.py}{20}{25}{python}{\texttt{Defender.update()}}{srcInvader06e1}

The call then takes place indirectly in \srcref[vref]{srcInvader06e2} at \zeiref{srcInvader06e03} by using the argument passed to the method.

Note: This also complies with the object-oriented design principle \Gls{dontasktell}.

\lstsource{SRC/00 Introduction/05 Sprite/invader/v05/invader.py}{74}{77}{python}{\texttt{Game.update()}}{srcInvader06e2}

%%%%%%%%%%%%%%%%%%%%%%%%%%%%%%%%%%%%%%%%%%%%%%%%%%%%%%%%%%%%%%%%%%%%%%%%%%%%%%%%
\subsubsection{Add Sprite Objects to a Group Right Away}
\begin{diskbox}
		\url{https://github.com/adamsralf/pygame_book/tree/main/src/00%20Introduction/05%20Sprite/invadergroup/v01}
\end{diskbox}

It is often very convenient to assign a sprite object to a group already at the time it is created. To do this, the signature of \texttt{\_\_init\_\_()} only needs to be adjusted accordingly.

The parameter

\verb+*groups+

together with the corresponding call to the constructor of the \gls{superclass}

\verb+super().__init__(*groups)+

ensures that the sprite object is immediately added to the given group or groups. Below is the complete source code. In \zeiref{srcInvadergroup0101}, the group is then simply passed to the constructor as the last argument.
 

\lstsource{SRC/00 Introduction/05 Sprite/invadergroup/v01/invadergroup.py}{1}{67}{python}{Add Sprite Objects to a Group Right Away}{srcInvaderGroup01a}

%%%%%%%%%%%%%%%%%%%%%%%%%%%%%%%%%%%%%%%%%%%%%%%%%%%%%%%%%%%%%%%%%%%%%%%%%%%%%%%%
\subsubsection{Delete Sprites from Groups}
\begin{diskbox}
	\url{https://github.com/adamsralf/pygame_book/tree/main/src/00%20Introduction/05%20Sprite/invadergroup/v02}
\end{diskbox}

If you follow the program logic in \srcref[vref]{srcInvaderGroup01a}, you will notice that a spaceship is created at the bottom edge four times per second and then flies upward. Once it reaches the top, it simply stops.

The latter behavior is usually rather pointless. A more natural approach is to remove all spaceships that have crossed an upper boundary. Think, for example, of projectiles that leave the playfield.

This can be achieved by calling \texttt{pygame.sprite.Sprite.kill()}\myindex{pyg}{\texttt{sprite}!\texttt{Sprite}!\texttt{kill()}}\randnotiz{kill()}, which instructs Pygame to remove the \texttt{Sprite} object from \emph{all} sprite groups. If no further references to the \texttt{Sprite} object exist, it will then be deleted by Python’s \gls{garbagecollector}.

You can see an example of this in \srcref[vref]{srcInvaderGroup02a}. Once the spaceship has completely passed the midpoint of the window, it is removed.


\lstsource{SRC/00 Introduction/05 Sprite/invadergroup/v02/invadergroup.py}{18}{22}{python}{Kill a Sprite}{srcInvaderGroup02a}

%%%%%%%%%%%%%%%%%%%%%%%%%%%%%%%%%%%%%%%%%%%%%%%%%%%%%%%%%%%%%%%%%%%%%%%%%%%%%%%%
\subsection{What was new?}
\begin{hintbox}
	From a behavioral point of view: \emph{nothing at all}. The existing application has merely been embedded into a flexible framework.
\end{hintbox}

\begin{pygbox}
\begin{itemize}
	\item \texttt{pygame.Rect.move()}:
	\myindex{pyg}{\texttt{Rect}!\texttt{move()}}\\
	\url{https://pyga.me/docs/ref/rect.html#pygame.Rect.move}

	\item \texttt{pygame.Rect.move\_ip()}:
	\myindex{pyg}{\texttt{Rect}!\texttt{move\_ip()}}\\
	\url{https://pyga.me/docs/ref/rect.html#pygame.Rect.move_ip}

	\item \texttt{pygame.sprite.Group}:
	\myindex{pyg}{\texttt{sprite}!\texttt{Group}}\\
	\url{https://pyga.me/docs/ref/sprite.html#pygame.sprite.Group}

	\item \texttt{pygame.sprite.GroupSingle}:
	\myindex{pyg}{\texttt{sprite}!\texttt{GroupSingle}}\\
	\url{https://pyga.me/docs/ref/sprite.html#pygame.sprite.GroupSingle}

	\item \texttt{pygame.sprite.GroupSingle.sprite}:
	\myindex{pyg}{\texttt{sprite}!\texttt{GroupSingle}!\texttt{sprite}}\\
    \url{https://pyga.me/docs/ref/sprite.html#pygame.sprite.GroupSingle}

	\item \texttt{pygame.sprite.Sprite}:
	\myindex{pyg}{\texttt{sprite}!\texttt{Sprite}}\\
	\url{https://pyga.me/docs/ref/sprite.html#pygame.sprite.Sprite}
	
	\item \texttt{pygame.sprite.Sprite-kill()}:
	\myindex{pyg}{\texttt{sprite}!\texttt{Sprite}!\texttt{kill()}}\\
	\url{https://pyga.me/docs/ref/sprite.html#pygame.sprite.Sprite.kill}

	\item \texttt{pygame.sprite.collide\_rect()}:
	\myindex{pyg}{\texttt{sprite}!\texttt{collide\_rect()}}\\
	\url{https://pyga.me/docs/ref/sprite.html#pygame.sprite.collide_rect}
	
	\item \texttt{pygame.sprite.spritecollide()}:
    \myindex{pyg}{\texttt{sprite}!\texttt{spritecollide()}}\\
    \url{https://pyga.me/docs/ref/sprite.html#pygame.sprite.spritecollide}
\end{itemize}
\end{pygbox}

%%%%%%%%%%%%%%%%%%%%%%%%%%%%%%%%%%%%%%%%%%%%%%%%%%%%%%%%%%%%%%%%%%%%%%%%%%%%%%%%
\subsection{Homework}

\begin{enumerate}
	\item Modify the source code from \secref[vref]{secCreateBitmaps} so that \texttt{Meadow}, \texttt{Sky}, \texttt{Tree}, \texttt{House}, and \texttt{Sun} are subclasses of \texttt{pygame.sprite.Sprite}.

	\item Manage all \texttt{Sprite} classes that implement an \texttt{update()} method in one \texttt{pygame\-.sprite\-.Group} object, and store the others in a separate group.
\end{enumerate}

	% !TeX spellcheck = en_US
\newpage
%%%%%%%%%%%%%%%%%%%%%%%%%%%%%%%%%%%%%%%%%%%%%%%%%%%%%%%%%%%%%%%%%%%%%%%%%%%
\section{Handling Keyboard Input}\index{Tastatur}\label{secTastatur}
%%%%%%%%%%%%%%%%%%%%%%%%%%%%%%%%%%%%%%%%%%%%%%%%%%%%%%%%%%%%%%%%%%%%%%%%%%%
\subsection{Introduction}\label{KeyboardByEvent}
\begin{diskbox}
	\url{https://github.com/adamsralf/pygame_book/tree/main/src/00%20Introduction/06%20Keyboard/keyboard/v01}
\end{diskbox}

%begin{wrapfigure}[10]{r}{3.1cm}%
%	\begin{center}%
%		\vspace{-1cm}%
%		\myfigure{invader07.png}{0.8}{Keyboard}{picInvader07}%
%	\end{center}%
%\end{wrapfigure}%
I~do not intend to cover the keyboard exhaustively here, but merely to illustrate the basic principle. The movement of the spaceship is controlled by keyboard events. Every time I~press a arrow key -- left, right, up, down -- the spaceship moves in the corresponding direction. If I~release the arrow key, the spaceship stops. The game can now also be exited using the Escape key (\Gls{boss}).

As a first step, a dictionary\index{Dictionary} of possible directions is created in \texttt{config.py}. 
\begin{hintbox}[Vector2D]
	These directions are managed as \texttt{Vector2} objects \myindex{pyg}{\texttt{math}!\texttt{Vector2D}}, since they are easier to use for mathematical operations.
\end{hintbox}

\lstsource{SRC/00 Introduction/06 Keyboard/keyboard/v01/config.py}{1}{99}{python}{Control Direction by Keys (1), \texttt{config.py}}{srcTastatur00aa} 


Next, we prepare the \texttt{Defender} class or modify it slightly (\srcref[vref]{srcTastatur00a}). The sprite is no longer placed at the bottom but centered (\zeiref{srcTastatur0001}), and the spaceship should now also be able to move vertically. For this, we either need two separate variables or a \texttt{Vector2} object. I~choose a \texttt{Vector2} object, where the first component represents the horizontal direction vector and the second the vertical direction. Each direction vector is set according to the semantics introduced earlier. 


I~would like to draw special attention to \zeiref{srcTastatur0008}. The methods \texttt{clamp\_ip()}\randnotiz{clamp\_ip()\newline clamp()}\myindex{pyg}{\texttt{rect}!\texttt{Rect}!\texttt{clamp\_ip()}|underline}\myindex{pyg}{\texttt{rect}!\texttt{FRect}!\texttt{clamp\_ip()}|underline} as well as \texttt{clamp()}\myindex{pyg}{\texttt{rect}!\texttt{Rect}!\texttt{clamp()}|underline}\myindex{pyg}{\texttt{rect}!\texttt{FRect}!\texttt{clamp()}|underline} provide a very convenient shortcut in programming. Both methods check whether the inner rectangle has crossed the boundary of the outer rectangle on any side and, if necessary, move it back to the edge of the outer rectangle.

Here is an equivalent check without using \texttt{clamp()}, shown only for the left edge of the outer rectangle:
\lstset{firstnumber=1}
\begin{lstlisting}
	if inner_rect.right < outer_rect.left:
		inner_rect.right = outer_rect.left + 1
\end{lstlisting}

The method \texttt{clamp()} performs this kind of check for all sides of the inner rectangle. Using \texttt{clamp()} or \texttt{clamp\_ip()}, you can therefore ensure that a sprite never leaves a defined play area.

The difference between the two methods is that \texttt{clamp()} does not modify the values of the inner rectangle but instead returns a new, adjusted rectangle, whereas \texttt{clamp\_ip()} modifies the values of the inner rectangle directly.

\lstsource{SRC/00 Introduction/06 Keyboard/keyboard/v01/keyboard.py}{8}{25}{python}{Control Direction by Keys (2), Class \texttt{Defender}}{srcTastatur00a} 

Let us now turn to the actual handling of keyboard input: Pressing a key can trigger the event types\index{event}\myindex{pyg}{\texttt{Event}} \texttt{pygame.KEYDOWN}\myindex{pyg}{\texttt{KEYDOWN}|underline} or \texttt{pygame.KEYUP}\myindex{pyg}{\texttt{KEYUP}|underline}\randnotiz{KEYDOWN\\KEYUP}. In our example (\zeiref{srcTastatur0003}), we are interested in which key is \emph{pressed}, so we use \texttt{KEYDOWN}. After that, we can determine which key was pressed via \texttt{pygame.event.key}\myindex{pyg}{\texttt{event}!\texttt{key}}\randnotiz{key}. For this purpose, Pygame provides a set of predefined constants in \texttt{pygame.key}\myindex{pyg}{\texttt{key}} (see \tabref[vref]{tabKey} and \tabref[vref]{tabKeyMod}).


\lstsource{SRC/00 Introduction/06 Keyboard/keyboard/v01/keyboard.py}{54}{72}{python}{Control Direction by Keys (3)), \texttt{Game}.\texttt{watch\_for\_events()}}{srcTastatur00d}

Let us start with the boss key. In \zeiref{srcTastatur0004}, the constant \texttt{K\_ESCAPE}\myindex{pyg}{\texttt{K\_ESCAPE}}\randnotiz{K\_ESCAPE} is used to check whether the pressed key is the \keys{\esc}. As with clicking the window’s close button, the flag of the main program loop is then simply set to \false. Try it out!

After that, the four arrow keys are handled starting at \zeiref{srcTastatur0005}ff. Using \randnotiz{K\_LEFT K\_RIGHT K\_UP K\_DOWN}\texttt{K\_LEFT}, \texttt{K\_RIGHT}, \texttt{K\_UP}, and \texttt{K\_DOWN}\myindex{pyg}{\texttt{K\_LEFT}}\myindex{pyg}{\texttt{K\_RIGHT}}\myindex{pyg}{\texttt{K\_UP}}\myindex{pyg}{\texttt{K\_DOWN}}, the corresponding arrow key is checked and the appropriate message is sent to the defender.

If one of the arrow keys is released (\texttt{pygame.KEYUP}\myindex{pyg}{\texttt{KEYUP}} in \zeiref{srcTastatur0006}), the spaceship is stopped.

This should be sufficient for now. The keyboard is only one possible way to control a game. Mouse input, game controllers, and joysticks are also supported in Pygame.

%%%%%%%%%%%%%%%%%%%%%%%%%%%%%%%%%%%%%%%%%%%%%%%%%%%%%%%%%%%%%%%%%%%%%%%%%%%
\subsection{More Input}
%%%%%%%%%%%%%%%%%%%%%%%%%%%%%%%%%%%%%%%%%%%%%%%%%%%%%%%%%%%%%%%%%%%%%%%%%%%
\subsubsection{Example: Shift and Related Keys}
\begin{diskbox}
	\url{https://github.com/adamsralf/pygame_book/tree/main/src/00%20Introduction/06%20Keyboard/keyboard/v02}
\end{diskbox}

With the help of \keys{\shift}, \keys{\ctrl}, or other modifier keys, additional meanings can be assigned to regular keys. But how can we detect that such keys are being pressed? In the source code of the previous example, only a reaction to \emph{one} key press per condition is implemented.

I~will now extend the example so that pressing a movement key together with the left Shift key makes the spaceship move faster, while pressing it together with the right Shift key makes it move slower.

To do this, we first adapt the \texttt{update()} method of the \texttt{Defender}. As you can see, the signals to speed up or slow down can now be handled as well. It is also possible to reset the speed to its normal value. The concrete values are more or less arbitrary, but they are chosen so that the difference in speed is easy to perceive.

\newpage
\lstsource{SRC/00 Introduction/06 Keyboard/keyboard/v02/keyboard.py}{20}{33}{python}{Control Direction by Keys (4)), \texttt{Defender}.\texttt{update()}}{srcTastatur01a}

Now we adapt \texttt{watch\_for\_events()}. To be able to detect simultaneous key presses, a slightly different mechanism must be used. Internally, a \texttt{modifier} bit
\randnotiz{event.mod}\index{event.mod} is set. This bitmask must be checked using the \gls{binaryand}. In our example, this is done in \zeiref{srcTastatur0102} using \texttt{KMOD\_LSHIFT}\randnotiz{KMOD\_LSHIFT KMOD\_RSHIFT}\myindex{pyg}{\texttt{KMOD\_LSHIFT}} and in \zeiref{srcTastatur0103} using \texttt{KMOD\_RSHIFT}\myindex{pyg}{\texttt{KMOD\_RSHIFT}}.

\lstsource{SRC/00 Introduction/06 Keyboard/keyboard/v02/keyboard.py}{61}{87}{python}{Control Direction by Keys (5)), \texttt{Game}.\texttt{watch\_for\_events()}}{srcTastatur01b}

\begin{hintbox}[Querying modifier keys]
	You cannot check the modifier keys directly. Instead, you have to use a binary AND operation to determine whether the corresponding bit is set in a mask or not.
\end{hintbox}

In this way, the state of multiple modifier keys can be encoded and queried within a single integer. For example, if you want to check whether the keys \keys{\shift} and \keys{\Alt} are pressed at the same time, you can do so as follows:

\lstset{firstnumber=69}
\begin{lstlisting}
	if (event.mod & pygame.KMOD_SHIFT) and (event.mod & pygame.KMOD_ALT):
		...
\end{lstlisting}

Or

\lstset{firstnumber=69}
\begin{lstlisting}
	if (event.mod & (pygame.KMOD_SHIFT event.mod | pygame.KMOD_ALT)):
	    ...
\end{lstlisting}

As a complement, it should also be noted that \texttt{KMOD\_NONE}\myindex{pyg}{\texttt{KMOD\_NONE}} can be used to check whether no modifier key is pressed at all\randnotiz{KMOD\-\_NONE} (see \zeiref{srcTastatur0101}).
 

%%%%%%%%%%%%%%%%%%%%%%%%%%%%%%%%%%%%%%%%%%%%%%%%%%%%%%%%%%%%%%%%%%%%%%%%%%%
\subsubsection{In Which Window was the Key Pressed?}\label{secMultipleWindowKeyboard}
\begin{diskbox}
	\url{https://github.com/adamsralf/pygame_book/tree/main/src/00%20Introduction/06%20Keyboard/keyboard/v03}
\end{diskbox}

When working with multiple windows, it is certainly important to determine in which window a keyboard input was made. There are many ways to achieve this, but the two approaches presented here are probably the most straightforward.

As a basis, we use the example from \secref[vref]{secMultipleWindows}. Two windows are displayed next to each other.

The variables \texttt{window\_first} and \texttt{window\_second} are handle to the two windows. Internally, these handles are essentially memory addresses (\gls{ram}) through which the windows can be accessed via their properties and methods. Naturally, this address is unique for each window.

When a keyboard event is triggered, the event also carries a handle to the window in which it occurred. This handle can be accessed via \texttt{event.window}\randnotiz{event.window}\myindex{pyg}{\texttt{Window}},  which is exactly what happens in \zeiref{srcKeyboard03a} and \zeiref{srcKeyboard03b}. By comparing these handles, it is therefore possible to determine unambiguously in which window the event was triggered.

Alternatively, one can use the \texttt{id}\randnotiz{Window.id}\myindex{pyg}{\texttt{Window}!\texttt{id}} attribute. This value is also unique for each window and reflects the order in which the windows were created.

\newpage
\lstsource{SRC/00 Introduction/06 Keyboard/keyboard/v03/keyboard.py}{4}{43}{python}{In which window was the key pressed?}{srcKeyboard03}

Running the program produces the following console output when the corresponding keys are pressed:

\lstset{firstnumber=1}
\begin{lstlisting}
ID 1: Main Window (Key Pressed: 'a')
ID 1: Main Window (Key Pressed: 's')
ID 1: Main Window (Key Pressed: '5')
ID 1: Main Window (Key Pressed: 'j')
ID 2: Side Window (Key Pressed: 'm')
ID 2: Side Window (Key Pressed: 'space')
ID 2: Side Window (Key Pressed: '3')
ID 2: Side Window (Key Pressed: '4')	
\end{lstlisting}

%%%%%%%%%%%%%%%%%%%%%%%%%%%%%%%%%%%%%%%%%%%%%%%%%%%%%%%%%%%%%%%%%%%%%%%%%%%
\newpage
\subsubsection{Example: Visualizing the Keyboard}
\begin{diskbox}
	\url{https://github.com/adamsralf/pygame_book/tree/main/src/00%20Introduction/06%20Keyboard/typewriter}
\end{diskbox}

\myebild{keyboard00}{0.55}{Typewriter}{picKeyboard00}

Just as a small exercise, let us write a simple program that visualizes a key press (see \abbref[vref]{picKeyboard00}). For this purpose, I~want to define the keyboard layout in \texttt{config.py} so that it can be easily adapted to different variants -- here a simple U.S. keyboard layout.
 
\lstsource{SRC/00 Introduction/06 Keyboard/typewriter/config.py}{0}{99}{python}{Typewriter, \texttt{config.py}}{srcTypewriter00Config}

The constructor of \texttt{KeySprite} receives its label (i.e., its meaning), its value according to \tabref[vref]{tabKey}, and its position. Based on this data, the sprite, the text label, its position, and its size are computed and processed.

A \emph{normal} key has exactly the width and height defined in \texttt{config.py} by the dictionary \texttt{KEY}. Therefore, in \zeiref{keyboard0201} the width is multiplied by the factor~1, and no additional spacing is needed.

Other keys have different widths; for example, the \keys{Space} key has a width of 10~keys. This is determined using a case distinction and stored in
\texttt{factor\_width}.

From this factor, we can also determine how many gaps between keys must be included in the width calculation. For a normal key, no additional gap is consumed. For two keys, there is exactly one gap between them; for three keys, there are two gaps, and so on. So the number of gaps is the number of normal key widths minus~1. However, since we also use factor values such as~$1.5$, we must always compute the next higher integer. The result is the calculation in \zeiref{keyboard0202}.

In the next line, the total width of the key can then be computed based on the two factors.

The attribute \texttt{pressed} is a flag that stores whether the key is currently pressed or not. After that, the rectangle is created using the given position, and the label is rendered. Text output using fonts will be explained in more detail in a later section.

In \texttt{update()}, the key is filled either gray or red. From a performance point of view, this is not ideal; strictly speaking, we would only need to redraw the key when the state of \texttt{pressed} changes, but let us keep it simple here for now.

\lstsource{SRC/00 Introduction/06 Keyboard/typewriter/typewriter.py}{8}{42}{python}{Typewriter, Class \texttt{KeySprite}}{srckeyboard02a}

The constructor and the \texttt{run()} method of \texttt{Game} should be self-explanatory. In the dictionary \texttt{keyboard}, all keys are stored using the identifiers defined in \tabref[vref]{tabKey} as dictionary keys.

\lstsource{SRC/00 Introduction/06 Keyboard/typewriter/typewriter.py}{45}{64}{python}{Typewriter, \texttt{Game.init()} and \texttt{Game.run()}}{srckeyboard02b}

The creation of this dictionary takes place in the method \texttt{generate\_sprites()}. First, I~define the starting position of the first key. It is placed in the upper-left corner and starts with twice the vertical key spacing -- a purely arbitrary choice, but one that results in a visually pleasing layout.

Next, the keyboard layout defined in \texttt{config.py} is traversed using a nested loop. In \zeiref{keyboard0203}, the corresponding \texttt{Pygame} key value is determined from the key label. This value is used as the dictionary key in \zeiref{keyboard0204}, allowing it to be easily compared with and processed from the parameters of keyboard events later on.

For each key, a corresponding sprite is created. The new horizontal position \texttt{left} is calculated by adding the key spacing to the actual width of the key. Once all keys in a row have been processed, \texttt{left} is reset to its initial value and the vertical position \texttt{top} is shifted downward accordingly.

Finally, the generated dictionary is returned. Alternatively, it could have been stored directly as a class attribute; however, the approach chosen here keeps the method self-contained.

\lstsource{SRC/00 Introduction/06 Keyboard/typewriter/typewriter.py}{120}{133}{python}{Typewriter, \texttt{Game.generate\_sprites()}}{srckeyboard02c}

The tricky part is the method \texttt{label2key()}. Its task is to determine the corresponding \texttt{Pygame} key code, as listed in \tabref[vref]{tabKey}, based on a given key label. For many keys, such as \keys{T}, we can directly use \texttt{pygame.key.key\_code()} in \zeiref{keyboard0205}\randnotiz{key\_code()}\myindex{pyg}{\texttt{key}!\texttt{key\_code()}}.

For all other keys, we construct our own mapping tables. The underlying logic is more about Python than about Pygame itself and is therefore left to the interested reader to explore. 
 
\lstsource{SRC/00 Introduction/06 Keyboard/typewriter/typewriter.py}{85}{118}{python}{Typewriter, \texttt{Game.label2key()}}{srckeyboard02d}

The implementations of the methods \texttt{update()} and \texttt{draw()} are straightforward and require no further explanation.

\newpage
\lstsource{SRC/00 Introduction/06 Keyboard/typewriter/typewriter.py}{77}{83}{python}{Typewriter, \texttt{Game.update()} and \texttt{Game.draw()}}{srckeyboard02e}


%%%%%%%%%%%%%%%%%%%%%%%%%%%%%%%%%%%%%%%%%%%%%%%%%%%%%%%%%%%%%%%%%%%%%%%%%%%
\subsubsection{Not by Event, but by Function}\label{secKeyPressed}
\begin{hintbox}[Another way to find out which key was pressed or released]
\begin{itemize}
	\item \texttt{pygame.key.get\_pressed()}\myindex{pyg}{\texttt{key}!\texttt{get\_pressed()}}
	\item \texttt{pygame.key.get\_just\_pressed()}\myindex{pyg}{\texttt{key}!\texttt{get\_just\_pressed()}}
	\item \texttt{pygame.key.get\_just\_released()}\myindex{pyg}{\texttt{key}!\texttt{get\_just\_released()}}
\end{itemize}
\end{hintbox}
These functions each return a dictionary containing all available keys, where every entry is associated with a Boolean flag:

\begin{itemize}
	\item the value is \true{} if the key is currently pressed (or was released, depending on the function),
	
	\item otherwise it is \false.
\end{itemize}

The \gls{infix} \emph{just} therefore refers to the time span within two frames.

So, if you want to check whether \keys{K} is currently pressed, you can simply write something like:

\lstset{firstnumber=1}
\begin{lstlisting}
	if pygame.key.get_pressed()[pygame.K_k]:
	   ...
\end{lstlisting}

Often you will see constructions like this:

\lstset{firstnumber=1}
\begin{lstlisting}
	while pygame.key.get_pressed()[pygame.K_LEFT]:
	   ...
\end{lstlisting}

With this approach, actions inside the loop can react immediately to a key being held down -- for example, \emph{walking to the right}. 

\begin{warningbox}
	However, you then get \emph{trapped} inside the loop. If the game is supposed to react to something else as well, or if other game objects need to be updated, this does not work. 
\end{warningbox}

\newpage
In that case, it is better to query the keyboard once per frame:
\lstset{firstnumber=1}
\begin{lstlisting}
	keys = pygame.key.get_pressed()
	if keys[pygame.K_LEFT]:
	   ...
\end{lstlisting}

This can be quite useful, but I~have never really grown comfortable with this style of logic. Handling events as described above (see \secref[vref]{KeyboardByEvent}) does not trap me in loops and also feels ``cleaner'' to me, even though I~cannot fully justify that feeling.

One important note from the documentation: with this function, you cannot determine the order in which keys were used. This is only possible via event handling.


%%%%%%%%%%%%%%%%%%%%%%%%%%%%%%%%%%%%%%%%%%%%%%%%%%%%%%%%%%%%%%%%%%%%%%%%%%%
\subsection*{What was new?}
\begin{hintbox}
	The keyboard sends event messages that can be intercepted and evaluated. First, a distinction is made as to what kind of keyboard action occurred (\texttt{event.type}), and then which key was involved (\texttt{event.key}). Using \texttt{event.mod}, it is possible to query bitwise which modifier keys on the keyboard were used.
\end{hintbox}


\begin{pygbox}
\begin{itemize}
  \item \texttt{pygame.rect.FRect.clamp()}:
  \myindex{pyg}{\texttt{rect}!\texttt{FRect}!\texttt{clamp()}}\\
   \url{https://pyga.me/docs/ref/rect.html#pygame.FRect.clamp}

  \item \texttt{pygame.rect.FRect.clamp\_ip()}:
  \myindex{pyg}{\texttt{rect}!\texttt{FRect}!\texttt{clamp\_ip()}}\\
  \url{https://pyga.me/docs/ref/rect.html#pygame.FRect.clamp_ip}

  \item \texttt{pygame.rect.Rect.clamp()}:
  \myindex{pyg}{\texttt{rect}!\texttt{Rect}!\texttt{clamp()}}\\
  \url{https://pyga.me/docs/ref/rect.html#pygame.Rect.clamp}

	\item \texttt{pygame.rect.Rect.clamp\_ip()}:
	\myindex{pyg}{\texttt{rect}!\texttt{Rect}!\texttt{clamp\_ip()}}\\
	\url{https://pyga.me/docs/ref/rect.html#pygame.Rect.clamp_ip}

	\item \texttt{pygame.key}:
	\myindex{pyg}{\texttt{KEY}|underline}\\ 
	\url{https://pyga.me/docs/ref/key.html}

	\item \texttt{pygame.key.get\_pressed()}:
	\myindex{pyg}{\texttt{KEY}!\texttt{get\_pressed()}}\\ 
	\url{https://pyga.me/docs/ref/key.html#pygame.key.get_pressed}

	\item \texttt{pygame.key.get\_just\_pressed()}:
	\myindex{pyg}{\texttt{KEY}!\texttt{get\_just\_pressed()}}\\ 
	\url{https://pyga.me/docs/ref/key.html#pygame.key.get_just_pressed}

	\item \texttt{pygame.key.get\_just\_released()}:
	\myindex{pyg}{\texttt{KEY}!\texttt{get\_just\_released()}}\\ 
	\url{https://pyga.me/docs/ref/key.html#pygame.key.get_just_released}

	\item \texttt{pygame.key.key\_code()}:
	\myindex{pyg}{\texttt{KEY}!\texttt{key\_code()}}\\ 
	\url{https://pyga.me/docs/ref/key.html#pygame.key.key_code}

	\item \texttt{pygame.KEYDOWN}, \texttt{pygame.KEYUP}:
	\myindex{pyg}{\texttt{KEYDOWN}}\myindex{pyg}{\texttt{KEYUP}}\\ \url{https://pyga.me/docs/ref/event.html}
	
\end{itemize}
\end{pygbox}


\begin{longtable}{lll}
	\caption{Predefined Keyboard Constants}\label{tabKey}\index{Keyboard!Constants} \\
	% Definition des Tabellenkopfes auf der ersten Seite
	\toprule
     Constant & Meaning & Description \\
	\midrule
	\endfirsthead % Erster Kopf zu Ende
	
	% Definition des Tabellenkopfes auf den folgenden Seiten
	\caption{Predefined Keyboard Constants (continued)}\\
    \toprule
     Constant & Meaning & Description \\
	\midrule
	\endhead % Zweiter Kopf ist zu Ende
	
	\midrule
	\multicolumn{3}{r}{\emph{continued on next page}} \\
	\endfoot

	\bottomrule
	\endlastfoot
	
	% Ab hier kommt der Inhalt der Tabelle
\myindex{pyg}{\texttt{K\_BACKSPACE}}    \texttt{K\_BACKSPACE}    &  \verb+\b+    &  backspace \\ 
\myindex{pyg}{\texttt{K\_TAB}}          \texttt{K\_TAB}          &  \verb+\t+    &  tabulator\\ 
\myindex{pyg}{\texttt{K\_CLEAR}}        \texttt{K\_CLEAR}        &  \verb++      &  clear\\ 
\myindex{pyg}{\texttt{K\_RETURN}}       \texttt{K\_RETURN}       &  \verb+\r+    &  return, enter\\ 
\myindex{pyg}{\texttt{K\_PAUSE}}        \texttt{K\_PAUSE}        &  \verb++      &  pause\\ 
\myindex{pyg}{\texttt{K\_ESCAPE}}       \texttt{K\_ESCAPE}       &  \verb+^[+    &  escape\\ 
\myindex{pyg}{\texttt{K\_SPACE}}        \texttt{K\_SPACE}        &  \verb+ +     &  space\\ 
\myindex{pyg}{\texttt{K\_EXCLAIM}}      \texttt{K\_EXCLAIM}      &  \verb+!+     &  exclaim\\ 
\myindex{pyg}{\texttt{K\_QUOTEDBL}}     \texttt{K\_QUOTEDBL}     &  \verb+"+     &  double quote\\ 
\myindex{pyg}{\texttt{K\_HASH}}         \texttt{K\_HASH}         &  \verb+#+     &  hash\\ 
\myindex{pyg}{\texttt{K\_DOLLAR}}       \texttt{K\_DOLLAR}       &  \verb+$+     &  dollar\\ 
\myindex{pyg}{\texttt{K\_AMPERSAND}}    \texttt{K\_AMPERSAND}    &  \verb+&+     &  ampersand\\ 
\myindex{pyg}{\texttt{K\_QUOTE}}        \texttt{K\_QUOTE}        &  \verb+'+     &  quote\\ 
\myindex{pyg}{\texttt{K\_LEFTPAREN}}    \texttt{K\_LEFTPAREN}    &  \verb+(+     &  left parenthesis\\ 
\myindex{pyg}{\texttt{K\_RIGHTPAREN}}   \texttt{K\_RIGHTPAREN}   &  \verb+)+     &  right parenthesis\\ 
\myindex{pyg}{\texttt{K\_ASTERISK}}     \texttt{K\_ASTERISK}     &  \verb+*+     &  asterisk\\ 
\myindex{pyg}{\texttt{K\_PLUS}}         \texttt{K\_PLUS}         &  \verb-+-     &  plus\\ 
\myindex{pyg}{\texttt{K\_COMMA}}        \texttt{K\_COMMA}        &  \verb+,+     &  comma\\ 
\myindex{pyg}{\texttt{K\_MINUS}}        \texttt{K\_MINUS}        &  \verb+-+     &  minus\\ 
\myindex{pyg}{\texttt{K\_PERIOD}}       \texttt{K\_PERIOD}       &  \verb+.+     &  period\\ 
\myindex{pyg}{\texttt{K\_SLASH}}        \texttt{K\_SLASH}        &  \verb+/+     &  slash\\ 
\myindex{pyg}{\texttt{K\_0}}            \texttt{K\_0}            &  \verb+0+     &  0\\ 
\myindex{pyg}{\texttt{K\_1}}            \texttt{K\_1}            &  \verb+1+     &  1\\ 
\myindex{pyg}{\texttt{K\_2}}            \texttt{K\_2}            &  \verb+2+     &  2\\ 
\myindex{pyg}{\texttt{K\_3}}            \texttt{K\_3}            &  \verb+3+     &  3\\ 
\myindex{pyg}{\texttt{K\_4}}            \texttt{K\_4}            &  \verb+4+     &  4\\ 
\myindex{pyg}{\texttt{K\_5}}            \texttt{K\_5}            &  \verb+5+     &  5\\ 
\myindex{pyg}{\texttt{K\_6}}            \texttt{K\_6}            &  \verb+6+     &  6\\ 
\myindex{pyg}{\texttt{K\_7}}            \texttt{K\_7}            &  \verb+7+     &  7\\ 
\myindex{pyg}{\texttt{K\_8}}            \texttt{K\_8}            &  \verb+8+     &  8\\ 
\myindex{pyg}{\texttt{K\_9}}            \texttt{K\_9}            &  \verb+9+     &  9\\ 
\myindex{pyg}{\texttt{K\_COLON}}        \texttt{K\_COLON}        &  \verb+:+     &  colon\\ 
\myindex{pyg}{\texttt{K\_SEMICOLON}}    \texttt{K\_SEMICOLON}    &  \verb+;+     &  semicolon\\ 
\myindex{pyg}{\texttt{K\_LESS}}         \texttt{K\_LESS}         &  \verb+<+     &  less-than\\ 
\myindex{pyg}{\texttt{K\_EQUALS}}       \texttt{K\_EQUALS}       &  \verb+=+     &  equals\\ 
\myindex{pyg}{\texttt{K\_GREATER}}      \texttt{K\_GREATER}      &  \verb+>+     &  greater-than\\ 
\myindex{pyg}{\texttt{K\_QUESTION}}     \texttt{K\_QUESTION}     &  \verb+?+     &  question mark\\ 
\myindex{pyg}{\texttt{K\_AT}}           \texttt{K\_AT}           &  \makeatletter \verb+@+ \makeatother &  at\\ 
\myindex{pyg}{\texttt{K\_LEFTBRACKET}}  \texttt{K\_LEFTBRACKET}  &  \verb+[+     &  left bracket\\ 
\myindex{pyg}{\texttt{K\_BACKSLASH}}    \texttt{K\_BACKSLASH}    &  \verb+\+     &  backslash\\ 
\myindex{pyg}{\texttt{K\_RIGHTBRACKET}} \texttt{K\_RIGHTBRACKET} &  \verb+]+     &  right bracket\\ 
\myindex{pyg}{\texttt{K\_CARET}}        \texttt{K\_CARET}        &  \verb+^+     &  caret\\ 
\myindex{pyg}{\texttt{K\_UNDERSCORE}}   \texttt{K\_UNDERSCORE}   &  \verb+_+     &  underscore\\ 
\myindex{pyg}{\texttt{K\_BACKQUOTE}}    \texttt{K\_BACKQUOTE}    &  \verb+`+     &  grave\\ 
\myindex{pyg}{\texttt{K\_a}}            \texttt{K\_a}            &  \verb+a+     &  a\\ 
\myindex{pyg}{\texttt{K\_b}}            \texttt{K\_b}            &  \verb+b+     &  b\\ 
\myindex{pyg}{\texttt{K\_c}}            \texttt{K\_c}            &  \verb+c+     &  c\\ 
\myindex{pyg}{\texttt{K\_d}}            \texttt{K\_d}            &  \verb+d+     &  d\\ 
\myindex{pyg}{\texttt{K\_e}}            \texttt{K\_e}            &  \verb+e+     &  e\\ 
\myindex{pyg}{\texttt{K\_f}}            \texttt{K\_f}            &  \verb+f+     &  f\\ 
\myindex{pyg}{\texttt{K\_g}}            \texttt{K\_g}            &  \verb+g+     &  g\\ 
\myindex{pyg}{\texttt{K\_h}}            \texttt{K\_h}            &  \verb+h+     &  h\\ 
\myindex{pyg}{\texttt{K\_i}}            \texttt{K\_i}            &  \verb+i+     &  i\\ 
\myindex{pyg}{\texttt{K\_j}}            \texttt{K\_j}            &  \verb+j+     &  j\\ 
\myindex{pyg}{\texttt{K\_k}}            \texttt{K\_k}            &  \verb+k+     &  k\\ 
\myindex{pyg}{\texttt{K\_l}}            \texttt{K\_l}            &  \verb+l+     &  l\\ 
\myindex{pyg}{\texttt{K\_m}}            \texttt{K\_m}            &  \verb+m+     &  m\\ 
\myindex{pyg}{\texttt{K\_n}}            \texttt{K\_n}            &  \verb+n+     &  n\\ 
\myindex{pyg}{\texttt{K\_o}}            \texttt{K\_o}            &  \verb+o+     &  o\\ 
\myindex{pyg}{\texttt{K\_p}}            \texttt{K\_p}            &  \verb+p+     &  p\\ 
\myindex{pyg}{\texttt{K\_q}}            \texttt{K\_q}            &  \verb+q+     &  q\\ 
\myindex{pyg}{\texttt{K\_r}}            \texttt{K\_r}            &  \verb+r+     &  r\\ 
\myindex{pyg}{\texttt{K\_s}}            \texttt{K\_s}            &  \verb+s+     &  s\\ 
\myindex{pyg}{\texttt{K\_t}}            \texttt{K\_t}            &  \verb+t+     &  t\\ 
\myindex{pyg}{\texttt{K\_u}}            \texttt{K\_u}            &  \verb+u+     &  u\\ 
\myindex{pyg}{\texttt{K\_v}}            \texttt{K\_v}            &  \verb+v+     &  v\\ 
\myindex{pyg}{\texttt{K\_w}}            \texttt{K\_w}            &  \verb+w+     &  w\\ 
\myindex{pyg}{\texttt{K\_x}}            \texttt{K\_x}            &  \verb+x+     &  x\\ 
\myindex{pyg}{\texttt{K\_y}}            \texttt{K\_y}            &  \verb+y+     &  y\\ 
\myindex{pyg}{\texttt{K\_z}}            \texttt{K\_z}            &  \verb+z+     &  z\\ 
\myindex{pyg}{\texttt{K\_DELETE}}       \texttt{K\_DELETE}       &  \verb+ +     &  delete\\ 
\myindex{pyg}{\texttt{K\_KP0}}          \texttt{K\_KP0}          &  \verb+ +     &  keypad  0\\ 
\myindex{pyg}{\texttt{K\_KP1}}          \texttt{K\_KP1}          &  \verb+ +     &  keypad  1\\ 
\myindex{pyg}{\texttt{K\_KP2}}          \texttt{K\_KP2}          &  \verb+ +     &  keypad  2\\ 
\myindex{pyg}{\texttt{K\_KP3}}          \texttt{K\_KP3}          &  \verb+ +     &  keypad  3\\ 
\myindex{pyg}{\texttt{K\_KP4}}          \texttt{K\_KP4}          &  \verb+ +     &  keypad  4\\ 
\myindex{pyg}{\texttt{K\_KP5}}          \texttt{K\_KP5}          &  \verb+ +     &  keypad  5\\ 
\myindex{pyg}{\texttt{K\_KP6}}          \texttt{K\_KP6}          &  \verb+ +     &  keypad  6\\ 
\myindex{pyg}{\texttt{K\_KP7}}          \texttt{K\_KP7}          &  \verb+ +     &  keypad  7\\ 
\myindex{pyg}{\texttt{K\_KP8}}          \texttt{K\_KP8}          &  \verb+ +     &  keypad  8\\ 
\myindex{pyg}{\texttt{K\_KP9}}          \texttt{K\_KP9}          &  \verb+ +     &  keypad  9\\ 
\myindex{pyg}{\texttt{K\_KP\_PERIOD}}   \texttt{K\_KP\_PERIOD}   &  \verb+.+     &  keypad  period\\ 
\myindex{pyg}{\texttt{K\_KP\_DIVIDE}}   \texttt{K\_KP\_DIVIDE}   &  \verb+/+     &  keypad  divide\\ 
\myindex{pyg}{\texttt{K\_KP\_MULTIPLY}} \texttt{K\_KP\_MULTIPLY} &  \verb+*+     &  Nummernfeld multiply\\ 
\myindex{pyg}{\texttt{K\_KP\_MINUS}}    \texttt{K\_KP\_MINUS}    &  \verb+-+     &  keypad  minus\\ 
\myindex{pyg}{\texttt{K\_KP\_PLUS}}     \texttt{K\_KP\_PLUS}     &  \verb-+-     &  keypad  plus\\ 
\myindex{pyg}{\texttt{K\_KP\_ENTER}}    \texttt{K\_KP\_ENTER}    &  \verb+\r+    &  keypad  return, enter\\ 
\myindex{pyg}{\texttt{K\_KP\_EQUALS}}   \texttt{K\_KP\_EQUALS}   &  \verb+=+     &  keypad  equals\\ 
\myindex{pyg}{\texttt{K\_UP}}           \texttt{K\_UP}           &  \verb+ +     &  up arrow\\ 
\myindex{pyg}{\texttt{K\_DOWN}}         \texttt{K\_DOWN}         &  \verb+ +     &  down arrow\\ 
\myindex{pyg}{\texttt{K\_RIGHT}}        \texttt{K\_RIGHT}        &  \verb+ +     &  right arrow\\ 
\myindex{pyg}{\texttt{K\_LEFT}}         \texttt{K\_LEFT}         &  \verb+ +     &  left arrow\\ 
\myindex{pyg}{\texttt{K\_INSERT}}       \texttt{K\_INSERT}       &  \verb+ +     &  insert\\ 
\myindex{pyg}{\texttt{K\_HOME}}         \texttt{K\_HOME}         &  \verb+ +     &  home\\ 
\myindex{pyg}{\texttt{K\_END}}          \texttt{K\_END}          &  \verb+ +     &  end\\ 
\myindex{pyg}{\texttt{K\_PAGEUP}}       \texttt{K\_PAGEUP}       &  \verb+ +     &  page up\\ 
\myindex{pyg}{\texttt{K\_PAGEDOWN}}     \texttt{K\_PAGEDOWN}     &  \verb+ +     &  page down\\ 
\myindex{pyg}{\texttt{K\_F1}}           \texttt{K\_F1}           &  \verb+ +     &  F1\\ 
\myindex{pyg}{\texttt{K\_F2}}           \texttt{K\_F2}           &  \verb+ +     &  F2\\ 
\myindex{pyg}{\texttt{K\_F3}}           \texttt{K\_F3}           &  \verb+ +     &  F3\\ 
\myindex{pyg}{\texttt{K\_F4}}           \texttt{K\_F4}           &  \verb+ +     &  F4\\ 
\myindex{pyg}{\texttt{K\_F5}}           \texttt{K\_F5}           &  \verb+ +     &  F5\\ 
\myindex{pyg}{\texttt{K\_F6}}           \texttt{K\_F6}           &  \verb+ +     &  F6\\ 
\myindex{pyg}{\texttt{K\_F7}}           \texttt{K\_F7}           &  \verb+ +     &  F7\\ 
\myindex{pyg}{\texttt{K\_F8}}           \texttt{K\_F8}           &  \verb+ +     &  F8\\ 
\myindex{pyg}{\texttt{K\_F9}}           \texttt{K\_F9}           &  \verb+ +     &  F9\\ 
\myindex{pyg}{\texttt{K\_F10}}          \texttt{K\_F10}          &  \verb+ +     &  F10\\ 
\myindex{pyg}{\texttt{K\_F11}}          \texttt{K\_F11}          &  \verb+ +     &  F11\\ 
\myindex{pyg}{\texttt{K\_F12}}          \texttt{K\_F12}          &  \verb+ +     &  F12\\ 
\myindex{pyg}{\texttt{K\_F13}}          \texttt{K\_F13}          &  \verb+ +     &  F13\\ 
\myindex{pyg}{\texttt{K\_F14}}          \texttt{K\_F14}          &  \verb+ +     &  F14\\ 
\myindex{pyg}{\texttt{K\_F15}}          \texttt{K\_F15}          &  \verb+ +     &  F15\\ 
\myindex{pyg}{\texttt{K\_NUMLOCK}}      \texttt{K\_NUMLOCK}      &  \verb+ +     &  numlock\\ 
\myindex{pyg}{\texttt{K\_CAPSLOCK}}     \texttt{K\_CAPSLOCK}     &  \verb+ +     &  capslock\\ 
\myindex{pyg}{\texttt{K\_SCROLLOCK}}    \texttt{K\_SCROLLOCK}    &  \verb+ +     &  scrollock\\ 
\myindex{pyg}{\texttt{K\_RSHIFT}}       \texttt{K\_RSHIFT}       &  \verb+ +     &  right shift\\ 
\myindex{pyg}{\texttt{K\_LSHIFT}}       \texttt{K\_LSHIFT}       &  \verb+ +     &  left shift\\ 
\myindex{pyg}{\texttt{K\_RCTRL}}        \texttt{K\_RCTRL}        &  \verb+ +     &  right control\\ 
\myindex{pyg}{\texttt{K\_LCTRL}}        \texttt{K\_LCTRL}        &  \verb+ +     &  left control\\ 
\myindex{pyg}{\texttt{K\_RALT}}         \texttt{K\_RALT}         &  \verb+ +     &  right alt\\ 
\myindex{pyg}{\texttt{K\_LALT}}         \texttt{K\_LALT}         &  \verb+ +     &  left alt\\ 
\myindex{pyg}{\texttt{K\_RMETA}}        \texttt{K\_RMETA}        &  \verb+ +     &  right meta\\ 
\myindex{pyg}{\texttt{K\_LMETA}}        \texttt{K\_LMETA}        &  \verb+ +     &  left meta\\ 
\myindex{pyg}{\texttt{K\_LSUPER}}       \texttt{K\_LSUPER}       &  \verb+ +     &  left Windows key\\ 
\myindex{pyg}{\texttt{K\_RSUPER}}       \texttt{K\_RSUPER}       &  \verb+ +     &  right windows key\\ 
\myindex{pyg}{\texttt{K\_MODE}}         \texttt{K\_MODE}         &  \verb+ +     &  mode shift/AltGr\\ 
\myindex{pyg}{\texttt{K\_HELP}}         \texttt{K\_HELP}         &  \verb+ +     &  help\\ 
\myindex{pyg}{\texttt{K\_PRINT}}        \texttt{K\_PRINT}        &  \verb+ +     &  print screen\\ 
\myindex{pyg}{\texttt{K\_SYSREQ}}       \texttt{K\_SYSREQ}       &  \verb+ +     &  sysreq\\ 
\myindex{pyg}{\texttt{K\_BREAK}}        \texttt{K\_BREAK}        &  \verb+ +     &  break\\ 
\myindex{pyg}{\texttt{K\_MENU}}         \texttt{K\_MENU}         &  \verb+ +     &  menu\\ 
\myindex{pyg}{\texttt{K\_POWER}}        \texttt{K\_POWER}        &  \verb+ +     &  power\\ 
\myindex{pyg}{\texttt{K\_EURO}}         \texttt{K\_EURO}         &  \verb+€+     &  Euro\\ 
\myindex{pyg}{\texttt{K\_AC\_BACK}}     \texttt{K\_AC\_BACK}     &  \verb+ +     &  Android back button\\ 
\end{longtable} 

\begin{longtable}{ll}
	\caption{Predefined Keyboard Modifier}\label{tabKeyMod}\index{Keyboard!Modifier} \\
	% Definition des Tabellenkopfes auf der ersten Seite
	\toprule
	Constant  & Description\\
	\midrule
	\endfirsthead % Erster Kopf zu Ende
	
	% Definition des Tabellenkopfes auf den folgenden Seiten
	\caption{Predefined Keyboard Modifier (continued)}\\
	\toprule
	Constant  & Description\\
	\midrule
	\endhead % Zweiter Kopf ist zu Ende
	
	\midrule
	\multicolumn{2}{r}{\emph{continued on next page}} \\
	\endfoot

	\bottomrule
	\endlastfoot
	% Ab hier kommt der Inhalt der Tabelle
    \myindex{pyg}{\texttt{KMOD\_NONE}}   \texttt{KMOD\_NONE}   &  no modifier keys pressed\\ 
    \myindex{pyg}{\texttt{KMOD\_LSHIFT}} \texttt{KMOD\_LSHIFT} &  left shift\\ 
    \myindex{pyg}{\texttt{KMOD\_RSHIFT}} \texttt{KMOD\_RSHIFT} &  right shift\\ 
    \myindex{pyg}{\texttt{KMOD\_SHIFT}}  \texttt{KMOD\_SHIFT}  &  left or right shift or both\\ 
    \myindex{pyg}{\texttt{KMOD\_LCTRL}}  \texttt{KMOD\_LCTRL}  &  left control\\ 
    \myindex{pyg}{\texttt{KMOD\_RCTRL}}  \texttt{KMOD\_RCTRL}  &  right control\\ 
    \myindex{pyg}{\texttt{KMOD\_CTRL}}   \texttt{KMOD\_CTRL}   &  left or right control or both\\ 
    \myindex{pyg}{\texttt{KMOD\_LALT}}   \texttt{KMOD\_LALT}   &  left alt\\ 
    \myindex{pyg}{\texttt{KMOD\_RALT}}   \texttt{KMOD\_RALT}   &  right alt\\ 
    \myindex{pyg}{\texttt{KMOD\_ALT}}    \texttt{KMOD\_ALT}    &  left or right alt or both\\ 
    \myindex{pyg}{\texttt{KMOD\_LMETA}}  \texttt{KMOD\_LMETA}  &  left meta\\ 
    \myindex{pyg}{\texttt{KMOD\_RMETA}}  \texttt{KMOD\_RMETA}  &  right meta\\ 
    \myindex{pyg}{\texttt{KMOD\_META}}   \texttt{KMOD\_META}   &  left or right meta or both\\ 
    \myindex{pyg}{\texttt{KMOD\_CAPS}}   \texttt{KMOD\_CAPS}   &  caps lock\\ 
    \myindex{pyg}{\texttt{KMOD\_NUM}}    \texttt{KMOD\_NUM}    &  num lock\\ 
    \myindex{pyg}{\texttt{KMOD\_MODE}}   \texttt{KMOD\_MODE}   &  AltGr\\ 
\end{longtable} 

%%%%%%%%%%%%%%%%%%%%%%%%%%%%%%%%%%%%%%%%%%%%%%%%%%%%%%%%%%%%%%%%%%%%%%%%%%%
\subsection{Homework}
\begin{enumerate}
	\item Place a ball bitmap with a non-uniform shape in the center of the window. The ball starts with a radius of \SI{100}{px}.	Pressing \keys{{+}}\randnotiz{scale()}\myindex{pyg}{\texttt{transform}!\texttt{scale()}} increases the size of the ball, and pressing \keys{-} decreases it. If the ball touches the border of the window, its size must not be	reduced any further. The minimum radius is \SI{10}{px}.
	
	\item Use the arrow keys to move the ball inside the window. The ball must not leave the window.
	
	\item Using\randnotiz{rotate()}\myindex{pyg}{\texttt{transform}!\texttt{rotate()}} \keys{\shift + \arrowkeyleft} rotates the ball by \ang{-90}, and \keys{\shift + \arrowkeyright} rotates it by \ang{90}.
	The function you need is \texttt{pygame.transform.rotate()}.
	
	\item Change the rotation from \ang{-90} to \ang{-10} and from \ang{90} to \ang{10}. Do you notice anything strange? Find out how to avoid this behavior. It is indeed a little tricky.
\end{enumerate}



	% !TeX spellcheck = en_US
\newpage
%%%%%%%%%%%%%%%%%%%%%%%%%%%%%%%%%%%%%%%%%%%%%%%%%%%%%%%%%%%%%%%%%%%%%%%%%%%
\section{Text output using fonts}\index{Font}
\subsection{Introduction}
\begin{diskbox}
	\url{https://github.com/adamsralf/pygame_book/tree/main/src/00%20Introduction/07%20Fonts}
\end{diskbox}

\myebild{font00}{0.8}{Simple text output using fonts}{picFont00}

In many games, information is not shown only in a symbolic way on the playfield (for example, three little figures to represent three lives), but also as written text. One way to achieve this is by displaying text using installed fonts.

The basic idea is simple: first, a \texttt{Font} object is created (see \zeiref{textoutput00a}). Then this font is used to generate a Surface object that contains the text -- the text is \glslink{render}{rendered}\randnotiz{render} onto a \texttt{Surface} object (see \zeiref{textoutput00b}). Once rendered, the text behaves like any other bitmap and can be blitted onto the screen.

\lstsource{SRC/00 Introduction/07 Fonts/textoutput00.py}{7}{35}{python}{Simple text output using fonts}{srcFonts00aa} 

But how does the contructor of \texttt{Font} know, which font I want? In \zeiref{textoutput00a} we use the predefined default font. The method \texttt{pygame.font.get\_default\_font()}\randnotiz{get\_default\_font()}\myindex{pyg}{\texttt{font}!\texttt{get\_default\_font()}} returns a unique string -- the internal name of the font. You can find in \secref[vref]{secFontInstalled} a program, which shows you all installed fonts and there internal names.

If you want to use a specific font, you can get all required information with \texttt{pygame\-.font\-.match\_font()}\randnotiz{match\_font()}\myindex{pyg}{\texttt{font}!\texttt{match\_font()}}:
\lstset{firstnumber=14}
\begin{lstlisting}
    font = pygame.font.Font(pygame.font.match_font("arial"), 24)
\end{lstlisting}

\begin{warningbox}[Don't wast computing time]
	It is a waste of computing time to render the text every frame. Try to cache the rendered \texttt{Surface}-Object and use some kind of dirty-flag to indicate, if the text was updated.
\end{warningbox}

%%%%%%%%%%%%%%%%%%%%%%%%%%%%%%%%%%%%%%%%%%%%%%%%%%%%%%%%%%%%%%%%%%%%%%%%%%%
\subsection{More Input}
%%%%%%%%%%%%%%%%%%%%%%%%%%%%%%%%%%%%%%%%%%%%%%%%%%%%%%%%%%%%%%%%%%%%%%%%%%%
\subsubsection{A More Sophisticated Approach}
\begin{diskbox}
	\url{https://github.com/adamsralf/pygame_book/tree/main/src/00%20Introduction/07%20Fonts}
\end{diskbox}
\myebild{font01}{0.8}{Text output using fonts}{picFont01}

For a small example, I have wrapped this functionality into a class. You can easily extend it, modify it, or adapt it to your own needs.

\lstsource{SRC/00 Introduction/07 Fonts/config.py}{1}{99}{python}{Text output using fonts (1), \texttt{config.py}}{srcFonts00a} 

And now the class \texttt{TextSprite}: do not let the \glslink{oo}{OO} approach confuse you. In fact, everything is quite simple. We need a \texttt{pygame.font.\-Font} object \myindex{pyg}{\texttt{font}!\texttt{Font}}\randnotiz{Font}. This object, in turn, requires two pieces of information: which installed \gls{font} it should use, and the font size in \glslink{pt}{$pt$}.

One way to obtain an installed font is the method \texttt{pygame.font.get\_default\_font()}\myindex{pyg}{\texttt{font}!\texttt{get\_default\_font()}}\randnotiz{get\_default\_font()}. Its call in \zeiref{srcTextoutputSimple01} returns the default font configured by the operating system. The font size (\texttt{fontsize}) can then be chosen freely according to our needs.


\lstsource{SRC/00 Introduction/07 Fonts/textoutput01.py}{8}{44}{python}{Text output using fonts (2), \texttt{TextSprite}}{srcFonts00b} 

Let us now take a closer look at the constructor. The attributes \texttt{image} and \texttt{rect} are initially created as dummy values; strictly speaking, this would not even be necessary. After storing the passed information about font size and font color\index{Font!size}\index{Font!color} in attributes, the \texttt{Font} object can be created. This is done by calling \texttt{fontsize\_update()} in \zeiref{srcTextoutputSimple02}. Passing the value~0 makes it clear that the size itself is not meant to be changed here, but that the object creation should take place.

Next, the actual text that is to be rendered as a label is stored, as well as the position where the center of the text should be placed. At this point, all required information is available, and by calling \texttt{render()} in \zeiref{srcTextoutputSimple03}, the \texttt{Surface} object is created using \texttt{pygame\-.font\-.ren\-der()}\myindex{pyg}{\texttt{font}!\texttt{Font}!\texttt{render()}}\randnotiz{render()} (\zeiref{srcTextoutputSimple04}). Afterwards, the rectangle of the bitmap is determined and its center is moved to the desired position.

Finally, there are the two methods \texttt{fontsize\_update()} and \texttt{fontcolor\_update()}. Both allow the font size and font color to be changed at runtime. Their semantics should be self-explanatory.

How can such a class be used in practice? Here is a simple example. A greeting is displayed in the center using the object \texttt{hello} (\zeiref{srcTextoutputSimple07}). Below it, the object \texttt{info} displays which font size and font color were used to render the greeting (\zeiref{srcTextoutputSimple07}).

\lstsource{SRC/00 Introduction/07 Fonts/textoutput01.py}{47}{91}{python}{Text output using fonts (3),  Main program}{srcFonts00c} 

The greeting can be resized using \keys{{+}} and \keys{{-}} (\zeiref{srcTextoutputSimple05}ff.). The keys \keys{r}, \keys{g}, and \keys{b} are used to manipulate the corresponding color channel. An uppercase letter increases the value (for example in \zeiref{srcTextoutputSimple09}), while the lowercase letter decreases it (for example in \zeiref{srcTextoutputSimple10}).

In \abbref[vref]{picFont01} you can see one possible visual result.

%%%%%%%%%%%%%%%%%%%%%%%%%%%%%%%%%%%%%%%%%%%%%%%%%%%%%%%%%%%%%%%%%%%%%%%%%%%
\subsubsection{List of all Installed Fonts}\label{secFontInstalled}
\begin{diskbox}
	\url{https://github.com/adamsralf/pygame_book/tree/main/src/00%20Introduction/07%20Fonts/fontslist}
\end{diskbox}
As another example, I would like to show you a small program that lists all installed fonts. This may be useful for getting ideas for visual design. The first part of the code should not cause any problems in terms of understanding.

\myebild{font02}{0.65}{List of all installed fonts}{picFont02}

\newpage
\lstsource{SRC/00 Introduction/07 Fonts/fontslist/config.py}{1}{99}{python}{List of all installed fonts (1), \texttt{config.py}}{srcFonts01aa} 

The class \texttt{TextSprite} was customized a little bit, but has still the same logic. 

\lstsource{SRC/00 Introduction/07 Fonts/fontslist/textoutput02.py}{6}{33}{python}{List of all installed fonts (2), \texttt{TextSprite}}{srcFonts01a} 

The class \texttt{BigImage} is responsible for managing all \texttt{FontSprite} images as one large bitmap. Later on, only a subsection of this bitmap is blitted onto the screen. This subsection depends on the current position within the list and is controlled by the attribute \texttt{offset}, which is updated in the method \texttt{update()} (\zeiref{srcTextoutputFontlist01}).

First, it is checked whether the upper or lower end of the bitmap has been reached. If this is the case, \texttt{top} or \texttt{bottom} is set accordingly, so that the entire screen is always filled. Otherwise, the \texttt{offset} rectangle is shifted up or down, and the corresponding subsection is determined using \texttt{pygame.Surface.subsurface()}\myindex{pyg}{\texttt{Surface}!\texttt{subsurface()}}\randnotiz{subsurface()}.


\lstsource{SRC/00 Introduction/07 Fonts/fontslist/textoutput02.py}{36}{55}{python}{List of all installed fonts (3), \texttt{BigImage}}{srcFonts01b} 

And now the main program. In the first part, a list of all installed font names is obtained via \texttt{pygame.font.get\_fonts()}\myindex{pyg}{\texttt{font}!\texttt{get\_fonts()}}\randnotiz{get\_fonts()} (\zeiref{srcTextoutputFontlist02}). Each of these names is passed to the constructor of \texttt{TextSprite}.

Using the method \texttt{pygame.font.match\_font()}\myindex{pyg}{\texttt{font}!\texttt{match\_font()}}\randnotiz{match\_font()} (\zeiref{srcTextoutputFontlist03}), the actual font file is then searched for on the system. This method takes advantage of the fact that the font file name can usually be derived from the font name and the file extension~\glslink{ttf}{\texttt{ttf}}.

\lstsource{SRC/00 Introduction/07 Fonts/fontslist/textoutput02.py}{58}{102}{python}{List of all installed fonts (4), \texttt{main()}}{srcFonts01c} 

In the \forSchleife{}, \texttt{TextSprite} objects are now created for all fonts, and their height and width are determined. All of these individual bitmaps are then blitted onto the large bitmap (\zeiref{srcTextoutputFontlist04}).

The main loop is now only responsible for scrolling (each time by one third of the screen height) and for terminating the program.

%%%%%%%%%%%%%%%%%%%%%%%%%%%%%%%%%%%%%%%%%%%%%%%%%%%%%%%%%%%%%%%%%%%%%%%%%%%
\subsubsection{Using Locally Installed Fonts}
\begin{diskbox}
	\begin{itemize}
		\item \url{https://github.com/adamsralf/pygame_book/tree/main/src/00%20Introduction/07%20Fonts/localfont}
		\item \url{https://github.com/adamsralf/pygame_book/blob/main/src/00%20Introduction/07%20Fonts/localfont/fonts/Alex%20Winterbottom%20License.txt}
	\end{itemize}
\end{diskbox}
\myebild{font03}{1.0}{Example of using a locally installed font}{picFont03}

In almost all aspects, using local fonts is identical to using system fonts. So what does this distinction actually mean? System fonts are fonts that have been registered and installed in the operating system via an installation process. Local fonts, on the other hand, are font files -- such as \texttt{ttf} files -- that are stored in a subdirectory of the game itself. In our example, this is the file \texttt{rothenbg.ttf} located in the \texttt{fonts} subdirectory.

There are a few practical reasons why local fonts are often preferred in games. By shipping a font file together with the game, you ensure that the visual appearance of the text is identical on all systems, independent of which fonts are installed on the player’s operating system.

In the example above (\srcref[vref]{srcFonts00b}, \zeiref{srcTextoutputSimple01}), the constructor of \texttt{Font} was informed which system font to use via the function \texttt{get\_default\_font()}\myindex{pyg}{\texttt{font}!\texttt{get\_default\_font()}}. In \srcref[vref]{srcFonts01a}, this was done by specifying a name known to the system -- that is, a string under which the font is registered in the operating system.

In the example \srcref[vref]{srcFonts03a}, however, a file name including a relative path is passed directly to the constructor of \texttt{Font}. In this case, \texttt{Font} looks for the corresponding file and uses it to build its \texttt{Font} object (see \zeiref{textoutput03a}).

From the programmer’s point of view, the difference between system fonts and local fonts is very small. In practice, it only affects how the constructor of \texttt{Font} is called. Once the \texttt{Font} object has been created, rendering and using text works in exactly the same way for both system and local fonts.


When using local fonts, it is important to pay attention to licensing. Not every font may be freely redistributed. Before including a font file in your project, you should always check whether its license allows redistribution as part of a game or application. Our example was taken from \url{https://www.fontsquirrel.com/fonts/list/tag/historical} and was kindly made available as freeware by Alex Winterbottom.

\lstsource{SRC/00 Introduction/07 Fonts/localfont/textoutput03.py}{7}{35}{python}{Using locally installed fonts}{srcFonts03a} 

In \abbref[vref]{picFont03} you can admire the result ;-)

%%%%%%%%%%%%%%%%%%%%%%%%%%%%%%%%%%%%%%%%%%%%%%%%%%%%%%%%%%%%%%%%%%%%%%%%%%%
\subsubsection{Text output}
\begin{diskbox}
	\url{https://github.com/adamsralf/pygame_book/tree/main/src/00%20Introduction/07%20Fonts/textbitmaps}
\end{diskbox}

Text output is often not done via fonts, but via a \gls{spritelib}. Such a library contains character sprites, symbols, or digits, usually in a special design that matches the style of the game. In \abbref[vref]{picSpritelib01} you can see a spritelib that provides sprites for a World War~II dogfight game. Among other things, it contains the sprites for the digits $0$--$9$ and the letters of the Latin alphabet.

\begin{hintbox}
	One advantage of this approach is that you do not have to rely on a specific game font being available on the target system. If you render text using a font such as \emph{Calibri}, that font must be installed on the player’s computer. 
\end{hintbox}
\begin{warningbox}
	A disadvantage is that bitmaps usually scale poorly, so you often do not have many different font sizes available.
\end{warningbox}

The idea is to \emph{punch out} the individual letters from the spritelib and store them in a suitable data structure. Whenever text needs to be displayed, the string is split into its characters, and the corresponding letter sprites are blitted from the data structure onto a target bitmap -- for example onto the screen. 

I will demonstrate this with a simple example. Our starting point is a spritelib that contains a character set in five different colors (see \abbref[vref]{picTextbitmaps01}).

The first part of \srcref[vref]{srcTextbitmaps00a} should look familiar and is only extended by a few convenience features. The file paths are now determined via the functions \texttt{filepath()} and \texttt{imagepath()}.

\lstsource{SRC/00 Introduction/07 Fonts/textbitmaps/config.py}{1}{99}{python}{Textbitmaps (1), \texttt{config.py}}{srcTextbitmaps00a} 
\clearpage
\myezweivbild{1945}{0.5}{Example of a spritelib}{picSpritelib01}%
			 {textbitmaps00}{0.5}{Text output using bitmaps}{picTextbitmaps01}
\clearpage

The class \texttt{Spritelib} is mainly used as a container. It loads the spritelib containing the letters and symbols and stores several parameters that are needed to extract individual letters or symbols precisely from the bitmap:

\begin{itemize}
	\item \texttt{nof}: Stores the number of rows and columns. In our case, the symbol set is arranged in the bitmap in 4~rows and 10~columns.	Since we are only interested in one color at a time, this information is sufficient.
	
	\item \texttt{letter}: Each sprite has a fixed width and height. In our case, this is particularly convenient because all sprites have the same dimensions. Take a look at the three squares around the letters \texttt{N}, \texttt{W}, and \texttt{X} in \abbref[vref]{picSpritelib02}. All sprites have a width and a height of \SI{18}{px}.
	
	\item \texttt{offset}: The first sprite in the top-left corner has a distance from the left edge and from the top edge of the bitmap. This can be seen clearly for the sprite of the digit \texttt{0} in \abbref{picSpritelib02}. There is a square around the bitmap and a gap between this square and the upper and left edges (marked by the green line). In our example, both offsets have a value of \SI{6}{px}.
	
	\item \texttt{distance}: Each sprite has a fixed distance to the next sprite to the right and to the one below. Fortunately, the sprites in our spritelib are arranged equidistantly, which simplifies things a lot. Using the sprite for \texttt{X} in \abbref{picSpritelib02} as an example, you can see these distances. In our case, they are \SI{14}{px} each.
\end{itemize}


\begin{figure}[H]
	\begin{center}
		\begin{tikzpicture}
			\node (myfirstpic) at (1cm,-1cm) {\includegraphics[scale=1.2]{symbol_red.png}};
			
			\draw
			(-4.2cm, 1.5cm) node (a) {}
			( 6.15cm,1.5cm) node (b) {}
			(-4.2cm, 1.2cm) node (c) {}
			(-4.2cm,-3.2cm) node (d) {}
			;
			\draw 
			(-0.73, -2.65) node (X) {}
			(-0.72, -2.44)[inner sep=0, outer sep=0] node (XL) {}
			(-0.52, -2.22)[inner sep=0, outer sep=0] node (XT) {}
			
			(-0.73, -1.67) node (N) {}
			
			(-1.71, -2.65) node (W) {}
			
			(-3.66,  0.28) node (O) {}
			(-3.45,  0.71) node (OT) {}
			(-3.65,  0.5) node (OL) {}
			;
			
			
			
			\draw[>-<, very thick, blue, densely dotted] 
			(a) --  
			node[above, blue, xshift=0cm] {nof['cols']}
			(b)
			;
			\draw[>-<, very thick, blue, densely dotted] 
			(c) --  
			node[left, blue, xshift=0cm] {nof['rows']}
			(d)
			;
			\draw (X) rectangle +(0.44cm, 0.44cm);
			\draw (N) rectangle +(0.44cm, 0.44cm);
			\draw (W) rectangle +(0.44cm, 0.44cm);
			\draw (O) rectangle +(0.44cm, 0.44cm);
			
			\draw[very thick, black] 
			(OL) -- 
			node[below, black, yshift=-0.2cm, xshift=0.4cm] {\tiny offset['h']}
			+(-0.24cm, 0.0cm);
			
			\draw[very thick, black]
			(OT) -- 
			node[right, black, yshift=0.0cm, xshift=0cm] {\tiny offset['v']}
			+( 0.0cm, 0.24cm);
			
			\draw[very thick, black] 
			(XL) -- 
			node[below, black, yshift=-0.1cm] {\footnotesize distance['h']}
			+(-0.54cm, 0.0cm);
			
			\draw[very thick, black] 
			(XT) --  
			node[right, black, xshift=0cm] {\footnotesize distance['v']} 
			+( 0.0cm,  0.54cm);
			
		\end{tikzpicture}
		\caption{Meaning of the variables in \texttt{Spritelib}}\label{picSpritelib02}
	\end{center}
\end{figure}


\lstsource{SRC/00 Introduction/07 Fonts/textbitmaps/textbitmaps.py}{6}{18}{python}{Textbitmaps (2), \texttt{Spritelib}}{srcTextbitmaps00b} 

Let us now move on to the actually interesting class: \texttt{Letters}. This class cuts out all sprites of a single color from the spritelib and makes them available as \texttt{Surface} objects in a \gls{dictionary}\randnotiz{Dictionary}\index{Dictionary}. This involves quite a bit of calculation, but do not let that intimidate you; in the end, it is nothing more than basic arithmetic.

Let us start with the constructor. The constructor has two parameters. The first parameter, \texttt{spritelib}, is a reference to the \texttt{Spritelib} object, which has loaded the original bitmap and provides several spacing and layout parameters. The second parameter, \texttt{colornumber}, allows us to extract the complete set of symbols for exactly one color later on: \texttt{0} stands for the white sprites, \texttt{1} for the yellow ones, and so on.


\lstsource{SRC/00 Introduction/07 Fonts/textbitmaps/textbitmaps.py}{21}{27}{python}{Textbitmaps (3), Constructor of \texttt{Letters}}{srcTextbitmaps00c} 

In the method \texttt{create\_letter\_bitmap()}, the individual sprites are now cut out and stored in a dictionary. The indices of this dictionary are defined in \zeiref{srcTextbitmaps0000}. Here, the order must of course match the order in which the sprites are cut out. The variable \texttt{index} ensures exactly this: with each loop iteration, the next \texttt{lettername} is used as the key for the dictionary.

In \zeiref{srcTextbitmaps0001}, the position -- i.\,e.\ the pixel coordinates -- of the first sprite is calculated. Try to follow the arithmetic yourself using the information given in \abbref[vref]{picSpritelib02}! Don’t worry: it is not difficult, just a bit lengthy.

Starting at \zeiref{srcTextbitmaps0002}, a nested \forSchleife{} begins. The outer loop iterates over all rows of the spritelib, and the inner loop over the columns. The goal of this construction is to create one \texttt{Rect} object for each sprite, in which the position and size of the sprite are stored. In \zeiref{srcTextbitmaps0003}, the top coordinate is calculated, and in \zeiref{srcTextbitmaps0005} the left coordinate. If you have understood \zeiref{srcTextbitmaps0001}, these two calculations should no longer pose any problems. 

Height and width in \zeiref{srcTextbitmaps0005} are straightforward, since all sprites always have the same dimensions. After that, the \texttt{Rect} object is created and used to cut out the bitmap with the help of \texttt{subsurface()}\myindex{pyg}{\texttt{Surface}!\texttt{subsurface()}}. This extracted bitmap is then stored in the dictionary under its symbol name.

\newpage
\lstinputlisting[caption={Textbitmaps (4): \texttt{create\_letter\_bitmap()} von \texttt{Letters}},label={srcTextbitmaps00d},firstline={29},lastline={85},linerange={29-33,66-86}, consecutivenumbers=false]{SRC/00 Introduction/07 Fonts/textbitmaps/textbitmaps.py}


The method \texttt{get\_text()} finally returns the matching sequence of bitmap sprites for a given text. To do this, it uses the method \texttt{get\_letter()}, which is necessary so that the program does not crash when an undefined letter or symbol is used. For example, if you type an \texttt{ü}, a square placeholder will be displayed.


\lstsource{SRC/00 Introduction/07 Fonts/textbitmaps/textbitmaps.py}{87}{100}{python}{Textbitmaps (5), \texttt{get\_letter()} and \texttt{get\_text()} von \texttt{Letters}}{srcTextbitmaps00e} 


The actual main program is encapsulated in the class \texttt{TextBitmaps}. Since the source code does not introduce anything fundamentally new, it should be largely self-explanatory. However, I would like to take a closer look at two specific lines:

\begin{itemize}
	\item \zeiref{srcTextbitmaps0008}: Here, \gls{slicing} of \glspl{array} is used. The value~\texttt{-1} causes the end index of the slice to start at the last element and then move one step to the left. The result is a new string that is shortened by its last character.
	
	\item \zeiref{srcTextbitmaps0009}: The attribute \texttt{unicode}\myindex{pyg}{\texttt{event}!\texttt{Event}!\texttt{unicode}}\randnotiz{unicode} provides, where applicable, the value of the pressed key in \gls{unicode} format. This means that meaningful letters, digits, and similar characters are added
	directly to the string.
\end{itemize}

\lstsource{SRC/00 Introduction/07 Fonts/textbitmaps/textbitmaps.py}{103}{139}{python}{Textbitmaps (6), \texttt{TextBitmaps}}{srcTextbitmaps00f}


%%%%%%%%%%%%%%%%%%%%%%%%%%%%%%%%%%%%%%%%%%%%%%%%%%%%%%%%%%%%%%%%%%%%%%%%%%%
\newpage
\subsection{What was new?}
\begin{hintbox}
	\begin{itemize}
		\item To produce text output, you can use either system-installed fonts or local font files. In the first step, a suitable font object is created. In the second step, this object is used to render a given text into a bitmap --- a \texttt{Surface} object. This bitmap can then be blitted to the desired position just like any other \texttt{Surface} object.
		
		\item But Text output is not only created using fonts, but also by means of spritelibs that contain character bitmaps. These bitmaps are cut out and then assembled into new composite bitmaps.
	\end{itemize}
\end{hintbox}

\begin{pygbox}
\begin{itemize}
	\item \texttt{pygame.event.Event.unicode}:
	\myindex{pyg}{\texttt{event}!\texttt{Event}!\texttt{unicode}}\\ \url{https://pyga.me/docs/ref/event.html}
	
	\item \texttt{pygame.font.Font}:
	\myindex{pyg}{\texttt{font}!\texttt{Font}}\\ \url{https://pyga.me/docs/ref/font.html}
	
	\item \texttt{pygame.font.get\_default\_font()}:
	\myindex{pyg}{\texttt{font}!\texttt{get\_default\_font()}}\\ \url{https://pyga.me/docs/ref/font.html#pygame.font.get_default_font}
	
	\item \texttt{pygame.font.get\_fonts()}:
	\myindex{pyg}{\texttt{font}!\texttt{get\_fonts()}}\\ \url{https://pyga.me/docs/ref/font.html#pygame.font.get_fonts}
	
	\item \texttt{pygame.font.match\_font()}:
	\myindex{pyg}{\texttt{font}!\texttt{match\_font()}}\\
	\url{https://pyga.me/docs/ref/font.html#pygame.font.match_font}
	
	\item \texttt{pygame.font.Font.render()}:
	\myindex{pyg}{\texttt{font}!\texttt{Font}!\texttt{render()}}\\ \url{https://pyga.me/docs/ref/font.html#pygame.font.Font.render}
	
	\texttt{pygame.Surface.subsurface()}:
	\myindex{pyg}{\texttt{Surface}!\texttt{subsurface()}}\\ \url{https://pyga.me/docs/ref/surface.html#pygame.Surface.subsurface}
	
	\item \texttt{pygame.Surface.subsurface()}:
	\myindex{pyg}{\texttt{Surface}!\texttt{subsurface()}}\\ \url{https://pyga.me/docs/ref/surface.html#pygame.Surface.subsurface}
\end{itemize}
\end{pygbox}


%%%%%%%%%%%%%%%%%%%%%%%%%%%%%%%%%%%%%%%%%%%%%%%%%%%%%%%%%%%%%%%%%%%%%%%%%%%
\newpage
\subsection{Homework}

\begin{wrapfigure}[13]{r}{6.5cm}%
	\vspace{-1em}%
	\myfigure{clock.png}{0.4}{Clock}{picClock00}%
\end{wrapfigure}Create a program that simulates a analog clock. The program contains the following status variables: number of lives, score, high score, the number of seconds since the program started, and a game title. Please position the information inside the window as follows:
	
	\begin{itemize}
		\item Title line: on the left the current date, in the center the game title, and on the right the current time
		\item Status line at the very bottom: on the left the number of lives, in the center score/high score, and on the right the number of seconds since the game started
	\end{itemize}
	
The values should update dynamically. The score increases by~1 every second. This score is written to the file \texttt{highscore.txt} if it is greater than the current high score. Date, time, and elapsed time are obtained from suitable functions of the \texttt{time} module. Draw an analog clock in the center with second hand and dial like in \abbref{picClock00}.




	% !TeX spellcheck = en_US
\newpage
%%%%%%%%%%%%%%%%%%%%%%%%%%%%%%%%%%%%%%%%%%%%%%%%%%%%%%%%%%%%%%%%%%%%%%%%%%%
\section{Collision Detection}\index{Collision}
%%%%%%%%%%%%%%%%%%%%%%%%%%%%%%%%%%%%%%%%%%%%%%%%%%%%%%%%%%%%%%%%%%%%%%%%%%%
\subsection{Introduction}
Collision detection is used very often in game programming: characters must not walk through obstacles, projectiles hit targets, balls bounce off walls, and so on. For this reason, Pygame provides a whole variety of collision detection methods:

\begin{itemize}
	\item \textbf{Rectangle overlap}\index{collision detection!rectangle}: When we looked at the \texttt{Sprite} class, we already saw that the attribute \texttt{rect} is required.	It contains the position and size of the surrounding rectangle.	If two sprites meet, it is checked whether their rectangles overlap.	This is a very \emph{cheap} detection method, because only a few comparisons are needed to decide whether two rectangles touch or overlap. This method does not consider the actual shape of the sprite, only its bounding rectangle. Here is an example implementation:
	\lstset{firstnumber=1}
\begin{lstlisting}
def rectangle_collision(rect1, rect2):
   return rect1.left < rect2.right and
          rect2.left < rect1.right and
          rect1.top < rect2.bottom and
          rect2.top < rect1.bottom
\end{lstlisting}

\begin{figure}[H]
\begin{center}
\tikzset{
    shape rechteck/.style= {
    draw,
    line width = 1pt,
    inner xsep = 0.0cm,
    inner ysep = 0.0cm,
   }
}
\begin{tikzpicture}
\tiny
\draw [->, name=xachse] (0cm, 6cm)  -- +(13cm, 0cm);
\draw [<-, name=yachse] (0cm, 0cm)  -- +(0cm, 6cm);

\pgfsetfillopacity{0.5}
\draw (4.0cm, 3.0cm) node[name=k1,shape=rectangle,shape rechteck, fill = yellow!30, minimum height = 4.0cm, minimum width = 3cm] {};
\draw (6.5cm, 2.0cm) node[name=k2,shape=rectangle,shape rechteck, fill = green!30, minimum height = 3.5cm, minimum width = 5cm] {};
\pgfsetfillopacity{1.0}

\draw[-, very thick, red, dotted]  let \p1 = (k1.north west) in (k1.north west) --  (\x1, 6.0cm);
\draw[-, very thick, red, dotted]  let \p1 = (k1.north west) in (k1.north west) --  (0cm, \y1);
\draw[-, very thick, blue, dotted] let \p1 = (k1.north east) in (k1.north east) --  (\x1, 6.0cm);
\draw[-, very thick, blue, dotted] let \p1 = (k1.south west) in (k1.south west) --  (0cm,\y1);

\draw[-, very thick, red, dotted]  let \p1 = (k2.north west) in (k2.north west) --  (\x1, 6.0cm);
\draw[-, very thick, red, dotted]  let \p1 = (k2.north west) in (k2.north west) --  (0cm, \y1);
\draw[-, very thick, blue, dotted] let \p1 = (k2.north east) in (k2.north east) --  (\x1, 6.0cm);
\draw[-, very thick, blue, dotted] let \p1 = (k2.south west) in (k2.south west) --  (0cm,\y1);

\path [name=x1, color=red] let \p1 = (k1.north west) in node  at (\x1,6.4cm) {$left_1$};
\path [name=x2, color=red] let \p1 = (k2.north west) in node  at (\x1,6.4cm) {$left_2$};
\path [name=x1, color=blue] let \p1 = (k1.north east) in node  at (\x1,6.4cm) {$right_1$};
\path [name=x2, color=blue] let \p1 = (k2.north east) in node  at (\x1,6.4cm) {$right_2$};
\path [name=y1, color=red] let \p1 = (k1.north west) in node  at (-0.5cm,\y1) {$top_1$};
\path [name=y2, color=red] let \p1 = (k2.north west) in node  at (-0.5cm,\y1) {$top_2$};
\path [name=y1, color=blue] let \p1 = (k1.south west) in node  at (-0.9cm,\y1) {$bottom_1$};
\path [name=y2, color=blue] let \p1 = (k2.south west) in node  at (-0.9cm,\y1) {$bottom_2$};
\end{tikzpicture}
\caption{Collision detection with rectangles}\label{picKollRect01}
\end{center}
\end{figure}


	\item \textbf{Circle overlap}\index{collision detection!circle}: For rather round sprites, it is recommended not to check rectangles, but to use an bounding circle for collision detection instead. This collision test is also quite fast, because only the distance between the centers has to be compared:  $\sqrt{(x_2-x_1)^2+(y_2-y_1)^2} < r_1+r_2$. For performance reason the check is usually computed as: $(x_2-x_1)^2+(y_2-y_1)^2 < (r_1+r_2)^2$

\begin{figure}[H]
\begin{center}
\tikzset{
    shape kreis/.style= {
    draw,
    fill = yellow!30,
    line width = 1pt,
    inner xsep = 0.0cm,
    inner ysep = 0.0cm,
   }
}
\begin{tikzpicture}
\draw [->, name=xachse] (0cm, 0cm)  -- +(13cm, 0cm);
\draw [->, name=yachse] (0cm, 0cm)  -- +(0cm, 6cm);

\draw (4.0cm, 3.5cm) node[name=k1,shape=circle,shape kreis,  minimum height = 4cm] {};
\draw (8.5cm, 2.5cm) node[name=k2,shape=circle,shape kreis,  minimum height = 3cm] {};

\draw[-, very thick, blue] 
 (k1.north west) --  node[above, blue, xshift=0cm] {$r_1$} (k1.center);
\draw[-, very thick, blue] 
 (k2.north east) --  node[above, blue, xshift=0cm] {$r_2$} (k2.center);

\draw[-, very thick, blue] 
 (k1.center) --  node[above, blue, sloped, xshift=0cm] {\footnotesize$\sqrt{(x_2-x_1)^2+(y_2-y_1)^2}$} (k2.center);

\draw[-, very thick, red, dotted] 
 (k1.center) --  +(0cm, -3.5cm);
\draw[-, very thick, red, dotted] 
 (k1.center) --  +(-4.0cm, 0cm);

\draw[-, very thick, red, dotted] 
 (k2.center) --  +(0cm, -2.5cm);
\draw[-, very thick, red, dotted] 
 (k2.center) --  +(-8.5cm, 0cm);

\path [name=x1, color=red] let \p1 = (k1) in node  at (\x1,-0.4cm) {$x_1$};
\path [name=x2, color=red] let \p1 = (k2) in node  at (\x1,-0.4cm) {$x_2$};
\path [name=y1, color=red] let \p1 = (k1) in node  at (-0.4cm,\y1) {$y_1$};
\path [name=y2, color=red] let \p1 = (k2) in node  at (-0.4cm,\y1) {$y_2$};
\end{tikzpicture}
\caption{Collision detection with circles}\label{picKollKreis01}
\end{center}
\end{figure}

	\item \textbf{Pixel overlap}\index{collision detection!pixel}: In pixel-perfect collision detection, every pixel of both sprites is checked to see whether they occupy the same position.  If \emph{yes}, the sprites overlap; if \emph{no}, they do not.  This is the most expensive collision test, but also the most accurate one.  

	To reduce the computational effort, the intersection rectangle of the two sprites is determined first.  As with rectangle collision detection, it is first checked whether the two rectangles overlap at all. If they do not, the test can stop immediately.  If they do, the intersection of the two rectangles is itself a rectangle (see \abbref[vref]{picKollMask01}).  

\begin{figure}[H]
	\begin{center}
		\begin{tikzpicture}[font=\small, x=0.55cm, y=0.55cm]
			
			\tikzset{
				px/.style={draw, very thin, minimum width=0.55cm, minimum height=0.55cm},
				one/.style={px, fill=white},
				zero/.style={px, fill=black!12},
				ma/.style={px, fill=blue!20},
				mb/.style={px, fill=red!20},
				mab/.style={px, fill=green!20},
				title/.style={align=center, font=\bfseries}
			}
			
			% ---------------------------
			% Helper macro: draw a 6x6 mask with a set of "1" pixels
			% (#1 shift x, #2 shift y, #3 node prefix, #4 list of x/y ones)
			% ---------------------------
			\newcommand{\drawmask}[6]{%
				\begin{scope}[shift={(#1,#2)}]
					% grid 6x6: coords x=0..5, y=0..5, drawn top-down
					\foreach \x in {0,...,#6} {
						\foreach \y in {0,...,#6} {
							\node[zero] (#3-\x-\y) at (\x,-\y) {};
						}
					}
					% set ones
					\foreach \x/\y in {#5} {
						\node[#4] at (\x,-\y) {};
					}
				\end{scope}
			}
			
			% ---------------------------
			% Two masks (A and B)
			% ---------------------------
			\node[title] at (2.5,1.4) {Mask A};
			\node[title] at (11.5,1.4) {Mask B};
			\node[title] at (20.5,1.4) {Mask A $\cap$ Mask B};
			
			% Mask A at (0,0)
			\drawmask{0}{0}{A}{one}{
				2/1,3/1,
				1/2,2/2,3/2,4/2,
				1/3,2/3,3/3,4/3,
				2/4,3/4
			}{5}
			
			% Mask B 
			\drawmask{9}{0}{B}{one}{
				0/0,
				0/1,1/1,
				0/2,1/2,2/2,
				0/3,1/3,2/3,3/3,
				0/4,1/4,2/4,3/4,4/4,
				0/5,1/5,2/5,3/5,4/5,5/5
			}{5}
			
			% Mask A intersect Mask B
			\drawmask{18}{0}{A}{ma}{
				2/1,3/1,
				1/2,2/2,3/2,4/2,
				1/3,2/3,3/3,4/3,
				2/4,3/4
			}{5}
			\drawmask{20}{-2}{B}{mb}{
				0/0,
				0/1,1/1,
				0/2,1/2,2/2,
				0/3,1/3,2/3,3/3,
				0/4,1/4,2/4,3/4,4/4,
				0/5,1/5,2/5,3/5,4/5,5/5
			}{5}	
			\drawmask{20}{-2}{B}{mab}{
				0/0,
				0/1,1/1,
				0/2,1/2
			}{2}
			\begin{scope}[shift={(18,0)}]
				\foreach \x/\y in {2/1,3/1,1/2,2/2,3/2,4/2,1/3,2/3,3/3,4/3,2/4,3/4} 
				{
					\node[ma] at (\x,-\y) {};
				}
			\end{scope}
			\begin{scope}[shift={(20,-2)}]
				\foreach \x/\y in {0/0,0/1,1/1,0/2,1/2,2/2,0/3,1/3,2/3,3/3,0/4,1/4,2/4,3/4,4/4,0/5,1/5,2/5,3/5,4/5,5/5} 
				{
					\node[mb] at (\x,-\y) {};
				}
			\end{scope}
			\begin{scope}[shift={(20,-2)}]
				\foreach \x/\y in {0/0,0/1,1/1,0/2,1/2} 
				{
					\node[mab] at (\x,-\y) {};
				}
				\draw[very thick] (-0.5,0.5) rectangle (3.5,-3.5);
				\node[anchor=west] at (0, -6.0) {\small \textcolor{blue!80}{A has 1 here}};
				\node[anchor=west] at (0, -7.0) {\small \textcolor{red!80}{B has 1 here}};
				\node[anchor=west] at (0, -8.0) {\small \textcolor{green!80!black}{collision pixels (A AND B)}};
			\end{scope}
			
		\end{tikzpicture}
		\caption{Collision detection using masks}\label{picKollMask01}
	\end{center}
\end{figure}

	If two pixels have the same position, they must lie inside this intersection rectangle. Therefore, the pixel test can be limited to this usually much smaller area (see \abbref[vref]{picKollMask02}).  

	Another problem with pixel-perfect collision detection is distinguishing background from foreground. How should the collision test know whether a blue pixel belongs to the object or to the background?  There are several approaches to this problem.  The simplest one is to create a black-and-white image for each sprite (a \gls{mask}\index{mask}\randnotiz{mask});  the white pixels are relevant, while the black pixels can be ignored.  The pixel collision test is then performed only on these masks.

\begin{figure}[H]
	\begin{center}
		\begin{tikzpicture}[font=\small, x=0.55cm, y=0.55cm]
			
			% ---------- Parameters ----------
			\def\N{8} % grid size (NxN)
			
			% ---------- Titles ----------
			\node[align=center] (t1) at (3.5,1.4) {\textbf{Sprite (pixels)}};
			\node[align=center] (t2) at (19.5,1.4) {\textbf{Mask (relevant / irrelevant)}};
			
			% ---------- Helper styles ----------
			\tikzset{
				px/.style={draw, very thin, minimum width=0.55cm, minimum height=0.55cm},
				lbl/.style={align=center}
			}
			
			% ---------- SPRITE GRID (left) ----------
			% Draw an 8x8 pixel grid. We'll "paint" a circle-ish object.
			\begin{scope}[shift={(0,0)}]
				% Base grid
				\foreach \x in {0,...,7} {
					\foreach \y in {0,...,7} {
						\node[px, fill=white] (S-\x-\y) at (\x, -\y) {};
					}
				}
				
				% "Object" pixels (colored) - a simple round-ish blob
				\foreach \x/\y in {
					3/1,4/1,
					2/2,5/2,
					1/3,6/3,
					1/4,6/4,
					2/5,5/5,
					3/6,4/6
				}{
					\node[px, fill=blue!30] at (\x, -\y) {};
				}
				\foreach \x/\y in {
					3/2,4/2,
					2/3,5/3,
					2/4,5/4,
					3/5,4/5
				}{
					\node[px, fill=green!30] at (\x, -\y) {};
				}
				\foreach \x/\y in {
					3/3,4/3,
					3/4,4/4
				}{
					\node[px, fill=red!30] at (\x, -\y) {};
				}
				
				% Label note
				\node[lbl] at (3.5, -9.2) {Colored pixels are the visible sprite.\\White cells are background / transparent.};
			\end{scope}
			
			% ---------- MASK GRID (right) ----------
			\begin{scope}[shift={(16,0)}]
				% Base grid
				\foreach \x in {0,...,7} {
					\foreach \y in {0,...,7} {
						\node[px, fill=black!10] (M-\x-\y) at (\x, -\y) {};
					}
				}
				
				% Relevant pixels (white = 1)
				\foreach \x/\y in {
					3/1,4/1,
					2/2,3/2,4/2,5/2,
					1/3,2/3,3/3,4/3,5/3,6/3,
					1/4,2/4,3/4,4/4,5/4,6/4,
					2/5,3/5,4/5,5/5,
					3/6,4/6
				}{
					\node[px, fill=white] at (\x, -\y) {};
				}
				
				% Label note
				\node[lbl] at (3.5, -9.2) {White = relevant pixel (1)\\Gray/black = irrelevant pixel (0)};
			\end{scope}
			
			% ---------- Arrow between sprite and mask ----------
			\draw[->, thick] (8.0,-3.5) -- node[above]{mask.from\_surface()} (15.0,-3.5);
			
			% ---------- Bit encoding explanation (bottom) ----------
			% We show one row of 8 pixels and the corresponding 8 bits -> 1 byte.
			\begin{scope}[shift={(2,-11.5)}]
				\node[anchor=west] at (0,0.8) {\textbf{Bitwise storage example (one row = 8 pixels = 1 byte)}};
				
				% Draw 8 "mask pixels" for a row example
				\foreach \i/\bit in {0/0,1/0,2/1,3/1,4/1,5/1,6/0,7/0} {
					\node[px, fill={\ifnum\bit=1 white\else black!10\fi}] (R\i) at (\i,0) {};
					\node at (\i,-0.9) {\ttfamily\bit};
				}
				
				\node[anchor=west] at (9,0.0) {\ttfamily 00111100};
				
				\draw[->, thick] (7.8,0) -- (8.6,0);
				\node[anchor=west] at (9,-0.9) {8 bits $\Rightarrow$ 1 byte};
			\end{scope}
			
		\end{tikzpicture}
		\caption{From sprite to mask}\label{picKollMask02}
	\end{center}
\end{figure}


\end{itemize}


Let us look at the collision detection behaviour in more detail. In \abbref[vref]{piccollision00} we see four sprites: a wall, a spaceship, a monster, and a projectile. None of the sprites are touching each other.

In \abbref[vref]{piccollision01} you can clearly see the effect of collision detection using the bounding rectangles (bounding boxes). For the wall, everything is perfect: the projectile hits the wall, and the color indicates that the program has detected the collision.

However, we can also see the disadvantage when looking at the spaceship. A collision is detected even though the two sprites do not actually touch. The reason is that the spaceship’s bounding rectangle also includes the empty corners, so the rectangles overlap and a collision is reported. The same effect can also be observed with the monster.
 

\myezweihbild%
    {collision00.png}{0.38}{Four sprites\\\ }{piccollision00}%
    {collision01.png}{0.38}{Collision detection using rectangles (montage)}{piccollision01}


The situation is different when we use collision detection based on bounding circles (\abbref[vref]{piccollision02}). Now the collision with the wall is no longer detected correctly, because the corners of the wall do not belong to the inner circle. For the spaceship, however, this method produces exactly the desired result, since the empty corners are not part of the bounding circle. If we move a little further to the right, the spaceship would also turn red, because a collision would then be detected. The monster still produces an incorrect result.

Finally, there is pixel-perfect collision detection (\abbref[vref]{piccollision03}).  The collision with the wall is detected correctly. Even more interesting are the results for the spaceship and the monster. Both correctly report no collision, because the projectile is inside the rectangle and the inner circle, but only on transparent pixels. Feel free to try it yourself: move the projectile slightly to the left or right, and you will immediately see the pixel-perfect collision detection in action through the color change.

\myezweihbild%
    {collision02.png}{0.38}{Collision detection using circles (montage)}{piccollision02}%
    {collision03.png}{0.38}{Collision detection using masks (montage)}{piccollision03}


%%%%%%%%%%%%%%%%%%%%%%%%%%%%%%%%%%%%%%%%%%%%%%%%%%%%%%%%%%%%%%%%%%%%%%%%%%%
\subsection{More Input}
%%%%%%%%%%%%%%%%%%%%%%%%%%%%%%%%%%%%%%%%%%%%%%%%%%%%%%%%%%%%%%%%%%%%%%%%%%%
\subsubsection{Three Types of Collision Detection (of a Bullet)}
\begin{diskbox}
	\url{https://github.com/adamsralf/pygame_book/tree/main/src/00%20Introduction/08%20Collision/v01}
\end{diskbox}

Let us now take a closer look at the corresponding source code. However, I will skip another discussion of the \texttt{config.py}.

\lstsource{SRC/00 Introduction/08 Collision/v01/config.py}{1}{99}{python}{Collision types (1), \texttt{config.py}}{srcCollision00a} 

Things become more interesting with the \texttt{Obstacle} class. This class is used for the wall, the spaceship, and the monster.  For rectangle-based collision detection, the surrounding rectangle is required. As usual, it is obtained in \zeiref{srcCollision01} using \texttt{pygame.Surface.get\-\_rect()}\myindex{pyg}{\texttt{Surface}!\texttt{get\_rect()}}\randnotiz{get\_rect()} and stored in the attribute \texttt{rect}\index{self.rect}\randnotiz{self.rect}.  

For sprites with implicit transparency or explicit transparency set via \texttt{set\_colorkey()}\myindex{pyg}{\texttt{Surface}!\texttt{set\_colorkey()}}, the mask can be created very easily using \texttt{pygame.mask.from\_surface()}\myindex{pyg}{\texttt{mask}!\texttt{from\_surface()}}\randnotiz{from\_surface()} (\zeiref{srcCollision02}). In order for the predefined collision detection functions to work, this mask must be stored in the \texttt{Sprite} object using the attribute \texttt{mask}\index{self.mask}\randnotiz{self.mask}.  

In \zeiref{srcCollision03}, the bounding radius is calculated. This is implemented in a somewhat unclean way. Strictly speaking, one should determine the minimum of width and height and divide it by two. 

\lstset{firstnumber=15}
\begin{lstlisting}
self.radius = min(self.rect.width, self.rect.height) // 2
\end{lstlisting}

As with the mask, the radius must also be stored in an attribute so that the predefined collision methods can work: \texttt{radius}\index{self.radius|underline}\randnotiz{self.radius}.  

The flag \texttt{hit} is only used to ensure that the correct image is displayed depending on the detected collision. As you have probably already noticed, two images are loaded for these sprites: one for the \emph{not hit} state and one for the \emph{hit} state.

\lstsource{SRC/00 Introduction/08 Collision/v01/collision.py}{8}{24}{python}{Collision types (2), \texttt{Obstacle}}{srcCollision00b} 

The \texttt{Bullet} class is similar in many ways to the \texttt{Obstacle} class. Since we also want to use this class for all three types of collision detection, we need the same three attributes here as well: \texttt{rect}, \texttt{radius}, and \texttt{mask}.  

In addition, the class contains a few lines of code to allow the bullet to move; this should be self-explanatory. Note: for the sake of simplicity, no boundary check has been implemented. There is no real need for it here.


\lstsource{SRC/00 Introduction/08 Collision/v01/collision.py}{27}{48}{python}{Collision types (3), \texttt{Bullet}}{srcCollision00c} 

And now the \texttt{Game} class. In the constructor, the usual things happen. There is nothing particularly noteworthy here.

\lstsource{SRC/00 Introduction/08 Collision/v01/collision.py}{51}{65}{python}{Collision types (4), Constructor of \texttt{Game}}{srcCollision00d} 

The methods \texttt{run()} and \texttt{watch\_for\_events()} also follow well-established patterns.

\lstsource{SRC/00 Introduction/08 Collision/v01/collision.py}{67}{99}{python}{Collision types (5), \texttt{run()} and \texttt{watch\_for\_events()} of \texttt{Game}}{srcCollision00e} 

The same applies to the methods \texttt{update()} and \texttt{draw()}.

\lstsource{SRC/00 Introduction/08 Collision/v01/collision.py}{101}{112}{python}{Collision types (6), \texttt{update()} and \texttt{draw()} of \texttt{Game}}{srcCollision00f} 

The method \texttt{resize()} is not related to collision detection itself. Its only purpose is to ensure that the \texttt{Obstacle} objects are distributed evenly across the width of the window.  

The first \forSchleife\ calculates the total width of all \texttt{Obstacle} objects. This information is needed to compute the spacing in \zeiref{srcCollision04}. To do this, the total obstacle width is subtracted from the window width. The remaining number of pixels can then be distributed across the gaps.  

And how many gaps do we have? There are two gaps between the three \texttt{Obstacle} objects, one gap to the left border, and one to the right border -- a total of four gaps. The resulting spacing is stored in \texttt{padding}.  

In the second \forSchleife, the left position of each \texttt{Obstacle} object can then be calculated and set accordingly.


\lstsource{SRC/00 Introduction/08 Collision/v01/collision.py}{114}{123}{python}{Collision types (7), \texttt{resize()} of \texttt{Game}}{srcCollision00g} 

\begin{hintbox} [Drum roll]
And now the actual collision detection. Depending on which collision method we have selected, the corresponding collision function is called inside the \forSchleife: \texttt{pygame.sprite.col\-lide\-\_circle()}\myindex{pyg}{\texttt{sprite}!\texttt{collide\_circle()}|underline},  
\texttt{pygame.sprite.col\-lide\-\_mask()}\myindex{pyg}{\texttt{sprite}!\texttt{collide\_mask()}}, or  
\texttt{pygame.sprite.col\-lide\-\_rect()}\myindex{pyg}{\texttt{sprite}!\texttt{collide\_rect()}}.  
\end{hintbox}
 
The semantics are actually quite simple. Each of these methods is given two \texttt{Sprite} objects and returns \true\ if a collision is detected; otherwise it returns \false. As already mentioned above, it is important to ensure that the method being used can find the information it requires in the sprite:

\begin{hintbox}[Required information]
	\begin{itemize}
	    \item \texttt{pygame.sprite.collide\_circle()} needs \texttt{self.radius}
	    \item \texttt{pygame.sprite.collide\_mask()} needs \texttt{self.mask}
	    \item \texttt{pygame.sprite.collide\_rect()} needs \texttt{self.rect}
	\end{itemize}
\end{hintbox}

\newpage
\lstsource{SRC/00 Introduction/08 Collision/v01/collision.py}{125}{134}{python}{Collision types (8), \texttt{check\_for\_collision()} of \texttt{Game}}{srcCollision00h} 

%%%%%%%%%%%%%%%%%%%%%%%%%%%%%%%%%%%%%%%%%%%%%%%%%%%%%%%%%%%%%%%%%%%%%%%%%%%
\subsubsection{Checking all Sprites in a List}
\begin{diskbox}
	\url{https://github.com/adamsralf/pygame_book/tree/main/src/00%20Introduction/08%20Collision/v03}
\end{diskbox}

Rectangle-based collision detection between a single sprite and a list of sprites -- that is, checking whether one sprite collides with any sprite in a \texttt{SpriteGroup} -- is used so often that a dedicated method exists for this purpose: \texttt{pygame.sprite.spritecollide()}\myindex{pyg}{\texttt{sprite}!\texttt{spritecollide()}}\randnotiz{spritecollide()}.  

The first parameter is a single \texttt{Sprite} object -- in this case, our fireball. The second parameter is the list of sprites in which a collision should be checked. The third parameter controls whether the colliding objects should be removed from the list. This is very useful, for example, when an obstacle should disappear upon contact.

Below a minimal example (printed only partially). In \zeiref{srcCollision05a} the key part happens: one sprite -- \texttt{player} -- is checked for collisions with many sprites -- \texttt{blocks}. Every sprite that the player collides with is removed from all groups using \texttt{kill()}, and is therefore most likely deleted completely.

\lstsource{SRC/00 Introduction/08 Collision/v03/collision.py}{31}{62}{python}{\texttt{spritecollide()}}{srcCollision00i} 


%%%%%%%%%%%%%%%%%%%%%%%%%%%%%%%%%%%%%%%%%%%%%%%%%%%%%%%%%%%%%%%%%%%%%%%%%%%
\subsubsection{Using Function Pointer/Collision Callback}
\begin{diskbox}
	\url{https://github.com/adamsralf/pygame_book/tree/main/src/00%20Introduction/08%20Collision/v02}
\end{diskbox}

The method \texttt{pygame.sprite.spritecollide()} has a fourth parameter as well. This parameter can be used to pass a \gls{functionpointer}\index{function pointer}\randnotiz{function pointer} or \emph{collision callback}\index{collision callback}\randnotiz{collision callback} to a different collision detection method. This function must accept two \texttt{Sprite} objects as parameters.  

This means you can either use your own custom collision function or one of the three predefined methods: \texttt{collide\_circle()}, \texttt{collide\_mask()}, or \texttt{collide\_rect()}. If nothing is specified here -- as in our source code -- \texttt{collide\_rect()} is used automatically.

\lstsource{SRC/00 Introduction/08 Collision/v02/collision.py}{125}{135}{python}{Dynamic collision callback}{srcCollision01a} 


%%%%%%%%%%%%%%%%%%%%%%%%%%%%%%%%%%%%%%%%%%%%%%%%%%%%%%%%%%%%%%%%%%%%%%%%%%%
\newpage
\subsection{What was new?}

\begin{hintbox}
There are three standard ways to test the collision of two sprites: 
\begin{itemize}
	\item checking whether their rectangles intersect, 
	\item whether their bounding circles intersect, or 
	\item whether the pixels of the objects overlap.
\end{itemize}

In order to perform these collision tests, a sprite must provide the required information: \texttt{rect}, \texttt{radius}, or \texttt{mask}.
\end{hintbox}

\begin{figure}[H]
	\begin{center}
		\begin{tikzpicture}[>=Stealth, font=\small]
			
			% Axes
			\draw[->, thick] (0,0) -- (10,0) node[below right] {Cost};
			\draw[->, thick] (0,0) -- (0,7)  node[above left] {Accuracy};
			
			% Axis hints
			\node[below] at (2.0,0) {high};
			\node[below] at (8.0,0) {low};
			\node[left]  at (0,1.0) {low};
			\node[left]  at (0,6.5) {high};	
			
			% Rect (fast, inaccurate)
			\node[draw, rounded corners, align=left, fill=gray!10] (rect) at (7.8,1.3) {%
				\textbf{Rect}\\
				\texttt{collide\_rect}\\
				+ fastest\\
				-- sometimes improper 
			};
			
			% Circle (medium)
			\node[draw, rounded corners, align=left, fill=gray!10] (circle) at (5.3,3.6) {%
				\textbf{Circle}\\
				\texttt{collide\_circle}\\
				+ fast\\
				-- improper in corners
			};
			
			% Mask (accurate, expensive)
			\node[draw, rounded corners, align=left, fill=gray!10] (mask) at (2.5,5.8) {%
				\textbf{Mask}\\
				\texttt{collide\_mask}\\
				+ super precise\\
				-- most expensive
			};
		\end{tikzpicture}
		\caption[Tradeoff Accuracy vs. Costs]{\Gls{tradeoff} Accuracy vs. Costs}\label{picKollVergleich01}
	\end{center}
\end{figure}

\begin{pygbox}
\begin{itemize}
	\item \texttt{pygame.mask.from\_surface()}:
	\myindex{pyg}{\texttt{mask}!\texttt{from\_surface()}}\\ 
    \url{https://pyga.me/docs/ref/mask.html#pygame.mask.from_surface}
	
	\item \texttt{pygame.sprite.collide\_circle()}:
	\myindex{pyg}{\texttt{sprite}!\texttt{collide\_circle()}}\\ 
    \url{https://pyga.me/docs/ref/sprite.html#pygame.sprite.collide_circle}

	\item \texttt{pygame.sprite.collide\_mask()}:
	\myindex{pyg}{\texttt{sprite}!\texttt{collide\_mask()}}\\ 
    \url{https://pyga.me/docs/ref/sprite.html#pygame.sprite.collide_mask}

	\item \texttt{pygame.sprite.collide\_rect()}:
	\myindex{pyg}{\texttt{sprite}!\texttt{collide\_rect()}}\\ 
    \url{https://pyga.me/docs/ref/sprite.html#pygame.sprite.collide_rect}
	
	\item \texttt{pygame.sprite.spritecollide()}:
	\myindex{pyg}{\texttt{sprite}!\texttt{spritecollide()}}\\ 
    \url{https://pyga.me/docs/ref/sprite.html#pygame.sprite.spritecollide}
	
\end{itemize}
\end{pygbox}

%%%%%%%%%%%%%%%%%%%%%%%%%%%%%%%%%%%%%%%%%%%%%%%%%%%%%%%%%%%%%%%%%%%%%%%%%%%
\subsection{Homework}

\myebild{collision04.png}{0.5}{A simple collision game}{piccollision04}%

Create a game with the following properties:

\begin{enumerate}
	\item A player sprite is displayed at the bottom center of the playfield.
	
	\item The player can move in all four directions using the keyboard, but cannot leave the playfield.
	
	\item Thirty obstacles are placed on the playfield.
	
	\item Ten obstacles each are detected using rectangle collision, circle collision, and mask-based collision detection.
	
	\item The obstacles are placed in such a way that they do not overlap with each other or with the player.
	
	\item The obstacles move downward at different speeds.
	
	\item There is a score counter that starts at~0. It is to display on the top edge of the window.
	
	\item If an obstacle leaves the playfield at the bottom, it reappears at the top and the score counter is increased by~1.
	
	\item If an obstacle hits the player, the game is lost.
	
	\item Bonus: Over time, the obstacles become faster and faster.
	
	\item Bonus: A Game Over message and a restart option.
\end{enumerate}



	% !TeX spellcheck = en_US
\newpage
%%%%%%%%%%%%%%%%%%%%%%%%%%%%%%%%%%%%%%%%%%%%%%%%%%%%%%%%%%%%%%%%%%%%%%%%%%%
\section{Time-based Actions}\index{Time-based}\label{secZeitstuerung}
%%%%%%%%%%%%%%%%%%%%%%%%%%%%%%%%%%%%%%%%%%%%%%%%%%%%%%%%%%%%%%%%%%%%%%%%%%%
\subsection{Introduction}
In games, time-based actions are needed in many situations: a bomb drops every half second, a shield is active for 10~seconds, after 3~jumps the \emph{jump} ability is not available for 5~minutes, animation frames should be displayed every $1/30$~second, and so on.

Let us first look at the screen output of \srcref[vref]{srcTime00a}ff. shown in \abbref[vref]{pictime00}. The fireballs are obviously released in very quick succession, so that they appear like a chain. Because the enemy is moving horizontally, this results in a slanted line -- which is clearly not the intended behaviour.

\myebild{time00.png}{0.8}{Fireball using frame-based movement}{pictime00}

Before we take a closer look at time control itself, let us briefly look at the program. The  \texttt{config.py} doesn't introduce anything new.


\lstsource{SRC/00 Introduction/09 Time/v01/config.py}{1}{99}{python}{Time-based actions (1), \texttt{config.py}}{srcTime00a} 

The \texttt{Enemy} class does not introduce anything particularly exciting either. With a spacing of \SI{10}{px}, the enemy continuously moves back and forth from left to right and vice versa.

\lstsource{SRC/00 Introduction/09 Time/v01/time.py}{9}{24}{python}{Time-based actions (2), Version 1.0: \texttt{Enemy}}{srcTime00b} 

The \texttt{Bullet} class is also largely a repetition of what we have seen before. What may be more interesting is \zeiref{srcTime0004}. The method \texttt{pygame.sprite.Sprite.kill()}\myindex{pyg}{\texttt{sprite}!\texttt{Sprite}!\texttt{kill()}|underline}\randnotiz{kill()} is not a true self-destruction mechanism. Instead, this method removes the \texttt{Sprite} object from all sprite groups. 

If all references to the object are lost as a result, the object is of course destroyed. However, if a reference still exists somewhere, the object will remain alive. In practice, \texttt{Sprite} objects are usually managed in groups (that is, in \texttt{pygame.sprite.Group} objects) and are therefore effectively destroyed by calling \texttt{kill()}.  

You can see this effect in \abbref[vref]{pictime00}: the fireball disappears about $30~px$ before reaching the bottom edge of the screen.

\lstsource{SRC/00 Introduction/09 Time/v01/time.py}{27}{40}{python}{Time-based actions (3), Version 1.0: \texttt{Bullet}}{srcTime00c} 

In the constructor of the \texttt{Game} class, a sprite group for the fireballs is created, as well as a \texttt{GroupSingle} object for the enemy.  In \texttt{run()}, the usual game loop tasks are carried out by calling the appropriate methods.  

I would like to briefly draw attention to \zeiref{srcTime0005}ff. By calling \texttt{pygame.time.Clock.tick()}\myindex{pyg}{\texttt{time}!\texttt{Clock}!\texttt{tick()}}\randnotiz{tick()}, the game loop is timed -- in this case to $1/60$ of a second -- and the \emph{delta time}\randnotiz{delta time}\index{delta time} is calculated afterwards.

\lstsource{SRC/00 Introduction/09 Time/v01/time.py}{43}{66}{python}{Time-based actions (4), Version 1.0: Constructor and \texttt{run()} of \texttt{Game}}{srcTime00d} 

The methods \texttt{watch\_for\_events()} and \texttt{draw()} also do not contain anything special.

\lstsource{SRC/00 Introduction/09 Time/v01/time.py}{68}{80}{python}{Time-based actions (5), Version 1.0:  \texttt{watch\_for\_events()} and \texttt{draw()} of \texttt{Game}}{srcTime00e} 

The \texttt{update()} method is only worth mentioning with regard to \zeiref{srcTime0006}, because a new fireball is created (dropped) there by calling the \texttt{new\_bullet()} method. The starting position is derived from the current position of the enemy. The horizontal center of the fireball and the enemy should be the same, while the vertical center is shifted slightly downward, which looks better visually.

\lstsource{SRC/00 Introduction/09 Time/v01/time.py}{82}{88}{python}{Time-based actions (6), Version 1.0:  \texttt{update()} and \texttt{new\_bullet()} of \texttt{Game}}{srcTime00f} 


Back to the actual problem. As we saw above, the application is timed to $1/60$ of a second by \texttt{FPS} and the call to \texttt{tick()} in \zeiref{srcTime0005}. In other words, up to 60~fireballs per second are currently created, which is nonsense.

A naive idea would be to reduce the frame rate. So, if I want to create a fireball only every half second, I would have to set the tick rate to~2. Try it! The result is disappointing: the entire game becomes slower. That is not the point.

A next -- and actually quite good -- idea is to introduce a counter. The idea is: if the tick rate is $1/60$, I count up to~30 and only then drop a fireball.

In the first step, two attributes are added to the \texttt{Game} class for this purpose (\zeiref{srcTime0101} and \zeiref{srcTime0102}).

\lstsource{SRC/00 Introduction/09 Time/v02/time.py}{52}{55}{python}{Time-based actions (7), Version 1.1: Konstruktor von \texttt{Game}}{srcTime01a} 


In the \texttt{new\_bullet()} method, these two values are now used to control the time interval between two drops. First, the counter is increased by~1 each time the method is called. Since the method is called once per iteration of the main program loop and each iteration is timed, this effectively counts the number of ticks.
 
If the counter exceeds its upper limit (30 in our example), half a second has passed since the last drop, and a new fireball is released.

Finally, the counter must be reset to~0, because we now have to wait for the next 30~ticks again. The result can be seen in \abbref[vref]{pictime01}: only two fireballs are visible now.

\lstsource{SRC/00 Introduction/09 Time/v02/time.py}{89}{93}{python}{Time-based actions (8), Version 1.1: \texttt{new\_bullet()} of \texttt{Game}}{srcTime01b} 

\myebild{time01.png}{0.8}{Fireball using counter-based movement}{pictime01}

The advantages of this approach are clear: it is easy to implement, and the speed of the game itself is not affected.

However, there is a decisive disadvantage: this approach only works if the tick rate does not change and always behaves as expected. In reality, this is not guaranteed. As we remember, calling \texttt{tick()} ensures that the loop is executed \emph{at most} 60~times per second. Under heavy load, it may run less often. In addition, many games determine the number of \emph{frames per second} dynamically in order to adapt to different hardware performance. Therefore, coupling time control to the tick rate is not a truly stable solution.

A better approach is to couple time control to a real time source. The method \texttt{pygame\-.time\-.get\_ticks()}\myindex{pyg}{\texttt{time}!\texttt{get\_ticks()}}\randnotiz{get\_ticks()} is very helpful here. This method returns the amount of time since the start of the game in \gls{ms}, and this value is independent of the performance of the hardware or the program.

Now the source code can be reworked. First, in \zeiref{srcTime0201}, the current number of~$ms$ since program start is measured, and in \zeiref{srcTime0202} it is defined how many~$ms$ a time interval should last. We want to drop a fireball every half second, so this value is~500.

\lstsource{SRC/00 Introduction/09 Time/v03/time.py}{51}{55}{python}{Time-based actions (9), Version 1.2: Construktor of \texttt{Game}}{srcTime02a} 

 After that, \texttt{new\_bullet()} checks whether the end of the interval has been reached. In \zeiref{srcTime0204}, the current time is measured again using \texttt{pygame.time.get\_ticks()}. If this value is greater than the previous interval start plus the interval duration -- which is the same as the interval end -- then \SI{500}{ms} have passed and a new fireball is dropped.  

Now only the new interval start has to be determined, which is done in \zeiref{srcTime0205}.

\lstsource{SRC/00 Introduction/09 Time/v03/time.py}{89}{92}{python}{Time-based actions (10), Version 1.2: \texttt{new\_bullet()} of \texttt{Game}}{srcTime02b} 


%%%%%%%%%%%%%%%%%%%%%%%%%%%%%%%%%%%%%%%%%%%%%%%%%%%%%%%%%%%%%%%%%%%%%%%%%%%
\subsection{More Input}

Note: Time control using events is introduced in \secref[vref]{eventtime}.

%%%%%%%%%%%%%%%%%%%%%%%%%%%%%%%%%%%%%%%%%%%%%%%%%%%%%%%%%%%%%%%%%%%%%%%%%%%
\subsubsection{The Class Timer}\label{secClassTimer}\index{time!class Timer}

Since we need this logic multiple times, I encapsulated it in the \texttt{Timer}~class\randnotiz{Class Timer}\index{Timer|underline}. The core of this class again consists of two attributes, which store the interval duration (\texttt{duration}) and the end of the interval (\texttt{next}). Unlike before, the interval start is no longer stored; instead, the interval end is saved -- which slightly reduces the required computation.

The optional parameter \texttt{with\_start} is particularly interesting. It allows us to control whether the code should wait until the first interval ends, or whether the very first call to \texttt{is\_next\_stop\_reached()} should already return \true.
   
What does this mean for our example? If \texttt{with\_start} is set to \true, the first fireball is dropped immediately in the very first loop iteration. If the value is \false, the first fireball is only dropped after \SI{500}{ms}.

In \texttt{is\_next\_stop\_reached()}, it is checked whether the end of the interval has been reached, and, if necessary, a new interval end is calculated. The method returns \true\ if the interval end has been reached or exceeded; otherwise, it returns \false.

\lstsource{SRC/00 Introduction/09 Time/v04/time.py}{9}{22}{python}{Time-based actions (11), Version 1.3: Class \texttt{Timer}}{srcTime03a} 

How is this timer used now?  First, an appropriate object is created in the constructor (\zeiref{srcTime0301}); the two variables used previously are no longer needed.

\lstsource{SRC/00 Introduction/09 Time/v04/time.py}{68}{70}{python}{Time-based actions (12), Version 1.3: creating a \texttt{Timer} object}{srcTime03b} 

The \texttt{new\_bullet()} method has now become simpler, since it no longer has to take care of the internal timer logic. It only checks in \zeiref{srcTime0302} whether the interval end has been reached -- and that’s it!


\lstsource{SRC/00 Introduction/09 Time/v04/time.py}{104}{106}{python}{Time-based actions (13), Version 1.3: Using a \texttt{Timer} object verwenden}{srcTime03c} 

For the sake of comparability -- which may have been a bit confusing just now -- I have placed the four workflows side by side in \abbref[vref]{picTimeVergleich01}.

\begin{figure}[H]
	\begin{center}\begin{tikzpicture}[
	font=\small,
	>=latex,
	line/.style={->, thick},
	block/.style={rectangle, draw, rounded corners, align=center, minimum width=3.0cm, minimum height=8mm},
	decision/.style={diamond, draw, aspect=2.2, align=center, inner sep=1pt, minimum width=3.0cm, minimum height=8mm},
	title/.style={rectangle, draw, fill=gray!10, rounded corners, align=center, minimum width=5.2cm, minimum height=7mm},
	note/.style={align=left, text width=5.0cm}
	]
	
	% --- Helper: base loop blocks as nodes inside each panel ---
	% Panel 1: time00 ---------------------------------------------------------
	\begin{scope}[shift={(0,0)}, scale=0.7, transform shape]
		\node[title] (t00) {time00: Spawn \textbf{every frame} (no timer)};
		\node[block, below=6mm of t00] (t00_events) {Events (QUIT, ESC, input)};
		\node[block, below=4mm of t00_events] (t00_update) {Update};
		\node[block, below=4mm of t00_update] (t00_spawn) {Spawn bullet\\\textbf{always} in update()};
		\node[block, below=4mm of t00_spawn] (t00_move) {Move sprites using $\Delta t$};
		\node[block, below=4mm of t00_move] (t00_draw) {Draw + flip};
		\node[block, below=4mm of t00_draw] (t00_tick) {clock.tick(FPS) + compute $\Delta t$};
		\node[note, below=3mm of t00_tick] {\textit{Characteristic: FPS-dependent, very high bullet count}};
		\draw[line] (t00_events) -- (t00_update);
		\draw[line] (t00_update) -- (t00_spawn);
		\draw[line] (t00_spawn) -- (t00_move);
		\draw[line] (t00_move) -- (t00_draw);
		\draw[line] (t00_draw) -- (t00_tick);
		\draw[line] (t00_tick.west) -- ++(-10mm,0) |- (t00_events.west);
	\end{scope}
	
	% Panel 2: time01 ---------------------------------------------------------
	\begin{scope}[shift={(7.2,0)}, scale=0.7, transform shape]
		\node[title] (t01) {time01: Spawn every \textbf{N Frames} (Counter)};
		\node[block, below=6mm of t01] (t01_events) {Events (QUIT, ESC, input)};
		\node[block, below=4mm of t01_events] (t01_update) {Update};
		\node[block, below=4mm of t01_update] (t01_inc) {counter \,+= 1};
		\node[decision, below=4mm of t01_inc] (t01_dec) {counter $\ge$ limit?};
		\node[block, below=4mm of t01_dec] (t01_move) {Move sprites using $\Delta t$};
		\node[block, right=13mm of t01_dec] (t01_spawn) {Spawn bullet\\reset counter};
		\node[block, below=4mm of t01_move] (t01_draw) {Draw + flip};
		\node[block, below=4mm of t01_draw] (t01_tick) {clock.tick(FPS) + compute $\Delta t$};
		\node[note, below=3mm of t01_tick] {\textit{Characteristic: FPS-dependent, but simple}};
		\draw[line] (t01_events) -- (t01_update);
		\draw[line] (t01_update) -- (t01_inc);
		\draw[line] (t01_inc) -- (t01_dec);
		\draw[line] (t01_dec) -- node[left]{no} (t01_move);
		\draw[line] (t01_dec.east) -- node[above]{yes} (t01_spawn.west);
		\draw[line] (t01_spawn) |- (t01_move);
		\draw[line] (t01_move) -- (t01_draw);
		\draw[line] (t01_draw) -- (t01_tick);
		\draw[line] (t01_tick.west) -- ++(-10mm,0) |- (t01_events.west);
	\end{scope}
	
	% Panel 3: time02 ---------------------------------------------------------
	\begin{scope}[shift={(0,-8.5)}, scale=0.7, transform shape]
		\node[title] (t02) {time02: Spawn per \textbf{Timestamp} (get\_ticks)};
		\node[block, below=6mm of t02] (t02_events) {Events (QUIT, ESC, input)};
		\node[block, below=4mm of t02_events] (t02_update) {Update};
		\node[decision, below=4mm of t02_update] (t02_dec) {ticks $\ge$ \\stamp + duration?};
		\node[block, right=13mm of t02_dec] (t02_spawn) {Spawn bullet\\stamp = ticks};
		\node[block, below=4mm of t02_dec] (t02_move) {Move sprites using $\Delta t$};
		\node[block, below=4mm of t02_move] (t02_draw) {Draw + flip};
		\node[block, below=4mm of t02_draw] (t02_tick) {clock.tick(FPS) + compute $\Delta t$};
		\node[note, below=3mm of t02_tick] {\textit{Characteristic: FPS-independent, but logic not encapsulated}};
		\draw[line] (t02_events) -- (t02_update);
		\draw[line] (t02_update) -- (t02_dec);
		\draw[line] (t02_dec) -- node[left]{no} (t02_move);
		\draw[line] (t02_dec.east) -- node[above]{yes} (t02_spawn.west);
		\draw[line] (t02_spawn) |- (t02_move);
		\draw[line] (t02_move) -- (t02_draw);
		\draw[line] (t02_draw) -- (t02_tick);
		\draw[line] (t02_tick.west) -- ++(-10mm,0) |- (t02_events.west);
	\end{scope}
	
	% Panel 4: time03 ---------------------------------------------------------
	\begin{scope}[shift={(7.2,-8.5)}, scale=0.7, transform shape]
		\node[title] (t03) {time03: Spawn per \textbf{Timer-Class}};
		\node[block, below=6mm of t03] (t03_events) {Events (QUIT, ESC, input)};
		\node[block, below=4mm of t03_events] (t03_update) {Update};
		\node[decision, below=4mm of t03_update] (t03_dec) {is\_next\_stop\_reached()?};
		\node[block, right=13mm of t03_dec] (t03_spawn) {Spawn bullet};
		\node[block, below=4mm of t03_dec] (t03_move) {Move sprites using $\Delta t$};
		\node[block, below=4mm of t03_move] (t03_draw) {Draw + flip};
		\node[block, below=4mm of t03_draw] (t03_tick) {clock.tick(FPS) + compute $\Delta t$};
		\node[note, below=3mm of t03_tick] {\textit{Characteristic: FPS-independent, but logic encapsulated (SRP)}};
		\draw[line] (t03_events) -- (t03_update);
		\draw[line] (t03_update) -- (t03_dec);
		\draw[line] (t03_dec) -- node[left]{no} (t03_move);
		\draw[line] (t03_dec.east) -- node[above]{yes} (t03_spawn.west);
		\draw[line] (t03_spawn) |- (t03_move);
		\draw[line] (t03_move) -- (t03_draw);
		\draw[line] (t03_draw) -- (t03_tick);
		\draw[line] (t03_tick.west) -- ++(-10mm,0) |- (t03_events.west);
	\end{scope}
	
\end{tikzpicture}
\caption{Comparison of the 4 algorithms}\label{picTimeVergleich01}
\end{center}
\end{figure}

%%%%%%%%%%%%%%%%%%%%%%%%%%%%%%%%%%%%%%%%%%%%%%%%%%%%%%%%%%%%%%%%%%%%%%%%%%%
\subsubsection{Accumulated Time}\index{time!accumulated}

Sometimes it is sufficient to locally detect the progression of a time interval. For this purpose, one can use the \texttt{Timer} class, but it is also possible to accumulate the elapsed time and then compare the sum against a threshold.

Here is a small example:

\lstset{firstnumber = 1}
\begin{lstlisting}
	elapsed += cfg.DELTATIME
	if elapsed >= 0.5:
		elapsed = 0.0
		spawn()	
\end{lstlisting}

By using \texttt{DELTATIME}, a frame-rate–independent and fairly precise measurement of elapsed time is achieved. This logic is well suited for periodic actions or for animations.

%%%%%%%%%%%%%%%%%%%%%%%%%%%%%%%%%%%%%%%%%%%%%%%%%%%%%%%%%%%%%%%%%%%%%%%%%%%
\subsubsection{Cool Down}\index{time!cool down}

If you want to let a certain amount of time pass -- for example for shields or the time between two shots -- the following small logic is commonly used:

\lstset{firstnumber = 1}
\begin{lstlisting}
	if now() - last_shot >= cooldown:
		shoot()
		last_shot = now()
\end{lstlisting}

%%%%%%%%%%%%%%%%%%%%%%%%%%%%%%%%%%%%%%%%%%%%%%%%%%%%%%%%%%%%%%%%%%%%%%%%%%%
\subsubsection{Start Delay}\index{time!start delay}

If you want to let a certain amount of time pass between two actions -- for example because a start screen should be visible for a certain time -- you can use the following approach:

\lstset{firstnumber = 1}
\begin{lstlisting}
	start_time = pygame.time.get_ticks()
	if pygame.time.get_ticks() - start_time > 3000:
		do_something()
\end{lstlisting}



%%%%%%%%%%%%%%%%%%%%%%%%%%%%%%%%%%%%%%%%%%%%%%%%%%%%%%%%%%%%%%%%%%%%%%%%%%%
\subsection{What was new?}
Time-based events or time intervals should be made independent of the frame rate and should be based on the actual elapsed time. Since this is a frequently used logic, it is encapsulated in a separate class.

The following Pygame elements were introduced:

\begin{itemize}
	\item \texttt{pygame.time.get\_ticks()}:
	\myindex{pyg}{\texttt{time}!\texttt{get\_ticks()}}\\ 
    \url{https://pyga.me/docs/ref/time.html#pygame.time.get_ticks}
	
	\item \texttt{pygame.sprite.Sprite.kill()}:
	\myindex{pyg}{\texttt{sprite}!\texttt{Sprite}!\texttt{kill()}}\\ 
	\url{https://pyga.me/docs/ref/sprite.html#pygame.sprite.Sprite.kill}

\end{itemize}
	% !TeX spellcheck = en_US
\newpage
%%%%%%%%%%%%%%%%%%%%%%%%%%%%%%%%%%%%%%%%%%%%%%%%%%%%%%%%%%%%%%%%%%%%%%%%%%%
\section{Mouse}\index{Mouse}\label{secMaus}
%%%%%%%%%%%%%%%%%%%%%%%%%%%%%%%%%%%%%%%%%%%%%%%%%%%%%%%%%%%%%%%%%%%%%%%%%%%
\subsection{Introduction}
\begin{diskbox}
	\url{https://github.com/adamsralf/pygame_book/tree/main/src/00%20Introduction/10%20Mouse/example01}
\end{diskbox}

While many games are controlled using the keyboard or a controller, the mouse is also frequently used. In this script, basic mouse actions such as \emph{clicking} and \emph{position queries} are covered. Our example implements the following functionalities:
\begin{itemize}
	\item A small transparent bubble appears in the center.
	\item When the mouse moves inside an inner rectangle, the bubble acts as the mouse cursor.
	\item When the mouse leaves the inner rectangle, the usual system mouse cursor appears.
	\item A left mouse click rotates the bubble \SI{90}{\degree} to the left.
	\item A right mouse click rotates the bubble \SI{90}{\degree} to the right.
	\item The mouse wheel is used to scale the size of the bubble.
	\item Clicking the mouse wheel terminates the application.
\end{itemize}

\myebild{mouse00.png}{0.5}{Example of actions with a mouse}{picMaus00}

The main action takes place in the \texttt{Game} class, since this is where the mouse actions are processed. Instead of using a separate \texttt{config.py} file, I~implemented here static variables and methods in the \texttt{Game} class -- that works as well.  In the constructor, the usual suspects are initialized, and in \zeiref{srcMaus0001} the \texttt{Ball} object is created.

\lstsource{SRC/00 Introduction/10 Mouse/example01/mouse.py}{55}{68}{python}{Mouse actions: statics and constructor of \texttt{Game}}{srcMaus00a}

The \texttt{run()} method also holds no surprises.

\lstsource{SRC/00 Introduction/10 Mouse/example01/mouse.py}{70}{80}{python}{Mouse actions -- \texttt{Game.run()}}{srcMaus00b} 

In \texttt{watch\_for\_events()}, we encounter the first interesting parts. Just as \texttt{KEYDOWN} and \texttt{KEYUP} mark the pressing and releasing of keys, there are corresponding events for the mouse as well: \randnotiz{MOUSEBUTTONDOWN} \randnotiz{MOUSEBUTTONUP}\texttt{MOUSEBUTTONDOWN}\myindex{pyg}{\texttt{MOUSEBUTTONDOWN}|underline} and \texttt{MOUSEBUTTONUP}\myindex{pyg}{\texttt{MOUSEBUTTONUP}|underline}.  

In \zeiref{srcMaus0002}, the value of \texttt{event.type} is checked and then it is determined which mouse button was pressed.

For this purpose, these two mouse events provide two attributes: \texttt{event.button} and \texttt{event.pos}. The numeric codes of \texttt{event.button}\myindex{pyg}{\texttt{event}!\texttt{button}|underline}\randnotiz{event.button} are shown in \tabref[vref]{tabMousebutton}. Interestingly, there are no predefined constants here, unlike with keyboard input. After the check, the corresponding messages are sent to the \texttt{Ball} object.

If the left mouse button is pressed (\zeiref{srcMaus0004}), a message is sent to the ball to rotate by  \SI{90}{\degree} to the left, and if the right mouse button is pressed, to rotate by \SI{90}{\degree} to the right (hence  \SI{-90}{\degree}, see \zeiref{srcMaus0006}).  

The mouse wheel\index{mouse wheel} is also handled like a mouse button. Depending on the direction of rotation, a different numeric code is returned (see \zeiref{srcMaus0007} and \zeiref{srcMaus0008}). If the mouse wheel is pressed -- that is, clicked -- the game should terminate. This is checked and implemented in \zeiref{srcMaus0005}.

Using \texttt{event.pos}\myindex{pyg}{\texttt{event}!\texttt{pos}}, one could immediately query the mouse position of this very moment -- which we do not do here.\randnotiz{event.pos}

\lstsource{SRC/00 Introduction/10 Mouse/example01/mouse.py}{82}{99}{python}{Mouse actions -- \texttt{Game.watch\_for\_events()}}{srcMaus00c} 

One requirement was that the system mouse cursor should only be visible outside the inner rectangle. Inside the rectangle, the ball is supposed to act as the mouse cursor. In \zeiref{srcMaus0003}, this is achieved using the method \texttt{pygame.mouse.set\_visible()}\randnotiz{set\_visible()}\myindex{pyg}{\texttt{mouse}!\texttt{set\_visible()}|underline}. This method controls whether the system mouse cursor -- in whatever visual form -- is shown or hidden.

The decision is based on whether the current mouse position lies inside the inner rectangle. The method \texttt{pygame.mouse.get\_pos()}\randnotiz{get\_pos()}\myindex{pyg}{\texttt{mouse}!\texttt{get\_pos()}|underline} returns the current mouse position. This position is then simply passed into an already familiar collision test: \texttt{pygame\-.Rect\-.col\-lide\-point()}\randnotiz{collidepoint()}\myindex{pyg}{\texttt{Rect}!\texttt{collidepoint()}|underline}. If the mouse position is inside the rectangle, this method returns \true; otherwise, it returns \false.
%; daher muss der Wahrheitswert noch mit \texttt{not} negiert werden.

\lstsource{SRC/00 Introduction/10 Mouse/example01/mouse.py}{101}{114}{python}{Mouse actions -- \texttt{Game.update()} and \texttt{Game.draw()}}{srcMaus00d} 

The only class left is \texttt{Ball}. Although it no longer contains any direct mouse actions, the \texttt{update()} method now looks quite different from what we have seen in the previous examples.  

\begin{warningbox}[And again: LSP]
	In earlier examples, methods such as \texttt{rotate()} or \texttt{resize()} were called directly from \texttt{watch\_for\_events()} or comparable methods of the \texttt{Game} class. This is perfectly fine in principle. However, problems arise once such subclasses of \texttt{pygame.sprite.Sprite} are added to a \texttt{pygame.sprite.Group} or a \texttt{pygame.sprite.GroupSingle}. These classes expect only \texttt{Sprite} objects as elements. Therefore, in terms of object-oriented programming, one should only use methods and attributes that are known to the parent class \texttt{pygame.sprite.Sprite} -- for example, \texttt{update()}.  Methods such as \texttt{rotate()} would be unknown to the sprite group.
\end{warningbox}

Consider, for example, \zeiref{srcInvader06d01} in \srcref[vref]{srcInvader06d}. The method \texttt{change\-\_di\-rec\-tion()} is completely unknown to the \texttt{GroupSingle} object \texttt{defender}, because it expects a \texttt{Sprite} and not a \texttt{Defender} object. Syntax checkers such as \gls{pylance} will report errors here.


\begin{hintbox}[Dispatching hub]
	One way to work around this problem is to use \texttt{update()} as a dispatching hub\index{dispatching hub}. In the class 		\texttt{pygame.sprite.Sprite}, this method is defined with the following signature:

	\verb+update(self, *args: Any, **kwargs: Any) -> None+
\end{hintbox}


In other words, any number of freely definable parameters can be passed to this method. This is exactly what happens in our \texttt{update()} method. For rotation, the parameter \texttt{rotate} is passed with the corresponding angle; for scaling, the parameter \texttt{scale}; and in \texttt{update()} of \texttt{Game}, the parameter \texttt{go} is passed with the value \true. 
 
Each caller can therefore define its own parameters spontaneously and assign values to them. The \texttt{update()} method in the subclass -- here \texttt{Ball} -- only needs to check for these parameters.

In the first step, it is checked whether a parameter was provided, as shown in \zeiref{srcMaus0009}, \zeiref{srcMaus0010}, \zeiref{srcMaus0011}, and \zeiref{srcMaus0012}. Afterwards, the parameter value is forwarded to the corresponding method of the subclass. This way, the sprite group does not need to access methods of the subclass directly, but can rely on the method provided by the parent class.

\newpage
\lstsource{SRC/00 Introduction/10 Mouse/example01/mouse.py}{9}{52}{python}{Mouse actions -- \texttt{Ball}}{srcMaus00e} 

\begin{warningbox}[Two final notes on \texttt{pygame.transform.rotate()}\myindex{pyg}{\texttt{transform}!\texttt{rotate()}|underline}]
\begin{itemize}
	\item Unlike many other systems that work with angles, the angle here is measured in \gls{degree} rather than in \gls{radian}.
	\item The approach used here works only because the bitmap represents a ball. In general, you should not repeatedly rotate the already rotated image. Instead, keep the original image unchanged and update only an angle variable. If you rotate the image over and over again, the contents may be scaled up or down depending on the bitmap’s shape, resulting in visual artifacts and severe pixel degradation.
\end{itemize}
\end{warningbox}

%%%%%%%%%%%%%%%%%%%%%%%%%%%%%%%%%%%%%%%%%%%%%%%%%%%%%%%%%%%%%%%%%%%%%%%%%%%
\subsection{More Input}
%%%%%%%%%%%%%%%%%%%%%%%%%%%%%%%%%%%%%%%%%%%%%%%%%%%%%%%%%%%%%%%%%%%%%%%%%%%
\subsubsection{In Which Window Took the Mouse Action Place?}
\begin{diskbox}
	\url{https://github.com/adamsralf/pygame_book/tree/main/src/00%20Introduction/10%20Mouse/example02}
\end{diskbox}

Completely analogous to the example and explanation in \secref[vref]{secMultipleWindowKeyboard},\texttt{id}\randnotiz{Window.id}\myindex{pyg}{\texttt{Window}!\texttt{id}} it is also possible to determine in which window a mouse action occurred.\randnotiz{event.window}\myindex{pyg}{\texttt{Window}}

\lstsource{SRC/00 Introduction/10 Mouse/example02/mouse.py}{4}{43}{python}{In which window took the mouse action place?}{srcMouse01a}

Running the program produces the following console output when the corresponding keys are pressed:

\lstset{firstnumber=1}
\begin{lstlisting}
ID 1: Main Window (Mouse Pressed: '1' at (55, 25))
ID 1: Main Window (Mouse Pressed: '3' at (101, 27))
ID 1: Main Window (Mouse Pressed: '2' at (171, 23))
ID 2: Side Window (Mouse Pressed: '1' at (74, 12))
ID 2: Side Window (Mouse Pressed: '2' at (123, 25))
ID 2: Side Window (Mouse Pressed: '3' at (224, 25))
\end{lstlisting}

%%%%%%%%%%%%%%%%%%%%%%%%%%%%%%%%%%%%%%%%%%%%%%%%%%%%%%%%%%%%%%%%%%%%%%%%%%%
\subsubsection{Surprise: No Double Click}

\begin{warningbox}
	In Pygamee, there is no built-in event for a mouse double click. Unlike classic desktop GUI frameworks, pygame only reports individual mouse button presses.
\end{warningbox}

This means you cannot ask directly whether a double click happened. Instead, you have to detect it yourself.

The usual approach is to measure the time between two mouse clicks. If the same mouse button is pressed twice within a short time window, it is treated as a double click. Optionally, the mouse position can also be checked to make sure the cursor has not moved too far between clicks.

This may sound a bit low-level at first, but it is actually very common in game programming. Many higher-level interactions are built by combining simple events with a bit of timing logic.

Once implemented, this technique works reliably and gives you full control over what \emph{double click} really means in your game or tool.

\lstset{firstnumber=1}
\begin{lstlisting}
import pygame
import time

DOUBLE_CLICK_TIME = 400  # ms

last_click_time = 0
last_button = None

for event in pygame.event.get():
	if event.type == pygame.MOUSEBUTTONDOWN:
		now = pygame.time.get_ticks()

		if (event.button == last_button and now - last_click_time < DOUBLE_CLICK_TIME):
			print("Double click detected!")

		last_click_time = now
		last_button = event.button
\end{lstlisting}

\begin{hintbox}[Options]
\begin{itemize}
	\item A \texttt{DOUBLE\_CLICK\_TIME} between 250 and \SI{300}{ms} feels very snappy.
	\item A \texttt{DOUBLE\_CLICK\_TIME} between 400 and \SI{500}{ms} is easier for beginners.
	\item In addition, you can check the distance between the \texttt{event.pos} values of both clicks.
\end{itemize}
\end{hintbox}

%%%%%%%%%%%%%%%%%%%%%%%%%%%%%%%%%%%%%%%%%%%%%%%%%%%%%%%%%%%%%%%%%%%%%%%%%%%
\subsubsection{Not by Event, but by Function}
In contrast to mouse button events, mouse button states can also be queried directly. Using the function \texttt{pygame.mouse.get\_pressed()}\randnotiz{get\_pressed()}\myindex{pyg}{\texttt{mouse}!\texttt{get\_pressed()}}, it is possible to check at any time which mouse buttons are currently being held down.

The function returns a tuple of boolean values that indicate whether the left, middle, or right mouse button is pressed. This allows continuous mouse input to be processed without relying on discrete events such as \texttt{MOUSEBUTTONDOWN} or \texttt{MOUSEBUTTONUP}.

This approach is particularly useful when mouse buttons should influence behaviour as long as they are held down, for example for dragging objects or continuous actions.

A completely analogous discussion can be found in \secref[vref]{secKeyPressed}.

%%%%%%%%%%%%%%%%%%%%%%%%%%%%%%%%%%%%%%%%%%%%%%%%%%%%%%%%%%%%%%%%%%%%%%%%%%%
\subsection{What was new?}
\begin{hintbox}
	Mouse actions are processed in a similar way to keyboard events. The mouse position can be queried easily. It is often simpler to hide the mouse cursor and let a bitmap follow the mouse position than to define a new mouse cursor.
\end{hintbox}

\begin{pygbox}
\begin{itemize}
	\item \texttt{pygame.constants}:
	\myindex{pyg}{\texttt{constants}}\\ 
	\url{https://pyga.me/docs/ref/locals.html}
	
	\item \texttt{pygame.mouse.get\_pressed()}
	\myindex{pyg}{\texttt{mouse}!\texttt{get\_pressed()}}\\ \url{https://pyga.me/docs/ref/mouse.html#pygame.mouse.get_pressed}
	
	\item \texttt{pygame.MOUSEBUTTONDOWN}, \texttt{pygame.MOUSEBUTTONDOWN}:
	\myindex{pyg}{\texttt{MOUSEBUTTONDOWN}}\myindex{pyg}{\texttt{MOUSEBUTTONUP}}\\ \url{https://pyga.me/docs/ref/event.html}
	
	\item List of mouse events: \tabref[vref]{tabMousebutton}
	\item \texttt{pygame.mouse.get\_pos()}:
	\myindex{pyg}{\texttt{mouse}!\texttt{get\_pos()}}\\ \url{https://pyga.me/docs/ref/mouse.html#pygame.mouse.get_pos}
	
	\item \texttt{pygame.mouse.set\_visible()}:
	\myindex{pyg}{\texttt{mouse}!\texttt{set\_visible()}}\\ \url{https://pyga.me/docs/ref/mouse.html#pygame.mouse.set_visible}
	
	\item \texttt{pygame.Rect.collidepoint()}:
	\myindex{pyg}{\texttt{Rect}!\texttt{collidepoint()}}\\ \url{https://pyga.me/docs/ref/rect.html#pygame.Rect.collidepoint}
	
	\item \texttt{pygame.transform.rotate()}:
	\myindex{pyg}{\texttt{transform}!\texttt{rotate()}}\\ \url{https://pyga.me/docs/ref/transform.html#pygame.transform.rotate}
	
\end{itemize}
\end{pygbox}

\begin{longtable}{ll}
	\caption{List of mouse events}\label{tabMousebutton}\myindex{pyg}{\texttt{event}!\texttt{button}} \\
	% Definition des Tabellenkopfes auf der ersten Seite
	\toprule
	Constant & Description \\
	\midrule
	\endfirsthead % Erster Kopf zu Ende
	
	% Definition des Tabellenkopfes auf den folgenden Seiten
	\caption{List of mouse events (continued)}\\
	\toprule
	Constant & Description \\
	\midrule
	\endhead % Zweiter Kopf ist zu Ende
	
	\midrule
	\multicolumn{2}{r}{\emph{continued on next page}} \\
	\endfoot
	
	\bottomrule
	\endlastfoot


	% Ab hier kommt der Inhalt der Tabelle
	\texttt{0} &  undefined \\ 
	\texttt{1} &  left mouse button\index{mouse!left button}\\ 
	\texttt{2} &  middle mouse button/mouse wheel\index{mouse!middle button}\index{mouse!wheel}\\ 
	\texttt{3} &  right mouse button\index{mouse!right button}\\ 
	\texttt{4} &  scroll the mouse wheel up\index{mouse!wheel up}\\ 
	\texttt{5} &  Scroll the mouse wheel down\index{mouse!wheel down}\\ 
\end{longtable} 

%%%%%%%%%%%%%%%%%%%%%%%%%%%%%%%%%%%%%%%%%%%%%%%%%%%%%%%%%%%%%%%%%%%%%%%%%%%
\subsection{Homework}
Time for a small game:

\begin{enumerate}
	\item Create a fairly large playfield with a space look.
	\item Your spaceship appears in the center.
	\item Control the spaceship using the mouse. If one of the \keys{\shift} keys is pressed, the mouse movement is interpreted as a control command. Hint: you need to compare the mouse position coordinates between two frames.
	\item Rocks (asteroids) of different sizes fly into the screen from all sides.
	\item The goal is to avoid being hit by a rock for as long as possible.
	\item Every 10~seconds, the score is increased.
	\item If I~collide with a rock, I~lose one life.
	\item At the bottom, the score and the remaining lives are displayed.
	\item I~start with three lives. Add new lives repeatedly when certain score values are reached.
	\item When a certain score is reached, I~can activate a shield for 15 seconds using the right mouse button. Whether the shield is available should be shown in the bottom line.
	\item BONUS: The nose of the spaceship always points in the direction of movement.
\end{enumerate}


	% !TeX spellcheck = en_US
\newpage
%%%%%%%%%%%%%%%%%%%%%%%%%%%%%%%%%%%%%%%%%%%%%%%%%%%%%%%%%%%%%%%%%%%%%%%%%%%
\section{Sound}\index{Sound}
Without background sounds and/or music, many games would simply be boring.  
Therefore, I~would like to present three different topics here: background music or ambient sounds, sound events, and stereo effects.

%%%%%%%%%%%%%%%%%%%%%%%%%%%%%%%%%%%%%%%%%%%%%%%%%%%%%%%%%%%%%%%%%%%%%%%%%%%
\subsection{Introduction}

%%%%%%%%%%%%%%%%%%%%%%%%%%%%%%%%%%%%%%%%%%%%%%%%%%%%%%%%%%%%%%%%%%%%%%%%%%%
\subsubsection{Sound: Music}\index{sound!music}
\begin{diskbox}
	\url{https://github.com/adamsralf/pygame_book/tree/main/src/00%20Introduction/11%20Sound/music}
\end{diskbox}

The first example covers the following features:
\begin{itemize}
	\item Background music is loaded and played in an endless loop.
	\item The volume can be adjusted using the mouse wheel.
	\item Pressing \keys{p} pauses the background music or resumes playback.
	\item Pressing \keys{j} fades out the background music.
\end{itemize}

I~will not explain the imports, \texttt{config.py}, and the other familiar building blocks in detail anymore, as they have already appeared many times before.

\lstsource{SRC/00 Introduction/11 Sound/music/config.py}{1}{28}{python}{Sound -- \texttt{config.py}}{srcSound00a} 

Before sound can be used, the corresponding subsystem must be initialized. This can be done explicitly using \texttt{pygame.mixer.init()}\randnotiz{init()}\myindex{pyg}{\texttt{mixer}!\texttt{init()}|underline}, or implicitly -- as in the source code at \zeiref{srcSound0002} -- by calling \texttt{pygame.init()}\myindex{pyg}{\texttt{init()}}.  

In the \texttt{sounds()} method, the preparatory steps for sound output are encapsulated. Background music\randnotiz{background music}\index{background music} is loaded into the mixer's internal memory using \texttt{pygame\-.mixer\-.music\-.load()}\myindex{pyg}{\texttt{mixer}!\texttt{music}!\texttt{load()}|underline}. However, loading the music does not start playback yet.  

Playback begins after the volume has been set in \zeiref{srcSound0004} using\randnotiz{set\_volume()} \texttt{pygame\-.mixer\-.music\-.set\-\_volume()}\myindex{pyg}{\texttt{mixer}!\texttt{music}!\texttt{set\_volume()}|underline}, by calling the method in \zeiref{srcSound0005}. The method \texttt{pygame\-.mixer\-.music\-.play()}\myindex{pyg}{\texttt{mixer}!\texttt{music}!\texttt{play()}|underline}\randnotiz{play()} accepts three parameters: 
\begin{itemize}
	\item The first parameter, \texttt{loops}, controls the number of repetitions; a value of~$-1$ means that the music is repeated indefinitely. 
	\item The second parameter, \texttt{start}, specifies the position at which playback should begin; the default is~$0.0$. 
	\item If the music should start quietly and then become louder (\gls{fade}\index{fade}\randnotiz{fade}), this can be achieved using the third parameter \texttt{fade}. Here, you can specify how many milliseconds are available for the fade-in; if nothing is specified, playback starts immediately at the target volume.
\end{itemize}

\lstsource{SRC/00 Introduction/11 Sound/music/sound.py}{9}{23}{python}{Sound -- Constructor and \texttt{sounds()} of \texttt{Game}}{srcSound00b} 

 The \texttt{watch\_for\_events()} method acts purely as a dispatcher. Depending on which key is pressed or which mouse element is used, the corresponding helper methods are called.

\lstsource{SRC/00 Introduction/11 Sound/music/sound.py}{25}{43}{python}{Sound -- \texttt{Game.watch\_for\_events()}}{srcSound00c} 

I~want to start the background music at some times and fade it out at others. This is handled by the helper method \texttt{music\_start\_stop()}. The background music is stopped using \texttt{pygame.mixer.music.fadeout()}\myindex{pyg}{\texttt{mixer}!\texttt{music}!\texttt{fadeout()}|underline}\randnotiz{fadeout()}. Here, you have to specify over how many milliseconds the music should gradually become quieter until it stops -- in our example, this is~\SI{5000}{ms}. The method \texttt{pygame.mixer.music.play()}\myindex{pyg}{\texttt{mixer}!\texttt{music}!\texttt{play()}} used to start the background music has already been explained above.

\lstsource{SRC/00 Introduction/11 Sound/music/sound.py}{46}{50}{python}{Sound -- \texttt{Game.music\_start\_stop()}}{srcSound00e} 

Pressing \keys{p} pauses the background music or resumes playback. The current state is stored in the attribute \texttt{pause}. This attribute then determines which of the two \texttt{music} methods is called in the \texttt{pause\_alter()} method -- either  \texttt{pygame.mixer.music.pause()}\myindex{pyg}{\texttt{mixer}!\texttt{music}!\texttt{pause()}|underline}\randnotiz{pause()} or \texttt{pygame.mixer.music.unpause()}\myindex{pyg}{\texttt{mixer}!\texttt{music}!\texttt{unpause()}|underline}\randnotiz{unpause()}. Finally, in \zeiref{srcSound0007}, the \texttt{pause} flag is toggled.

\lstsource{SRC/00 Introduction/11 Sound/music/sound.py}{52}{57}{python}{Sound -- \texttt{Game.pause\_alter()}}{srcSound00f} 

As the final feature, volume control is introduced. It is encapsulated in the \texttt{volume\-\_alter()} method. Instead of passing an absolute volume value to this method, a delta value is provided.

First, this value is added to the \texttt{volume} variable. Afterwards, the value is clamped to the interval $[0, 1]$, and finally the new volume is set using \texttt{pygame.mixer.music.set\-\_volume()}\myindex{pyg}{\texttt{mixer}!\texttt{music}!\texttt{set\_volume()}|underline}\randnotiz{set\_volume()}.

\lstsource{SRC/00 Introduction/11 Sound/music/sound.py}{59}{63}{python}{Sound -- \texttt{volume\_alter()} von \texttt{Game}}{srcSound00g} 

And finally, we deal with the remaining bits.

\lstsource{SRC/00 Introduction/11 Sound/music/sound.py}{65}{85}{python}{Sound -- \texttt{draw()}, \texttt{update()}, \texttt{run()} of \texttt{Game}}{srcSound00h} 

%%%%%%%%%%%%%%%%%%%%%%%%%%%%%%%%%%%%%%%%%%%%%%%%%%%%%%%%%%%%%%%%%%%%%%%%%%%
\subsubsection{Sound: Events}\index{sound!event}
\begin{diskbox}
	\url{https://github.com/adamsralf/pygame_book/tree/main/src/00%20Introduction/11%20Sound/effects}
\end{diskbox}

For sound effects\index{sound effects}\randnotiz{sound effects}, a separate \texttt{Sound} object is created in each case (\zeiref{srcSound0101}ff.). The constructor of \texttt{pygame.mixer.Sound}\myindex{pyg}{\texttt{mixer}!\texttt{Sound}|underline} is given the file name including the path. If you already have an open file object, you can pass that instead; in this case, however, you should provide a second parameter specifying the sound encoding, for example \texttt{.\gls{ogg}}\index{ogg} or \texttt{.\gls{mp3}}\index{mp3}. As with background music, loading a sound is not the same as playing it.

\lstsource{SRC/00 Introduction/11 Sound/effects/sound.py}{19}{21}{python}{Sound -- \texttt{Game.sound()}}{srcSound01a} 

In \zeiref{srcSound0106}, the current volume is first stored in a variable using
\texttt{pygame\-.mixer\-.Sound\-.get\_\-volume()}\randnotiz{get\_volume()}\myindex{pyg}{\texttt{mixer}!\texttt{Sound}!\texttt{get\_volume()}}. To ensure that both sounds are played at the same volume, this value is modified and then applied to both sounds using
\texttt{pygame.mixer.Sound.set\_volume()}\randnotiz{set\_volume()}\myindex{pyg}{\texttt{mixer}!\texttt{Sound}!\texttt{set\_volume()}}.

\newpage
\lstsource{SRC/00 Introduction/11 Sound/effects/sound.py}{40}{45}{python}{Sound -- \texttt{Game.volume\_alter()}}{srcSound01b} 

For simplicity, the sounds are played directly in \texttt{watch\_for\_events()} (see \zeiref{srcSound0103} and \zeiref{srcSound0104}). The actual playback is done using \texttt{pygame.mixer.Sound.play()}\myindex{pyg}{\texttt{mixer}!\texttt{Sound}!\texttt{play()}|underline}\randnotiz{play()}. You can see that the \texttt{play()} method is called on the corresponding \texttt{Sound} object.

The \texttt{play()} method provides three optional arguments:
\begin{itemize}
	\item \texttt{loops}: number of repetitions ($-1$ means infinite playback, default)
	\item \texttt{maxtime}: maximum playback time in milliseconds ($0$ means unlimited, default)
	\item \texttt{fade\_ms}: duration of the fade-in in milliseconds (default: $0$)
\end{itemize}

If -- as in this case --no arguments are provided, the sound starts playing immediately and stops automatically after it has finished playing. Any other sounds that are currently being played by other \texttt{Sound} objects are not interrupted. This means that multiple sounds can be played at the same time.

\lstsource{SRC/00 Introduction/11 Sound/effects/sound.py}{23}{38}{python}{Sound -- \texttt{Game.watch\_for\_events()()}}{srcSound01c} 


%%%%%%%%%%%%%%%%%%%%%%%%%%%%%%%%%%%%%%%%%%%%%%%%%%%%%%%%%%%%%%%%
\newpage
\subsection{More Input}

%%%%%%%%%%%%%%%%%%%%%%%%%%%%%%%%%%%%%%%%%%%%%%%%%%%%%%%%%%%%%%%%
\subsubsection{Stereo}\label{secStereo}
\begin{diskbox}
	\url{https://github.com/adamsralf/pygame_book/tree/main/src/00%20Introduction/11%20Sound/stereo}
\end{diskbox}

A small example is intended to illustrate the use of channels and \gls{stereo} effects. The topic is too extensive to be presented in full detail, but I~hope that this chapter provides a helpful introduction.

In \abbref[vref]{picStereo00}, you can see a tank driving from left to right or from right to left. While driving, it can fire up to 5~shots. It would be nice if the driving sound indicated acoustically where the tank is currently located. That is, if the tank is more to the right, the driving sound or the shot should be louder on the right speaker than on the left speaker (\gls{stereopanning}\randnotiz{stereo panning}). When driving from right to left, the driving sound would therefore move along with the tank.

\myebild{stereo00.png}{0.7}{Example Stereo Sound}{picStereo00}

First, the necessary boilerplate, which should not require any further explanation:

\lstsource{SRC/00 Introduction/11 Sound/stereo/config.py}{1}{43}{python}{Sound-Stereo -- \texttt{config.py}}{srcSound01aa} 

\lstsource{SRC/00 Introduction/11 Sound/stereo/sound.py}{12}{23}{python}{Sound-Stereo -- Class \texttt{Ground}}{srcSound01ab} 

In \zeiref{srcSound0101}, a \texttt{Sound} object\myindex{pyg}{\texttt{mixer}!\texttt{Sound}}\randnotiz{Sound object} is created. This object is played to emphasize the movement of the tank with appropriate sounds. In the following line (\zeiref{srcSound0201}), the helper method \texttt{stereo()} is called (see below), and then the playback of the driving sound starts in an infinite loop (\zeiref{srcSound0203}). 

It is noticeable that the output is not started using \texttt{pygame.mixer.Sound.play()}\myindex{pyg}{\texttt{mixer}!\texttt{Sound}!\texttt{play()}}. Normally, this would be a good choice, since this command automatically selects one of the eight available Pygame sound channels\index{channel}\randnotiz{channel}.

\begin{hintbox}[Selecting channels manually]
	However, it is also possible to address a Pygame channel directly and thus gain more control over the sound behavior. In \zeiref{srcSound0204}, a free \texttt{pygame.mixer.Channel} object\myindex{pyg}{\texttt{mixer}!\texttt{Channel}|underline} is determined for this purpose. The method \texttt{pygame.mixer.find\-\_chan\-nel()} \myindex{pyg}{\texttt{mixer}!\texttt{find\_channel()}|underline} returns the first pygame channel and stores it in the attribute \texttt{channel}.
\end{hintbox} 

Playback in \zeiref{srcSound0203} is then no longer started via a method of the \texttt{Sound} object, but by using \texttt{pygame.mixer.Channel.play()}\myindex{pyg}{\texttt{mixer}!\texttt{Channel}!\texttt{play()}|underline}\randnotiz{play()}.

This makes it possible to adjust volume and stereo panning dynamically while the sound is playing.

\newpage
\lstsource{SRC/00 Introduction/11 Sound/stereo/sound.py}{26}{50}{python}{Sound-Stereo -- Constructor of \texttt{Tank}}{srcSound02b} 

The \texttt{update()} method is shown here only for completeness. It does not contain any code related to sound playback.

\lstsource{SRC/00 Introduction/11 Sound/stereo/sound.py}{52}{75}{python}{Sound-Stereo -- \texttt{Tank.update()}}{srcSound02c} 

The \texttt{stereo()} method is surprisingly simple. The method \texttt{pygame.mixer.Channel\-.set-\_vol\-ume()}\myindex{pyg}{\texttt{mixer}!\texttt{Channel}!\texttt{set\_volume()}|underline}\randnotiz{set\_volume()} provides two parameters: \emph{left} and \emph{right}. Both parameters have a value range of~$[0, 1]$.

As discussed before, we want the right speaker to play the engine sound louder the further to the right the tank is positioned, and vice versa (\gls{stereopanning}) \randnotiz{Stereo Panning}. To achieve this, I~calculate the relative horizontal position of the tank’s center with respect to the window width in \zeiref{srcSound0203}. This calculation also yields a value in the interval~$[0, 1]$.

Once this value is known, the relative value for the left speaker can be determined in the following line by $left = 1 - right $ . After that, both values are passed to the \texttt{set\_volume()} method.

\begin{hintbox}[Hint]
	The method \texttt{pygame.mixer.Channel\-.set\-\_vol\-ume()} allows different volume levels to be specified for the left and right Pygame channels, whereas the methods \texttt{pygame\-.mix\-er\-.Sound\-.set\-\_vol\-ume()}\myindex{pyg}{\texttt{mixer}!\texttt{Sound}!\texttt{set\_volume()}} and \texttt{pygame.mixer.music\-.set\-\_vol\-ume()}\myindex{pyg}{\texttt{mixer}!\texttt{music}!\texttt{set\_volume()}} do not. 
\end{hintbox}
 
 \lstsource{SRC/00 Introduction/11 Sound/stereo/sound.py}{77}{80}{python}{Sound-Stereo -- \texttt{Tank.stereo()}}{srcSound02d} 

What else could this effect be used for? For example, think of two people talking to each other, sound sources in a room, and so on. Whenever audio is meant to make localization easier, or when individual sounds should stand out or be easier to distinguish, different volume levels -- i.e., stereo -- are a good option.

Nothing related to sound output happens in \texttt{turn()} and \texttt{update\_imageindex()}.

\lstsource{SRC/00 Introduction/11 Sound/stereo/sound.py}{82}{89}{python}{Sound-Stereo -- \texttt{Tank.turn()} and \texttt{Tank.update\_imageindex()}}{srcSound02e} 

The sound output of the \texttt{Bullet} could also have been implemented in the \texttt{Tank} class. However, I~find it more natural to place it in \texttt{Bullet}. After all, it might later be extended to include an impact sound or an explosion.

Before the constructor, the static variable \texttt{sound\_fire} is defined in \zeiref{srcSound0206}. Although there are many bullets, they all use the same firing sound. Reading this sound file repeatedly and creating a new object each time would therefore waste memory and reduce performance. Instead, starting at \zeiref{srcSound0207}, a kind of \gls{singleton} check is performed. This ensures that the sound file is read and the corresponding object is created exactly once.

After that, a free channel is searched for, just as with the tank, and the volume of the left and right speakers is determined based on the position.
Finally, the sound is played based on the horizontal position of the bullet.

\lstsource{SRC/00 Introduction/11 Sound/stereo/sound.py}{92}{126}{python}{Sound-Stereo -- Class \texttt{Bullet}}{srcSound02g} 

The remaining source code is shown here only for the sake of completeness.

\lstsource{SRC/00 Introduction/11 Sound/stereo/sound.py}{129}{186}{python}{Sound-Stereo -- Rest}{srcSound02h} 


%%%%%%%%%%%%%%%%%%%%%%%%%%%%%%%%%%%%%%%%%%%%%%%%%%%%%%%%%%%%%%%%
\subsubsection{Sound Formats and Technical Basics}

Pygame does not support all audio formats equally well. The following formats have proven to be particularly reliable:

\begin{itemize}
	\item[\texttt{.wav}] -- uncompressed, fast to load, ideal for sound effects
	\item[\texttt{.ogg}] -- compressed, well suited for music
	\item[\texttt{.mp3}] -- limited support, not recommended
\end{itemize}

\newpage
\begin{warningbox}[Don'ts]
	\begin{itemize}
		\item Mono sounds are often used for sound effects because they require less memory and can be positioned spatially more effectively.
		\item Large files and high sample rates can negatively affect loading times and performance.
	\end{itemize}
\end{warningbox}

%%%%%%%%%%%%%%%%%%%%%%%%%%%%%%%%%%%%%%%%%%%%%%%%%%%%%%%%%%%%%%%%
\subsubsection{Volume Hierarchies and Sound Mixing}

Games often use multiple volume levels:
\begin{enumerate}
	\item Master volume (everything)
	\item Music volume
	\item Effect volume
\end{enumerate}

These can be combined:
\lstset{firstnumber=1}
\begin{lstlisting}
	master_volume = 0.8
	effects_volume = 0.5
	
	sound.set_volume(master_volume * effects_volume)
\end{lstlisting}

This makes it easy to implement audio settings for game menus later on.

%%%%%%%%%%%%%%%%%%%%%%%%%%%%%%%%%%%%%%%%%%%%%%%%%%%%%%%%%%%%%%%%
\subsubsection{Mono Sounds and Stereo Panning}

Mono sounds are particularly suitable for position-dependent audio. Only mono sounds can be cleanly distributed between the left and right speakers.

\lstset{firstnumber=1}
\begin{lstlisting}
	channel.set_volume(left, right)
\end{lstlisting}

\begin{warningbox}[Don'ts]
	Stereo sounds already contain spatial information and may produce unexpected results when additional panning is applied.
\end{warningbox}

%%%%%%%%%%%%%%%%%%%%%%%%%%%%%%%%%%%%%%%%%%%%%%%%%%%%%%%%%%%%%%%%
\newpage
\subsubsection{Sound Lifetime and Resource Management}

\begin{hintbox}
	Sounds should not be reloaded for every event. Instead, they should be loaded once and reused.
\end{hintbox}

\lstset{firstnumber=1}
\begin{lstlisting}
	class Bullet:
		sound_fire = None
	
		def __init__(self):
		if sound_fire is None:
			sound_fire = pygame.mixer.Sound("fire.wav")
\end{lstlisting}


This saves memory and avoids unnecessary loading times.

%%%%%%%%%%%%%%%%%%%%%%%%%%%%%%%%%%%%%%%%%%%%%%%%%%%%%%%%%%%%%%%%
\subsubsection{Event-driven Sound Output}
\begin{hintbox}
	Sounds should be played in an event-driven manner, not frame-based.
\end{hintbox}

It is incorrect to call \texttt{sound.play()} directly or indirectly inside the update loop. Instead, sounds should be triggered by events:

\lstset{firstnumber=1}
\begin{lstlisting}
	elif event.key == K_SPACE:
		sound.play()
\end{lstlisting}


%%%%%%%%%%%%%%%%%%%%%%%%%%%%%%%%%%%%%%%%%%%%%%%%%%%%%%%%%%%%%%%%
\subsubsection{Looping and Transitions}

Loops are used for continuous sounds (engines, wind, music):\index{sound!endless loops} 

\lstset{firstnumber=1}
\begin{lstlisting}
	channel.play(sound, loops=-1)
\end{lstlisting}

Smooth transitions can be achieved using fade-in and fade-out effects:\index{sound!fade-in}\index{sound!fade-out}

\lstset{firstnumber=1}
\begin{lstlisting}
	sound.fadeout(1000)   # 1 second
\end{lstlisting}


%%%%%%%%%%%%%%%%%%%%%%%%%%%%%%%%%%%%%%%%%%%%%%%%%%%%%%%%%%%%%%%%
\subsubsection{Muting and Pausing}

Many games offer an option to mute or pause sound globally.

\lstset{firstnumber=1}
\begin{lstlisting}
	pygame.mixer.pause()
	...
	pygame.mixer.unpause()
\end{lstlisting}


Alternatively, this can be done via volume control:

\lstset{firstnumber=1}
\begin{lstlisting}
	pygame.mixer.music.set_volume(0)
\end{lstlisting}


This is especially important for pause menus or when the game window loses focus.

%%%%%%%%%%%%%%%%%%%%%%%%%%%%%%%%%%%%%%%%%%%%%%%%%%%%%%%%%%%%%%%%
\subsubsection{Typical Errors and Debugging}

\begin{warningbox}[Common problems with sound in Pygame include:]
\begin{itemize}
	\item Mixer not initialized
	\item Sound played too frequently
	\item No free channels available
	\item Distorted sound (incorrect format)
	\item Meaningless increase of the number of sound channels, such as \texttt{pygame\-.mixer\-.set\-\_num\-\_channels(16)}
\end{itemize}
\end{warningbox}



Checking the current state often helps:

\lstset{firstnumber=1}
\begin{lstlisting}
	print(pygame.mixer.music.get_busy())
\end{lstlisting}
%%%%%%%%%%%%%%%%%%%%%%%%%%%%%%%%%%%%%%%%%%%%%%%%%%%%%%%%%%%%%%%%
\subsection{What was new?}
\begin{hintbox}
	Two options are available for sound support. One option is background music, while the other uses individual sounds played on different channels and, if possible, distributed across the left and right speakers.
\end{hintbox}

\begin{pygbox}
\begin{itemize}
	\item\texttt{pygame.mixer.Channel} :
	\myindex{pyg}{\texttt{mixer}!\texttt{Channel}}\\ \url{https://pyga.me/docs/ref/music.html#pygame.mixer.Channel}

	\item \texttt{pygame.mixer.Channel.play()}:
	\myindex{pyg}{\texttt{mixer}!\texttt{Channel}!\texttt{play()}}\\ \url{https://pyga.me/docs/ref/mixer.html#pygame.mixer.Channel.play}

	\item \texttt{pygame.mixer.Channel.set\_volume()}:
	\myindex{pyg}{\texttt{mixer}!\texttt{Channel}!\texttt{set\_volume()}}\\ \url{https://pyga.me/docs/ref/mixer.html#pygame.mixer.Channel.set_volume}

	\item\texttt{pygame.mixer.find\_channel()} :
	\myindex{pyg}{\texttt{mixer}!\texttt{find\_channel()}}\\ \url{https://pyga.me/docs/ref/music.html#pygame.mixer.find_channel}

	\item \texttt{pygame.mixer.init()}:
	\myindex{pyg}{\texttt{mixer}!\texttt{init()}}\\ \url{https://pyga.me/docs/ref/mixer.html#pygame.mixer.init}

	\item \texttt{pygame.mixer.set\_num\_channels()}:
	\myindex{pyg}{\texttt{mixer}!\texttt{set\_num\_channels()}}\\ \url{https://pyga.me/docs/ref/mixer.html#pygame.mixer.set_num_channels}

	\item\texttt{pygame.mixer.music.fadeout()}:
	\myindex{pyg}{\texttt{mixer}!\texttt{music}!\texttt{fadeout()}}\\ \url{https://pyga.me/docs/ref/music.html#pygame.mixer.music.fadeout}

	\item\texttt{pygame.mixer.music.get\_busy()} :
	\myindex{pyg}{\texttt{mixer}!\texttt{music}!\texttt{get\_busy()}}\\ \url{https://pyga.me/docs/ref/music.html#pygame.mixer.music.get_busy}

	\item \texttt{pygame.mixer.music.get\_volume()}:
	\myindex{pyg}{\texttt{mixer}!\texttt{music}!\texttt{get\_volume()}}\\ \url{https://pyga.me/docs/ref/music.html#pygame.mixer.music.get_volume}

    \item \texttt{pygame.mixer.music.load()}:
	\myindex{pyg}{\texttt{mixer}!\texttt{music}!\texttt{load()}}\\ \url{https://pyga.me/docs/ref/music.html#pygame.mixer.music.load}

	\item \texttt{pygame.mixer.music.pause()}:
	\myindex{pyg}{\texttt{mixer}!\texttt{music}!\texttt{pause()}}\\ \url{https://pyga.me/docs/ref/music.html#pygame.mixer.music.pause}

	\item \texttt{pygame.mixer.music.play()}:
	\myindex{pyg}{\texttt{mixer}!\texttt{music}!\texttt{play()}}\\ \url{https://pyga.me/docs/ref/music.html#pygame.mixer.music.play}

	\item \texttt{pygame.mixer.music.set\_volume()}:
	\myindex{pyg}{\texttt{mixer}!\texttt{music}!\texttt{set\_volume()}}\\ \url{https://pyga.me/docs/ref/music.html#pygame.mixer.music.set_volume}

	\item \texttt{pygame.mixer.music.unpause()}:
	\myindex{pyg}{\texttt{mixer}!\texttt{music}!\texttt{unpause()}}\\ \url{https://pyga.me/docs/ref/music.html#pygame.mixer.music.unpause}

	\item \texttt{pygame.mixer.Sound}:
	\myindex{pyg}{\texttt{mixer}!\texttt{Sound}}\\ \url{https://pyga.me/docs/ref/mixer.html#pygame.mixer.Sound}

	\item \texttt{pygame.mixer.Sound.get\_volume()}:
	\myindex{pyg}{\texttt{mixer}!\texttt{Sound}!\texttt{get\_volume()}}\\ \url{https://pyga.me/docs/ref/mixer.html#pygame.mixer.Sound.get_volume}

	\item \texttt{pygame.mixer.Sound.play()}:
	\myindex{pyg}{\texttt{mixer}!\texttt{Sound}!\texttt{play()}}\\ \url{https://pyga.me/docs/ref/mixer.html#pygame.mixer.Sound.play}

	\item \texttt{pygame.mixer.Sound.set\_volume()}:
	\myindex{pyg}{\texttt{mixer}!\texttt{Sound}!\texttt{set\_volume()}}\\ \url{https://pyga.me/docs/ref/mixer.html#pygame.mixer.Sound.set_volume}

\end{itemize}
\end{pygbox}

	% !TeX spellcheck = en_US
\newpage
%%%%%%%%%%%%%%%%%%%%%%%%%%%%%%%%%%%%%%%%%%%%%%%%%%%%%%%%%%%%%%%%%%%%%%%%%%%
\section{Events}\label{secEvents}\index{Event}
%%%%%%%%%%%%%%%%%%%%%%%%%%%%%%%%%%%%%%%%%%%%%%%%%%%%%%%%%%%%%%%%%%%%%%%%%%%
\subsection{Introduction}
We have already used events (\gls{event}) in two places without examining them in more detail.  On the one hand, this happened when we talked about the keyboard in \kapref[vref]{secTastatur}, and on the other hand when we discussed the mouse in \kapref[vref]{secMaus}. 

\begin{hintbox}[Here, we will take a closer look at three aspects]
\begin{itemize}
	\item What information is actually contained in an event?
	\item How can I create an event myself?
	\item How can events be generated periodically?
\end{itemize}
\end{hintbox}
 

%%%%%%%%%%%%%%%%%%%%%%%%%%%%%%%%%%%%%%%%%%%%%%%%%%%%%%%%%%%%%%%%%%%%%%%%%%%
\subsubsection{What Information is Contained in an Event?}
\begin{diskbox}
	\url{https://github.com/adamsralf/pygame_book/tree/main/src/00%20Introduction/12%20Events/event00}
\end{diskbox}
The program shown in \srcref[vref]{srcEvent00a} simply creates a gray window and prints the event to the console using \texttt{print()} in \zeiref{srcEvents0001}.

\lstsource{SRC/00 Introduction/12 Events/event00/events.py}{29}{36}{python}{Events -- outputting information}{srcEvent00a} 

If you now move the mouse back and forth, press a few keys, or close the application, something like the following will appear in the console. Many redundant lines have been removed here:

\newpage
\lstsource{console00.txt}{1}{999}{python}{Events -- console output}{srcConsole00a} 

At first, it becomes apparent that the event information is provided in the form of a dictionary. The first entry (the number with a hyphen followed by a name) can be accessed via \texttt{event.type}\myindex{pyg}{\texttt{Event}!\texttt{type}}. So that you do not have to memorize these numbers, Pygame provides corresponding constants; an overview for the keyboard can be found in \tabref[vref]{tabKey}, and for the mouse in \tabref[vref]{tabMousebutton}.

The key/value pairs inside the curly braces contain the information associated with the event.  For keyboard events, this includes, for example, the representation as a Unicode character or its Unicode number. Mouse events are sensibly provided with the position and the button number. Clicking the \emph{window close} button in the upper right corner triggers several events; the last four of the list are shown here.

We will soon see that, for user-defined events, this information can be defined according to our own requirements.


%%%%%%%%%%%%%%%%%%%%%%%%%%%%%%%%%%%%%%%%%%%%%%%%%%%%%%%%%%%%%%%%%%%%%%%%%%%
\subsubsection{How can I Create and Use User-defined Events?}
\begin{diskbox}
	\url{https://github.com/adamsralf/pygame_book/tree/main/src/00%20Introduction/12%20Events/event01}
\end{diskbox}
As an example, I will use two simple buttons here, each of which should generate an event when the left mouse button is clicked. Inside the screen, \texttt{NOFSTARTPARTICLES} many particles move around. Using the \texttt{Stop} and \texttt{Start} buttons, the particles can be stopped and started again.

\myebild{event01.png}{0.7}{User-defined events}{picEvent01}

As an additional feature, a kind of counter is implemented. The boxes in the center absorb the particles and count them. The logic works as follows: Each time a particle hits a box, a counting event is triggered. In this process, a~1 is always added to the box on the far right.

When the rightmost box reaches the value~10, it generates an overflow to the next digit to its left and resets itself to~0. This process continues from right to left. In this way, the boxes display the total number of particles that have already been absorbed.

Now let us look at the whole setup in detail. In the console output above (see page~\pageref{srcConsole00a}), a unique number can be seen for each event, which can be used to identify the event. 

\begin{hintbox}
	Pygame reserves a range of numbers for user-defined events between the constants \texttt{pygame.USEREVENT}\myindex{pyg}{\texttt{USEREVENT}|underline} and \texttt{pygame.NUMEVENTS~-~1}\myindex{pyg}{\texttt{NUMEVENTS}|underline}. 
\end{hintbox}
For each user-defined event, such a unique number must be assigned. The simplest approach is to define these centrally using \texttt{USEREVENT~+~}$n$. You can find corresponding examples in \zeiref{srcEvent0101} and \zeiref{srcEvent0102}.  

I encapsulate these definitions in a static class for no other reason than that it allows me to make good use of the editor’s auto-completion (\zeiref{srcEvent0100}). 

\lstsource{SRC/00 Introduction/12 Events/event01/config.py}{1}{99}{python}{Events (2) -- \texttt{config.py}}{srcEvent01a}

The \texttt{Button} class should also be understandable for the most part. The first interesting section can be found in \zeiref{srcEvent0103}. Here, a new \texttt{pygame.event.Event}\randnotiz{Event} object\myindex{pyg}{\texttt{event}!\texttt{Event}|underline} is created. As the first parameter, the previously mentioned ID must be specified. After that, any number of additional pieces of information can be passed as event data. In our example, the button text is included so that it can later be determined which button was pressed.

Afterwards, in \zeiref{srcEvent0104}, the event is dispatched using \texttt{pygame.event.post()}\randnotiz{post()}\myindex{pyg}{\texttt{event}!\texttt{post()}|underline}.

\lstsource{SRC/00 Introduction/12 Events/event01/events.py}{12}{27}{python}{Events (2) -- Class \texttt{Button}}{srcEvent01b}

The \texttt{Particle} class consists of a lot of source code with little that is new. Particles of random size, color, direction, and speed move across the screen and may bounce off the edges. They do not contain any event-specific functionality. The attribute \texttt{halted} is used to stop the particle or let it move again after the buttons have been pressed.

\lstsource{SRC/00 Introduction/12 Events/event01/events.py}{30}{61}{python}{Events (2) -- Class \texttt{Particle}}{srcEvent01c}

With \texttt{Box}, a kind of digit box is implemented. The constructor receives a position and an index as parameters. The meaning of the parameter \texttt{position} should be clear. Using \texttt{index}, it can later be determined which box caused an overflow to the next higher power of ten.

In \texttt{update()}, the internal counter \texttt{count} is increased by~1 each time. If the value~10 is reached (\zeiref{srcEvent0105}), an event is generated and the index is passed as event data. This allows the main program to determine which box now needs to receive an \texttt{update()} call.

\lstsource{SRC/00 Introduction/12 Events/event01/events.py}{64}{89}{python}{Events (2) -- Klasse \texttt{Box}}{srcEvent01d}

And now the main program: In the constructor, the buttons, boxes, and particles are created and assigned to sprite groups.

\lstsource{SRC/00 Introduction/12 Events/event01/events.py}{92}{108}{python}{Events (2) --  Constructor of \texttt{Game}}{srcEvent01e}

The \texttt{run()} method is almost boring.

\lstsource{SRC/00 Introduction/12 Events/event01/events.py}{110}{120}{python}{Events (2) --  \texttt{Game.run()}}{srcEvent01f}

\begin{hintbox}
	User-defined events are handled in exactly the same way as predefined ones. 
\end{hintbox}

First, you check the \texttt{type}, and then you process the event data. In \zeiref{srcEvent0106}, it is checked whether one of the two buttons was pressed. Afterwards, the message is forwarded to the particles via the event field \texttt{text}, telling them whether they should stop or keep moving. The same idea is used starting at \zeiref{srcEvent0107}. First, it is checked whether a box has overflowed, and then the next box is informed -- using the event field \texttt{index} -- that it has to increase by~1.

\lstsource{SRC/00 Introduction/12 Events/event01/events.py}{122}{136}{python}{Events (2) --  \texttt{Game.watch\_for\_events()}}{srcEvent01g}

The rest is shown here for completeness.

\lstsource{SRC/00 Introduction/12 Events/event01/events.py}{138}{164}{python}{Events (2) --  The Rest of \texttt{Game}}{srcEvent01h}

Finally, I would like to briefly explain the whole mechanism of the counter again using \abbref[vref]{picCounter01}, in order to make it clearer how the event \texttt{MyEvent.OVER\-FLOW} works in this context.

\begin{itemize}
	\item $t_0$: First, a standard collision check determines that a particle has hit the rectangle of the rightmost field (see \abbref[vref]{picEvent01}, the box with the~6). As a result, the method \verb+update(counter="inc")+ is called.
	
	\item $t_1$: This causes the event \texttt{MyEvents.OVERFLOW} to be triggered in the \texttt{Box} class with the index value~0. This event is caught in \texttt{watch\_for\_event()} and forwarded to the appropriate box -- that is, the one with index \texttt{index+1} -- together with the instruction to also execute \verb+update(counter="inc")+ there.
	
	\item $t_2$: Since this box currently contains the value~9, \texttt{MyEvents.OVERFLOW} is triggered again in this box, but now with the next index value, namely~1. This event is again caught in \texttt{watch\_for\_event()} and forwarded to box~2 with \verb+update(counter="inc")+.
	
	\item $t_3$: The value of the leftmost box is currently~0 and is increased by~1 by the call to \verb+update(counter="inc")+. Since no overflow is generated in this case, the chain of events stops here.
\end{itemize}


\begin{figure}[H]
	\begin{center}
		\begin{tikzpicture}[font=\small,>=Latex]
			
			% Styles
			\tikzset{
				digit/.style={draw, rounded corners=2pt, minimum width=1.2cm, minimum height=1.0cm, align=center},
				label/.style={font=\scriptsize, text=black!70},
				arrow/.style={->, thick}
			}
			
			\node[label] at (1.0,0.9) {$10^2$};
			\node[label] at (2.8,0.9) {$10^1$};
			\node[label] at (4.6,0.9) {$10^0$};
			
			
			% --- t0 ---
			\node[label] at      (0.0,0) {$t_0$};
			\node[digit] (h1) at (1.0,0) {0};
			\node[digit] (z1) at (2.8,0) {9};
			\node[digit] (e1) at (4.6,0) {(9+1)};
			
			% --- Pfeil ---
			\draw[arrow] (10.0,0) -- (5.5,0) node[midway, above] {\texttt{update(counter="inc")}};
			
			% --- t1 ---
			\node[label] at      (0.0,-2.0) {$t_1$};
			\node[digit] (h2) at (1.0,-2.0) {0};
			\node[digit] (z2) at (2.8,-2.0) {(9+1)};
			\node[digit] (e2) at (4.6,-2.0) {0};
			
			% --- t2 ---
			\node[label] at      (0.0,-4.0) {$t_2$};
			\node[digit] (h3) at (1.0,-4.0) {(0+1)};
			\node[digit] (z3) at (2.8,-4.0) {0};
			\node[digit] (e3) at (4.6,-4.0) {0};
			
			% --- t3 ---
			\node[label] at      (0.0,-5.5) {$t_3$};
			\node[digit] (h4) at (1.0,-5.5) {1};
			\node[digit] (z4) at (2.8,-5.5) {0};
			\node[digit] (e4) at (4.6,-5.5) {0};
			
			% --- Übertrags-Pfeile ---
			\draw[arrow, bend right=30] (e2.north) to node[above, label] {OVERFLOW} (z2.north);
			\draw[arrow, bend right=30] (z3.north) to node[above, label] {OVERFLOW} (h3.north);
		\end{tikzpicture}
		\caption{How the counter works}\label{picCounter01}
	\end{center}
\end{figure}


%%%%%%%%%%%%%%%%%%%%%%%%%%%%%%%%%%%%%%%%%%%%%%%%%%%%%%%%%%%%%%%%%%%%%%%%%%%
\subsection{More Input}
%%%%%%%%%%%%%%%%%%%%%%%%%%%%%%%%%%%%%%%%%%%%%%%%%%%%%%%%%%%%%%%%%%%%%%%%%%%
\subsubsection{How can periodic events be generated?}\label{eventtime}
\begin{diskbox}
	\url{https://github.com/adamsralf/pygame_book/tree/main/src/00%20Introduction/12%20Events/event02}
\end{diskbox}

This is actually quite simple. The previous example is extended so that new particles are created at intervals of \SI{500}{ms}. To achieve this, a new ID \texttt{NEWPARTICLES} is first defined for the user event.

\lstsource{SRC/00 Introduction/12 Events/event02/config.py}{4}{99}{python}{Events (3) --  \texttt{config.py}}{srcEvent02a}

In the constructor of \texttt{Game}, a periodic timer is set in \zeiref{srcEvent0201} using \texttt{pygame.time.set\-\_timer()}\randnotiz{set\_timer()}\myindex{pyg}{\texttt{time}!\texttt{set\_timer()}|underline}. This timer fires the corresponding event ID every~\SI{500}{ms}.

\lstsource{SRC/00 Introduction/12 Events/event02/events.py}{117}{119}{python}{Events (3) -- \texttt{Timer} in the constructor of \texttt{Game}}{srcEvent02b}

Like the other events, this one is caught in \texttt{watch\_for\_event()} (\zeiref{srcEvent0202}) and processed. In this case, this is done by calling the method \texttt{generate\_particles()}.

\lstsource{SRC/00 Introduction/12 Events/event02/events.py}{148}{149}{python}{Events (3) -- Catching a periodical event}{srcEvent02c}


%%%%%%%%%%%%%%%%%%%%%%%%%%%%%%%%%%%%%%%%%%%%%%%%%%%%%%%%%%%%%%%%%%%%%%%%%%%
\subsubsection{Structuring the Event Loop Correctly}

\begin{hintbox}[Best practice]
In each frame, the event loop should follow a clear and consistent structure:
\begin{enumerate}
	\item Retrieve all events from the event queue
	\item Process the events
	\item Update the game state
	\item Render the scene
\end{enumerate}
\end{hintbox}

\lstset{firstnumber=1}
\begin{lstlisting}
...
	for event in pygame.event.get():
		if event.type == pygame.QUIT:
			running = False
	handle_event(event)
...
	update()
	draw()
\end{lstlisting}

\begin{warningbox}[Common mistakes]
	\begin{itemize}
		\item Failing to regularly process events can cause the application window to become unresponsive.
		\item Forgetting to call \texttt{pygame.event.get()} or processing events only sporadically.\myindex{pyg}{\texttt{event}!\texttt{get()}}
		\item  Using \texttt{event.wait()} in the main loop, which can block rendering and updates.\myindex{pyg}{\texttt{event}!\texttt{wait()}}
	\end{itemize}
\end{warningbox}


%%%%%%%%%%%%%%%%%%%%%%%%%%%%%%%%%%%%%%%%%%%%%%%%%%%%%%%%%%%%%%%%%%%%%%%%%%%
\subsubsection{Choosing the Right Event Retrieval Method}

\begin{hintbox}[Pygame provides different ways to access events:]
\begin{itemize}
	\item \texttt{pygame.event.get()}\myindex{pyg}{\texttt{event}!\texttt{get()}} retrieves all pending events (standard choice for games)
	\item \texttt{pygame.event.poll()}\myindex{pyg}{\texttt{event}!\texttt{poll()}} retrieves a single event
	\item \texttt{pygame.event.wait()}\myindex{pyg}{\texttt{event}!\texttt{wait()}} blocks until an event occurs
\end{itemize}
\end{hintbox}

%%%%%%%%%%%%%%%%%%%%%%%%%%%%%%%%%%%%%%%%%%%%%%%%%%%%%%%%%%%%%%%%%%%%%%%%%%%
\subsubsection{Avoid Generating Events Every Frame}

\begin{warningbox}[Events should represent state changes, not continuous states]
	Common mistake: Posting custom events or playing sounds inside the \texttt{update()} method every frame, which can flood the event queue and cause performance issues. Generate events only when a condition changes (edge-triggered behavior).
\end{warningbox}


\lstset{firstnumber=1}
\begin{lstlisting}
	if not was_pressed and is_pressed:
		pygame.event.post(pygame.event.Event(MyEvents.FIRE))
		was_pressed = is_pressed
\end{lstlisting}


%%%%%%%%%%%%%%%%%%%%%%%%%%%%%%%%%%%%%%%%%%%%%%%%%%%%%%%%%%%%%%%%%%%%%%%%%%%
\subsubsection{Defining Event Data Clearly and Consistently}

When creating custom events, meaningful and consistent event data should be attached.

\lstset{firstnumber=1}
\begin{lstlisting}
	pygame.event.post(pygame.event.Event(MyEvents.BUTTON, {"action": "start"}))
\end{lstlisting}

Use clear, descriptive keys (e.g. action, index, pos) and stick to a consistent naming scheme (snake\_case). Using inconsistent field names across different events, making event handling error-prone.

%%%%%%%%%%%%%%%%%%%%%%%%%%%%%%%%%%%%%%%%%%%%%%%%%%%%%%%%%%%%%%%%%%%%%%%%%%%
\newpage
\subsubsection{Managing User-Defined Event IDs}

\begin{hintbox}[Best practice]
	Define all custom event IDs centrally, for example in a configuration file.
\end{hintbox}

\lstset{firstnumber=1}
\begin{lstlisting}
BUTTON_EVENT   = pygame.USEREVENT + 1
OVERFLOW_EVENT = pygame.USEREVENT + 2
NEWPARTICLES   = pygame.USEREVENT + 3
\end{lstlisting}


%%%%%%%%%%%%%%%%%%%%%%%%%%%%%%%%%%%%%%%%%%%%%%%%%%%%%%%%%%%%%%%%%%%%%%%%%%%
\subsubsection{Use \texttt{set\_timer()} Correctly!}
\begin{hintbox}[Best practice]
	Set timers only once (e.g. in the constructor) and disable timers when they are no longer needed\myindex{pyg}{\texttt{time}!\texttt{set\_timer()}}.
\end{hintbox}

\lstset{firstnumber=1}
\begin{lstlisting}
	pygame.time.set_timer(NEWPARTICLES, 0)
\end{lstlisting}

Setting the same timer multiple times and forgetting to disable timers when restarting a game or switching scenes.

%%%%%%%%%%%%%%%%%%%%%%%%%%%%%%%%%%%%%%%%%%%%%%%%%%%%%%%%%%%%%%%%%%%%%%%%%%%
\subsubsection{Filtering Events for Performance}

In event-heavy applications, it may be useful to restrict which events are allowed.

\lstset{firstnumber=1}
\begin{lstlisting}
    pygame.event.set_allowed([pygame.QUIT, 
                              pygame.KEYDOWN, 
                              pygame.MOUSEBUTTONDOWN, 
                              NEWPARTICLES])
\end{lstlisting} 

Use event filtering sparingly and only when necessary. Blocking too many events and accidentally preventing essential input from being processed.

%%%%%%%%%%%%%%%%%%%%%%%%%%%%%%%%%%%%%%%%%%%%%%%%%%%%%%%%%%%%%%%%%%%%%%%%%%%
\subsubsection{Event-Based Input vs. State-Based Input}

There are two complementary approaches to input handling:
\begin{enumerate}
	\item Event-based input: Reacts to discrete events (\texttt{KEYDOWN}, \texttt{KEYUP}) and is ideal for actions like shooting, opening menus, or triggering sounds.
	
	\item State-based input: Uses continuous state queries (\texttt{key.get\_pressed()}) and can be used for movement and continuous control.
\end{enumerate}

Combine both approaches appropriately. 

%%%%%%%%%%%%%%%%%%%%%%%%%%%%%%%%%%%%%%%%%%%%%%%%%%%%%%%%%%%%%%%%%%%%%%%%%%%
\subsubsection{Window Focus and Application State}

\begin{hintbox}[Best practice]
	Games should respond appropriately when the window loses focus or is minimized. Pause the game or mute sound when focus is lost and always handle the QUIT event reliably.
\end{hintbox}

%%%%%%%%%%%%%%%%%%%%%%%%%%%%%%%%%%%%%%%%%%%%%%%%%%%%%%%%%%%%%%%%%%%%%%%%%%%
\subsubsection{Debugging Events Effectively}

Printing events to the console is useful during development but should be done selectively.

\lstset{firstnumber=1}
\begin{lstlisting}
	for event in pygame.event.get():
		if event.type != pygame.MOUSEMOTION:
			print(event)
\end{lstlisting}			

\begin{warningbox}[Common mistake]
	Logging every mouse movement event, which can overwhelm the console and reduce performance.
\end{warningbox}


%%%%%%%%%%%%%%%%%%%%%%%%%%%%%%%%%%%%%%%%%%%%%%%%%%%%%%%%%%%%%%%%%%%%%%%%%%%
\subsubsection{Structuring Event Handling Code}

\begin{hintbox}[Best practice]
	As projects grow, large if–elif blocks become hard to maintain. Use handler functions or dispatch tables.
\end{hintbox}

\lstset{firstnumber=1}
\begin{lstlisting}
	handlers = {
		pygame.KEYDOWN: handle_keydown,
		pygame.MOUSEBUTTONDOWN: handle_mouse,
		NEWPARTICLES: handle_newparticles,
	}

	for event in pygame.event.get():
		handlers.get(event.type, handle_default)(event)
\end{lstlisting}

This approach improves readability and scalability.

%%%%%%%%%%%%%%%%%%%%%%%%%%%%%%%%%%%%%%%%%%%%%%%%%%%%%%%%%%%%%%%%%%%%%%%%%%%
\subsection{What was new?}
\begin{hintbox}
The advantage of user-defined events becomes very clear here. If this were implemented in a different way, the objects would have to know about each other. For example, all boxes would have to know their predecessor or successor via references in order to report an overflow. While this can also be a valid approach, using events decouples the classes, and the main program can control and organize the forwarding of information via the event data.

In particular, clicking on the buttons can be implemented very easily using events.
\end{hintbox}

\begin{pygbox}
\begin{itemize}
	\item \texttt{USEREVENT}
    \myindex{pyg}{\texttt{USEREVENT}}:\\
    \url{https://pyga.me/docs/ref/event.html#pygame.event}

    \item \texttt{NUMEVENTS}
    \myindex{pyg}{\texttt{NUMEVENTS}}:\\
    \url{https://pyga.me/docs/ref/event.html#pygame.event}

	\item \texttt{pygame.event.Event}:
	\myindex{pyg}{\texttt{event}!\texttt{Event}}\\
	\url{https://pyga.me/docs/ref/event.html#pygame.event.Event}
	
	\item \texttt{pygame.event.get()}
    \myindex{pyg}{\texttt{event}!\texttt{get()}}:\\
    \url{https://pyga.me/docs/ref/event.html#pygame.event.get}

	\item \texttt{pygame.event.poll()}
	\myindex{pyg}{\texttt{event}!\texttt{poll()}}:\\
	\url{https://pyga.me/docs/ref/event.html#pygame.event.poll}

	\item \texttt{pygame.event.post()}:
	\myindex{pyg}{\texttt{event}!\texttt{post()}}\\
	\url{https://pyga.me/docs/ref/event.html#pygame.event.post}
	
	\item \texttt{pygame.event.wait()}:
	\myindex{pyg}{\texttt{event}!\texttt{wait()}}\\
	\url{https://pyga.me/docs/ref/event.html#pygame.event.wait}

	\item \texttt{pygame.time.set\_allowed()}:
	\myindex{pyg}{\texttt{time}!\texttt{set\_allowed()}}\\
	\url{https://pyga.me/docs/ref/time.html#pygame.time.set_allowed}

	\item \texttt{pygame.time.set\_timer()}:
	\myindex{pyg}{\texttt{time}!\texttt{set\_timer()}}\\
	\url{https://pyga.me/docs/ref/time.html#pygame.time.set_timer}

	\item \texttt{pygame.WINDOWPOS\_CENTERED}:
	\myindex{pyg}{\texttt{locals}!\texttt{WINDOWPOS\_CENTERED}}\\
	\url{https://pyga.me/docs/ref/window.html#pygame.Window.position}
\end{itemize}
\end{pygbox}



\chapter{Techniques}\label{secTechniques}
	% !TeX spellcheck = en_US
%%%%%%%%%%%%%%%%%%%%%%%%%%%%%%%%%%%%%%%%%%%%%%%%%%%%%%%%%%%%%%%%%%%%%%%%%%%
\section{Animation}\index{Animation}
An animation is essentially a small \emph{movie} inside a game. Examples of useful animations include movements, explosions, pulsing effects, and changes in appearance. Here, I would like to present two examples: a small movement and an explosion.

%%%%%%%%%%%%%%%%%%%%%%%%%%%%%%%%%%%%%%%%%%%%%%%%%%%%%%%%%%%%%%%%%%%%%%%%%%%
\subsection{The running cat}

\myebild{animation00.png}{0.8}{Animation of a cat: frame sprites}{picAnimation00}

You can see the individual frames of the movement example in \abbref[vref]{picAnimation00}. If these individual sprites are displayed one after another at a certain speed, they appear as a smooth movement. The following rule applies: the more individual frames are used, the smoother the animation appears.



At first the \texttt{config.py}:

\lstsource{SRC/01 Techniques/01 Animation/config.py}{1}{99}{python}{The running cat, \texttt{config.py}}{srcAnimation00a} 

The source code in \srcref[vref]{srcAnimation00a} differs from the chapter above (see \secref[vref]{secClassTimer} by only one feature. The \texttt{Timer} class has been extended by the method \texttt{change\_duration()}. This method makes it possible to change the duration of the time interval at runtime, with a lower limit of~\SI{0}{ms}. We will use this feature shortly to manually adjust the animation speed.

\lstsource{SRC/01 Techniques/01 Animation/animation00.py}{9}{27}{python}{The running cat (1), Version 1.0: \texttt{Timer}}{srcAnimation00a} 

If we want to animate something, this animation does not require just a single sprite for display, but several. For this reason, in addition to the \texttt{image} attribute, I introduced another one: the list \texttt{images}. Using a \forSchleife\ starting at \zeiref{srcAnimation0001}, I now load all bitmaps of the animation into this list.

We now need an attribute that keeps track of which of the 6~sprites should currently be displayed: \texttt{imageindex}. If the images are stored in the \texttt{images} array in the same order in which they are supposed to be displayed, \texttt{imageindex} only needs to be incremented. We also need a \texttt{Timer} object so that the animation does not run absurdly fast -- we start here with \SI{100}{ms}.

In the \texttt{update()} method, the \texttt{imageindex} attribute is incremented by~1 depending on the \texttt{Timer} object, and the corresponding bitmap is then assigned to the \texttt{image} attribute so that the familiar \texttt{Sprite} features can be used. The method \texttt{change\_animation\_time()} simply forwards its parameter to the \texttt{Timer} object. With this, all preparatory steps are essentially complete.

\lstsource{SRC/01 Techniques/01 Animation/animation00.py}{30}{56}{python}{The running cat (2), Version 1.0: \texttt{Cat}}{srcAnimation00b} 

The \texttt{CatAnimation} class is merely the usual encapsulation of the main program. In \zeiref{srcAnimation0002}, the \texttt{Cat} object is created and placed into a \texttt{GroupSingle}.

\lstsource{SRC/01 Techniques/01 Animation/animation00.py}{59}{82}{python}{The running cat (3), Version 1.0: Constructor and \texttt{run()}}{srcAnimation00c} 

In \texttt{watch\_for\_events()}, the only noteworthy aspect is that \keys{{+}} and the \keys{-} key are used to manipulate the animation speed. To increase the animation speed, the time interval of the \texttt{Timer} object has to be reduced, hence~\texttt{-10}. To slow down the animation, the time interval of the \texttt{Timer} object has to be increased, hence~\texttt{+10}.

\lstsource{SRC/01 Techniques/01 Animation/animation00.py}{84}{94}{python}{The running cat (4), Version 1.0: \texttt{watch\_for\_events()}}{srcAnimation00d} 

The remaining source code (\srcref[vref]{srcAnimation00e}) should be self-explanatory. When you start the program, an animated cat movement will be displayed. Feel free to try changing the animation speed.

\lstsource{SRC/01 Techniques/01 Animation/animation00.py}{96}{107}{python}{The running cat (5), Version 1.0: \texttt{update()} and \texttt{draw()}}{srcAnimation00e} 

%%%%%%%%%%%%%%%%%%%%%%%%%%%%%%%%%%%%%%%%%%%%%%%%%%%%%%%%%%%%%%%%%%%%%%%%%%%
\subsection{The Class Animation}

As with time control, I am bothered by the fact that the animation logic is spread across the \texttt{Cat} class, which in my opinion violates the Single Responsibility Principle (SRP). So let us simply build a dedicated animation class (see \srcref[vref]{srcAnimation01a}).

Let us take a look at the constructor parameters:
\begin{itemize}
	\item \textbf{namelist}: A list of file names without path information. These are resolved automatically using the entries in \texttt{config.py}. The order of the file names must correspond to the animation order.
	
	\item \textbf{endless}: This flag controls whether the animation repeats indefinitely.  
	\true\ means that after the last sprite, the animation starts again with the first one.  
	\false\ means that the last sprite remains displayed.
	
	\item \textbf{animationtime}: The delay between individual sprites in~\unit{ms}.
	
	\item \textbf{colorkey}: This parameter handles the case where sprites may not have transparency and therefore require an explicit transparency color (see page~\pageref{pageTransparenz}).  
	If no value is provided, the transparency of the loaded sprite is kept as is.  
	If a color value is provided, it is applied using \texttt{set\_colorkey()} in \zeiref{srcAnimation0101}.
\end{itemize}

In the \texttt{next()} method, the next \texttt{imageindex} is calculated and the corresponding sprite is returned. For this purpose, the internal \texttt{Timer} object is used so that the sprites appear with a defined time interval. The \texttt{imageindex} attribute is increased by~$1$ and then checked to see whether the end of the sprite list has been reached. If the animation is set to \emph{endless}, the \texttt{imageindex} is reset to~$0$; otherwise, the last image of the list is displayed permanently.

Question to the audience: Why was \texttt{imageindex} initialized to~$-1$ in the constructor?

A feature that is often needed has been implemented in the \texttt{is\_ended()} method. Frequently, the code that triggered the animation needs to know whether the animation has finished. We will make use of this later on.

\lstsource{SRC/01 Techniques/01 Animation/animation01.py}{10}{38}{python}{The running cat (6), Version 1.1: \texttt{Animation}}{srcAnimation01a} 

This simplifies the \texttt{Cat} class, allowing it to focus again on its -- admittedly still non-existent -- game logic. The \texttt{Animation} object is created here in \zeiref{srcAnimation0102}. The file names can be generated very easily, since they are numbered consecutively. The cat is supposed to run endlessly, with a time interval of \SI{100}{ms} between the sprites. In \texttt{update()}, the \texttt{next()} method is then simply called.


\lstsource{SRC/01 Techniques/01 Animation/animation01.py}{62}{78}{python}{The running cat (7), Version 1.1: \texttt{Cat}}{srcAnimation01b} 

%%%%%%%%%%%%%%%%%%%%%%%%%%%%%%%%%%%%%%%%%%%%%%%%%%%%%%%%%%%%%%%%%%%%%%%%%%%
\subsection{The Exploding Rock}

My second example spawns rocks (meteors) at random positions and at random time intervals. Each rock is also given a certain lifetime — again chosen randomly. After that, it explodes. This explosion is animated.

Let us first take a look at the \texttt{Rock} class. In \zeiref{srcAnimation0201}, a random number is generated, which is then used in the following line to load one of four possible rock bitmaps. After that, the coordinates of the rock’s center are determined using a random number generator, while keeping a certain distance from the screen borders. In \zeiref{srcAnimation0202}, the \texttt{Animation} object is created. Here again, the file names of the animation bitmaps are loaded in the order of the animation. You can see these bitmaps in \abbref[vref]{picAnimation01}.

Since the animation should not repeat, the corresponding parameter is set to \false\ here. After the explosion, the rock is supposed to disappear. The delay between the individual frames is set to \SI{50}{ms}. In \zeiref{srcAnimation0203}, the lifetime of the rock is again determined randomly and a corresponding \texttt{Timer} object is created -- as you can see, these are quite useful and can be reused often. The flag \texttt{bumm} is a marker that indicates whether the rock is currently exploding.

The \texttt{update()} method has now become quite interesting. First, the \texttt{Timer} object is used to check whether the end of the lifetime has been reached. If not, nothing happens here, although one could implement movement or some other meaningful behaviour in the \texttt{else} branch. If the lifetime has been reached, the corresponding flag is set. Depending on this, the animation is then started.

What is the purpose of the three lines starting at \zeiref{srcAnimation0204}? They serve purely visual purposes. The dimensions of the explosion sprites are not always the same, and the \texttt{rect} object always aligns them to the upper-left corner, which would result in a visible jitter. To avoid this, the old center position is stored, the new rectangle of the next animation sprite is calculated, and its center is set to the previous position. This keeps the animation nicely aligned to the original center of the rock.

Finally, it is checked whether the animation has finished. If so, the sprite is no longer needed and can be removed from the sprite group using \texttt{kill()}\randnotiz{kill()}\myindex{pyg}{\texttt{sprite}!\texttt{Sprite}!\texttt{kill()}}.


\lstsource{SRC/01 Techniques/01 Animation/animation02.py}{63}{85}{python}{The exploding rock (1): \texttt{Rock}}{srcAnimation02a} 

\myebild{animation01.png}{0.8}{The exploding rock: frame sprites}{picAnimation01}

The \texttt{ExplosionAnimation} class should no longer pose any difficulty for you. There are only a few places that I would like to briefly address. In \zeiref{srcAnimation0205}, a \texttt{Timer} object is created that is supposed to spawn two rocks per second, and in \zeiref{srcAnimation0206} this timer is checked.


\lstsource{SRC/01 Techniques/01 Animation/animation02.py}{88}{129}{python}{The exploding roc (2): \texttt{ExplosionAnimation}}{srcAnimation02b} 

Note: There is also the source file \texttt{animation03.py}. In this variant, the rocks move and explode when they collide with each other. Take a look!


	% !TeX spellcheck = en_US
\newpage
\section{Tiles Are Beautiful}
%%%%%%%%%%%%%%%%%%%%%%%%%%%%%%%%%%%%%%%%%%%%%%%%%%%%%%%%%%%%%%%%%%%%%
\begin{diskbox}
	\url{https://github.com/adamsralf/pygame_book/tree/main/src/01%20Techniques/02%20Tilemap}
\end{diskbox}
Very often, the visual appearance of games consists of many small and large tiles that are assembled in an appropriate way. These tiles are usually combined into larger bitmaps (\gls{spritelib}\index{Spritelib}\randnotiz{Spritelib}) and then have to be cut out correctly by the game developer. In \abbref[vref]{picTileMap01}, you can find such a simple sprite library.

I~will now show you how tiles can be cut out of a sprite library and used to assemble your own worlds. The required information is stored in a \Gls{csvdatei}. I~will also use several example levels (not to be confused with difficulty levels or floors within the game world) in order to address different types of sprites in different ways.

\begin{hintbox}[Hint]
	There is an excellent tool that helps you create such levels and integrate them into a game. It is called \Gls{tiled} and can be downloaded from \hyperlink{https://www.mapeditor.org/}{MapEditor}. Since there are already sufficiently detailed and high-quality introductions available for this software, I~will omit a description here.
	
\end{hintbox}

%%%%%%%%%%%%%%%%%%%%%%%%%%%%%%%%%%%%%%%%%%%%%%%%%%%%%%%%%%%%%%%%%%%%%%%%%%%
\subsection{Our Example}

\myezweihbild{TileMap01.png}{0.40}{Sprite library (original)}{picTileMap01}{TileMap02.png}{0.40}{Sprite library (prepared)}{picTileMap02}

In \abbref{picTileMap01}, we can see such a sprite library. It contains tiles for building a landscape consisting of lakes, meadows, and forests, along with some \gls{gadget}\randnotiz{Gadget}. In \abbref{picTileMap02}, I~have made the individual tiles visible by adding a grid and numbering them. The numbers will become important again later. Our tiles have a width and a height of \SI{32}{px} each; they are arranged in \SI{16}{columns} and \SI{12}{rows}.

The goal is a game surface as shown in \abbref[vref]{picTileMap03}. The playing field -- coincidentally -- also has \SI{16}{columns} and \SI{12}{rows}.

\myebild{TileMap03.png}{0.50}{Example of tile based playground}{picTileMap03}

\subsection{A Green Meadow}

In the first step, I~will show how a single tile can be used to fill the entire game area. So let us get started. In \srcref{srcForest01}, we first find the usual suspects. The parameters \texttt{TILESIZE}, \texttt{TILEMAP\_NOF\_COLS}, \texttt{TILEMAP\_NOF\_ROWS}, and \texttt{TILEMAP\_WINDOW} should also be self-explanatory.

\lstsource{SRC/01 Techniques/02 Tilemap/config.py}{1}{14}{python}{Forest -- \texttt{config.py}}{srcForest01}

Also \texttt{Game} is very easy to understand.

\lstsource{SRC/01 Techniques/02 Tilemap/forest01.py}{78}{63}{python}{Forest -- \texttt{Game}}{srcForest02}

So far, everything happens inside the \texttt{WindowGame} class, which will of course change later on. In the constructor, the window is created with all its parameters and the sprite library is loaded. In \texttt{draw()}, a single tile is now cut out of the sprite library and stored in the variable \texttt{image}. The crucial part here is the parameters passed to \texttt{subsurface()}\myindex{pyg}{\texttt{surface}!\texttt{Surface}!\texttt{subsurface()}}\randnotiz{subsurface()}. Starting at position $(0,0)$ in the sprite library -- i.\,e.\ the top-left corner -- a rectangle of size $\SI{32}{px} \times \SI{32}{px}$ is cut out. This corresponds to the yellow-bordered rectangle~0 in \abbref[vref]{picTileMap02}.

Using the two \texttt{for} loops, this image is then distributed across the entire surface. In each loop iteration, the variable \texttt{position} is calculated from the row number \texttt{row} ($[0 - 12]$) and the column number \texttt{col} ($[0 - 16]$).  Multiplying the column number by the tile width yields the x-position, and multiplying the row number by the tile height yields the y-position at which the tile should be drawn.

The result then looks as shown in \abbref[vref]{picTileMap04}.

\newpage
\lstsource{SRC/01 Techniques/02 Tilemap/forest01.py}{7}{24}{python}{Forest -- \texttt{WindowGame}}{srcForest03}

However, I~would like to be able to select and display any arbitrary tile. For example, instead of a monotonous meadow, we might want it to be composed of the tiles with the numbers $0, 1, 2, 3, 4, 16, 17, 18, 19, 20, 32, 33, 34, 35, 36$. This practically calls for a separate class.

In the constructor (see \srcref[vref]{srcForest04}), only the bitmap of the sprite library is loaded. The method \texttt{sub\-sur\-face()} expects the tile number as a parameter.  The tiles are assumed to be numbered from left to right and from top to bottom. This makes it fairly easy to compute the column number and the row number from the tile number.

The column number is the value that remains after all complete rows have been removed. Example: The tile number is~34. I~want to determine the column number. All complete rows above tile~34 are irrelevant. Therefore, with 16~columns, from

\[34 \rightarrow 18 \rightarrow 2.\]

Mathematically, this is the remainder (\gls{modulo}) of an integer division. Do you remember? Elementary school? \emph{15 divided by 6 is 2 remainder 3}.  Or, in our case,
\[34 \bmod 16 = 2.\]

This is exactly what happens in \zeiref{srcforest0201}; the only difference is that the result is then multiplied by the number of pixels per column in order to obtain the left position of the tile.

The row number -- that is, the row in which tile~34 is located -- is determined in a similar, but slightly different way. In other words: how many complete rows -- i.\,e.\ 16 tiles -- are contained in the tile number? For this, we use integer division:
\[34 \div 16 = 2.\]

And since in computer science we always start counting at~0, the tile is indeed located in row~2. Take a look at \abbref[vref]{picTileMap02}! In \zeiref{srcforest0201}, the row number is then multiplied by the row height to obtain the top position of the tile.  The rest should be self-explanatory.

\lstsource{SRC/01 Techniques/02 Tilemap/forest02.py}{8}{16}{python}{Forest -- \texttt{Spritelib}}{srcForest04}

Now we integrate the new class into \texttt{WindowGame}. The constructor is extended by determining random tile numbers:

\lstsource{SRC/01 Techniques/02 Tilemap/forest02.py}{21}{31}{python}{Forest -- Extension of the Constructor of \texttt{WindowGame}}{srcForest05}

In \texttt{WindowGame.draw()}, we now only need to implement access to the tile number. The result then looks as shown in \abbref[vref]{picTileMap05}.

\lstsource{SRC/01 Techniques/02 Tilemap/forest02.py}{33}{41}{python}{Forest -- Selecting specific tile numbers in \texttt{Game.draw()}}{srcForest06}

\myezweihbild{TileMap04.png}{0.50}{Forest playground (1)}{picTileMap04}{TileMap05.png}{0.50}{Forest playground (2)}{picTileMap05}


So, what can we do already?
\begin{enumerate}
	\item We can provide a sprite library to all components of the game via a separate class.
	\item We can determine the row -- and thus the vertical pixel position in the sprite library -- from the tile number.
	\item We can determine the column -- and thus the horizontal pixel position in the sprite library -- from the tile number.
	\item We can compute the upper-left pixel position in the target window -- that is, the game playground -- from a row and column number.
\end{enumerate}

That is already quite a lot.

\subsection{Tile Numbers and Two-Dimensional Arrays}

In most cases, tiles should not be determined by a random generator. Instead, the tile numbers should be specified explicitly in some way. Two-dimensional arrays are well suited for this purpose, since they allow convenient access using row and column indices.

Nothing could be simpler, you might think. And the reader would be right ;-)

First, let us replace the random selection with a meaningful assignment using a two-dimensional array. The tile numbers are no longer random, but already correspond to those that we want to see later.

\lstsource{SRC/01 Techniques/02 Tilemap/forest03.py}{18}{39}{python}{Forest -- Tile numbers in a 2D array}{srcForest09}

The output is now simplified, as can be seen starting at \zeiref{srcforest0301}.

\lstsource{SRC/01 Techniques/02 Tilemap/forest03.py}{42}{50}{python}{Forest -- \texttt{WindowGame.draw()} using a 2D array}{srcForest10}

\begin{warningbox}[The result is a bit unsatisfying.]
\begin{itemize}
	\item Since there is only space for a single tile number at each tile position, background tiles are missing. As a result, an ugly black border becomes visible around these tiles.
	\item The semantic meaning of the tiles is completely lost. This may be acceptable for pure background tiles, but what if some tiles represent ground, others obstacles, and yet others gadgets with a special meaning?
\end{itemize}
\end{warningbox}
I~usually get by with these three semantic layers and encapsulate the whole concept in a \texttt{Map} class.

In the constructor, tile numbers are specified only for these three layers. For collision detection, the main program—or wherever it makes sense—can then always check whether a tile number is contained in \texttt{layer\_data[0]}, \texttt{layer\_data[1]}, or \texttt{layer\_data[2]}. The result with regard to the black backgrounds has also improved, as can be seen in \abbref[vref]{picTileMap07}; since I~can now first draw a green meadow as a background in level~0 and later place a tent or a tree in level~1 or~2.

\lstsource{SRC/01 Techniques/02 Tilemap/forest04.py}{19}{63}{python}{Forest -- \texttt{Map} with a 2D array}{srcForest11}

In \texttt{WindowGame}, the object of the class is now created:

\lstsource{SRC/01 Techniques/02 Tilemap/forest04.py}{73}{74}{python}{Forest -- \texttt{Map} object in \texttt{WindowGame}}{srcForest12}

In \texttt{draw()}, only the \texttt{Map} object is accessed from now on. Unassigned tiles (see \zeiref{srcforest0401}) are skipped in the process.

\lstsource{SRC/01 Techniques/02 Tilemap/forest04.py}{76}{87}{python}{Forest -- \texttt{WindowGame.draw()} using a \texttt{Map} object}{srcForest13}

\myezweihbild{TileMap06.png}{0.50}{Forest playground (3)}{picTileMap06}{TileMap07.png}{0.50}{Forest playground (4)}{picTileMap07}

\begin{warningbox}
	Hard-coding the tile numbers directly in the source code is, of course, a nightmare. 
\end{warningbox}

It is much better to store them in separate CSV files. The appeal of this approach is that it allows the use of external programs such as \emph{Tiled} for level design. Even self-made tools like \texttt{forest00.py} (which can be found in the GitHub repository of this script) are often sufficient and much more efficient than specifying the data directly in the source code.

So let us quickly switch to using CSV files. First, the \texttt{csv} module needs to be imported.

\lstsource{SRC/01 Techniques/02 Tilemap/forest05.py}{1}{1}{python}{Forest -- \texttt{csv} import}{srcForest14}

In the constructor, the CSV files located in the \texttt{level} subdirectory are now loaded. Please note that creating the \texttt{Map} object now requires specifying how many levels are available as CSV files.

\lstsource{SRC/01 Techniques/02 Tilemap/forest05.py}{20}{28}{python}{Forest -- Constructor of \texttt{Map} using \texttt{csv}}{srcForest15}

That should be sufficient at this point :-)











	% !TeX spellcheck = en_US
\newpage
\section{Very Large Worlds}
%%%%%%%%%%%%%%%%%%%%%%%%%%%%%%%%%%%%%%%%%%%%%%%%%%%%%%%%%%%%%%%%%%%%%

In many games, the playable universe is too large to be displayed entirely within a single game window or across the whole screen. Therefore, solutions are required to determine how a section of the game world should be displayed relative to the player’s position.

%%%%%%%%%%%%%%%%%%%%%%%%%%%%%%%%%%%%%%%%%%%%%%%%%%%%%%%%%%%%%%%%%%%%%
\subsection{A Large Example World}
\begin{diskbox}
	\begin{itemize}
		\item \url{https://github.com/adamsralf/pygame_book/tree/main/src/01%20Techniques/03%20WorldScrolling/V01}
		\item \url{https://github.com/adamsralf/pygame_book/tree/main/src/01%20Techniques/03%20WorldScrolling/V02}
	\end{itemize}
\end{diskbox}

Let us take a look at a simple example without any actual gameplay mechanics, so that it does not become too distracting. Our world consists of a large number of square tiles that differ only in their color. The closer a tile is to the center, the whiter its yellow color becomes.

The complete source code is split across several files to maintain clarity. Let us start with the file \texttt{globals.py}. By now, \texttt{FPS} and \texttt{DELTATIME} should be self-explanatory. The constant \texttt{TILESIZE\_WORLD} defines the width and height of a tile in the large world; in our case \SI{24}{px}. \texttt{NOF\_COLS} and \texttt{NOF\_ROWS} specify the number of columns and rows—that is, how many tiles per row and per column exist in the large world. As a result, \texttt{WORLD} becomes a rectangle with a width of \SI{2160}{px} and a height of \SI{1680}{px} in \zeiref{globalsv0101}; this is larger than what can be displayed on most monitors.

The setting \texttt{TILE\_WITH\_BORDER} controls whether the tiles should have an inner border. A value of~0 means \emph{no}, while a value $>0$ specifies the width of the inner border. This makes the individual tiles visible again; otherwise, the world would appear as a large color gradient.

Finally, there is \texttt{NOF\_MOBS}. I~let a few blue rectangles move aimlessly around the world so that something is happening visually and the scene appears a bit more representative of a real game; after all, real games usually contain more than just static elements and the player.


\lstsource{SRC/01 Techniques/03 WorldScrolling/V01/config.py}{1}{99}{python}{Big World --  \texttt{config.py}}{srcSettings01}

Next, we take a look at the three classes \texttt{Tile}, \texttt{Player}, and \texttt{Mob} in \texttt{objects.py}. The \texttt{Tile} class is representative of any type of sprite placed in the world. These can be static background elements, walls in the foreground, or other movable objects. For our considerations here, this distinction does not matter.

In the constructor of \texttt{Tile}, an image with the size \texttt{TILESIZE\_WORLD} is created. Starting at \zeiref{tilev0101}, the relative distance of the tile to the center is calculated. This is possible because the position is passed as a parameter. The value range of \texttt{rel\_dist\_center} lies within the interval~$[0,1]$. The blue component for the yellow color is then computed in \zeiref{tilev0102} and used to color the tile in the following line.

The \texttt{Player} class represents a simple wanderer through the world. It is drawn as a simple red circle centered within a tile-sized image. The radius is chosen such that the circle fills the tile as much as possible. In \texttt{update()}, either the position of the tile in the world is set directly or a new position is calculated based on the chosen direction. Note that the position refers not to the top-left corner, but to the center of the circle.

Finally, there is the \texttt{Mob} class. A blue rectangle is placed at a random position -- at a reasonable distance from the borders. The direction vectors are chosen from a uniform distribution, and the speed is also randomly selected between \SI{100}{px} and \SI{500}{px}. The size of the rectangle is likewise determined by a random padding. In \texttt{update()}, the mob is moved. If it has completely wandered out of the world, it reappears at the opposite edge.

\lstsource{SRC/01 Techniques/03 WorldScrolling/V01/objects.py}{1}{99}{python}{Big World --  \texttt{Tile}, \texttt{Player}, and \texttt{Mob}}{srcObjects01}

Our output window -- i.\,e.\ the first rather unworthy attempt -- is defined in the file \texttt{windows.py}. Not much happens here. A window with the size defined in \texttt{config.py} is created. Finally, the window title is adjusted so that it conveys a bit of information about its properties. In \texttt{draw()}, the window is filled with black, and then all world objects -- background tiles as well as moving objects, including the player -- are rendered. I~added the method \texttt{save()} only for this script, so that I~can capture images of the current states, for example to show them here.


\lstsource{SRC/01 Techniques/03 WorldScrolling/V01/windows.py}{1}{999}{python}{Big World --  \texttt{WindowPlain}}{srcWindows01}

In \texttt{camera\_demo.py}, the class \texttt{Game} is now defined and the call to \texttt{main()} is performed. In the constructor of \texttt{Game}, the functions \texttt{create\_tiles()} and \texttt{create\_mobs()} are called. These functions create the tiles and the moving objects and assign them the correct positions within the world; more on this later. In addition, the \texttt{WindowPlain} object is created as the output window, as well as the player -- i.\,e.\ a \texttt{Player} object.

In \texttt{run()}, the basic structure of the main program loop that I~typically use in my games is defined. There are plenty of explanations for this structure earlier and later in the script.

\newpage
\lstsource{SRC/01 Techniques/03 WorldScrolling/V01/camera_demo.py}{9}{33}{python}{Big World -- Constructor and \texttt{run()} of \texttt{Game}}{srcCamerademo01}

The methods shown in \srcref{srcCamerademo02} also require no further explanation.

\lstsource{SRC/01 Techniques/03 WorldScrolling/V01/camera_demo.py}{35}{67}{python}{Big World -- \texttt{watch\_for\_events()}, \texttt{update()}, and \texttt{draw()} of \texttt{Game}}{srcCamerademo02}

The methods \texttt{create\_tiles()} and \texttt{create\_mobs()} create the game objects -- that is, in our case, the static tiles and the moving game elements. The method \texttt{save()} triggers saving both the entire world and the game window as PNG files. As mentioned above, this functionality exists only to allow images to be included in this script (for example, \abbref[vref]{picPlainWindow03}).

\lstsource{SRC/01 Techniques/03 WorldScrolling/V01/camera_demo.py}{69}{84}{python}{Big World -- \texttt{create\_tiles()}, \texttt{create\_mobs()}, and \texttt{save()} of \texttt{Game}}{srcCamerademo03}

If we take a look at the screen outputs of the current source code in \abbref{picPlainWindow02} and \abbref{picPlainWindow01}, the fundamental problem becomes immediately apparent. The window \texttt{Plain\-Wind\-ow} is far too small to display the entire world (see \abbref[vref]{picPlainWindow03}). We can only see the top-left corner of the large world -- once without borders and once with borders. The visible section is so small that even the color gradient is hardly recognizable. Here, the borders help to make the many tiles visible. Later on, we will no longer need them.

\myezweihbild{plainwindow02.png}{0.35}{Tiles without borders}{picPlainWindow02}{plainwindow01.png}{0.35}{Tiles with borders}{picPlainWindow01}

\begin{figure}[htb]
	\begin{tikzpicture}
		%Bildschirm Koordinatensystem
		\draw (0,10) node [rotate=45] (oo) {(0,0)};
		\draw[->] (oo) -- ++(14.5,0) node[right]{x};
		\draw[->] (oo) -- ++(0,-9.5) node[below]{y};		
		
		%Rechteck Welt
		\node[anchor=north west, inner sep=0] (image) at (oo) {\includegraphics[width=14cm, height=9cm]{world_image_01.png}};
		\draw (14,1) node[rotate=45] (bottomright_w) {(2160,1680)};
		\draw (oo) rectangle (bottomright_w);
		
		%Rechteck Plain
		\draw (3.5,10 - 2.5) node[rotate=45] (bottomright_p) {(580,420)};
		\draw [line width=1.5pt] (oo) rectangle (bottomright_p);
	\end{tikzpicture}
	\caption{PlainWindow as a Viewport of the World}\label{picPlainWindow03}
\end{figure}

\index{Visibility culling}Before continuing with the different top-down views, I~would like to address a \randnotiz{Visibility culling} performance issue. To do so, I~extend the window title to display the actual \texttt{FPS} and increase \texttt{FPS} in \texttt{config.py} to \SI{600}{fps} -- not because this is a realistic value for a game, but because I~want to determine how many frames are actually achieved. I~then store this actually achieved number of frames in a text file.

If we take a look at the method \texttt{draw()} in \srcref[vref]{srcWindows01}, we see that all tiles and all moving objects are rendered into the target window -- that is, drawn -- even though the vast majority of these objects are not visible within the window at all.

One possible approach is to first check all sprites, or rather their rectangles, to see whether they are actually located inside the output window. The list~\texttt{a} is created by iterating over all sprites in the group and, for each one, using \texttt{colliderect()}\myindex{pyg}{\texttt{FRect}!\texttt{colliderect()}} to check whether it lies within or at least touches the rectangle of the window.

\lstsource{SRC/01 Techniques/03 WorldScrolling/V02/windows.py}{18}{28}{python}{Big World --  \texttt{WindowPlain.draw()} with Visibility culling}{srcWindows02}

Afterwards, I~performed the same performance measurement as before, and the result can be seen in \abbref[vref]{picPerformancemitohneprüfung01}. When visibility checking is enabled, significantly more frames per second are achieved than without it. Conclusion: it is worth using.

\myebild{PerformaceMitOhnePrüfung.pdf}{1.0}{Performance without and with visibility culling}{picPerformancemitohneprüfung01} 

%%%%%%%%%%%%%%%%%%%%%%%%%%%%%%%%%%%%%%%%%%%%%%%%%%%%%%%%%%%%%%%%%%%%%
\subsection{Top-Down View / Bird's-Eye View}
\begin{diskbox}
	\begin{itemize}
		\item \url{https://github.com/adamsralf/pygame_book/tree/main/src/01%20Techniques/03%20WorldScrolling/V03}
		\item \url{https://github.com/adamsralf/pygame_book/tree/main/src/01%20Techniques/03%20WorldScrolling/V04}
	\end{itemize}
\end{diskbox}
First of all, one might want to have a complete overview of the entire world. This is not strictly necessary. It is not uncommon for a game to never allow the player to see the whole world at once. However, many games offer a top-down view (Bird's-Eye View)\index{BirdEyeView}.

A first naive approach would be to scale down the world using \texttt{transform.scale\_by()}\myindex{pyg}{\texttt{transform}!\texttt{scale\_by()}} (see \zeiref{windowsv0301}). 

\begin{hintbox}[This approach has the following advantages and disadvantages]
\begin{itemize}
	\item Advantage: Very easy to implement.
	\item Disadvantage: In every frame, the entire oversized bitmap has to be created.
	\item Disadvantage: Scaling can produce undesirable artifacts, since we have no control over which pixels are lost during scaling.
	\item Disadvantage: Scaling can create objects that are only one pixel in size or even smaller, and therefore hardly visible or not visible at all.
\end{itemize}
\end{hintbox}

\lstsource{SRC/01 Techniques/03 WorldScrolling/V03/windows.py}{35}{55}{python}{Big World --  \texttt{WindowBirdEyeView}}{srcWindows03}

%\myebild{birdeye_image_01.png}{0.35}{BirdEye (scaled)}{picBirdEyeImage01}

%\begin{wrapfigure}[14]{r}{8.5cm}%
%	\begin{center}%
%		\vspace{-1cm}%
%		\myfigure{birdeye_image_01.png}{0.45}{BirdEye (scaled)}{picBirdEyeImage01}%
%	\end{center}%
%\end{wrapfigure}%
I~therefore recommend that each sprite provides two or possibly even several variants of its image. This makes it easier to identify objects quickly in the Bird’s-Eye View as well. 

Keep in mind that the purpose of the top-down view is not to allow the game to be played in a full overview after all, but rather to provide orientation, to locate important points of interest, and possibly to identify friends and enemies.

\begin{enumerate}
	\item Each stationary tile is simply displayed in a scaled-down version.
	\item The player is represented by a smaller red circle.
	\item The other moving objects are displayed as equally sized blue squares.
\end{enumerate}

One more thing I~would like to have is the ability to recognize the visible section of the \texttt{PlainWindow} as a rectangle within the Bird’s-Eye View.

Let us implement this: First, the Bird’s-Eye View window is integrated into the main program (see \srcref{srcCamerademo04} and \srcref{srcCamerademo05}). Please note that at no point in the source code is a surface for the entire world created at its original size anymore.

\lstsource{SRC/01 Techniques/03 WorldScrolling/V04/camera_demo.py}{16}{18}{python}{Big World -- \texttt{BirdEyeView} in the constructor of \texttt{Game}}{srcCamerademo04}

In \texttt{draw()}, an additional option is prepared, namely the visualization of which part of the entire world is currently covered by \texttt{PlainWindow}. To do this, I~pass the rectangle of \texttt{PlainWindow} and the border color to the \texttt{draw()} method of \texttt{BirdEyeView} in \zeiref{camerav0401} (see \abbref[vref]{picBirdEyeImage02}).

\lstsource{SRC/01 Techniques/03 WorldScrolling/V04/camera_demo.py}{64}{68}{python}{Big World -- \texttt{BirdEyeView} in \texttt{Game.draw()}}{srcCamerademo05}

This data is then used in \texttt{draw()} of \texttt{BirdEyeView}, and the rectangle(s) are drawn starting at \zeiref{windowsv0401}.

\lstsource{SRC/01 Techniques/03 WorldScrolling/V04/windows.py}{49}{57}{python}{Big World --  \texttt{BirdEyeView.draw()}}{srcWindows04}

%\myebild{birdeye_image_02.png}{0.35}{Bird’s-Eye View (simplified and with visibility indicator)}{picBirdEyeImage02}

One more note on performance: Due to the preparations carried out -- namely the one-time creation of a smaller, symbolic representation of the game elements -- a significant performance improvement was achieved as well. Analogous to the measurements above, I~performed a corresponding benchmark over \SI{660}{frames}. The result can be read from \abbref[vref]{picPerformaceMitOhneVorbereitung01}.

%\myezweihbild{birdeye_image_01.png}{0.35}{BirdEye (scaled)\newline \newline}{picBirdEyeImage01}{birdeye_image_02.png}{0.35}{Bird’s-Eye View (simplified and with visibility indicator)}{picBirdEyeImage02}

\begin{figure}[hbtp]%
	\centering%
	\begin{minipage}[b]{6.5cm}%
		\centering%
		\includegraphics[scale=0.35]{birdeye_image_01.png}%
		\caption[BirdEye (scaled)]{BirdEye (scaled)\newline \newline}\label{birdeye_image_02.png}%
	\end{minipage}%
	\hfil%
	\begin{minipage}[b]{6.5cm}%
		\centering%
		\includegraphics[scale=0.35]{birdeye_image_02.png}%
		\caption{Bird’s-Eye View (simplified and with visibility indicator)}\label{picBirdEyeImage02}%
	\end{minipage}%
\end{figure}%


\myebild{PerformaceMitOhneVorbereitung.pdf}{1.0}{Performance without and with preprocessing}{picPerformaceMitOhneVorbereitung01} 


%%%%%%%%%%%%%%%%%%%%%%%%%%%%%%%%%%%%%%%%%%%%%%%%%%%%%%%%%%%%%%%%%%%%%
\newpage
\subsection{Player Centered Camera}
\begin{diskbox}
	\begin{itemize}
		\item \url{https://github.com/adamsralf/pygame_book/tree/main/src/01%20Techniques/03%20WorldScrolling/V05}
		\item \url{https://github.com/adamsralf/pygame_book/tree/main/src/01%20Techniques/03%20WorldScrolling/V06}
		\item \url{https://github.com/adamsralf/pygame_book/tree/main/src/01%20Techniques/03%20WorldScrolling/V07}
	\end{itemize}
\end{diskbox}

As the next type of view, I~would like to introduce the \emph{Player-Centered Camera}.  
In this view, the player appears to be fixed at the center, while the elements of the game move according to the player’s subjective direction of movement.

The basic idea behind a solution is actually quite simple -- so do not be intimidated by the mathematics! Each player has a position in the world measured from the top-left corner, given by the vector $\vec{P}_W = (P_{Wx}, P_{Wy})$. The point $P_W$ has a distance to the center of the world—that is, the large game world. Why is this important? Because the player is later supposed to appear at the center of the window.

Therefore, we need to find a correction value (\emph{offset}\index{Offset}\randnotiz{Offset}) that transforms the player’s world coordinates into the center coordinates of the camera view window. This offset must be subtracted from the world coordinates, since the coordinates of the view window are much smaller than those of the game world.  

Let $\vec{P}_V$ be the player position in the camera view window and $\vec{O}_{ff}$ the correction value. In other words: the position in the world minus the correction value yields the position in the window. Let us now transform this relationship so that we can compute the correction value.

\begin{align}
	\vec{P}_W - \vec{O}_{ff}  &=& \vec{P}_V                   &\hspace{0.5cm}\| + \vec{O}_{ff}\label{off00}\\
	\vec{P}_W                 &=& \vec{P}_V  + \vec{O}_{ff}   &\hspace{0.5cm}\| - \vec{P}_{V}\nonumber\\
	\vec{P}_W - \vec{P}_V     &=& \vec{O}_{ff}                &\hspace{0.5cm}\| reverse order\nonumber\\
	\vec{O}_{ff}              &=& \vec{P}_W - \vec{P}_V       &\hspace{0.5cm}\| coordinate notation\nonumber\\
	\left(\begin{array}{c} O_{ffx} \\ O_{ffy} \end{array}\right) &=& \left(\begin{array}{c} P_{Wx} \\ P_{Wy} \end{array}\right)  - \left(\begin{array}{c} P_{Vx} \\ P_{Vy} \end{array}\right) &\hspace{0.5cm}\| vector subtraction\nonumber\\
	\left(\begin{array}{c} O_{ffx} \\ O_{ffy} \end{array}\right) &=& \left(\begin{array}{c} P_{Wx} - P_{Vx}\\  P_{Wy} - P_{Vy}\end{array}\right)&\hspace{0.5cm}\label{off01}
\end{align}

\Gleiref{off02} reflects the fact that the new position of our player is supposed to be exactly the center of the window, that is, half the width and half the height:

\begin{align}
	\left(\begin{array}{c} P_{Vx}  \\ P_{Vy}  \end{array}\right) &=& \left(\begin{array}{c} S_{Vx} / 2\\ S_{Vy} / 2 \end{array}\right) & \hspace{0.5cm}\label{off02}
\end{align}

Now we substitute \gleiref{off02} into \gleiref{off01}:

\begin{align}
	\left(\begin{array}{c} O_{ffx} \\ O_{ffy} \end{array}\right) &=& \left(\begin{array}{c}   P_{Wx} - S_{Vx} / 2\\  P_{Wy} - S_{Vy} / 2\end{array}\right) & \hspace{0.5cm}\label{off03}
\end{align}

With this, we have indeed derived -- using nothing more than straightforward mathematics -- the formula for computing the translation, that is, the offset.

It is time for a bit of source code. Let us first prepare everything in the main program. Even though we have not implemented the class yet, we can simply treat the new window like the other two and work with copy/paste. In \zeiref{camerav0501}, the window is created. Please note that this has to happen after creating the \texttt{Player}, because we need its position.

\lstsource{SRC/01 Techniques/03 WorldScrolling/V05/camera_demo.py}{11}{21}{python}{Big World --  Centered Camera in \texttt{Game}}{srcCamerademo06}

In the \texttt{draw()} method, three additions are necessary.  
In \zeiref{camerav0502}, the rectangle for the Bird’s-Eye View is added, showing the visible section of the new window.  
After that, \texttt{draw()} is called as with the other windows and the title line is updated.

\lstsource{SRC/01 Techniques/03 WorldScrolling/V05/camera_demo.py}{66}{74}{python}{Big World --  Centered Camera in \texttt{Game.draw()}}{srcCamerademo07}

And \texttt{save()} also needs to be extended by \zeiref{camerav0504}:

\lstsource{SRC/01 Techniques/03 WorldScrolling/V05/camera_demo.py}{89}{92}{python}{Big World --  Centered Camera in \texttt{Game.save()}}{srcCamerademo09}

Now let us move on to the fun part: the \texttt{WindowCenteredCamera} class.  
First, the self-explanatory \texttt{\_\_init\_\_()}. In addition, two attributes are defined here: \texttt{self.offset} and \texttt{self.player}.  
Using the offset, I~will later compute the new coordinates, and the player is used to compute the offset.

 
\lstsource{SRC/01 Techniques/03 WorldScrolling/V05/windows.py}{72}{82}{python}{Big World -- Constructor of \texttt{WindowCenteredCamera}}{srcWindows05}

The offset is computed in the method \texttt{scroll()}. Keep \gleiref[vref]{off03} next to the source code.  The implementation should be self-explanatory, as it serves as an example of how easily mathematical expressions can be translated into source code.

Why is the last line actually needed? It is not required for scrolling itself. However, by providing this value, I~can inspect the rectangle of the visible world section in the Bird’s-Eye View.

\lstsource{SRC/01 Techniques/03 WorldScrolling/V05/windows.py}{97}{100}{python}{Big World -- \texttt{WindowCenteredCamera.scroll()}}{srcWindows06}

Now we have everything in place to implement the remaining parts. Let us start with two helper methods so that the coordinate transformations do not have to be implemented multiple times. In \texttt{world2camera()}, the coordinates of the game objects in the large world are transformed into the coordinates of the centered view. This calculation corresponds exactly to the initial idea shown in \gleiref[vref]{off00}.

\lstsource{SRC/01 Techniques/03 WorldScrolling/V05/windows.py}{102}{103}{python}{Big World -- \texttt{WindowCenteredCamera.world2camera()}}{srcWindows07}

What remains is the method \texttt{draw()}. This method looks almost the same as the \texttt{draw()} method of the other class. However, here the coordinates of the game objects are transformed using \texttt{camera2world()} before the visibility check. Take a moment to think about why this is necessary!

\lstsource{SRC/01 Techniques/03 WorldScrolling/V05/windows.py}{84}{92}{python}{Big World -- \texttt{WindowCenteredCamera.draw()}}{srcWindows08}

If we now run the source code and move the player to the top-left corner of the game world, we obtain the views shown in \abbref[vref]{picBirdEyeImage04} and \abbref[vref]{picCenteredImage01}. In the left image, the screen section visible in the right image can be identified by the green rectangle. The green rectangle only appears to be smaller than the blue one; in fact, three quarters of the view lie outside the visible area of the Bird’s-Eye View.

\myezweihbild{birdeye_image_04.png}{0.35}{Bird’s-Eye View: Green = Centered}{picBirdEyeImage04}{centered_image_01.png}{0.35}{Centered Camera -- with border error}{picCenteredImage01}

We therefore need to adjust the method \texttt{scroll()} so that the borders are not exceeded. We now see that the offset is limited to~0 at the top and on the left, meaning it cannot become negative. This would otherwise indicate that we are extending beyond the world to the top or to the left. The same logic applies to the right and bottom edges. Here, it is checked whether the right edge of the object exceeds the right edge of the world, and analogously whether this also happens at the bottom. This procedure is known as \gls{clamp}\index{Clamp}\randnotiz{Clamp}.

\newpage
\lstsource{SRC/01 Techniques/03 WorldScrolling/V06/windows.py}{97}{106}{python}{Big World -- \texttt{WindowCenteredCamera.scroll()} with clamping}{srcWindows09}

In \abbref[vref]{picCenteredImage02}, we no longer see any border artifacts.  
Instead, the player’s position has shifted from the center toward the edge—exactly as intended.  
Note: In \abbref[vref]{picBirdEyeImage05}, the blue border of \texttt{PlainWindow} can no longer be seen, since both views now cover the same section.

\myezweihbild{birdeye_image_05.png}{0.35}{BirdEyeView: Grün=Centered}{picBirdEyeImage05}{centered_image_02.png}{0.35}{Centered Camera -- without border error}{picCenteredImage02}

\begin{warningbox}[There is one more thing I~would like to address]
	In the method \texttt{draw()} (see \srcref[vref]{srcWindows08}), the method \texttt{world2camera()} is called for every single game object; that is, thousands of coordinates are transformed. Would it not be more efficient to transform the world coordinates just once and then compare them with the coordinates of the game objects?
\end{warningbox}

To this end, we introduce a new method, appropriately called \texttt{camera2world()}. It is, so to speak, the inverse of \texttt{world2camera()}.

\lstsource{SRC/01 Techniques/03 WorldScrolling/V07/windows.py}{111}{112}{python}{Big World -- \texttt{WindowCenteredCamera.camera2world()} with clamping}{srcWindows10}

I~now integrate this into \texttt{draw()}. I~have left the old computation commented out above, so that the difference can be seen more clearly. Once again, simple reasoning has led to a performance gain (see \abbref[vref]{picPerformaceTransformation01}).

\lstsource{SRC/01 Techniques/03 WorldScrolling/V07/windows.py}{84}{95}{python}{Big World -- \texttt{WindowCenteredCamera.draw()} with clamping}{srcWindows11}

\myebild{PerformaceTransformation.pdf}{1.0}{Performance with element-based and world-based transformation}{picPerformaceTransformation01} 

\newpage
%%%%%%%%%%%%%%%%%%%%%%%%%%%%%%%%%%%%%%%%%%%%%%%%%%%%%%%%%%%%%%%%%%%%%
\subsection{Page Scrolling/Edge Scrolling}
\begin{diskbox}
	\url{https://github.com/adamsralf/pygame_book/tree/main/src/01%20Techniques/03%20WorldScrolling/V08}
\end{diskbox}
Keeping the player permanently centered can lead to a restless and confusing visual effect, depending on the game’s visual style. It is also associated with many transformations that are often unnecessary for the gameplay itself. Page-wise scrolling (edge scrolling\randnotiz{Edge}\index{Scrolling!Edge}\index{Edge Scrolling} or page scrolling\randnotiz{Page}\index{Scrolling!Page}\index{Page Scrolling}) represents a good compromise. The player initially moves normally within the visible game area. Only when a minimum distance (\gls{padding}) to one of the borders is undershot does the background shift in the corresponding direction -- that is, the view scrolls to the next page.

Let us take a look at the constructor of \texttt{WindowPagewise}. Lost of the elements have already been explained above. What is new is the attribute \texttt{inner\_rect} and the parameter \texttt{padding}. This parameter controls the distance between the inner rectangle and the boundaries of the view—that is, the space between the inner brown rectangle and the outer green rectangle in \abbref[vref]{picBirdEyeImage06}. The parameter is -- purely arbitrarily -- an integer here and serves as a factor for calculating the distance. As a second factor, I~chose the width and height of the player. Semantically, this means that a value of~2 results in a padding of two player widths or heights.

\lstsource{SRC/01 Techniques/03 WorldScrolling/V08/windows.py}{112}{130}{python}{Big World -- Constructor of \texttt{WindowPagewise}}{srcWindows12}

The actual work happens in the method \texttt{scroll()}. Here as well, the basic logic is fairly simple. If the player’s rectangle lies within the inner rectangle, no scrolling needs to take place at all; the player simply moves normally. Once the player leaves the inner area, scrolling has to be performed.   How do we test whether the player is still inside the inner rectangle? By checking whether the player no longer collides with the inner rectangle (\zeiref{camerav0801}).

\lstsource{SRC/01 Techniques/03 WorldScrolling/V08/windows.py}{146}{153}{python}{Big World -- \texttt{WindowPagewise.scroll()}}{srcWindows13}

\begin{figure}[hbtp]%
	\centering%
	\begin{minipage}[b]{6.5cm}%
		\centering%
		\includegraphics[scale=0.35]{birdeye_image_06.png}%
		\caption[Bird’s-Eye: Centered, Pagewise, InnerRect]{Bird’s-Eye: \textcolor{green}{Centered}, \textcolor{red}{Pagewise}, \textcolor{brown}{InnerRect}}\label{picBirdEyeImage06}%
	\end{minipage}%
	\hfil%
	\begin{minipage}[b]{6.5cm}%
		\centering%
		\includegraphics[scale=0.35]{pagewise_image_01.png}%
		\caption[Page / Edge Scrolling]{Page / Edge Scrolling\newline}\label{picPagewiseImage01}%
	\end{minipage}%
\end{figure}%

Integrating the new class is done in exactly the same way as integrating \texttt{Wind\-ow\-Cen\-tered\-Cam\-era}. Only the visualization of the rectangles in the Bird’s-Eye View requires some explanation here: In \zeiref{camerav0802}, the rectangle of the view is rendered in red. The second rectangle, shown in brown, is the inner rectangle whose boundary triggers scrolling when it is crossed. The coordinates for these rectangles are computed in advance in \zeiref{camerav0805}.

\lstsource{SRC/01 Techniques/03 WorldScrolling/V08/camera_demo.py}{68}{84}{python}{Big World -- \texttt{Game.draw()}}{srcCamerademo10}


%%%%%%%%%%%%%%%%%%%%%%%%%%%%%%%%%%%%%%%%%%%%%%%%%%%%%%%%%%%%%%%%%%%%%
\subsection{Auto Scrolling/Endless Scrolling}
\begin{diskbox}
	\url{https://github.com/adamsralf/pygame_book/tree/main/src/01%20Techniques/03%20WorldScrolling/V09}
\end{diskbox}
Another variant is auto scrolling\index{Scrolling!Auto}\index{Auto Scrolling}\index{Scrolling!Endless}\index{Endless Scrolling}. In this approach, the background moves automatically and continuously in a fixed direction, while the player usually remains centered and can only evade vertically—for example by jumping. It is also quite common for the background to move downward, requiring the player to jump to higher platforms.

Let us therefore take a look at the new class \texttt{WindowAuto}. What is new or special here is that the constructor receives a direction parameter. This consists of two numbers: one for horizontal movement and one for vertical movement. The values represent the speed in \unit{px/s}, while the sign determines the direction: positive values indicate movement to the right and downward, negative values movement to the left and upward.

\lstsource{SRC/01 Techniques/03 WorldScrolling/V09/windows.py}{163}{176}{python}{Big World -- Constructor of \texttt{WindowAuto}}{srcWindows14}

Here as well, essentially only the method \texttt{scroll()} needs to be adapted. Only the first line is of interest. According to the task, the offset is adjusted using the specified \texttt{direction}. Multiplying by \texttt{DELTATIME} allows the unit \unit{px/s} to be used instead of \unit{px/frame}.

\lstsource{SRC/01 Techniques/03 WorldScrolling/V09/windows.py}{193}{198}{python}{Big World -- \texttt{WindowAuto.scroll()}}{srcWindows15}

For the sake of completeness: do not forget to adjust the name in \texttt{save()} ;-)


%%%%%%%%%%%%%%%%%%%%%%%%%%%%%%%%%%%%%%%%%%%%%%%%%%%%%%%%%%%%%%%%%%%%%
\subsection{As a Strategy Pattern}
\begin{diskbox}
	\url{https://github.com/adamsralf/pygame_book/tree/main/src/01%20Techniques/03%20WorldScrolling/Pattern}
\end{diskbox}

Although I~hope that all the techniques presented so far have been explained clearly, it is worthwhile to further decouple the algorithms from the concrete game scenario of my \emph{large world}. After all, these techniques occur repeatedly, making it worthwhile to formulate a reusable solution.

If we compare the three scenarios closely, we find that they differ only in the method \texttt{scroll()}, with one or two additional attributes being required in each case. In other words, we can encapsulate the algorithms in separate classes and then formulate a solution using the \gls{strategypattern}. The resulting architecture is shown in \abbref[vref]{picStrategCamera01}.

\begin{figure}[htb]
	\centering
	\scriptsize
	\begin{tikzpicture}%[show background grid]

		%--------------------------------------------------------------
		% Klassen
		%--------------------------------------------------------------
		
		% Context: Camera
		\begin{class}[text width=6cm]{Camera}{-6,6}
			\attribute{- offset : Vector2}
			\attribute{- rect : FRect}
			\attribute{- scroller : CamScroll}
			\operation{+ set\_scroller(scroller : CamScroll) : void}
			\operation{+ scroll() : void}
			\operation{+ world2camera(rect : FRect) : FRect}
			\operation{+ camera2world(rect : FRect) : FRect}
		\end{class}
		
		% Strategy (abstrakt)
		\begin{abstractclass}[text width=6cm]{CamScroll}{2,6}
			\attribute{- camera : Camera}
			\attribute{- player : FRect}
			\attribute{- rect\_world : FRect}
			\attribute{- rect\_view : FRect}
			\operation{+ scroll() : void}
		\end{abstractclass}
		
		% Concrete Strategy: CenteredCamera
		\begin{class}[text width=2.5cm]{CenteredCamera}{-3.5,2}
			\inherit{CamScroll}
			\operation{+ scroll() : void}
		\end{class}
		
		% Concrete Strategy: AutoCamera
		\begin{class}[text width=3.0cm]{AutoCamera}{0,2}
			\inherit{CamScroll}
			\attribute{- direction : Vector2}
			\operation{+ scroll() : void}
		\end{class}
		
		% Concrete Strategy: PagewiseCamera
		\begin{class}[text width=3.9cm]{PagewiseCamera}{4,2}
			\inherit{CamScroll}
			\attribute{- inner : Vector2}
			\attribute{- inner\_rect : FRect}
			\operation{+ scroll() : void}
		\end{class}

		%--------------------------------------------------------------
		% Beziehungen
		%--------------------------------------------------------------

		% Strategy-Pattern: Camera --(Komposition)--> CamScroll
		 \composition{Camera}{ }{0..1}{CamScroll}
  \end{tikzpicture}
   \normalsize
 \caption{Strategy pattern applied on \texttt{Camera} and scroll strategies in \texttt{cameraview.py}}\label{picStrategCamera01}
\end{figure}

The class \texttt{Camera} is the class that we will later use in our game. It contains all attributes and methods required to render an oversized world. It also contains a \emph{placeholder} for the actual scrolling behavior: the attribute \texttt{scroller}. In the method \texttt{scroll()}, the scrolling behavior of the behavior class is then invoked.


%\lstsource{SRC/01 Techniques/03 WorldScrolling/Pattern/cameraview.py}{171}{999}{python}{Scroll Pattern -- \texttt{Camera}}{srcCameraview01}

\texttt{CamScroll(ABC)} is the abstract interface class for the behavior classes and essentially consists only of the abstract method \texttt{scroll()}. The four attributes are required for computing the scrolling behavior.

%\lstsource{SRC/01 Techniques/03 WorldScrolling/Pattern/cameraview.py}{24}{57}{python}{Scroll Pattern -- \texttt{CamScroll(ABC)}}{srcCameraview02}

The three concrete behavior classes are \texttt{CenteredCamera}, \texttt{AutoCamera}, and \texttt{Page\-wise\-Camera}. Here, the method \texttt{scroll()} is implemented. The logic of these implementations corresponds exactly to the approaches shown above and should therefore be easy to understand.

%\lstsource{SRC/01 Techniques/03 WorldScrolling/Pattern/cameraview.py}{60}{167}{python}{Scroll Pattern -- \texttt{CenteredCamera}, \texttt{AutoCamera}, and \texttt{PagewiseCamera}}{srcCameraview03}
%
%An example implementation can be found in \srcref[vref]{srcCameraviewTest01}. It essentially corresponds to the examples implemented above.

%\lstsource{SRC/01 Techniques/03 WorldScrolling/Pattern/cameraview_test.py}{1}{999}{python}{Scroll Pattern -- Example implementation}{srcCameraviewTest01}

A video demonstration can be found here: \href{https://youtu.be/A2uXPimynnc}{https://youtu.be/A2uXPimynnc}.  

Further resources include:
\begin{itemize}
	\item \href{https://www.youtube.com/watch?v=XmSv2V69Y7A}{https://www.youtube.com/watch?v=XmSv2V69Y7A}
	\item \href{https://www.youtube.com/watch?v=ARt6DLP38-Y}{https://www.youtube.com/watch?v=ARt6DLP38-Y}
	\item \href{https://www.youtube.com/watch?v=FDJU8lIObVE}{https://www.youtube.com/watch?v=FDJU8lIObVE}
\end{itemize}


	%TODO: Zustandautomaten
	%TODO: Vorbeiziehende Gebirge (Hintergründe) mit 3D-Optik
\chapter{Examples}\label{secExamples}
	% !TeX spellcheck = en_US
\section{Pong}\index{Pong}
%%%%%%%%%%%%%%%%%%%%%%%%%%%%%%%%%%%%%%%%%%%%%%%%%%%%%%%%%%%%%%%%%%%%%

The ultimate beginner classic. This game has been played in countless variations since 1972.  Because the rules are simple and easy to understand, it is perfectly suited as a first programming project.

We will develop this game step by step in a systematic way, assuming that the techniques from \kapref{secBasics} are already familiar. I will deliberately omit docstring comments in the source code, since everything is explained in the text and including them here would only make the listings unnecessarily long.  They are, of course, included in the final version.

Note: At the very beginning, I once asked \glslink{chatgpt}{ChatGPT} to generate a Pong game for me. It was quite impressive to see that it produced a fully working game.


%%%%%%%%%%%%%%%%%%%%%%%%%%%%%%%%%%%%%%%%%%%%%%%%%%%%%%%%%%%%%%%%%%%%%
\subsection{\Reqref{req0201Standard}: Standards}
\begin{diskbox}
	\url{https://github.com/adamsralf/pygame_book/tree/main/src/02%20Examples/01%20Pong/v01}
\end{diskbox}
\br{Standard functionality}{req0201Standard}
\begin{enumerate}
	\item The window has an appropriate size.\label{req0201StandardGröße}
	\item The background is a dark red playing field with a dashed center line.\label{req0201StandardHintergrund}
	\item The game can be exited using the \keys{\esc} key or by clicking the red ``X''.\label{req0201StandardBeenden}
	\item The game runs at a speed independent of the \emph{FPS}.\label{req0201StandardFps}
\end{enumerate}
\er

And off we go. Here, the \texttt{config.py}. I assume that you have sufficient Python knowledge to extend it as needed.

\lstsource{SRC/02 Examples/01 Pong/v01/config.py}{1}{10}{python}{Pong (\Reqref{req0201Standard}) -- \texttt{config.py}}
{srcPong01a}


\begin{wrapfigure}[9]{r}{6.5cm}%
	\begin{center}%
		\vspace{-1em}%
		\myfigure{pong00.png}{0.2}{Pong: the background}{picPong00}%
	\end{center}%
\end{wrapfigure}%
The background is not loaded from a bitmap this time, but created on the fly. There is no deep reason for this -- apart from showing that bitmaps do not always have to come from image files (see \secref[vref]{secCreateBitmaps}). Instead, they can be generated dynamically as well.

To do this, a \texttt{Surface} object\myindex{pyg}{\texttt{Surface}} with the size of the screen is created first. It is then filled with a dark red color, meant to resemble a clay court. In \texttt{paint\_net()}, starting at \zeiref{srcPong0101}, the net is drawn as a sequence of small white rectangles.

\lstsource{SRC/02 Examples/01 Pong/v01/pong01.py}{8}{23}{python}{Pong (\Reqref{req0201Standard}) -- the class \texttt{Background}}
{srcPong01b}

The class \texttt{Game} consists of the basic building blocks that we have already seen in \kapref{secBasics}. In \texttt{\_\_init\_\_()}, Pygame is initialized, the window and the clock are created, and the control flag for the main game loop is set up. The background is stored in a \texttt{Group\-Single} object\myindex{pyg}{\texttt{sprite}!\texttt{GroupSingle}}. The remaining methods should be fairly self-explanatory.

\lstsource{SRC/02 Examples/01 Pong/v01/pong01.py}{26}{60}{python}{Pong (\Reqref{req0201Standard}) -- the class \texttt{Game}}
{srcPong01c}

%For the sake of completeness:
%
%\lstsource{SRC/02 Examples/01 Pong/v01/pong01.py}{63}{999}{python}{Pong (\Reqref{req0201Standard}) -- the class \texttt{Game}}
%{srcPong01d}

At this point, the application is not functional yet, but it already displays the background as you can see in \abbref[vref]{picPong00}.


%%%%%%%%%%%%%%%%%%%%%%%%%%%%%%%%%%%%%%%%%%%%%%%%%%%%%%%%%%%%%%%%%%%%%
\subsection{\Reqref{req0201Schläger}: The Paddles}
\begin{diskbox}
	\url{https://github.com/adamsralf/pygame_book/tree/main/src/02%20Examples/01%20Pong/v02}
\end{diskbox}
\br{Paddles}{req0201Schläger}
\begin{enumerate}
	\item There is one rectangular paddle on the left side and one on the right side.\label{req0201SchlägerZwei}
	\item The paddles have a width of \SI{15}{px} and a height of one tenth of the screen height.\label{req0201SchlägerGröße}
	\item The paddles have a speed of $\frac{\text{screen height}}{2}~px/s$.\label{req0201SchlägerGeschwindigkeit}
	\item Each paddle is positioned at a distance of \SI{50}{px} from the left or right edge, measured from its center.\label{req0201SchlägerOrt}
	\item The left paddle is moved upward using \keys{w} and downward using \keys{s}.\label{req0201SchlägerTastenLinks}
	\item The right paddle is moved upward using \keys{\arrowkeyup} and downward using  \keys{\arrowkeydown}.\label{req0201SchlägerTastenRechts}
	\item The paddles cannot leave the playing field.\label{req0201SchlägerGefangen}
\end{enumerate}
\er


In \zeiref{srcPong0201}, the size of the paddle is calculated (requirements~\ref{req0201Schläger}.\ref{req0201SchlägerZwei} and~\ref{req0201Schläger}.\ref{req0201SchlägerGröße}). Starting at \zeiref{srcPong0202}, the paddles are positioned. Vertically, they always start in the center of the screen. Horizontally, their start position depends on whether we are dealing with the left or the right paddle. In both cases, they are placed slightly away from the edge, exactly as specified in requirement~\ref{req0201Schläger}.\ref{req0201SchlägerOrt}.

The paddle speed is set in \zeiref{srcPong0206} according to requirement~\ref{req0201Schläger}.\ref{req0201SchlägerGeschwindigkeit}.  
Just like the background, this bitmap is not loaded from a file but created directly in the code (\zeiref{srcPong0203}) and filled with a bright yellow color.

\lstsource{SRC/02 Examples/01 Pong/v02/pong02.py}{26}{42}{python}{Pong (\Reqref{req0201Schläger}) -- The constructor of \texttt{Paddle}}
{srcPong02a}

The method \texttt{update()} is responsible for distributing the tasks. With regard to movement, the attribute \texttt{self.direction} is adjusted accordingly (starting at \zeiref{srcPong0205}). If the paddle is supposed to change its position, the method \texttt{move()} is called in \zeiref{srcPong0204}.

\lstsource{SRC/02 Examples/01 Pong/v02/pong02.py}{44}{50}{python}{Pong (\Reqref{req0201Schläger}) -- \texttt{Paddle.update()}}{srcPong02b}


All that remains is the method \texttt{move()}. It looks more complicated than it actually is. After checking whether there is anything to do at all, the new vertical position is calculated in \zeiref{srcPong0207} (the horizontal position remains unchanged). After that, it is checked whether the paddle has left the playing field. If so, the paddle is moved back to the top or bottom edge accordingly.

\lstsource{SRC/02 Examples/01 Pong/v02/pong02.py}{52}{58}{python}{Pong (\Reqref{req0201Schläger}) -- \texttt{Paddle.move()}}
{srcPong02c}

Now the paddles need to be integrated into the \texttt{Game} class. In \zeiref{srcPong0208}, a sprite group\myindex{pyg}{\texttt{sprite}!\texttt{Group}} is created first, which will hold all sprites except the background. After that, the two paddles are created and immediately added to the sprite group via constructor arguments.

\lstsource{SRC/02 Examples/01 Pong/v02/pong02.py}{61}{72}{python}{Pong (\Reqref{req0201Schläger}) -- Constructor of \texttt{Game}}
{srcPong02d}

In \texttt{update()} and \texttt{draw()}, the only thing that happens is the corresponding method call on the sprite group and now the paddles finally show up on screen.

\lstsource{SRC/02 Examples/01 Pong/v02/pong02.py}{86}{92}{python}{Pong (\Reqref{req0201Schläger}) -- \texttt{Game.update()} and \texttt{Game.draw()}}
{srcPong02e}

And now the keyboard events are handled. Pressing a key triggers a movement (starting at \zeiref{srcPong0211}), while releasing the key causes the corresponding paddle to stop (starting at \zeiref{srcPong0212}).

In each case, the method \texttt{Paddle.update()} is called with an appropriate parameter:  for movement with \verb+action="up"+ or \verb+action="down"+, and for stopping with \verb+action="halt"+.

\lstsource{SRC/02 Examples/01 Pong/v02/pong02.py}{94}{113}{python}{Pong (\Reqref{req0201Schläger}) -- \texttt{Game.watch\_for\_events()}
}{srcPong02f}


%%%%%%%%%%%%%%%%%%%%%%%%%%%%%%%%%%%%%%%%%%%%%%%%%%%%%%%%%%%%%%%%%%%%%
\subsection{\Reqref{req0201Ball}: The Ball}
\begin{diskbox}
	\url{https://github.com/adamsralf/pygame_book/tree/main/src/02%20Examples/01%20Pong/v03}
\end{diskbox}
\br{Ball}{req0201Ball}
\begin{enumerate}
	\item The ball is a circle with a radius of \SI{10}{px}.\label{req0201BallGröße}
	\item Its speed is $\frac{\text{screen width}}{3}~px/s$.\label{req0201BallGeschwindigkeit}
	\item It starts in the center of the screen with a random horizontal and vertical direction.\label{req0201BallStart}
	\item It bounces off the top and bottom edges of the screen.\label{req0201BallObenUnten}
	\item When it touches the left edge, it is reset to the center. The same happens when it touches the right edge.\label{req0201BallRechtsLinks}
	\item If the right edge is hit, player~1 scores a point; if the left edge is hit, player~2 scores a point.\label{req0201BallPunkt}
\end{enumerate}
\er

Since we need to keep track of the players’ scores according to requirement~\ref{req0201Ball}.\ref{req0201BallPunkt}, a corresponding array is added to \texttt{config.py} (\zeiref{srcPong0300}).
 
\lstsource{SRC/02 Examples/01 Pong/v03/config.py}{1}{99}{python}{Pong (\Reqref{req0201Ball}) -- \texttt{config.py}
}{srcPong03A}

In accordance with requirements~\ref{req0201Ball}.\ref{req0201BallGröße} and~\ref{req0201Ball}.\ref{req0201BallGeschwindigkeit}, the size and speed of the ball are defined in \zeiref{srcPong0301} and \zeiref{srcPong0302}. Since the ball needs to be restarted frequently, the initialization of its starting position and direction is moved into the separate method \texttt{service()} (\zeiref{srcPong0303}).

\newpage
\lstsource{SRC/02 Examples/01 Pong/v03/pong03.py}{62}{71}{python}{Pong (\Reqref{req0201Ball}) -- Constructor of \texttt{Ball}}
{srcPong03a}

In \texttt{update()}, the responsibilities are distributed.

\lstsource{SRC/02 Examples/01 Pong/v03/pong03.py}{73}{79}{python}{Pong (\Reqref{req0201Ball}) -- \texttt{Ball.update()}}
{srcPong03b}

Let us now take a closer look at the helper methods, one by one. We start with \texttt{move()}. As expected, the position is updated using the velocity values. After that, starting at \zeiref{srcPong0304}, it is checked whether the ball has reached any of the four edges of the screen.

If the top or bottom edge is hit (requirement~\ref{req0201Ball}.\ref{req0201BallObenUnten}), the sign of the vertical velocity is inverted by calling \texttt{vertical\_flip()} (\srcref[vref]{srcPong03e}). After the flip, the ball is clamped to the top or bottom edge, since it may already have crossed the boundary.

Things are different when the ball reaches the left or right edge. In that case, the ball is served again according to requirement~\ref{req0201Ball}.\ref{req0201BallRechtsLinks} (see \srcref[vref]{srcPong03d}), and -- as specified in requirement~\ref{req0201Ball}.\ref{req0201BallPunkt} -- the appropriate player’s score is increased.

\lstsource{SRC/02 Examples/01 Pong/v03/pong03.py}{81}{94}{python}{Pong (\Reqref{req0201Ball}) -- \texttt{Ball.move()}}
{srcPong03c}

When serving, the center of the ball is set to the center of the screen (requirement~\ref{req0201Ball}.\ref{req0201BallStart}). After that, the signs of the two velocity components are chosen randomly, which determines the direction of movement (left or right, and up or down). Since we do not have a score display yet, a temporary console output is implemented in \zeiref{srcPong0305}.

\lstsource{SRC/02 Examples/01 Pong/v03/pong03.py}{96}{99}{python}{Pong (\Reqref{req0201Ball}) -- \texttt{Ball.service()}}{srcPong03d}

The direction change is simply a sign flip. The method \texttt{flip\_hori\-zon\-tal()} is not used yet, but we will need it later when we want the ball to bounce off the paddle.

\lstsource{SRC/02 Examples/01 Pong/v03/pong03.py}{101}{105}{python}{Pong (\Reqref{req0201Ball}) -- The flip methods of \texttt{Ball}}{srcPong03e}


\begin{warningbox}[\hspace{1cm}Typical reflection pitfalls when handling the ball]

\begin{itemize}
	\item \textbf{Sticky edge / multiple flips per frame}\index{sticky edge}  
	
	If the ball is still inside the wall after a flip (because it has already crossed the boundary), the velocity is inverted again in the next frame.  
	The result is a ball that appears to jitter or stick to the edge.
	
	\emph{Fix:}  
	After flipping the velocity, clamp the position explicitly (e.g.\ \texttt{rect.top = 1} or \texttt{rect.bottom = WINDOW.HEIGHT - 1}).
	
	\item \textbf{Checking the wrong reference (center vs.\ rect)}  
	
	A common mistake is to compute movement using the ball center or a position vector, but perform collision checks against \texttt{rect.left/right/top/bottom} (or vice versa). This usually leads to off-by-radius errors.
	
	\emph{Fix:}  
	Be consistent: either check collisions exclusively using \texttt{rect.*},  
	or use \texttt{center} together with \texttt{$\pm$ radius} -- but do not mix both approaches.
	
	\item \textbf{Flipping the wrong axis}  
	
	A classic error: when hitting the top or bottom wall, \texttt{speed.x} is inverted instead of \texttt{speed.y} (or the other way around).
	
	\emph{Fix:}
	\begin{itemize}
		\item Top / bottom collision: \texttt{speed.y *= -1}
		\item Left / right collision: \texttt{x *= -1} (or trigger a service/reset)
	\end{itemize}
	
	\item \textbf{Tunneling at high speed}  
	
	With large \texttt{DELTATIME} values or high velocities, the ball may jump over a wall between two frames and never register a collision.
	
	\emph{Fix (simple):}  
	After moving the ball, check whether it has crossed a boundary and clamp it back.
	
	\emph{Fix (robust):}  
	Split the movement into smaller steps (sub-stepping) or use swept collision detection.
	
	\item \textbf{Incorrect handling of multiple collisions in one frame}  
	
	If the ball hits, for example, a corner (top wall and paddle edge at the same time), naively flipping both axes can cancel out the reflection entirely.
	
	\emph{Fix:}  
	Prioritize collisions (e.g.\ wall before paddle),  
	or decide based on the smaller penetration depth.
	
	\item \textbf{Not pushing the ball out of the paddle after a hit}  
	
	If the ball remains inside the paddle rectangle after a bounce, it will flip direction again in the next frame and appear to vibrate.
	
	\emph{Fix:}  
	After a paddle collision, move the ball explicitly in front of the paddle edge (clamp), and only then invert \texttt{vx}.
\end{itemize}
\end{warningbox}




%%%%%%%%%%%%%%%%%%%%%%%%%%%%%%%%%%%%%%%%%%%%%%%%%%%%%%%%%%%%%%%%%%%%%
\subsection{\Reqref{req0201Punkte}: Scoring}
\begin{diskbox}
	\url{https://github.com/adamsralf/pygame_book/tree/main/src/02%20Examples/01%20Pong/v04}
\end{diskbox}
\br{Scoring}{req0201Punkte}
\begin{enumerate}
	\item The score is displayed centered at the top of the screen.\label{req0201PunkteOben}
\end{enumerate}
\er

For displaying the score, the class \texttt{Score} is used. In the end, it is just another sprite—but one that needs to be recreated from time to time, namely whenever the score changes. Since the current score is now stored in \zeiref{srcPong0405}, it can be removed from \texttt{config.py}.

\newpage
\lstsource{SRC/02 Examples/01 Pong/v04/pong04.py}{114}{122}{python}{Pong (\Reqref{req0201Punkte}) -- Constructor of \texttt{Score}}
{srcPong04a}

In this method, the current score is rendered using a font object and then positioned accordingly.

\lstsource{SRC/02 Examples/01 Pong/v04/pong04.py}{130}{132}{python}{Pong (\Reqref{req0201Punkte}) -- \texttt{Score.render()}}
{srcPong04b}

In \texttt{update()}, the appropriate score value is updated and \texttt{render()} is called.

\lstsource{SRC/02 Examples/01 Pong/v04/pong04.py}{124}{128}{python}{Pong (\Reqref{req0201Punkte}) -- \texttt{Score.update()}}
{srcPong04c}

What is still missing is a trigger for updating the score display. This is a perfect opportunity to introduce a user-defined event. Starting at \zeiref{srcPong0400}, everything required for such a user event is implemented. First, an event ID is defined, followed by the corresponding \texttt{pygame.e\-vent\-.E\-vent} object\myindex{pyg}{\texttt{event}!\texttt{Event()}}.

\lstsource{SRC/02 Examples/01 Pong/v04/config.py}{1}{99}{python}{Pong (\Reqref{req0201Punkte}) -- \texttt{MyEvent}}{srcPong04d}

Now the \texttt{Ball} class only has to trigger the appropriate event, and \texttt{Game} needs to handle it. Here are the required changes in \texttt{Ball}. Inside the method \texttt{move()}, the relevant code sections are replaced. For example, in \zeiref{srcPong0402} the number of the player who scores the point is packed into the event, and in \zeiref{srcPong0403} the event is dispatched.

\newpage
\lstsource{SRC/02 Examples/01 Pong/v04/pong04.py}{86}{101}{python}{Pong (\Reqref{req0201Punkte}) -- \texttt{Ball.move()}}{srcPong04g}

Now all that remains is to catch the user-defined event inside \texttt{watch\_for\_events()} (starting at \zeiref{srcPong0404}).

\lstsource{SRC/02 Examples/01 Pong/v04/pong04.py}{190}{191}{python}{Pong (\Reqref{req0201Punkte}) -- \texttt{Ball.watch\_for\_events()}}{srcPong04h}

\begin{hintbox}[Why is a user-defined event more elegant than direct access?]
\begin{itemize}
	\item \textbf{No direct access from the ball to the score:} 
	If the \texttt{Ball} were to call \texttt{Score.render()} or \texttt{Score.update()} directly, it would need to know about the \texttt{Score} object -- or even the \texttt{Game} class. This creates unnecessary dependencies and tightly couples classes that should remain independent.
	
	\item \textbf{Clear separation of responsibilities:}
	The \texttt{Ball} only knows one thing:	\emph{A point was scored by player X.} The \texttt{Game}, on the other hand, decides what that means in practice. Update the score data, re-render the score display, maybe play a sound, reset the ball, or start the next serve.	Each class focuses on its own responsibility (SRP)\index{SRP}.
	
	\item \textbf{A clean extension point:}
	Later on, additional reactions can easily be attached to the same event -- such as sound effects, particle effects, a short pause, a change in serve direction, or logging -- without touching the \texttt{Ball} code again.
	
	\item \textbf{Better testability and maintainability:}
	All scoring-related behavior is handled centrally in \texttt{Game.watch\_for\_events()}, instead of being scattered across multiple classes.
	This makes the code easier to understand, test, and maintain.
	
\end{itemize}
\end{hintbox}

The ball reports what happened and the game decides what to do about it.

%%%%%%%%%%%%%%%%%%%%%%%%%%%%%%%%%%%%%%%%%%%%%%%%%%%%%%%%%%%%%%%%%%%%%
\newpage
\subsection{\Reqref{req0201Schlag}: Paddle hit}
\begin{diskbox}
	\url{https://github.com/adamsralf/pygame_book/tree/main/src/02%20Examples/01%20Pong/v05}
\end{diskbox}
At first glance, the game already looks finished -- but it is still not really playable, because the paddles are not doing anything yet.

\myebild{pong01.png}{0.3}{Pong: paddles, ball, and score}{picPong01}

\br{Paddle hit}{req0201Schlag}
\begin{enumerate}
	\item When the ball touches a paddle, it bounces off and is returned to the opponent’s side of the field.\label{req0201SchlagZurück}
	\item Each time the ball hits a paddle, its directional velocities are increased by a small random amount.\label{req0201SchlagGeschwindigkeit}
\end{enumerate}
\er

To achieve this, we add the method \texttt{check\_collision()} to the \texttt{Game} class. This method checks whether the ball has hit one of the paddles. A good choice here is the method \texttt{pygame\-.sprite\-.coll\-ide\_rect()}\myindex{pyg}{\texttt{sprite}!\texttt{collide\_rect()}}.  

If a collision is detected, the previously unused method \texttt{horizontal\_flip()} (see \srcref[vref]{srcPong03e}) is triggered via \texttt{update()}. Afterwards, the positions are adjusted so that the ball and the paddle no longer overlap. In addition, the method \texttt{respeed()} is called via \texttt{update()} to fulfill requirement~\ref{req0201Schlag}.\ref{req0201SchlagGeschwindigkeit}.

\lstsource{SRC/02 Examples/01 Pong/v05/pong05.py}{199}{205}{python}{Pong (\Reqref{req0201Schlag}) -- \texttt{Game.check\_collision()}}
{srcPong05a}

In \texttt{respeed()}, small random values are added to the velocity components. Via the attribute \texttt{speed}, this variation is indirectly tied to the screen size.

\lstsource{SRC/02 Examples/01 Pong/v05/pong05.py}{114}{116}{python}{Pong (\Reqref{req0201Schlag}) -- \texttt{Ball.respeed()}}
{srcPong05b}

Now the game finally becomes playable.

%%%%%%%%%%%%%%%%%%%%%%%%%%%%%%%%%%%%%%%%%%%%%%%%%%%%%%%%%%%%%%%%%%%%%
\subsection{\Reqref{req0201Computer}: Computer-controlled player}
\begin{diskbox}
	\url{https://github.com/adamsralf/pygame_book/tree/main/src/02%20Examples/01%20Pong/v06}
\end{diskbox}
Strictly speaking, we would be finished at this point -- but I would like to add a computer-controlled player. This allows the game to be played against the computer, or simply to let the computer play against itself for hours.

\br{Computer player}{req0201Computer}
\begin{enumerate}
	\item Pressing \keys{1} toggles control of the left paddle between human and computer.\label{req0201Computer1}
	
	\item Pressing \keys{2} toggles control of the right paddle between human and computer.\label{req0201Computer2}
	
	\item When control is switched back to manual, the paddle should initially remain stationary.\label{req0201ComputerHalt}
\end{enumerate}
\er

In \texttt{config.py}, a dictionary of flags is defined in \zeiref{srcPong0602}. These flags control, for each player, whether the paddle is controlled manually or by the computer.

\lstsource{SRC/02 Examples/01 Pong/v06/config.py}{1}{99}{python}{Pong (\Reqref{req0201Computer}) -- \texttt{config.py}}{srcPong06a}

In the \texttt{update()} method, starting at \zeiref{srcPong0603}, the flags are checked to determine whether a paddle is controlled by the computer.  
If so, a corresponding controller method is called.

\lstsource{SRC/02 Examples/01 Pong/v06/pong06.py}{185}{190}{python}{Pong (\Reqref{req0201Computer}) -- \texttt{Game.update()}}{srcPong06b}

Let us now take a look at the controller method. The basic idea is simple: the paddle moves upward as long as the center of the ball is above the center of the paddle, and it moves downward as long as the ball’s center is below the paddle’s center.  

There is no need to move all the way to the very top or bottom.  The last few pixels can be ignored, since a collision will usually be triggered before that anyway.

\subsubsection*{Why does this simple computer player work so well?}

At first glance, this controller logic looks almost trivial: the paddle simply follows the vertical position of the ball. Surprisingly, this already produces a reasonably strong computer opponent.

The reason is that Pong is a very simple game in terms of physics and decision-making. The ball moves along a straight line between collisions, and its vertical position is the most important piece of information needed to intercept it. By continuously aligning the paddle’s center with the ball’s center, the computer ensures that the paddle is usually in the right place at the right time.

Another advantage of this approach is that it is stable and predictable. The paddle does not overreact, oscillate wildly, or make unnecessary movements. Since the paddle speed is limited, it also cannot instantly teleport to the ball’s position, which keeps the game fair.

Finally, stopping the paddle slightly before reaching the exact ball position is intentional. This avoids jitter and unnecessary micro-movements, and in practice a collision will occur anyway once the ball reaches the paddle.

In short: For simple games like Pong, a straightforward \emph{follow the ball}\randnotiz{follow the ball} strategy is often more than sufficient -- and a great example of how simple rules can lead to convincing behavior.
 
\lstsource{SRC/02 Examples/01 Pong/v06/pong06.py}{242}{248}{python}{Pong (\Reqref{req0201Computer}) -- \texttt{Game.paddlecontroler()}}{srcPong06c}

In \texttt{watch\_for\_events()}, more extensive changes are required. First, manual control for a paddle must be disabled whenever that paddle is set to computer control. So, before calling the corresponding \texttt{update()} method, we first check whether the computer player currently has control. An example can be found in \zeiref{srcPong0604}.

\myebild{pong02.png}{0.3}{Pong: paddle color indicates AI mode (left AI, right manual)}{picPong02}

One remaining detail is requirement~\ref{req0201Computer}.\ref{req0201ComputerHalt}. For this, the corresponding flag is checked as shown in \zeiref{srcPong0605}, and the paddle is sent a halt signal.

\lstsource{SRC/02 Examples/01 Pong/v06/pong06.py}{197}{232}{python}{Pong (\Reqref{req0201Computer}) -- \texttt{Game.watch\_for\_events()}}{srcPong06d}

%%%%%%%%%%%%%%%%%%%%%%%%%%%%%%%%%%%%%%%%%%%%%%%%%%%%%%%%%%%%%%%%%%%%%
\subsection{\Reqref{req0201Sound}: Sound}
\begin{diskbox}
	\url{https://github.com/adamsralf/pygame_book/tree/main/src/02%20Examples/01%20Pong/v07}
\end{diskbox}
A bit of sound would make the game feel much more lively.

\br{Sound}{req0201Sound}
\begin{enumerate}
	\item Hitting the ball with a paddle should be accompanied by an appropriate sound effect.\label{req0201Sound1}
	
	\item Bouncing off the top and bottom edges should also be accompanied by a suitable sound effect.\label{req0201Sound2}
	
	\item Sound should be toggleable on and off using the \keys{F2}.\label{req0201Sound3}
\end{enumerate}
\er

As a first step, we extend \texttt{Settings} by adding the flag \texttt{SOUNDFLAG} in \zeiref{srcPong0701}. This flag controls whether sound should be played or not and provides access to the sound files.

\lstsource{SRC/02 Examples/01 Pong/v07/config.py}{1}{99}{python}{Pong (\Reqref{req0201Sound}) -- \texttt{config.py}}{srcPong07a}

The actual sound playback is implemented in the \texttt{Ball} class. In the constructor, starting at \zeiref{srcPong0702}, the sound effects are loaded and a channel is selected through which the sounds will be played.

\lstsource{SRC/02 Examples/01 Pong/v07/pong07.py}{81}{95}{python}{Pong (\Reqref{req0201Sound}) -- Constructor of \texttt{Ball}}{srcPong07b}

 The first sound effect is implemented for paddle collisions in \texttt{horizontal\_flip()}. After checking whether sound output is enabled at all, it is determined whether the ball is bouncing off the left or the right paddle. This is done indirectly by checking the current horizontal direction of the ball (\zeiref{srcPong0703}). Based on this information\randnotiz{stereo panning}\index{stereo panning} (see \secref[vref]{secStereo}, the volume of the sound is adjusted so that it creates the impression that the bounce happens to the left or right of the listener.

\lstsource{SRC/02 Examples/01 Pong/v07/pong07.py}{130}{139}{python}{Pong (\Reqref{req0201Sound}) -- \texttt{Ball.horizontal\_flip()}}{srcPong07d}

This sound effect becomes a bit more dynamic in \texttt{vertical\_flip()}. In \zeiref{srcPong0704}, the relative horizontal position of the ball is calculated. If the center of the ball is on the left side, \texttt{rel\_pos} will be close to~0; if the ball is far to the right, the value will be close to~1.  

These values can then be used directly as the left and right volume levels when calling \texttt{set\_volume()}, creating a simple but effective stereo panning effect.

\lstsource{SRC/02 Examples/01 Pong/v07/pong07.py}{141}{146}{python}{Pong (\Reqref{req0201Sound}) -- \texttt{Ball.vertical\_flip()}}{srcPong07e}

All that remains is toggling sound output on and off inside \texttt{watch\_for\_events()} in \zeiref{srcPong0705} using the function key \keys{F2}.

\lstsource{SRC/02 Examples/01 Pong/v07/pong07.py}{213}{227}{python}{Pong (\Reqref{req0201Sound}) -- \texttt{Ball.watch\_for\_events()}}{srcPong07f}

%%%%%%%%%%%%%%%%%%%%%%%%%%%%%%%%%%%%%%%%%%%%%%%%%%%%%%%%%%%%%%%%%%%%%
\subsection{\Reqref{req0201PauseHelp}: Pause and Help Screen}
\begin{diskbox}
	\url{https://github.com/adamsralf/pygame_book/tree/main/src/02%20Examples/01%20Pong/v08}
\end{diskbox}
\br{Pause and help}{req0201PauseHelp}
\begin{enumerate}
	\item Pressing \keys{p} pauses all activity and stops the game.  
	Pressing \keys{p} again resumes the game.\label{req0201PauseHelp1}
	
	\item Pressing \keys{h} pauses the game and displays a help text.  
	Pressing \keys{h} again resumes the game.\label{req0201PauseHelp2}
\end{enumerate}
\er

For the pause functionality, we create a separate class—perhaps a bit overengineered, but nicely self-contained. The essential part can be found in \zeiref{srcPong0801}. There, a semi-transparent gray overlay is created using a \texttt{Surface} object with the same size as the screen. The surface is filled with a gray color whose alpha channel\index{Alpha channel} is set to~200, allowing the background to shine through.

\lstsource{SRC/02 Examples/01 Pong/v08/pong08.py}{28}{33}{python}{Pong (\Reqref{req0201PauseHelp}) -- \texttt{Pause}}{srcPong08a}

The help screen is implemented in an analogous way. The only difference is that an additional text is blitted onto the surface. The text is split into a left and a right column to improve readability.

\lstsource{SRC/02 Examples/01 Pong/v08/pong08.py}{36}{50}{python}{Pong (\Reqref{req0201PauseHelp}) -- \texttt{Help}}{srcPong08b}

In the constructor of \texttt{Game}, two flags now need to be created to represent the respective modes (\zeiref{srcPong0803} and \zeiref{srcPong0804}). After that, the two overlay objects are created and assigned to a \texttt{pygame.Group.Single} object.

\lstsource{SRC/02 Examples/01 Pong/v08/pong08.py}{199}{216}{python}{Pong (\Reqref{req0201PauseHelp}) -- Constructor of \texttt{Game}}{srcPong08c}

Once everything is prepared, the \texttt{update()} method is modified so that the actual game logic is only executed when neither pause mode nor help mode is active (\zeiref{srcPong0805}). If one of these modes is enabled, the game state is effectively frozen: positions, movements, and collisions are no longer updated, while the current screen remains visible.

\lstsource{SRC/02 Examples/01 Pong/v08/pong08.py}{230}{236}{python}{Pong (\Reqref{req0201PauseHelp}) -- \texttt{Game.update()}}{srcPong08d}

In \texttt{draw()}, the currently active mode is checked as well. If the game is paused or the help screen is active, the corresponding sprite is rendered on top of the game scene. Otherwise, only the current game state is rendered as usual.

\lstsource{SRC/02 Examples/01 Pong/v08/pong08.py}{238}{245}{python}{Pong (\Reqref{req0201PauseHelp}) -- \texttt{Game.draw()}}{srcPong08e}

\myebild{pong03.png}{0.3}{Pong: Help screen}{picPong03}

\subsubsection*{Pause vs.\ Help -- what happens technically?}

Both the pause mode and the help screen are based on the same fundamental idea: the game is still rendered visually, but the actual game simulation is stopped.

\begin{itemize}
	\item \textbf{Pause}:
	All movement and state-changing calculations are suspended.	The current game situation is frozen and merely covered by a semi-transparent overlay.
	
	\item \textbf{Help}:
	Technically identical to the pause mode, but extended by an additional text overlay. The player receives information about controls and gameplay while the game state itself remains unchanged.
\end{itemize}

The crucial point is that in both modes the method \texttt{update()} does not modify any game objects. Rendering continues, which keeps the game visually \emph{alive} while it is logically paused.

\begin{hintbox}[This approach has several advantages]
\begin{itemize}
	\item No special-case logic inside individual sprite classes
	\item A clear separation between \emph{game state} and \emph{presentation}
	\item Easy to extend (e.\,g.\ for menus, dialogs, or settings)
\end{itemize}

Rule of thumb: Pausing does not mean drawing nothing -- it means changing nothing.
\end{hintbox}


	% !TeX spellcheck = en_US
\newpage
\section{Bubbles}\index{Bubbles}
%%%%%%%%%%%%%%%%%%%%%%%%%%%%%%%%%%%%%%%%%%%%%%%%%%%%%%%%%%%%%%%%%%%%%

In this chapter, the game \emph{Bubbles} is discussed as an example. We will develop this game step by step in a systematic way. I will assume that the techniques introduced in \kapref{secGoals} are already familiar. I will deliberately omit docstring comments in the source code, since everything is explained in the text and the listings would otherwise become unnecessarily long. In the final version, however, these comments are included.

\begin{hintbox}[Thank you]
	I would like to point out right away that the idea for the game did not come from me. A student once presented it as a mobile version at an \Gls{ita} fair. Unfortunately, I can no longer remember the student’s name, but I would like to take this opportunity to say a sincere \emph{thank you}.
\end{hintbox}

The game can be extended almost without limits: bubble popping animations, high score lists, and much more. But as is so often the case, the better is the enemy of the good. I hope you enjoy studying this example.



%%%%%%%%%%%%%%%%%%%%%%%%%%%%%%%%%%%%%%%%%%%%%%%%%%%%%%%%%%%%%%%%%%%%%
\subsection{\Reqref{reqStandard}: Standards}
\begin{diskbox}
	\url{https://github.com/adamsralf/pygame_book/tree/main/src/02%20Examples/02%20Bubbles/v01}
\end{diskbox}

\br{Standard functionality}{reqStandard}
\begin{enumerate}
	\item The window has an appropriate size.\label{reqStandardGröße}
	\item The background is either a suitable bitmap or a solid color.\label{reqStandardHintergrund}
	\item The game can be exited using \keys{\esc} or by clicking the red “X”.\label{reqStandardBeenden}
	\item All bitmaps are converted and scaled appropriately after loading.\label{reqStandardSprite}
	\item All bitmaps -- except for the background -- are transparent.\label{reqStandardTransparenz}
	\item All bitmaps are stored in \texttt{pygame.sprite.Group} or \texttt{py\-game\-.sprite\-.Group\-Single} objects.\label{reqStandardGruppe}
	\item The game has a frame-rate-independent execution speed.\label{reqStandardFps}
\end{enumerate}
\er

\Reqref[vref]{reqStandard} does not only define specific requirements, but also general ones. For this reason, it will appear again in later implementations.

At this point, the preamble is presented once. I assume that you have sufficient Python knowledge to extend it as needed. The static configuration values of the game are stored, as usual, in the separate \texttt{config.py} file.

It is required that the window has an appropriate size. With $\SI{1220}{px} \times \SI{1002}{px}$, the window is large enough to distribute the bubbles, yet small enough to allow quick mouse movement. Everything else has already been discussed in detail in previous chapters (e.g.\ \texttt{FPS}, \texttt{DELTATIME}, or \texttt{PATH}) and will therefore not be explained further here.

\myebild{aquarium01}{0.15}{Bubbles: background image}{picAquarium01}

\lstsource{SRC/02 Examples/02 Bubbles/v01/config.py}{1}{99}{python}{Bubbles (\reqref{reqStandard}.\ref{reqStandardGröße}) -- \texttt{config.py}
}{srcBubbles01a}

The \texttt{Background} class is a subclass of \texttt{Sprite}. It is only loaded and scaled to the appropriate size. Since the background never changes, there is no need to implement an \texttt{update()} method. Creating a dedicated subclass for this is somewhat like using a sledgehammer to crack a nut. We could just as well have implemented it directly as a \texttt{Sprite} object. I chose this approach purely for the sake of clarity. The background image can be seen in \abbref[vref]{picAquarium01}.

%\newpage
\lstsource{SRC/02 Examples/02 Bubbles/v01/bubbles.py}{7}{13}{python}{Bubbles (\reqref{reqStandard}.\ref{reqStandardHintergrund}) -- \texttt{Background
}}{srcBubbles01c}

In the \texttt{Game} class, the usual Pygame suspects are initialized or created in \texttt{\_\_init\_\_()}: \texttt{init()}\myindex{pyg}{\texttt{init()}}, \texttt{Window()}\myindex{pyg}{\texttt{Window}}, and \texttt{clock()}\myindex{pyg}{\texttt{time}!\texttt{Clock}}. The flag \texttt{running} for the main game loop is also initialized. The methods \texttt{run()}, \texttt{watch\_for\_events()}, \texttt{up\-date()}, and \texttt{draw()} contain only basic functionality and therefore do not need to be explained further at this point.

\lstsource{SRC/02 Examples/02 Bubbles/v01/bubbles.py}{16}{52}{python}{Bubbles (\reqref{reqStandard}) -- \texttt{Game}}{srcBubbles01d}  

However, these methods already define the overall flow of the game. All further properties of the game are merely extensions of this flow and no longer change it. Finally, the call is made (see \srcref[vref]{srcBubbles01s}). With this, all subitems of \reqref[vref]{reqStandard} that apply here are fulfilled.

\lstsource{SRC/02 Examples/02 Bubbles/v01/bubbles.py}{55}{999}{python}{Bubbles (\reqref{reqStandard}) -- invocation}{srcBubbles01s}
%%%%%%%%%%%%%%%%%%%%%%%%%%%%%%%%%%%%%%%%%%%%%%%%%%%%%%%%%%%%%%%%%%%%%
\subsection{\Reqref{reqBlasenErscheinen}: Bubbles appear}
\begin{diskbox}
	\url{https://github.com/adamsralf/pygame_book/tree/main/src/02%20Examples/02%20Bubbles/v02}
\end{diskbox}


\br{Bubbles appear}{reqBlasenErscheinen}
\begin{enumerate}
	\item A bubble appears at a random position.\label{reqBlasenErscheinenZufall}
	\item At the beginning, this happens every half second.\label{reqBlasenErscheinenIntervall}
	\item It has an initial radius of \SI{15}{px}.\label{reqBlasenErscheinenRadius}
	\item It keeps a minimum distance of \SI{10}{px} from the edges.\label{reqBlasenErscheinenAbstand}
	\item It keeps a minimum distance of \SI{10}{px} from all other bubbles.\label{reqBlasenErscheinenMindestabstand}
\end{enumerate}
\er


\begin{wrapfigure}[6]{r}{3.1cm}%
	\begin{center}%
		\vspace{-1cm}%
		\myfigure{blase1.png}{0.05}{Bubble}{picBlase1}%
	\end{center}%
\end{wrapfigure}%
For the bubble, we use the already transparent graphic from \abbref{picBlase1}. The random position still needs to be restricted. The aquarium does not fill the entire screen (see \abbref[vref]{picAquarium01}); instead, it sits inside something like a TV frame. So we have to define a playing area (\emph{playground}). The bubbles should only appear inside this area.

The playing area is a rectangle\index{rectangle}\index{self.rect} with an offset from the left and top edges of the screen -- \texttt{left}\myindex{pyg}{\texttt{Rect}!\texttt{left}} and \texttt{top}\myindex{pyg}{\texttt{Rect}!\texttt{top}} -- and a width (\texttt{width}\myindex{pyg}{\texttt{Rect}!\texttt{width}}) and height (\texttt{height}\myindex{pyg}{\texttt{Rect}!\texttt{height}}). The corresponding values are defined in \zeiref{srcBubble0203}. The distance to the border of the playing area and the minimum distance between bubbles are defined in \zeiref{srcBubble0202} as \SI{10}{px}, in accordance with \reqref{reqBlasenErscheinen}.\ref{reqBlasenErscheinenAbstand}. The initial radius\index{self.radius} -- and therefore the minimum radius -- is set to \SI{15}{px} in \zeiref{srcBubble0201} because of \reqref{reqBlasenErscheinen}.\ref{reqBlasenErscheinenRadius}. While playing, I noticed that smaller initial radii are simply too hard to see.

\lstsource{SRC/02 Examples/02 Bubbles/v02/config.py}{14}{16}{python}{Bubbles (\reqref{reqBlasenErscheinen}) -- additions in \texttt{config.py}}{srcBubbles02a}

The \texttt{Timer} class is exactly the one described above in \kapref[vref]{secZeitsteuerung}\index{Timer}; everything is explained there.

\lstsource{SRC/02 Examples/02 Bubbles/v02/bubbles.py}{9}{21}{python}{Bubbles (\reqref{reqBlasenErscheinen}) -- \texttt{Timer}}{srcBubbles02b}  

Let us now take a look at the \texttt{Bubble} class. The constructor is self-explanatory; it only handles the usual suspects: \texttt{image}, \texttt{rect}, and \texttt{radius}. The \texttt{update()} method is currently empty, since no changes are required yet. However, the \texttt{randompos()} method is needed because of \reqref{reqBlasenErscheinen}.\ref{reqBlasenErscheinenZufall}. It calculates a new bubble center and assigns it to \texttt{rect}. If necessary, this method must be repeated until a free area is found (see \reqref{reqBlasenErscheinen}.\ref{reqBlasenErscheinenAbstand} and~\reqref{reqBlasenErscheinen}.\ref{reqBlasenErscheinenMindestabstand}).

\lstsource{SRC/02 Examples/02 Bubbles/v02/bubbles.py}{33}{49}{python}{Bubbles (\reqref{reqBlasenErscheinen}) -- \texttt{Bubble}}{srcBubbles02c}  

The \texttt{Game} class now has to be extended accordingly. In \zeiref{srcBubble0204}, the \texttt{Background} object is created. \zeiref{srcBubble0205} creates a \texttt{Timer} object with an interval of \SI{500}{ms}, where no bubbles are generated during the first interval (see \reqref{reqBlasenErscheinen}.\ref{reqBlasenErscheinenIntervall}).

\lstsource{SRC/02 Examples/02 Bubbles/v02/bubbles.py}{52}{61}{python}{Bubbles (\reqref{reqBlasenErscheinen}) -- Constructor of \texttt{Game}}{srcBubbles02d}  

In the \texttt{draw()} method, only the \texttt{draw()} methods of the sprite groups are called. The \texttt{update()} method has also been adjusted; it now calls the \texttt{spawn\_bubble()} method and thus delegates the task of creating new bubbles.

\lstsource{SRC/02 Examples/02 Bubbles/v02/bubbles.py}{71}{77}{python}{Bubbles (\reqref{reqBlasenErscheinen}) -- \texttt{draw()} and \texttt{update()} of \texttt{Game}}{srcBubbles02e}  

The basic idea behind \texttt{spawn\_bubble()} is to keep guessing a position for a new bubble until a free area is found.
\begin{warningbox}[Avoid endloss loop]
	To avoid ending up in an \gls{endlosschleife}, the number of attempts is limited to~100. If no free area is found, the bubble is not added to the sprite group -- it is simply discarded.
\end{warningbox}

For this purpose, the radius is temporarily increased (\zeiref{srcBubble0207}) and then reduced back to its original value after the collision check (\zeiref{srcBubble0208}). 

This is an example showing that a method reference is passed to \texttt{pygame.sprite\-.sprite\-collide()}\myindex{pyg}{\texttt{sprite}!\texttt{spritecollide()}}\randnotiz{sprite\-collide()} -- in this case \texttt{pygame.sprite.collide\_circle()}\myindex{pyg}{\texttt{sprite}!\texttt{collide\_circle()}}\randnotiz{collide\_\-circle()} -- and that the usual rectangle-based collision check is therefore not used.

\lstsource{SRC/02 Examples/02 Bubbles/v02/bubbles.py}{79}{89}{python}{Bubbles (\reqref{reqBlasenErscheinen}) -- \texttt{spawn\_bubble()} of \texttt{Game}}{srcBubbles02f}  

The result can be seen in \abbref[vref]{picAquarium02}. The bubbles are evenly distributed across the playing area, and the required minimum distance to the edges and between the bubbles is maintained.

\myebild{aquarium02}{0.32}{Bubbles: the bubbles have a minimum distance at the start}{picAquarium02}

%%%%%%%%%%%%%%%%%%%%%%%%%%%%%%%%%%%%%%%%%%%%%%%%%%%%%%%%%%%%%%%%%%%%%
\subsection{\Reqref{reqBlasenanzahl}: Number of bubbles}
\begin{diskbox}
	\url{https://github.com/adamsralf/pygame_book/tree/main/src/02%20Examples/02%20Bubbles/v03}
\end{diskbox}

\br{Number of bubbles}{reqBlasenanzahl}
	The maximum number of bubbles shall depend on the size of the playing area.
\er

I want to define the maximum number in the \texttt{Game} class. Based on the available area, an upper limit is calculated:

\lstsource{SRC/02 Examples/02 Bubbles/v03/config.py}{17}{17}{python}{Bubbles (\reqref{reqBlasenanzahl}) -- additions in \texttt{config.py}}{srcBubbles03a}  

This upper limit from \zeiref{srcBubble0301} is checked in \zeiref{srcBubble0306}. A new bubble is only created if the maximum number has not yet been reached.

\lstsource{SRC/02 Examples/02 Bubbles/v03/bubbles.py}{79}{90}{python}{Bubbles (\reqref{reqBlasenanzahl}) -- additions in \texttt{Game.spawn\_bubbles()}}{srcBubbles03b}  

The rest of the program remains unchanged.

%%%%%%%%%%%%%%%%%%%%%%%%%%%%%%%%%%%%%%%%%%%%%%%%%%%%%%%%%%%%%%%%%%%%%
\subsection{\Reqref{reqBlasenWachstum}: Bubble growth}
\begin{diskbox}
	\url{https://github.com/adamsralf/pygame_book/tree/main/src/02%20Examples/02%20Bubbles/v04}
\end{diskbox}

\br{Bubble growth}{reqBlasenWachstum}
\begin{enumerate}
	\item Bubbles of different sizes are managed in a container.\label{reqBlasenWachstumContainer}
	
	\item The maximum radius of a bubble is \SI{240}{px}.\label{reqBlasenWachstumMax}
\end{enumerate}
\er

The purpose of \reqref{reqBlasenWachstum}.\ref{reqBlasenWachstumContainer} is to save computing time. During the game, bubbles repeatedly start with a certain radius and then grow. Scaling the bitmap to the required size every single time would waste processing power -- after all, the same bubble appears multiple times with identical radii. For this reason, it makes sense to scale the bubble once to all possible radii and store the results in a dictionary. The key used is the respective radius (see \zeiref{srcBubble0403}).\myindex{pyg}{\texttt{transform}!\texttt{scale()}}\randnotiz{scale()}

The \texttt{get()} method then returns the appropriately scaled and ready-to-use image for a given radius. Before that, lines~\ref{srcBubble0404} and~\ref{srcBubble0405} check whether the radius lies within the valid range. If the radius is too large, the maximum value is used; if it is too small, the minimum value is applied instead.

\lstsource{SRC/02 Examples/02 Bubbles/v04/bubbles.py}{33}{42}{python}{Bubbles (\reqref{reqBlasenWachstum}.\ref{reqBlasenWachstumContainer}) -- \texttt{BubbleContainer}}{srcBubbles04a}  

So far, only a start value -- and thus a lower bound -- for the bubble radius has been defined in \texttt{Game}. This definition is now extended in \zeiref{srcBubble0400} in accordance with \reqref{reqBlasenWachstum}.\ref{reqBlasenWachstumMax} by adding a maximum radius.

\lstsource{SRC/02 Examples/02 Bubbles/v04/config.py}{14}{14}{python}{Bubbles (\reqref{reqBlasenWachstum}.\ref{reqBlasenWachstumMax}) -- extension of \texttt{config.py}}{srcBubbles04b}

The \texttt{BubbleContainer} is passed to the constructor of \texttt{Bubble}, allowing this class to retrieve images from it. A direct example of this can be found in \zeiref{srcBubble0407}. The \texttt{image} attribute is set according to the current \texttt{radius}.

The \texttt{update()} method is no longer empty. Its main purpose is to make the bubble grow. To achieve this, the radius is continuously increased, which results in increasingly larger images being loaded from the \texttt{BubbleContainer} and displayed (\zeiref{srcBubble0412}). The new radius is calculated in \zeiref{srcBubble0410}. In the same line, this value is compared with the maximum radius from \texttt{config.py}, and the minimum of the two is selected. This logic prevents the radius from becoming too large.

\begin{hintbox}[Center-based scaling]
	But what is the purpose of lines~\ref{srcBubble0411} and~\ref{srcBubble0413}\index{center-based scaling}? The reference point of an image in a sprite is its top-left corner. If the bubble grows, it would therefore expand to the right and downward; the left and top edges would remain fixed, which looks awkward. To avoid this, we store the old center point, load the new image, create the corresponding \texttt{Rect} object, and then move it back to the old center. This way, the bubble visually grows outward from its center in all directions.
\end{hintbox}

\lstsource{SRC/02 Examples/02 Bubbles/v04/bubbles.py}{45}{70}{python}{Bubbles (\reqref{reqBlasenWachstum}) -- extension of \texttt{Bubble}}{srcBubbles04c}  

The \texttt{update()} method in \texttt{Game} only needs to be extended by calling all \texttt{update()} methods of the bubbles. This can be done very conveniently using the sprite group mechanism. Just like with \texttt{draw()}, \texttt{update()} can be called for the entire group in a single step (see \zeiref{srcBubble0414}).

\lstsource{SRC/02 Examples/02 Bubbles/v04/bubbles.py}{98}{100}{python}{Bubbles (\reqref{reqBlasenWachstum}) -- extension of \texttt{Game.update()}}{srcBubbles04d}  

Create the BubbleContainer

\lstsource{SRC/02 Examples/02 Bubbles/v04/bubbles.py}{79}{81}{python}{Bubbles (\reqref{reqBlasenWachstum}) -- extension of the constructor of \texttt{Game}}{srcBubbles04f}  

And in the \texttt{spawn\_bubble()} method, the constructor call of \texttt{Bubble} is extended by passing the \texttt{BubbleContainer}.

\lstsource{SRC/02 Examples/02 Bubbles/v04/bubbles.py}{102}{106}{python}{Bubbles (\reqref{reqBlasenWachstum}) -- extension of \texttt{Game.spawn\_bubble()}}{srcBubbles04e}  

The bubbles now grow outward from their center. The result might then look like the one shown in \abbref[vref]{picAquarium03}.

\myebild{aquarium03}{0.19}{Bubbles -- the bubbles have grown and merged}{picAquarium03}

%%%%%%%%%%%%%%%%%%%%%%%%%%%%%%%%%%%%%%%%%%%%%%%%%%%%%%%%%%%%%%%%%%%%%
\newpage
\subsection{\Reqref{reqMauscursor}: Mouse cursor}
\begin{diskbox}
	\url{https://github.com/adamsralf/pygame_book/tree/main/src/02%20Examples/02%20Bubbles/v05}
\end{diskbox}

\br{Mouse cursor}{reqMauscursor}
	If the mouse is inside a bubble, its appearance should change.
\er

This requirement is intended to provide visual feedback to the player. It allows them to recognize more quickly whether they have already reached a bubble. Pygame itself does not provide a method or function to test whether a point lies inside a circle. However, \abbref[vref]{picKollKreis02} provides a simple approach to solving this problem.

The value~$d$ represents the distance in pixels between the center of the circle $(x_1, y_1)$ and the point $(x_2, y_2)$. If~$d \leq r$, the point lies inside the circle or touches it. However, we do not actually need the distance itself. Put simply, we only need to know whether the expression on the left side of the inequality is smaller than the one on the right side. We can therefore avoid the expensive square root operation and instead check $(x_2 - x_1)^2 + (y_2 - y_1)^2 \leq r^2$. We therefore extend \texttt{Bubble} with an appropriate method.

\begin{figure}[H]
\begin{center}\index{Pythagoras, Satz von}
\tikzset{
    shape kreis/.style= {
    draw,
    fill = yellow!30,
    line width = 1pt,
    inner xsep = 0.0cm,
    inner ysep = 0.0cm,
   }
}

\begin{tikzpicture}
\draw [->, name=xachse] (0cm, 6cm)  -- +(13cm, 0cm);
\draw [<-, name=yachse] (0cm, 0cm)  -- +(0cm, 6cm);

\draw (4.0cm, 2.5cm) node[name=k1,shape=circle,shape kreis,  minimum height = 4cm] {};
\draw (9.5cm, 4.0cm) node[name=k2,shape=circle,shape kreis,  minimum height = 0.1cm] {};

\draw[-, very thick, blue] 
 (k1.north west) --  node[below, blue, xshift=0cm] {$r$} (k1.center);

\draw[-, very thick, blue] 
 (k1.center) --  node[below, blue, sloped, xshift=0cm] {\footnotesize$d=\sqrt{(x_2-x_1)^2+(y_2-y_1)^2}$} (k2.center);

\draw[-, very thick, red, dotted] 
 (k1.center) --  +(0cm, +3.5cm);
\draw[-, very thick, red, dotted] 
 (k1.center) --  +(-4.0cm, 0cm);

\draw[-, very thick, red, dotted] 
 (k2.center) --  +(0cm, +2.0cm);
\draw[-, very thick, red, dotted] 
 (k2.center) --  +(-9.5cm, 0cm);

\path [name=x1, color=red] let \p1 = (k1) in node  at (\x1,+6.4cm) {$x_1$};
\path [name=x2, color=red] let \p1 = (k2) in node  at (\x1,+6.4cm) {$x_2$};
\path [name=y1, color=red] let \p1 = (k1) in node  at (-0.4cm,\y1) {$y_1$};
\path [name=y2, color=red] let \p1 = (k2) in node  at (-0.4cm,\y1) {$y_2$};
\end{tikzpicture}
\caption[collision detection -- point inside a circle?]{collision detection -- point inside a circle (\gls{pythagoras})?}\label{picKollKreis02}
\end{center}
\end{figure}

\lstsource{SRC/02 Examples/02 Bubbles/v05/bubbles.py}{117}{122}{python}{Bubbles (\reqref{reqMauscursor}) -- \texttt{Game.collidepoint()}}{srcbubbles05a}  

With the help of this method, the solution is no longer a problem. The variable \texttt{is\_over} is a flag that keeps track of whether the mouse coordinates are inside a bubble or not. The normal case is that the mouse is not inside any bubble, so the variable is initialized with \false.

After that, the current mouse position is obtained using \texttt{pygame.mouse.\-get\-\_pos()}\myindex{pyg}{\texttt{mouse}!\texttt{get\_pos()}}. This mouse position is passed to the \texttt{Bubble.collidepoint()} method in \zeiref{srcBubble0502}. If a bubble is found that collides with the mouse, the flag is set to \true\ and the loop is terminated using \texttt{break}. This saves some processing time, since not all remaining bubbles have to be checked. Depending on the flag, the mouse cursor is then set accordingly.\myindex{pyg}{\texttt{SYSTEM\_CURSOR\_HAND}}\myindex{pyg}{\texttt{SYSTEM\_CURSOR\_CROSSHAIR}}

\lstsource{SRC/02 Examples/02 Bubbles/v05/bubbles.py}{124}{134}{python}{Bubbles (\reqref{reqMauscursor}) -- \texttt{Game.set\_mousecursor()}}{srcbubbles05b}  

The \texttt{update()} method in \texttt{Game} still needs to be extended by adding the call to the collision check.

\lstsource{SRC/02 Examples/02 Bubbles/v05/bubbles.py}{99}{102}{python}{Bubbles (\reqref{reqMauscursor}) -- \texttt{update()} in \texttt{Game}}{srcbubbles05c}  

Try running the program. Place the mouse in a lower-left corner outside a bubble and wait until the growing bubble touches the mouse.

%%%%%%%%%%%%%%%%%%%%%%%%%%%%%%%%%%%%%%%%%%%%%%%%%%%%%%%%%%%%%%%%%%%%%
\subsection{\Reqref{reqBlasenzerplatzen}: Bubbles burst}
\begin{diskbox}
	\url{https://github.com/adamsralf/pygame_book/tree/main/src/02%20Examples/02%20Bubbles/v06}
\end{diskbox}

\br{Bubbles burst}{reqBlasenzerplatzen}
	When a left mouse click occurs inside a bubble, the bubble should burst.
\er

Most of the work required to implement this requirement has already been done with the implementation of the \texttt{Bubble.collidepoint()} method. We only need to use this method in a clever way -- in fact, only a few remaining steps are necessary. In \texttt{watch\_for\_events()}, the left mouse click is detected first (\zeiref{srcBubble0601})\myindex{pyg}{\texttt{MOUSEBUTTONDOWN}}\myindex{pyg}{\texttt{mouse}!\texttt{get\_pos()}}\randnotiz{MOUSEBUTTONDOWN get\_pos()} and the current mouse position is passed to the newly created \texttt{sting()} method (\zeiref{srcBubble0602}).

\begin{hintbox}[\texttt{watch\_\-for\_\-events()} as a dispatcher]
	As a general rule, implement as little logic as possible in \texttt{watch\_\-for\_\-events()}. This method acts as a dispatcher; the actual processing should always be delegated to separate methods.
\end{hintbox}

\lstsource{SRC/02 Examples/02 Bubbles/v06/bubbles.py}{85}{94}{python}{Bubbles (\reqref{reqBlasenzerplatzen}) -- \texttt{Game.watch\_for\_event()}}{srcbubbles06a}  

The \texttt{sting()} method is now very simple. All \texttt{Bubble} objects are iterated over and checked to see whether the mouse position lies within their radius (\zeiref{srcBubble0603}). If the answer is \emph{yes}, the corresponding object is removed from the sprite group using \texttt{kill()}\myindex{pyg}{\texttt{sprite}!\texttt{Sprite}!\texttt{kill()}}\randnotiz{kill()}.

\lstsource{SRC/02 Examples/02 Bubbles/v06/bubbles.py}{138}{141}{python}{Bubbles (\reqref{reqBlasenzerplatzen}) -- \texttt{Game.sting()}}{srcbubbles06b}  

%%%%%%%%%%%%%%%%%%%%%%%%%%%%%%%%%%%%%%%%%%%%%%%%%%%%%%%%%%%%%%%%%%%%%
\subsection{\Reqref{reqPunktestand}: Score}
\begin{diskbox}
	\url{https://github.com/adamsralf/pygame_book/tree/main/src/02%20Examples/02%20Bubbles/v07}
\end{diskbox}

\br{Score}{reqPunktestand}
\begin{enumerate}
	\item The game starts with 0~points.\label{reqPunktestandNull}
	\item When a bubble bursts, the score is increased proportionally to its radius.\label{reqPunktestandRadius}
	\item The score is displayed in the lower part of the screen.\label{reqPunktestandAnzeige}
\end{enumerate}
\er

Popping bubbles should, of course, be rewarded with points. To do this, the score has to be calculated and displayed. The simplest way to keep track of the score is to use a variable in \texttt{config.py} or a global variable. I prefer option~1 (see \srcref[vref]{srcbubbles07a}).

\lstsource{SRC/02 Examples/02 Bubbles/v07/config.py}{18}{18}{python}{Bubbles (\reqref{reqPunktestand}.\ref{reqPunktestandNull}) -- extension of \texttt{config.py}}{srcbubbles07a}  

Since popping a bubble no longer only makes it disappear but also updates the score, I added a new method to the \texttt{Bubble} class. In \zeiref{srcBubble0702}, the radius of the bubble is simply added to the score.

\lstsource{SRC/02 Examples/02 Bubbles/v07/bubbles.py}{74}{76}{python}{Bubbles (\reqref{reqPunktestand}.\ref{reqPunktestandRadius}) -- \texttt{Bubble.stung()}}{srcbubbles07b}  

The call to \texttt{stung()} is triggered by an adjusted \texttt{update()} method.

\lstsource{SRC/02 Examples/02 Bubbles/v07/bubbles.py}{55}{66}{python}{Bubbles (\reqref{reqPunktestand}.\ref{reqPunktestandRadius}) -- \texttt{Bubble.update()}}{srcbubbles07b1}  

The \texttt{sting()} and \texttt{update()} methods in \texttt{Game} have to be adjusted accordingly (see \zeiref{srcBubble0704} and \zeiref{srcBubble0705}).

\lstsource{SRC/02 Examples/02 Bubbles/v07/bubbles.py}{159}{162}{python}{Bubbles (\reqref{reqPunktestand}.\ref{reqPunktestandRadius}) -- \texttt{Game.sting()}}{srcbubbles07c}  

\lstsource{SRC/02 Examples/02 Bubbles/v07/bubbles.py}{122}{125}{python}{Bubbles (\reqref{reqPunktestand}.\ref{reqPunktestandRadius}) -- \texttt{Game.update()}}{srcbubbles07d}  

This leaves \reqref{reqPunktestand}.\ref{reqPunktestandAnzeige}. Similar to the playing area, I want to define the dimensions of the lower section as an output box in \texttt{config.py}.\randnotiz{Rect}\myindex{pyg}{\texttt{Rect}}

\lstsource{SRC/02 Examples/02 Bubbles/v07/bubbles.py}{19}{19}{python}{Bubbles (\reqref{reqPunktestand}.\ref{reqPunktestandAnzeige}) -- extension of \texttt{config.py}}{srcbubbles07e}  

For displaying the score itself, I once again create a small class that encapsulates this task: \texttt{Points}. In the constructor, a \texttt{Font}\myindex{pyg}{\texttt{font}!\texttt{Font}}\randnotiz{Font} object is created, which is then used in \texttt{update()} to render the score. The position of the text output is determined from the values defined in \texttt{config.py}. The rest is handled for me by the \texttt{Sprite} class.

\lstsource{SRC/02 Examples/02 Bubbles/v07/bubbles.py}{79}{90}{python}{Bubbles (\reqref{reqPunktestand}.\ref{reqPunktestandAnzeige}) -- Class \texttt{Points}}{srcbubbles07f}  

A few extensions remain in \texttt{Game}. In the constructor, the \texttt{Points} object is added to the \texttt{Group} object\myindex{pyg}{\texttt{sprite}!\texttt{Group}}.

\lstsource{SRC/02 Examples/02 Bubbles/v07/bubbles.py}{102}{104}{python}{Bubbles (\reqref{reqPunktestand}.\ref{reqPunktestandAnzeige}) -- extension of the \texttt{Game} constructor}{srcbubbles07g}  

\myebild{aquarium04}{0.30}{Bubbles -- score display}{picAquarium04}

In \abbref[vref]{picAquarium04}, you can see the score display in the lower part of the screen. This area could later also be used for a list of the top ten scores or other types of output.

%%%%%%%%%%%%%%%%%%%%%%%%%%%%%%%%%%%%%%%%%%%%%%%%%%%%%%%%%%%%%%%%%%%%%
\subsection{\Reqref{reqSpielende}: Game over}
\begin{diskbox}
	\url{https://github.com/adamsralf/pygame_book/tree/main/src/02%20Examples/02%20Bubbles/v08}
\end{diskbox}

\br{Game over}{reqSpielende}
\begin{enumerate}
	\item If two bubbles touch, the game is lost.\label{reqSpielendeBlase}
	\item If a bubble touches the edge, the game is lost.\label{reqSpielendeRand}
\end{enumerate}
\er

Note: To make the game playable, I set the growth speed of a bubble to 10.

\lstsource{SRC/02 Examples/02 Bubbles/v08/bubbles.py}{53}{53}{python}{Bubbles (\reqref{reqSpielende}) -- \texttt{Bubble.speed}}{srcbubbles08x}  

The basic structure of our game makes it fairly easy to implement this requirement by extending the \texttt{update()} method in \texttt{Game}.

\lstsource{SRC/02 Examples/02 Bubbles/v08/bubbles.py}{125}{131}{python}{Bubbles (\reqref{reqSpielende}) -- extension of \texttt{Game.update()}}{srcbubbles08a}

In the new method \texttt{check\_\-bubble\-coll\-ision()}, it is checked whether bubbles touch each other or whether a bubble collides with the edge. This method is simply used as a decision maker (\zeiref{srcBubble0801}) to determine whether the game should end. If the answer is \emph{yes}, the flag of the main game loop is set; if the answer is \emph{no}, the remaining game logic is executed as usual. The two nested \forSchleife s starting at \zeiref{srcBubble0802} iterate over the bubble group twice and avoid two problems:\myindex{pyg}{\texttt{sprite}!\texttt{Group}!\texttt{sprites()}}\randnotiz{sprites()}

\begin{itemize}
	\item A bubble must not be compared with itself. Therefore, the index of the inner loop always starts one position after the current index of the outer loop, and the outer loop index ends before the last element of the bubble group.
	
	\item If bubble~1 has already been compared with bubble~2, bubble~2 should not be compared again with bubble~1. This is also achieved by the shifted index.
\end{itemize}

\myezweihbild{aquarium05}{0.65}{Bubbles -- collision with the edge}{picAquarium05}{aquarium06}{0.70}{Bubbles -- bubble collision}{picAquarium06}


In \zeiref{srcBubble0803}, \reqref{reqSpielende}.\ref{reqSpielendeBlase} is checked. For this purpose, circle-based collision detection using \texttt{collide\_circle()} is applied. In \zeiref{srcBubble0804} and \zeiref{srcBubble0805}, \reqref{reqSpielende}.\ref{reqSpielendeRand} is implemented. This makes use of the fact that the playing area is a rectangle and that the sprite itself also has a rectangular shape. The method \texttt{pygame.rect.Rect.\-con\-tains()}\myindex{pyg}{\texttt{rect}!\texttt{Rect}!\texttt{contains()}}\randnotiz{contains()} checks whether one rectangle is completely contained within another. If this is not the case -- meaning the bubble leaves the playing area -- a collision is detected.

\lstsource{SRC/02 Examples/02 Bubbles/v08/bubbles.py}{170}{182}{python}{Bubbles (\reqref{reqSpielende}) -- \texttt{Game.check\_bubblecollision()}}{srcbubbles08b}  

In \abbref[vref]{picAquarium05}, the collision of a bubble with the edge is shown. To make this easier to see, helper lines have been drawn. You can clearly see that the rectangle of the bubble is no longer contained within the rectangle of the playing area. \abbref[vref]{picAquarium06} shows the collision of two bubbles. Here as well, helper lines are drawn. These helper lines are displayed when you remove the three comment characters in \texttt{Game.draw()}.

\lstsource{SRC/02 Examples/02 Bubbles/v08/bubbles.py}{117}{123}{python}{Bubbles (\reqref{reqSpielende}) -- helper lines in \texttt{Game}}{srcbubbles08c}


%%%%%%%%%%%%%%%%%%%%%%%%%%%%%%%%%%%%%%%%%%%%%%%%%%%%%%%%%%%%%%%%%%%%%
\subsection{\Reqref{reqZeitanpassungen}: Time-based adjustments}
\begin{diskbox}
	\url{https://github.com/adamsralf/pygame_book/tree/main/src/02%20Examples/02%20Bubbles/v09}
\end{diskbox}

\br{Time-based adjustments}{reqZeitanpassungen}
	The bubbles should grow faster over time.
\er

Since the bubbles are supposed to grow faster over time, I want to pass the growth speed to them as a constructor parameter. In \zeiref{srcBubble0900}, this parameter is stored in an attribute.

\lstsource{SRC/02 Examples/02 Bubbles/v09/bubbles.py}{46}{54}{python}{Bubbles (\reqref{reqZeitanpassungen}) -- \texttt{Bubble}}{srcBubbles09a}  

These are all the required changes in \texttt{Bubble}; everything else happens in \texttt{Game}. In \zeiref{srcBubble0901}, a timer\randnotiz{Timer}\index{Timer} is created that emits a signal every \SI{1000}{ms}. Below that, the initial growth speed of the bubbles is set to \SI{10}{px}.

\lstsource{SRC/02 Examples/02 Bubbles/v09/bubbles.py}{101}{105}{python}{Bubbles (\reqref{reqZeitanpassungen}) -- adjustment of the \texttt{Game} constructor}{srcBubbles09b}

In \texttt{spawn\_bubble()}, the timer is checked and, if necessary, the bubble growth speed is increased (\zeiref{srcBubble0902}). The maximum growth speed is limited to~\SI{100}{px/s}; anything faster does not seem playable to me. Each timer signal increases the speed by~\SI{5}{px/s}. This is done in this method so that the new speed is available for newly created bubbles.

\lstsource{SRC/02 Examples/02 Bubbles/v09/bubbles.py}{133}{147}{python}{Bubbles (\reqref{reqZeitanpassungen}) -- \texttt{Game.spawn\_bubble()}}{srcBubbles09c}  

If you now try out the game, you will notice an easy start and a moderate increase in game difficulty.


%%%%%%%%%%%%%%%%%%%%%%%%%%%%%%%%%%%%%%%%%%%%%%%%%%%%%%%%%%%%%%%%%%%%%
\subsection{\Reqref{reqKollisionanzeigen}: Display collision}
\begin{diskbox}
	\url{https://github.com/adamsralf/pygame_book/tree/main/src/02%20Examples/02%20Bubbles/v10}
\end{diskbox}

\br{Display collision}{reqKollisionanzeigen}
	If bubbles collide with the edge or with each other, they should change color and remain visible for~\SI{5}{s} before the application terminates.
\er

\begin{wrapfigure}[6]{r}{3.5cm}%
	\begin{center}%
		\vspace{-1cm}%
		\myfigure{blase2.png}{0.05}{Bubble~2}{picBlase2}%
	\end{center}%
\end{wrapfigure}%
So far, the game ends so quickly that I cannot really check whether I actually lost for a valid reason or whether the program is misbehaving. With this requirement, I want to be able to see the two colliding bubbles, or the bubble that touches the edge, in a different color. For this purpose, I colored the bubble red (see \abbref{picBlase2}).


To achieve this, a second \texttt{BubbleContainer} with scaled red bubbles is required. To make access easier, these containers are stored in \texttt{Game} as a static dictionary.

In \zeiref{srcBubble1001}, such a dictionary is created. Under a key, I can now store arbitrary \texttt{BubbleContainer} objects.
 
\lstsource{SRC/02 Examples/02 Bubbles/v10/bubbles.py}{101}{102}{python}{Bubbles (\reqref{reqKollisionanzeigen}) -- extension of \texttt{Game}}{srcbubbles10a}

The constructor of \texttt{BubbleContainer} now receives a filename as a parameter, allowing different graphics to be used as a basis.

\lstsource{SRC/02 Examples/02 Bubbles/v10/bubbles.py}{34}{38}{python}{Bubbles (\reqref{reqKollisionanzeigen}) -- change to the constructor of \texttt{BubbleContainer}}{srcbubbles10b}

The constructor of \texttt{Game} now populates the static dictionary \texttt{BUBBLE\_CONTAINER} (\zeiref{srcBubble1003} and \zeiref{srcBubble1004}).

\newpage
\lstsource{SRC/02 Examples/02 Bubbles/v10/bubbles.py}{108}{111}{python}{Bubbles (\reqref{reqKollisionanzeigen}) -- change to the constructor of \texttt{Game}}{srcbubbles10c}

Several changes are now required in \texttt{Bubble}. The new attribute \texttt{mode} (\zeiref{srcBubble1005}) determines the color of the bubble. Whenever an image is loaded from the \texttt{Bubble\-Con\-tainer}, this attribute controls which of the two \texttt{Bubble\-Con\-tainer} instances is used as the data source. As an example, \zeiref{srcBubble1006} in \texttt{update()} can be mentioned here.

\lstsource{SRC/02 Examples/02 Bubbles/v10/bubbles.py}{46}{69}{python}{Bubbles (\reqref{reqKollisionanzeigen}) -- constructor of \texttt{Bubble} and \texttt{update()}}{srcbubbles10d}

If the mode changes, the alternative color has to be reloaded. This is handled by the \texttt{set\_mode()} method in \texttt{Bubble}.

\lstsource{SRC/02 Examples/02 Bubbles/v10/bubbles.py}{71}{74}{python}{Bubbles (\reqref{reqKollisionanzeigen}) -- \texttt{Bubble.set\_mode()}}{srcbubbles10e}  

\newpage
\begin{wrapfigure}[15]{l}{11.0cm}%
	\begin{center}%
		\vspace{-1.0em}%
		\myfigure{aquarium07.png}{0.35}{Bubbles -- displaying a collision}{picAquarium07}%
	\end{center}%
\end{wrapfigure}%
Now, in the case of a collision -- that is, when the game ends -- the mode simply needs to be changed.

In \abbref[vref]{picAquarium07}, you can see how the two colliding bubbles appear in red. An example of how this is implemented can be found in \zeiref{srcBubble1008}. In a nested \forSchleife, all sprites are iterated over.

A naive approach would cause problems if both loops compared all sprites from start to finish. First, because due to symmetry you would check $B=A$ even though you have already checked $A=B$. Second, because you would end up comparing each sprite with itself, which would lead to a false collision detection.

\begin{figure}[H]
	\begin{center}
		\begin{tikzpicture}[
			font=\small,
			cell/.style={draw, minimum width=7mm, minimum height=7mm, align=center},
			idx/.style={font=\scriptsize},
			ok/.style={fill=green!12},
			skip/.style={fill=gray!15},
			self/.style={fill=red!12},
			arrow/.style={-Latex, thick},
			note/.style={align=left, text width=6.2cm}
			]
			
			% --- Parameters (visual only): we illustrate N=6 bubbles (indices 0..5)
			\def\N{6}
			
			% --- Grid origin
			\coordinate (O) at (0,-1.2);
			
			% --- Axis labels
			\node[idx, anchor=east] at ($(O)+(0,0)$) {};
			\node[idx, anchor=south] at ($(O)+(0,0)$) {};
			
			\node[idx, anchor=south]           at ($(O)+(2.60cm,   0.50cm)$) {inner index \texttt{$index_2$} $\rightarrow$};
			\node[idx, anchor=east, rotate=90] at ($(O)+(-0.50cm, -0.90cm)$) {$\leftarrow$ outer index \texttt{$index_1$}};
			
			% --- Draw header row/col indices
			\foreach \j in {0,...,5} {
				\node[idx, anchor=south] at ($(O)+(0.9cm+0.7cm*\j, -0.2cm)$) {\j};
				\node[idx, anchor=east]  at ($(O)+(0.55cm,         -0.7cm*\j-0.6cm)$) {\j};
			}
			
			% --- Draw cells: rows=i (index1), cols=j (index2)
			% Convention:
			%  - diagonal (i=j): self comparison -> forbidden
			%  - below diagonal (j<i): duplicates (would compare (2,1) after (1,2)) -> skipped
			%  - above diagonal (j>i): valid comparisons -> executed
			\foreach \i in {0,...,5} {
				\foreach \j in {0,...,5} {
					\pgfmathtruncatemacro{\x}{\j}
					\pgfmathtruncatemacro{\y}{\i}
					\coordinate (C\i\j) at ($(O)+(0.9cm+0.7cm*\j, -0.7cm*\i-0.6cm)$);
					
					\ifnum\j=\i
					\node[cell,self] at (C\i\j) {$\times$};
					\else
					\ifnum\j<\i
					\node[cell,skip] at (C\i\j) {};
					\else
					\node[cell,ok] at (C\i\j) {$\large \bullet$};
					\fi
					\fi
				}
			}
			
			% --- Legend
			\node[draw, rounded corners, inner sep=6pt, anchor=west] (legend) at ($(O)+(5.2cm, -2.5cm)$) {
				\begin{tabular}{@{}l@{\;\;}l@{}}
					\tikz\draw[fill=green!12,draw] (0,0) rectangle (0.25,0.25); & executed ($j>i$)\\
					\tikz\draw[fill=red!12,draw]   (0,0) rectangle (0.25,0.25); & illegal self-check ($j=i$)\\
					\tikz\draw[fill=gray!15,draw]  (0,0) rectangle (0.25,0.25); & duplicate pair ($j<i$)\\
				\end{tabular}
			};
		\end{tikzpicture}
		\caption{Why does the inner loop start at \texttt{$index_1+1$}?}\label{picForFor01}
	\end{center}
\end{figure}

In \abbref[vref]{picForFor01}, the algorithm used is illustrated. For each outer index \texttt{i=$index_1$}, the inner loop runs \texttt{j=$index_2$} from $i+1$ to $N-1$: \texttt{for $index_2$ in range($index_1+1$, N)}

\begin{itemize}
	\item Starting at $i+1$ skips the diagonal ($j=i$), so a bubble is never compared with itself.
	\item It also skips everything left of the diagonal ($j<i$), which would repeat comparisons
	(e.g.\ $(1,2)$ and later $(2,1)$).
	\item The outer loop stops at $N-2$ (i.e.\ \texttt{range(0, N-1)}), because when $i=N-1$
	there is no valid $j>i$ left to compare.
\end{itemize}


After that, it is checked whether both sprites are of type \texttt{Bubble}. Only then is a collision check performed — before that, it would not be worthwhile. If a collision is detected, both bubbles are colored red and the function is exited.

But what if the two bubbles do not touch each other? In that case, there is no need to check whether either of them touches the edge. If one of them does, that bubble is also colored red and the function is exited.


\lstsource{SRC/02 Examples/02 Bubbles/v10/bubbles.py}{183}{199}{python}{Bubbles (\reqref{reqKollisionanzeigen}) -- \texttt{Game.check\_bubblecollision()}}{srcbubbles10f}  

To give me enough time to see the collision, I want to wait for~\SI{2}{s} at the end. The method \texttt{pygame.time.wait()}\myindex{pyg}{\texttt{time}!\texttt{wait()}}\randnotiz{wait()} pauses the application for the specified duration (\zeiref{srcBubble1007}).

\lstsource{SRC/02 Examples/02 Bubbles/v10/bubbles.py}{201}{213}{python}{Bubbles (\reqref{reqKollisionanzeigen}) -- waiting time in \texttt{run()}}{srcbubbles10g}


%%%%%%%%%%%%%%%%%%%%%%%%%%%%%%%%%%%%%%%%%%%%%%%%%%%%%%%%%%%%%%%%%%%%%
\newpage
\subsection{\Reqref{reqPause}: Pause}
\begin{diskbox}
	\url{https://github.com/adamsralf/pygame_book/tree/main/src/02%20Examples/02%20Bubbles/v11}
\end{diskbox}

\br{Pause}{reqPause}
	The game enters or leaves pause mode by pressing the right mouse button or \keys{p}. The current game state is frozen and displayed in a \emph{grayed-out} form.
\er

The idea behind this requirement is that a necessary interruption should not automatically mean that the player loses the game. In \abbref[vref]{picAquarium08}, you can see what the pause screen should look like.

\myebild{aquarium08}{0.30}{Bubbles -- pause screen}{picAquarium08}

In the constructor of \texttt{Game}, the flag \texttt{pausing} is defined. It later controls whether the game is currently in pause mode or not.

\newpage
\lstsource{SRC/02 Examples/02 Bubbles/v11/bubbles.py}{125}{125}{python}{Bubbles (\reqref{reqPause}) -- constructor in \texttt{Game}}{srcbubbles11b}

In \texttt{watch\_for\_events()}, it is now checked whether the \texttt{P} key (\zeiref{srcBubble1103}) or the right mouse button (\zeiref{srcBubble1104}) has been pressed. In both cases, the new \texttt{setpause()} method is called.

\lstsource{SRC/02 Examples/02 Bubbles/v11/bubbles.py}{128}{141}{python}{Bubbles (\reqref{reqPause}) -- \texttt{Game.watch\_for\_events()}}{srcbubbles11c}  

Für die Darstellung der Pause, habe ich die -- vielleicht etwas überflüssige -- Klasse \texttt{Pause} implementiert. 

\lstsource{SRC/02 Examples/02 Bubbles/v11/bubbles.py}{45}{50}{python}{Bubbles (\reqref{reqPause}) -- Class \texttt{Pause}}{srcbubbles11d}  

In the constructor of \texttt{Game}, an object of the \texttt{Pause} class is created so that it can be used in \texttt{draw()}.

\lstsource{SRC/02 Examples/02 Bubbles/v11/bubbles.py}{126}{126}{python}{Bubbles (\reqref{reqPause}) -- constructor in \texttt{Game}}{srcbubbles11g}

However, the \texttt{setpause()} method still needs to be explained. This method either adds the \texttt{Pause} object to the list of sprites or removes it again, depending on whether the game is currently in pause mode or not. Afterwards, the boolean value of the flag is negated (\gls{toggling}).

\lstsource{SRC/02 Examples/02 Bubbles/v11/bubbles.py}{157}{162}{python}{Bubbles (\reqref{reqPause}) -- \texttt{Game.setpause()}}{srcbubbles11h}  

Nothing more is required, since the rest is handled by the usual \texttt{update()} and \texttt{draw()} mechanisms.

%%%%%%%%%%%%%%%%%%%%%%%%%%%%%%%%%%%%%%%%%%%%%%%%%%%%%%%%%%%%%%%%%%%%%
\subsection{\Reqref{reqNeustart}: Restart}
\begin{diskbox}
	\url{https://github.com/adamsralf/pygame_book/tree/main/src/02%20Examples/02%20Bubbles/v12}
\end{diskbox}

\br{Restart}{reqNeustart}
	At the end of the game, the player should be asked whether they want to restart the game or not.
\er

The basic idea of the implementation is to define the state of the game using two flags\index{flag}\randnotiz{flag}. As with the pause feature, we need a flag that controls whether the semi-transparent foreground is placed over the game (\texttt{restarting}). This is always the case when the bubble collision check detects a collision.

The other flag -- \texttt{do\_start} -- indicates whether the player wants to restart. At the relevant points in \texttt{update()} and \texttt{draw()}, these flags are then evaluated.

The task of displaying a confirmation dialog in the foreground is essentially already solved by the \texttt{Pause} class. I can therefore generalize this class by renaming it to \texttt{Message} and passing the filename to the constructor as a string parameter (\zeiref{srcBubble1202}).

\lstsource{SRC/02 Examples/02 Bubbles/v12/bubbles.py}{45}{50}{python}{Bubbles (\reqref{reqNeustart}) -- from \texttt{Pause} to \texttt{Message}}{srcbubbles12b}

\myebild{aquarium09}{0.3}{Bubbles -- restart screen}{picAquarium09}

In \texttt{Game}, starting at \zeiref{srcBubble1203}, the adjustments required for restarting the game are implemented in the constructor. Essentially, two \texttt{Message} objects are created for pause and restart, and all attributes that need to be reset on a start or restart are handled in the new \texttt{restart()} method.


\lstsource{SRC/02 Examples/02 Bubbles/v12/bubbles.py}{121}{124}{python}{Bubbles (\reqref{reqNeustart}) -- restructuring of the \texttt{Game} constructor}{srcbubbles12c}

The score is reset, the sprite group containing the bubbles is cleared, the timers are reinitialized, the bubble growth speed is reset to its initial value, and the two flags described above are set to \false.

\lstsource{SRC/02 Examples/02 Bubbles/v12/bubbles.py}{163}{171}{python}{Bubbles (\reqref{reqNeustart}) -- \texttt{Game.restart()}}{srcbubbles12d}

The method is called in \texttt{update()} when the corresponding flag \texttt{do\_start} is set. In addition, \texttt{update()} inserts the restart screen into the sprite group and sets the flag \texttt{restarting} to \true\ when a collision is detected.

\newpage
\lstsource{SRC/02 Examples/02 Bubbles/v12/bubbles.py}{150}{161}{python}{Bubbles (\reqref{reqNeustart}) -- \texttt{Game.update()}}{srcbubbles12e}  

The response to the restart screen is queried in \texttt{watch\_for\_events()} and translated into the corresponding flag values. If the player responds with~\keys{y} (\zeiref{srcBubble1205}), the game has to be restarted, so \texttt{do\_start} is set to \true. If the player presses~\emph{N}, the game should end, which is why the flag of the main game loop is set to \false\ (\zeiref{srcBubble1206}).

\lstsource{SRC/02 Examples/02 Bubbles/v12/bubbles.py}{135}{138}{python}{Bubbles (\reqref{reqNeustart}) -- extension of \texttt{watch\_for\_events()}}{srcbubbles12f}
  
Since we now display a semi-transparent foreground at the end of the game, there is no longer any need for a two-second pause to inspect the colliding bubbles (see \abbref[vref]{picAquarium09}).


%%%%%%%%%%%%%%%%%%%%%%%%%%%%%%%%%%%%%%%%%%%%%%%%%%%%%%%%%%%%%%%%%%%%%
\subsection{\Reqref{reqSound}: Sound}
\begin{diskbox}
	\url{https://github.com/adamsralf/pygame_book/tree/main/src/02%20Examples/02%20Bubbles/v13}
\end{diskbox}

\br{Sound}{reqSound}
\begin{enumerate}
	\item The appearance of bubbles is accompanied by a sound.\label{reqSoundErscheinen}
	\item Popping a bubble is accompanied by a sound.\label{reqSoundZerstechen}
	\item Touching a bubble is accompanied by a sound.\label{reqSoundBerühren}
\end{enumerate}
\er

Finally, a small sound accompaniment is added. Similar to the bubble sprites, I do not want to lose performance by repeatedly loading sound files. Therefore, the sounds are stored in a static dictionary (\zeiref{srcBubble1301}).

\lstsource{SRC/02 Examples/02 Bubbles/v13/bubbles.py}{109}{111}{python}{Bubbles (\reqref{reqSound}) -- \texttt{SOUND\_CONTAINER}}{srcbubbles13a}  

In the constructor of \texttt{Game}, the dictionary is populated with objects of the \texttt{Sound} class. The class used for sound effects is \texttt{pygame.mixer.Sound}\myindex{pyg}{\texttt{mixer}!\texttt{Sound}}\randnotiz{Sound} (see \zeiref{srcBubble1302} ff.).

\lstsource{SRC/02 Examples/02 Bubbles/v13/bubbles.py}{113}{122}{python}{Bubbles (\reqref{reqSound}) -- populating \texttt{SOUND\_CONTAINER}}{srcbubbles13b}

Now the sounds only need to be played at the appropriate places using \texttt{pygame.mixer\-.Sound\-.play()}\myindex{pyg}{\texttt{mixer}!\texttt{Sound}!\texttt{play()}}\randnotiz{play()}. First, the sound that is played when a new bubble appears: in \texttt{spawn\_\-bubb\-le()} in \zeiref{srcBubble1303}.

\lstsource{SRC/02 Examples/02 Bubbles/v13/bubbles.py}{198}{201}{python}{Bubbles  (\reqref{reqSound}.\ref{reqSoundErscheinen}) -- \texttt{spawn\_bubble()}}{srcbubbles13c}

Then, when a bubble bursts in \texttt{sting()} (\zeiref{srcBubble1304}):

\lstsource{SRC/02 Examples/02 Bubbles/v13/bubbles.py}{222}{226}{python}{Bubbles (\reqref{reqSound}.\ref{reqSoundZerstechen}) -- \texttt{sting()}}{srcbubbles13d}

Finally, the sound is played when a collision with other bubbles or with the edge occurs in \texttt{update()}. Here, it must also be taken into account whether the game is currently displaying the restart prompt. If the answer is \emph{yes}, the sound must not be played again; otherwise, the touch sound would be played continuously.

\lstsource{SRC/02 Examples/02 Bubbles/v13/bubbles.py}{155}{167}{python}{Bubbles (\reqref{reqSound}.\ref{reqSoundBerühren}) -- \texttt{update()}}{srcbubbles13e}  

And that’s it :-)

%%%%%%%%%%%%%%%%%%%%%%%%%%%%%%%%%%%%%%%%%%%%%%%%%%%%%%%%%%%%%%%%%%%%%
\subsection{Or maybe not?}
\begin{diskbox}
	\url{https://github.com/adamsralf/pygame_book/tree/main/src/02%20Examples/02%20Bubbles/v14}
\end{diskbox}

Pause, restart, and game over are currently still implemented rather sloppily; for example, the end of the game is delayed by playing a sound instead of using a properly programmed delay. In one version (see \srcref[vref]{srcbubbles10g}), I even froze the entire program for several seconds using \texttt{pygame.time.wait()}\myindex{pyg}{\texttt{time}!\texttt{wait()}} -- brrr :-(

It is far more advantageous to control the game using states. What exactly this means should become clear when looking at the solution.

As a first step, let us think about which states the game can actually have:

\begin{itemize}
	\item \texttt{PLAYING}: I am happily popping bubbles and collecting points.
	\item \texttt{PAUSED}: \keys{p} has been pressed and the game should stop temporarily, but not terminate.
	\item \texttt{WAITING}: The game has ended, but remains active for a few more seconds, for example to display a farewell message.
	\item \texttt{GAME\_OVER}: The game is exited. In our example, little or nothing happens here, but one might still want to close files or \glspl{socket}.
\end{itemize}

Let us simply add an enumeration for this:

\lstsource{SRC/02 Examples/02 Bubbles/v14/bubbles.py}{11}{15}{python}{Bubbles game state -- \texttt{GameState(Enum)}}{srcBubbles14a}  

Since we now have to manage more than just a single state indicating whether the game is paused or not, the attribute \texttt{self.state} is created in the constructor of \texttt{Game}. In addition, two attributes are introduced that will later be used to measure whether the game remains in a waiting state for \SI{10}{s} before it finally terminates.

\lstsource{SRC/02 Examples/02 Bubbles/v14/bubbles.py}{134}{138}{python}{Bubbles game state -- Constructor of \texttt{Game}}{srcBubbles14b}  

A real refactoring is now required for \texttt{watch\_for\_events()}. It no longer only checks for events, but also takes into account the current state of the game.

Consider pressing \keys{p} as an example: if the game state is currently \texttt{PLAYING}, the game switches to pause mode by calling \texttt{set\_pause()}. If the game is in the \texttt{PAUSED} state, calling \texttt{resume()} switches the game back to running mode.

\lstsource{SRC/02 Examples/02 Bubbles/v14/bubbles.py}{142}{167}{python}{Bubbles game state -- \texttt{Game.watch\_for\_events()}}{srcBubbles14c}  

Let us now take a closer look at the three helper methods, each of which is tailored to a specific state: \texttt{set\_pause()}, \texttt{set\_resume()}, and \texttt{set\_game\_over()}. All three methods consist of two parts: in the first part, the state change itself is performed, and in the second part, an action required by this state change is executed. In some cases, a message screen is added to the sprite group; in others, it is removed again. In the third method, the time at which the \SI{10}{seconds} waiting period begins is stored in \texttt{wait\_start\_time}.

\lstsource{SRC/02 Examples/02 Bubbles/v14/bubbles.py}{201}{211}{python}{Bubbles game state -- \texttt{set\_pause()}, \texttt{set\_resume()}, and \texttt{set\_game\_over()} of \texttt{Game}}{srcBubbles14d}  

Just as the game states are taken into account in \texttt{watch\_for\_events()}, they must also be considered in \texttt{update()}.

\lstsource{SRC/02 Examples/02 Bubbles/v14/bubbles.py}{174}{189}{python}{Bubbles game state -- \texttt{Game.update()}}{srcBubbles14e}  

What remains is the helper method that checks whether the \SI{10}{seconds} have elapsed: \texttt{check\_waiting\_timeout()}. In this method, the elapsed time is compared with \texttt{wait\_du\-ra\-tion}. If the time has elapsed, the game state is set to \texttt{GAME\_OVER}, so that the game can terminate cleanly.

\lstsource{SRC/02 Examples/02 Bubbles/v14/bubbles.py}{274}{279}{python}{Bubbles game state -- \texttt{Game.check\_waiting\_timeout()}}{srcBubbles14f}  

In \abbref[vref]{picBubbleState}, we can once again visually trace these state transitions.

\begin{figure}[H]
	\begin{center}
	\begin{tikzpicture}[
		state/.style={
			rectangle,
			rounded corners=0.5cm,
			draw=arrowblack,
			fill=stateblue,
			minimum width=2cm,
			minimum height=1cm,
			text centered,
			font=\small\bfseries
		},
		initial/.style={
			circle,
			draw=arrowblack,
			fill=arrowblack,
			minimum size=0.3cm,
			inner sep=0pt
		},
		final/.style={
			circle,
			draw=arrowblack,
			fill=white,
			minimum size=0.6cm,
			inner sep=0pt,
			line width=2pt
		},
		finalinner/.style={
			circle,
			fill=arrowblack,
			minimum size=0.2cm,
			inner sep=0pt
		},
		arrow/.style={
			->,
			thick,
			font=\small
		},
		loopabove/.style={
			min distance=2.5cm,
			loop,
			arrow,
			above=0.8cm,
			looseness=7
		},
		loopbelow/.style={
			min distance=2.5cm,
			loop,
			arrow,
			below=-0.8cm,
			looseness=7
		},
		arrowabove/.style={
			min distance=2.5cm,
			arrow,
			above=1.0cm,
			out=20, 
			in=160
		},
		arrowbelow/.style={
			min distance=2.5cm,
			arrow,
			below=-1.0cm,
			out=-160, 
			in=-20
		},
		arrowleft/.style={
			min distance=2.5cm,
			arrow,
			out=150, 
			in=-150
		},
		arrowright/.style={
			min distance=2.5cm,
			arrow,
			out=-30, 
			in=30
		}
		]
		
		% Startzustand
		\node[initial] (start) at (0, 0) {};
		
		% Zustände
		\node[state] (playing) at (5.0, 0) {PLAYING};
		\node[state] (paused) at (13.0, 0) {PAUSED};
		\node[state] (waiting) at (5.0, -4) {WAITING};
		\node[state] (gameover) at (5.0, -7) {GAME\_OVER};
		
		% Endzustand (Bullseye)
		\node[final] (end) at (5.0, -9) {};
		\node[finalinner] at (5.0, -9) {};
		
		% Übergänge
		% Start -> PLAYING
		\draw[arrow] (start) -- node[above, fill=white] {} (playing);
		
		% PLAYING <-> PAUSED
		\draw[arrowabove] (playing) to node[above, fill=white] {\keys{p} / right click} (paused);
		\draw[arrowbelow] (paused) to node[below, fill=white] {\keys{p} / right click} (playing);
		
		% PLAYING -> WAITING
		\draw[arrowright] (playing) to node[below right, fill=white] {collision} (waiting);
		
		% WAITING -> GAME_OVER
		\draw[arrow] (waiting) -- node[right, fill=white] {\keys{n}} (gameover);

		% WAITING -> PLAYING (Neustart)
		\draw[arrowleft] (waiting) to node[below left, fill=white] {\keys{j} / restart()} (playing);

		% GAME_OVER -> End
		\draw[arrow] (gameover) -- node[right, fill=white] {\SI{10}{sec}} (end);
		
		
		% Linksklick-Aktion im PLAYING-Zustand (Loop)
		\draw[loopabove] (playing) to node[above, fill=white] {left click / sting()} (playing);
		
	\end{tikzpicture}
		\caption{Bubbles -- state diagram}\label{picBubbleState}
\end{center}
\end{figure}
	



	% !TeX spellcheck = en_US
\newpage
\section{Moonlander}\index{Moonlander}
%%%%%%%%%%%%%%%%%%%%%%%%%%%%%%%%%%%%%%%%%%%%%%%%%%%%%%%%%%%%%%%%%%%%%
\begin{wrapfigure}[9]{r}{5.0cm}%
\begin{center}%
	\vspace{-8ex}%
	\myfigure{moonlander01}{0.3}{Moonlander (1)}{picMoonlander01}%
\end{center}%
\end{wrapfigure}%
In this chapter, we will build a Moon Lander game. For this project, I want to avoid pre-made sprites and create all graphics entirely from drawing primitives (see \abschnittref[vref]{secGrafikprimitive}).

We will develop this game systematically, step by step, assuming that the techniques from \kapref{secBasics} are already familiar. I will omit docstring comments in the source code, since everything is explained in the text and the listings would otherwise become unnecessarily long. In the final version, they are included.

%%%%%%%%%%%%%%%%%%%%%%%%%%%%%%%%%%%%%%%%%%%%%%%%%%%%%%%%%%%%%%%%%%%%%
\subsection{\Reqref{reqMoonStandard}: Standards}

\br{Standard functionality}{reqMoonStandard}
\begin{enumerate}
	\item The window has a size of $600\times \SI{800}{px}$.\label{reqMoonStandardGröße}
	\item The background is divided into a black sky, a blue Earth in the upper-right corner, and the lunar surface.\label{reqMoonStandardHintergrund}
	\item The game can be exited using \keys{\esc} or by clicking the red~X.\label{reqMoonStandardBeenden}
	\item Pressing \keys{R} triggers a restart.
	\item The game runs at a speed independent of the frame rate.\label{reqMoonStandardFps}
\end{enumerate}
\er


\Reqref{reqMoonStandard}.\ref{reqMoonStandardGröße} is already defined in the preamble. In addition, \texttt{FPS} and the associated \texttt{DELTATIME} are defined in \texttt{config.py}. The constant \texttt{HORIZONT} specifies where the lunar surface ends and the black night sky begins.

\lstsource{SRC/02 Examples/03 Moonlander/v01/config.py}{1}{10}{python}{Moonlander (\reqref{reqMoonStandard}.\ref{reqMoonStandardGröße}) -- \texttt{config.py}}{srcMoon01a}


I implement \reqref{reqMoonStandard}.\ref{reqMoonStandardHintergrund} using three classes: \texttt{Sky}, \texttt{Moon}, and \texttt{Earth}. Let us start with the \texttt{Sky} class. It has a fairly simple basic structure. In the constructor, a reference to the window is passed in and the size of the sky is stored as a \texttt{Rect} object. Space is left at the bottom for the lunar surface. The \texttt{draw()} method then draws a black rectangle at the appropriate position.

\lstsource{SRC/02 Examples/03 Moonlander/v01/moonlander.py}{7}{17}{python}{Moonlander (\reqref{reqMoonStandard}.\ref{reqMoonStandardHintergrund}) -- Class \texttt{Sky}}{srcMoon01b}

The \texttt{Moon} class works in exactly the same way (see \srcref[vref]{srcMoon01c}). The only differences are the different position and the different color -- gray in this case.

\lstsource{SRC/02 Examples/03 Moonlander/v01/moonlander.py}{19}{30}{python}{Moonlander (\reqref{reqMoonStandard}.\ref{reqMoonStandardHintergrund}) -- Class \texttt{Moon}}{srcMoon01c}

The \texttt{Earth} class draws a blue sphere in the upper-right corner of the screen (see \srcref[vref]{srcMoon01d}).

\lstsource{SRC/02 Examples/03 Moonlander/v01/moonlander.py}{32}{44}{python}{Moonlander (\reqref{reqMoonStandard}.\ref{reqMoonStandardHintergrund}) -- Class \texttt{Earth}}{srcMoon01d}

As usual, the game is encapsulated in its own class: \texttt{Game}. The three objects only need to be integrated into the standard structure of \texttt{Game}. This is also where quitting and restarting the game are implemented.

In \srcref[vref]{srcMoon01d}, Pygame is initialized in the constructor of \texttt{Game}, a window is created, the window’s screen surface is obtained, and a \texttt{Clock} object is created for the delta-time logic (see \abschnittref[vref]{secDeltatime}).

\lstsource{SRC/02 Examples/03 Moonlander/v01/moonlander.py}{46}{51}{python}{Moonlander (\reqref{reqMoonStandard}) -- Constructor of \texttt{Game}}{srcMoon01e}

The structure of the \texttt{run()} method follows the examples shown above. Its core consists of calling the event handler, updating the game objects, and drawing the game objects; in addition, the delta-time logic is applied.

\lstsource{SRC/02 Examples/03 Moonlander/v01/moonlander.py}{54}{65}{python}{Moonlander (\reqref{reqMoonStandard}) -- \texttt{Game.run()}}{srcMoon01f}

The event handler should no longer come as a surprise. Using \texttt{QUIT} or \keys{\esc} ends the game, and pressing \keys{r} triggers a restart.

\lstsource{SRC/02 Examples/03 Moonlander/v01/moonlander.py}{67}{75}{python}{Moonlander (\reqref{reqMoonStandard}) -- \texttt{Game.watch\_for\_events()}}{srcMoon01g}

At the moment, the \texttt{update()} method serves only as a placeholder for functionality that will be added later.

\lstsource{SRC/02 Examples/03 Moonlander/v01/moonlander.py}{77}{78}{python}{Moonlander (\reqref{reqMoonStandard}) -- \texttt{Game.update()}}{srcMoon01h}

In \texttt{draw()}, the drawing methods of the game objects are called, and the window buffer is flipped.

\lstsource{SRC/02 Examples/03 Moonlander/v01/moonlander.py}{80}{84}{python}{Moonlander (\reqref{reqMoonStandard}) -- \texttt{Game.draw()}}{srcMoon01i}

The restart does not reset the state of the individual game objects. Instead, the objects are recreated entirely. This is the simplest way to implement a restart, but it is not suitable for every type of game.

\lstsource{SRC/02 Examples/03 Moonlander/v01/moonlander.py}{86}{90}{python}{Moonlander (\reqref{reqMoonStandard}) -- \texttt{Game.restart()}}{srcMoon01j}

All that remains is the actual call that starts the game.

\lstsource{SRC/02 Examples/03 Moonlander/v01/moonlander.py}{92}{999}{python}{Moonlander (\reqref{reqMoonStandard}) -- \texttt{main()}}{srcMoon01k}

After starting the program, a scene like the one shown in \abbref[vref]{picMoonlander01} appears.

%%%%%%%%%%%%%%%%%%%%%%%%%%%%%%%%%%%%%%%%%%%%%%%%%%%%%%%%%%%%%%%%%%%%%
\subsection{\Reqref{reqMondoberfläche}: Lunar surface}

So far, the lunar surface is just a gray rectangle. However, I want a gray mountainous landscape to enhance the visual appeal.

\br{Lunar surface}{reqMondoberfläche}
	The lunar surface consists of consecutively arranged mountain ranges.
\er

As a first step, I extend the constructor of \texttt{Moon} by adding the number of mountain ranges. Each mountain range is initially represented by a gray rectangle. The variations in height will be added later.

The actual lunar surface (landing area in \zeiref{moonlander02a01}) remains a rectangle with a height of \texttt{HORIZONT}. In \texttt{self.layers}, the information for each mountain range is stored as a list.

Starting at \zeiref{moonlander02a02}, the mountain ranges (the default is~5) are created. First, the color of each mountain range is defined (\zeiref{moonlander02a03}). Starting from a base color value of~180, an amount depending on the layer index is subtracted. The larger the layer index, the more is subtracted from~180. In terms of color, this means that the mountain range becomes darker. The farther away a mountain range (the layer) is, the darker it appears.


\begin{wrapfigure}[6]{r}{5.0cm}%
	\begin{center}%
		\vspace{-3ex}%
		\myfigure{moonlander02.png}{0.3}{Moonlander (2)}{picGebirge1}%
	\end{center}%
\end{wrapfigure}%
The height of a mountain range (\texttt{y}) is calculated by moving upward at least~$\SI{10}{px}$ from the upper edge of the landing area. This value is then increased by a random number between~5 and~30, so that the heights of the mountain ranges are not always the same. To ensure that the mountain ranges in the background always stand out nicely, this value is additionally multiplied by the layer index. Finally, \texttt{draw()} is adjusted (\srcref[vref]{srcMoon02ab}) so that the mountain ranges are drawn as rectangles. The lunar surface should look roughly like the one shown in \abbref[vref]{picGebirge1}.

\lstsource{SRC/02 Examples/03 Moonlander/v02/moonlandera.py}{20}{31}{python}{Moonlander (\reqref{reqMondoberfläche}) -- Constructor of \texttt{Moon}}{srcMoon02aa}

\lstsource{SRC/02 Examples/03 Moonlander/v02/moonlandera.py}{33}{42}{python}{Moonlander (\reqref{reqMondoberfläche}) -- \texttt{Moon.draw()}}{srcMoon02ab}

Now it is time to add the mountain peaks. The basic idea is to generate random height variations around the upper edge of each mountain range and subtract them from the height of that upper edge.

As a first step, the constructor of \texttt{Moon} is extended by the parameter \texttt{peaks}. In \zeiref{moonlander02b01}, the distance between two height variations is calculated and stored in \texttt{dist}. This value could also be randomized further, but for some reason I did not feel like doing that here.

\begin{wrapfigure}[6]{r}{5.0cm}%
	\begin{center}%
		\vspace{-1cm}%
		\myfigure{moonlander03.png}{0.3}{Moonlander (3)}{picGebirge2}%
	\end{center}%
\end{wrapfigure}%
Now, within the loop, a peak or a valley is generated for each mountain range. The determination of the color (the shade of gray) remains unchanged. In \texttt{lofPeaks}, the peaks are stored as a list of points. The first point is always located at the far left on the upper edge of the landing area (\zeiref{moonlander02b04}). This point serves as the starting point of our closed polygon.

After that, the list of peaks is extended with additional random points using a loop. In \zeiref{moonlander02b06}, a height variation between $\SI{-5}{px}$ and $\SI{10}{px}$ is chosen at random and subtracted from the upper edge. In the following line, the next peak is shifted to the right by adding \texttt{dist}. Once this inner \forSchleife\ has finished, the list of height points is complete and can be added to the corresponding layer in \zeiref{moonlander02b08}. Before doing so, however, the final point of the polygon chain must still be generated and added.

\lstsource{SRC/02 Examples/03 Moonlander/v02/moonlanderb.py}{20}{40}{python}{Moonlander (\reqref{reqMondoberfläche}) -- constructor of \texttt{Moon} with peaks}{srcMoon02ba}

The \texttt{draw()} method has now become pleasantly simple. For each mountain range, \texttt{draw.\-poly\-gon()} is called; the actual work is done in the constructor. The result can be admired in \abbref[vref]{picGebirge2}.

\lstsource{SRC/02 Examples/03 Moonlander/v02/moonlanderb.py}{42}{49}{python}{Moonlander (\reqref{reqMondoberfläche}) -- \texttt{Moon.draw()} with peaks}{srcMoon02bb}

\begin{wrapfigure}[5]{r}{5.0cm}%
	\begin{center}%
		\vspace{-1cm}%
		\myfigure{moonlander04.png}{0.3}{Moonlander (4)}{picGebirge3}%
	\end{center}%
\end{wrapfigure}%
Finally, I want to give the mountains a bit more contour. To do this, the single polygon is split into many smaller ones, where each polygon spans from one peak to the next. First, it becomes apparent that the number of peaks now varies for each mountain range (\zeiref{moonlander02c01}); this makes the ranges appear less like a checkerboard. Of course, the distance between the peaks then also has to be recalculated (\zeiref{moonlander02c02}).

To make the source code easier to understand, I separated the generation of the peaks from the calculation of the corresponding polygons. Something genuinely new only happens starting at \zeiref{moonlander02c03}. For each peak, four points are now determined: the starting peak, the peak to its right, the point directly below it down to the surface, and finally a point on the surface back to the left underneath the starting peak. In addition, a subtle shade of gray is chosen at random. These four points are stored as a polygon together with the corresponding color in the list \texttt{layers}.

\lstsource{SRC/02 Examples/03 Moonlander/v02/moonlanderc.py}{20}{50}{python}{Moonlander (\reqref{reqMondoberfläche}) -- constructor of \texttt{Moon} with contour}{srcMoon02ca}

After a few more minor and easy-to-understand changes in \texttt{draw()}, the Moon is complete (\abbref[vref]{picGebirge3}).

\lstsource{SRC/02 Examples/03 Moonlander/v02/moonlanderc.py}{52}{59}{python}{Moonlander (\reqref{reqMondoberfläche}) -- \texttt{Moon.draw()} with contour}{srcMoon02cb}

Redrawing the mountain ranges from scratch every time is certainly a huge waste of computing time. A common technique to avoid this is to draw the image once onto a bitmap (\texttt{py\-game.sur\-face.Sur\-face}) and then simply blit this bitmap each frame. 

Note that \texttt{rect} is now needed for the entire bitmap and has therefore first been renamed to \texttt{landingarea}. It is also no longer necessary to keep \texttt{layers} and \texttt{landingarea} as attributes of the class, since this information is no longer required after the bitmap has been created.

\lstsource{SRC/02 Examples/03 Moonlander/v02/moonlanderd.py}{20}{66}{python}{Moonlander (\reqref{reqMondoberfläche}) -- \texttt{Moon} as bitmap}{srcMoon02d}


%%%%%%%%%%%%%%%%%%%%%%%%%%%%%%%%%%%%%%%%%%%%%%%%%%%%%%%%%%%%%%%%%%%%%
\subsection{\Reqref{reqEarth}: Earth}

The Earth as a simple blue spot? That would be far too unattractive!

\br{Earth}{reqEarth}
\begin{enumerate}
	\item The Earth should have an atmospheric glow.\label{reqEarthKranz}
	\item Landmasses should be visible on the Earth.\label{reqEarthLand}
\end{enumerate}
\er

First, the Earth is also converted into a bitmap in order to improve performance. The procedure is analogous to \srcref[vref]{srcMoon02d}.

\lstsource{SRC/02 Examples/03 Moonlander/v02/moonlandere.py}{68}{85}{python}{Moonlander (\reqref{reqEarth}) -- \texttt{Earth} as bitmap}{srcMoon02e}

%\begin{wrapfigure}[7]{r}{5.0cm}%
%	\begin{center}%
%		\vspace{-1cm}%
%		\myfigure{earth01.png}{0.5}{Moonlander (5)}{picErde1}%
%	\end{center}%
%\end{wrapfigure}%
Next, the atmospheric glow is created. The basic idea is to draw circles from the inside to the outside with increasing transparency. To achieve this, a loop counts down from~20 to~1. This counter is multiplied by~10 in \zeiref{moonlander02f01} and subtracted from~210, resulting in a sequence like $(10, 20, \ldots, 200)$. Correspondingly, the radius of these circles increases steadily in \zeiref{moonlander02f02}. Finally, the slightly reduced Earth itself is drawn (see \abbref[vref]{picErde1}.

\lstsource{SRC/02 Examples/03 Moonlander/v02/moonlanderf.py}{79}{85}{python}{Moonlander (\reqref{reqEarth}.\ref{reqEarthKranz}) -- \texttt{Earth} with atmospheric glow}{srcMoon02fa}

%\begin{wrapfigure}[7]{r}{5.0cm}%
%	\begin{center}%
%		\vspace{-1cm}%
%		\myfigure{earth02.png}{0.5}{Moonlander (6)}{picErde2}%
%	\end{center}%
%\end{wrapfigure}%
I had the polygon data for the landmasses generated by ChatGPT (what a blessing!). For the sake of readability, this data has been moved to an external file (\texttt{continent\_poly\-gons.py}). It consists of a list of lists of points. The inner lists represent the landmasses as closed polygon paths. First, the polygon data is imported as a module:

\lstsource{SRC/02 Examples/03 Moonlander/v02/moonlanderg.py}{7}{7}{python}{Moonlander (\reqref{reqEarth}.\ref{reqEarthLand}) -- importing the polygon data}{srcMoon02ga}

The actual drawing is then fairly straightforward. The coordinates only need to be aligned with the center of the Earth and scaled to half the size so that they fit inside the circle (see \abbref[vref]{picErde2}).

\lstsource{SRC/02 Examples/03 Moonlander/v02/moonlanderg.py}{91}{93}{python}{Moonlander (\reqref{reqEarth}.\ref{reqEarthLand}) -- \texttt{Earth} with continents}{srcMoon02gb}

That will be enough for the Earth. On to the next effect.

\myezweihbild{earth01.png}{0.9}{Moonlander (5) -- glowing}{picErde1}{earth02.png}{0.9}{Moonlander (6) -- continent}{picErde2}

%%%%%%%%%%%%%%%%%%%%%%%%%%%%%%%%%%%%%%%%%%%%%%%%%%%%%%%%%%%%%%%%%%%%%
\subsection{\Reqref{reqStars}: Stars}

Outer space is neither black nor empty.

\br{Stars}{reqStars}
\begin{enumerate}
	\item Stars of different sizes should be visible in the background.\label{reqStarsAnzeigen}
	\item The stars should change in brightness and size. This should create a kind of twinkling effect.\label{reqStarsFunkeln}
\end{enumerate}
\er

First, the constructor of \texttt{Sky} is extended by a parameter specifying the number of stars; the default value is 200 stars. In \zeiref{moonlander02h01}, a list for the stars is created. Afterwards, the loop fills this list with the corresponding number of entries. Position, size, and color are determined randomly.

\lstsource{SRC/02 Examples/03 Moonlander/v02/moonlanderh.py}{10}{24}{python}{Moonlander (\reqref{reqStars}.\ref{reqStarsAnzeigen}) -- Constructor of \texttt{Sky}}{srcMoon02ha}

In \texttt{draw()}, the entries of the list are used to render the stars.

\lstsource{SRC/02 Examples/03 Moonlander/v02/moonlanderh.py}{26}{29}{python}{Moonlander (\reqref{reqStars}.\ref{reqStarsAnzeigen}) -- \texttt{Sky.draw()}}{srcMoon02hb}

%\myebild{moonlander05}{0.3}{Moonlander -- Sternenhimmel}{picMoonlander05}
%\begin{wrapfigure}[10]{r}{6.0cm}%
%	\begin{center}%
%		\vspace{-1cm}%
%		\myfigure{moonlander05.png}{0.2}{Sternenhimmel}{picMoonlander05}%
%	\end{center}%
%\end{wrapfigure}%

Creating the twinkling effect is a bit more interesting. As preparation, each star is assigned a random value in \zeiref{moonlander02i01} that specifies after how many frames a change in brightness should occur. At $\SI{60}{fps}$, this corresponds to roughly $3.3$ to $\SI{10}{s}$.

\lstsource{SRC/02 Examples/03 Moonlander/v02/moonlanderi.py}{19}{26}{python}{Moonlander (\reqref{reqStars}.\ref{reqStarsFunkeln}) -- twinkling stars (1)}{srcMoon02ia}

\begin{wrapfigure}[14]{r}{6.0cm}%
	\begin{center}%
		\vspace{-6ex}%
		\myfigure{moonlander05.png}{0.3}{Moonlander (7)}{picMoonlander05}%
	\end{center}%
\end{wrapfigure}%
Since the state of the game object now changes over time, the \texttt{update()} method is required. This method recalculates the brightness and size of the stars. In \zeiref{moonlander02i02}, a counter is increased by~1 in every frame (that is, on each call). The value is then processed using the modulo operator. In this way, the counter always stays within a fixed range, preventing an overflow — that is, exceeding the valid range of an integer.

Within the loop, all stars are now processed. If the value of \texttt{counter} modulo \texttt{duration} is~0, exactly \texttt{duration} frames have passed and the color and size must be updated. The \texttt{draw()} method remains unchanged.

\lstsource{SRC/02 Examples/03 Moonlander/v02/moonlanderi.py}{28}{33}{python}{Moonlander (\reqref{reqStars}.\ref{reqStarsFunkeln}) -- twinkling stars (2)}{srcMoon02ib}

Finally, the previously unused \texttt{update()} method in \texttt{Game} is extended with the corresponding method call, and everything should work as expected (see \abbref[vref]{picMoonlander05}).

\lstsource{SRC/02 Examples/03 Moonlander/v02/moonlanderi.py}{148}{149}{python}{Moonlander (\reqref{reqStars}.\ref{reqStarsFunkeln}) -- twinkling stars (3) \texttt{Game.update()}}{srcMoon02ic}

%%%%%%%%%%%%%%%%%%%%%%%%%%%%%%%%%%%%%%%%%%%%%%%%%%%%%%%%%%%%%%%%%%%%%
\subsection{\Reqref{reqLander}: Lander}

\br{Lander}{reqLander}
\begin{enumerate}
	\item The lander consists of an antenna, a crew module, a base with connectors to the crew module, and landing legs with pads.\label{reqLanderAufbau}
	\item Pressing the \keys{\SPACE} key displays a thrust exhaust. \label{reqLanderAusstoß}
	\item The lander starts roughly in the middle and fairly high up, but not directly at the top edge.\label{reqLanderStart}
\end{enumerate}
\er

The first thing to notice is that I do not use just a single \texttt{Surface} object, but two. The idea behind this is to create one sprite for the lander with thrust and one without thrust. In \texttt{draw()}, the attribute \texttt{thrusting} (\zeiref{moonlander0302}) is then used to control which of the two sprites is blitted to the screen. For the sake of clarity, the drawing of the lander is encapsulated in the method \texttt{create\_lander()} (\zeiref{moonlander0301}).

\lstsource{SRC/02 Examples/03 Moonlander/v03/moonlander.py}{118}{127}{python}{Moonlander (\reqref{reqLander}.\ref{reqLanderAufbau}) -- Constructor of \texttt{Lander}}{srcMoon03a}

Explaining each individual drawing step would certainly be somewhat tedious and would not provide much additional learning value. The easiest way to understand the source code is to change individual details and observe the effect. Nevertheless, I would like to address one specific aspect.

In the first step, all drawing operations are performed on the surface \texttt{surface}. This results in a lander without thrust. Starting at \zeiref{moonlander0302}, the surface with thrust is created. For this purpose, the lander without thrust is first copied onto \texttt{surface\_thrusting} using \texttt{blit()}. After that, an additional thrust flame is drawn onto \texttt{surface\_thrusting}. As a result, two \texttt{Surface} objects are available for rendering the lander. Both can be seen in \abbref{moonlander06} and \abbref[vref]{moonlander07}.

\lstsource{SRC/02 Examples/03 Moonlander/v03/moonlander.py}{129}{191}{python}{Moonlander (\reqref{reqLander}.\ref{reqLanderAufbau}) -- \texttt{Lander.create\_lander()}}{srcMoon03b}

\myezweihbild{moonlander06.png}{1.0}{Moonlander (8) -- lander without thrust}{moonlander06}%
             {moonlander07.png}{1.0}{Moonlander (9) -- lander with thrust}{moonlander07}

In \texttt{update()}, the \texttt{thrusting} flag is controlled and set from outside the class.

\lstsource{SRC/02 Examples/03 Moonlander/v03/moonlander.py}{193}{198}{python}{Moonlander (\reqref{reqLander}.\ref{reqLanderAufbau}) -- \texttt{Lander.update()}}{srcMoon03c}

In \texttt{Game}, \texttt{watch\_for\_events()} must be adapted to handle the thrust control. Pressing \keys{\SPACE} activates the lander’s \emph{thrust} mode, and releasing the key deactivates it again.

\lstsource{SRC/02 Examples/03 Moonlander/v03/moonlander.py}{226}{239}{python}{Moonlander (\reqref{reqLander}.\ref{reqLanderAusstoß}) -- \texttt{Game.watch\_for\_events()}}{srcMoon03d}

In \texttt{Lander.draw()}, the output switches between the two surfaces depending on the current \emph{thrust} mode.

\lstsource{SRC/02 Examples/03 Moonlander/v03/moonlander.py}{200}{204}{python}{Moonlander (\reqref{reqLander}.\ref{reqLanderAusstoß}) -- \texttt{Lander.draw()}}{srcMoon03e}

All that remains is to define the starting position, which is fairly simple. In the constructor of \texttt{Lander}, the appropriate position is determined (see \zeiref{moonlander0304} and \zeiref{moonlander0305}). The result should look like the one shown in \abbref[vref]{picMoonlander08}.

\myebild{moonlander08}{0.5}{Moonlander (10) -- the lander}{picMoonlander08}

%%%%%%%%%%%%%%%%%%%%%%%%%%%%%%%%%%%%%%%%%%%%%%%%%%%%%%%%%%%%%%%%%%%%%
\subsection{\Reqref{reqGraviation}: Gravitation and landing}

\br{Gravitation and landing}{reqGraviation}
\begin{enumerate}
	\item The lander is accelerated by the Moon’s gravity at $\SI{1.62}{m/s^2}$.\label{reqLanderGravitation}
	\item When the pads of the landing legs touch the lunar surface, the lander comes to a stop. \label{reqLanderAufsetzen}
\end{enumerate}
\er

For this purpose, several parameters are defined in \texttt{config.py}, starting at \zeiref{moonlander0401}. I included both lunar and Earth gravity. Of course, you are completely free to let the lander touch down on Venus or even Jupiter instead.

\lstsource{SRC/02 Examples/03 Moonlander/v04/config.py}{3}{99}{python}{Moonlander (\reqref{reqGraviation}.\ref{reqLanderGravitation}) -- physical constants}{srcMoon04a}

In the constructor of the lander, its vertical velocity is defined in \zeiref{moonlander0402}. At the start of the game, this value is always set to~0. This is admittedly unrealistic, since the lander is already in the middle of its descent -- but let’s not worry about that for now.

\lstsource{SRC/02 Examples/03 Moonlander/v04/moonlander.py}{128}{128}{python}{Moonlander (\reqref{reqGraviation}.\ref{reqLanderGravitation}) -- extension of the \texttt{Lander} constructor}{srcMoon04b}

The \texttt{update()} method of \texttt{Lander} is extended by adding the action \texttt{move}. The actual calculation of the new position is encapsulated in the method \texttt{move()}.

\lstsource{SRC/02 Examples/03 Moonlander/v04/moonlander.py}{200}{201}{python}{Moonlander (\reqref{reqGraviation}.\ref{reqLanderGravitation}) -- extension of \texttt{Lander.update()}}{srcMoon04c}

First, the new velocity is calculated based on gravity. After that, the change in position is computed using this velocity. If the lower boundary is crossed (the pads of the landing legs), the lander is aligned with the lunar surface and then remains there.

\lstsource{SRC/02 Examples/03 Moonlander/v04/moonlander.py}{209}{213}{python}{Moonlander (\reqref{reqGraviation}.\ref{reqLanderAufsetzen}) -- \texttt{Lander.move()}}{srcMoon04d}

Finally, \texttt{Game.update()} still needs to be adapted.

\lstsource{SRC/02 Examples/03 Moonlander/v04/moonlander.py}{250}{252}{python}{Moonlander (\reqref{reqGraviation}) -- \texttt{Game.update()}}{srcMoon04e}


%%%%%%%%%%%%%%%%%%%%%%%%%%%%%%%%%%%%%%%%%%%%%%%%%%%%%%%%%%%%%%%%%%%%%
\subsection{\Reqref{reqGegenschub}: Counter-thrust}

\br{Counter-thrust}{reqGegenschub}
	If counter-thrust is activated using \keys{\SPACE}, this thrust should affect the lander’s descent speed. The counter-thrust should be $\SI{-3}{m/s^2}$.
\er

The counter-thrust is rather arbitrarily set to $\SI{-3}{m/s^2}$. The negative sign is used because this thrust acts in the exact opposite direction of the Moon’s gravity.

\lstsource{SRC/02 Examples/03 Moonlander/v05/config.py}{3}{99}{python}{Moonlander (\reqref{reqGegenschub}) -- magnitude of the counter-thrust}{srcMoon05a}

In \texttt{move()}, the counter-thrust is now included in the velocity calculation. To do this, it is first checked whether counter-thrust has been activated by pressing \keys{\SPACE}.

\lstsource{SRC/02 Examples/03 Moonlander/v05/moonlander.py}{209}{215}{python}{Moonlander (\reqref{reqGegenschub}) -- adjustment of \texttt{Lander.move()}}{srcMoon05b}

With this in place, the player can now influence the descent speed of the lander using counter-thrust.

%%%%%%%%%%%%%%%%%%%%%%%%%%%%%%%%%%%%%%%%%%%%%%%%%%%%%%%%%%%%%%%%%%%%%
\subsection{\Reqref{reqTreibstoff}: Fuel}

\br{Fuel}{reqTreibstoff}
\begin{enumerate}
	\item The lander has a limited fuel supply.\label{reqTreibstoffvorrat}
	\item Depending on the difficulty level, different fuel amounts are available.\label{reqTreibstofflevel}
	\item Fuel consumption is 20 units per second.\label{reqTreibstoffverbrauch}
	\item When the fuel supply is empty, no counter-thrust can be generated.\label{reqTreibstoffende}
\end{enumerate}
\er

First, the game constants are defined in \texttt{config.py}. \texttt{THRUST} represents the counter-thrust, but not in the unit~$\unit{m/s^2}$; instead, it is given in~$\unit{px/s^2}$. The possible fuel supplies for \reqref{reqTreibstoff}.\ref{reqTreibstofflevel} are stored in the dictionary \texttt{LEVEL} in \zeiref{moonlander0602}.

\lstsource{SRC/02 Examples/03 Moonlander/v06/config.py}{15}{15}{python}{Moonlander (\reqref{reqTreibstoff}) -- preparations in \texttt{config.py}}{srcMoon06a}

In the constructor of \texttt{Lander}, the initial fuel supply is defined, and the current fuel level is stored in the attribute \texttt{fuel}, initialized with this starting value.

\lstsource{SRC/02 Examples/03 Moonlander/v06/moonlander.py}{129}{131}{python}{Moonlander (\reqref{reqTreibstoff}) -- adjustment in the constructor of \texttt{Lander}}{srcMoon06b}

In \zeiref{moonlander0608}, it is now checked before calculating the counter-thrust whether there is still any fuel left in the tank, and in \zeiref{moonlander0606} the consumed fuel is subtracted from the tank. If the tank is empty, the counter-thrust mode must be switched off and, to prevent a negative fuel value, the fuel level must be clamped to~$0$.

\lstsource{SRC/02 Examples/03 Moonlander/v06/moonlander.py}{219}{229}{python}{Moonlander (\reqref{reqTreibstoff}) -- \texttt{Lander.move()}}{srcMoon06c}

To verify that the fuel supply starts correctly, is reduced properly when counter-thrust is applied, and that thrust is disabled once the tank is empty, I added a \texttt{print()} statement to \texttt{Lander.draw()} in \zeiref{moonlander0607}.

\lstsource{SRC/02 Examples/03 Moonlander/v06/moonlander.py}{212}{217}{python}{Moonlander (\reqref{reqTreibstoff}) -- \texttt{Lander.draw()}}{srcMoon06d}

Give it a try -- it should work!


%%%%%%%%%%%%%%%%%%%%%%%%%%%%%%%%%%%%%%%%%%%%%%%%%%%%%%%%%%%%%%%%%%%%%
\subsection{\Reqref{reqStatus}: Status display}

\br{Status display}{reqStatus}
\begin{enumerate}
	\item A separate status display is required for the lander.\label{reqStatusWindow}
	\item Velocity and altitude are displayed as text including their units.\label{reqStatusGH}
	\item If counter-thrust is active, a colored bar is shown.\label{reqStatusSchub}
	\item The fuel supply is displayed as a progress bar.\label{reqStatusTreibstoff}
\end{enumerate}
\er

All essential changes related to this feature take place in the \texttt{Lander} class. Since I want the position of the separate status display to depend on the position of the main window, the constructor signature has to be changed. Instead of passing a \texttt{Surface} object, a \texttt{Window} object is now passed in \zeiref{moonlander0701}.

\lstsource{SRC/02 Examples/03 Moonlander/v07/moonlander.py}{118}{132}{python}{Moonlander (\reqref{reqStatus}) -- \texttt{Lander.draw()}}{srcMoon07a}

The separate window is created in \texttt{create\_status\_window()}. First, a window of appropriate size is created and the corresponding \texttt{Surface} object is obtained. I want the status window to be positioned to the right of the main window and aligned with its top edge. To achieve this, I take the top edge of the main window and assign this value to the top edge of the status window. Then I take the left edge of the main window, add the width of the main window to obtain its right edge, and finally add an additional~$\SI{10}{px}$ of spacing.

\lstsource{SRC/02 Examples/03 Moonlander/v07/moonlander.py}{198}{203}{python}{Moonlander (\reqref{reqStatus}) -- \texttt{Lander.create\_status\_window()}}{srcMoon07b}

In the final line, the \texttt{draw()} method is extended by a call to \texttt{draw\_status()}. As a result, each time \texttt{draw()} is executed, not only the lander in the main window is redrawn, but the status window is updated as well.

\lstsource{SRC/02 Examples/03 Moonlander/v07/moonlander.py}{220}{225}{python}{Moonlander (\reqref{reqStatus}) -- \texttt{Lander.draw()}}{srcMoon07c} 

In \texttt{draw\_status()}, the window is first filled with a black background. Starting at \zeiref{moonlander0705}, the status display for altitude and fuel is rendered as text. From \zeiref{moonlander0706} onward, two bars are drawn. The first bar is only shown when the lander is currently applying counter-thrust. The second bar consists of two rectangles: a gray bar is drawn across the full width of the window, and a green bar is drawn from the left, proportionally scaled according to the remaining fuel supply.

\lstsource{SRC/02 Examples/03 Moonlander/v07/moonlander.py}{227}{248}{python}{Moonlander (\reqref{reqStatus}) -- \texttt{Lander.draw\_status()}}{srcMoon07d} 

A few adjustments in \texttt{Game} are still required due to rendering output in multiple windows. One of these concerns event handling. When multiple windows are open in Pygame, the event \texttt{pygame.WINDOWCLOSE} must be processed (\zeiref{moonlander0703}). In this case, the flag of the main game loop has to be set to \false{}, and the window associated with the event must be explicitly destroyed using \texttt{destroy()}.

\lstsource{SRC/02 Examples/03 Moonlander/v07/moonlander.py}{284}{300}{python}{Moonlander (\reqref{reqStatus}) -- \texttt{Game.watch\_for\_events()}}{srcMoon07e} 

In \texttt{restart()}, the call to the constructor is also adjusted in \zeiref{moonlander0704}.

\lstsource{SRC/02 Examples/03 Moonlander/v07/moonlander.py}{213}{218}{python}{Moonlander (\reqref{reqStatus}) -- \texttt{Game.restart()}}{srcMoon07f} 

The result then looks like the one shown in \abbref[vref]{picMoonlander09}.

\myebild{moonlander09}{0.6}{Moonlander (11) -- now with status window}{picMoonlander09}


%%%%%%%%%%%%%%%%%%%%%%%%%%%%%%%%%%%%%%%%%%%%%%%%%%%%%%%%%%%%%%%%%%%%%
\subsection{\Reqref{reqEnde}: Game over and restart}

\br{Game over and restart}{reqEnde}
\begin{enumerate}
	\item If the lunar module lands with a velocity of $<\SI{5}{px/s}$, the landing is considered safe.\label{reqEndeGewonnen}
	\item If it lands at a higher velocity, it is considered destroyed.\label{reqEndeVerloren}
	\item The user is asked whether they want to quit the game with \keys{q} or restart it with \keys{r}.\label{reqEndeNeustart}
\end{enumerate}
\er

We prepare \reqref{reqEnde}.\ref{reqEndeNeustart} by encapsulating the display of the prompt in a simple class called \texttt{Question}. A \texttt{Surface} object containing the appropriate text is created and positioned accordingly. In \texttt{draw()}, this \texttt{Surface} object is then simply rendered on top of the lunar surface at the bottom of the screen.

\lstsource{SRC/02 Examples/03 Moonlander/v08/moonlander.py}{303}{313}{python}{Moonlander (\reqref{reqEnde}) -- \texttt{question}}{srcMoon08a} 

How are quitting or restarting the game actually triggered? There are many possible approaches. In this case, I decided to use \emph{events} created with \texttt{pygame.event.Event()}\myindex{pyg}{\texttt{event}!\texttt{Event}}. The basic idea is that touching the lunar surface triggers an event: \texttt{LANDED} if the descent speed is low enough, otherwise \texttt{CRASHED}.

\lstsource{SRC/02 Examples/03 Moonlander/v08/moonlander.py}{11}{13}{python}{Moonlander (\reqref{reqEnde}) -- \texttt{MyEvents}}{srcMoon08b} 

This requires \texttt{watch\_for\_events()} to be rewritten. In \zeiref{moonlander0802} and~\zeiref{moonlander0803}, the two events are intercepted. In both cases, the new flag \texttt{landing} is set to \false{}. This allows me to determine, for example, whether thrust may still be activated at all or whether the prompt for quitting or restarting the game should be displayed. In addition, an \texttt{update()} call is forwarded to the \texttt{Lander} so that it, too, is informed about its new state—for instance, to display an appropriate message in the status window.

For this reason, \zeiref{moonlander0804} first checks whether the lander is still in the landing phase before allowing thrust to be activated.

The responses to the prompt are handled starting at \zeiref{moonlander0805} and \zeiref{moonlander0806}. If \keys{q} is pressed, the flag of the main program loop is simply set to \false{}. If \keys{r} is pressed, a restart is triggered by calling \texttt{restart()}.

\lstsource{SRC/02 Examples/03 Moonlander/v08/moonlander.py}{337}{363}{python}{Moonlander (\reqref{reqEnde}) -- \texttt{Game.watch\_for\_events()}}{srcMoon08c} 

In \texttt{Game}, the attribute \texttt{landing} is added to record whether the lander is still in the landing phase or has already touched the lunar surface.

\lstsource{SRC/02 Examples/03 Moonlander/v08/moonlander.py}{321}{321}{python}{Moonlander (\reqref{reqEnde}) -- \texttt{Game.landing}}{srcMoon08d} 

Finally, \texttt{restart()} is extended in \zeiref{moonlander0807} to reset the \texttt{landing} flag.

\lstsource{SRC/02 Examples/03 Moonlander/v08/moonlander.py}{379}{386}{python}{Moonlander (\reqref{reqEnde}) -- \texttt{Game.restart()}}{srcMoon08e} 


\Reqref{reqEnde}.\ref{reqEndeGewonnen} and \reqref{reqEnde}.\ref{reqEndeVerloren} are implemented in the new method \texttt{check\_landing()} in \texttt{Lander}. When the lander reaches the lunar surface, its velocity is checked. If the descent speed is too high, the event \texttt{CRASHED} is triggered; otherwise, the event \texttt{LANDED} is emitted. The handling of these events itself has already been discussed above in \texttt{watch\_for\_events()} (\srcref[vref]{srcMoon08c}).

\lstsource{SRC/02 Examples/03 Moonlander/v08/moonlander.py}{294}{301}{python}{Moonlander (\reqref{reqEnde}) -- \texttt{Lander.check\_landing()}}{srcMoon08f} 

Finally, \texttt{Game.update()} must be extended to include a call to \texttt{check\_landing()}.

\lstsource{SRC/02 Examples/03 Moonlander/v08/moonlander.py}{365}{364}{python}{Moonlander (\reqref{reqEnde}) -- \texttt{Game.update()}}{srcMoon08g} 

A few adjustments in \texttt{Lander} are still required. First, in \zeiref{moonlander0807}, the attribute \texttt{mode} is introduced. It keeps track of which of the three states the lunar module is currently in: \texttt{landing}, \texttt{landed}, or \texttt{crashed}.

\lstsource{SRC/02 Examples/03 Moonlander/v08/moonlander.py}{125}{141}{python}{Moonlander (\reqref{reqEnde}) -- \texttt{Lander.mode}}{srcMoon08h} 

This attribute is set or updated in \texttt{update()}, starting at \zeiref{moonlander0808}. Once the ground has been touched -- i.e. when the state is either \texttt{landed} or \texttt{crashed} -- the thrust is switched off.

\lstsource{SRC/02 Examples/03 Moonlander/v08/moonlander.py}{221}{233}{python}{Moonlander (\reqref{reqEnde}) -- \texttt{Lander.update()}}{srcMoon08i} 

The status display is now extended to show the game-over state as well. Its appearance can be examined in \abbref[vref]{picMoonlander10}.

\lstsource{SRC/02 Examples/03 Moonlander/v08/moonlander.py}{248}{277}{python}{Moonlander (\reqref{reqEnde}) -- \texttt{Lander.draw\_status()}}{srcMoon08j} 

\myebild{moonlander10}{0.6}{Moonlander (12) -- quit or restart?}{picMoonlander10}


%%%%%%%%%%%%%%%%%%%%%%%%%%%%%%%%%%%%%%%%%%%%%%%%%%%%%%%%%%%%%%%%%%%%%
\subsection{\Reqref{reqAutopilot}: Autopilot}

\br{Autopilot}{reqAutopilot}
	The autopilot can be switched on or off using \keys{h}.
\er

First, something that actually has nothing to do with \reqref{reqAutopilot}. I want the physical values to be a bit closer to real-world figures. The thrust is set to $\SI{-2.1}{m/s^2}$ and the safe landing speed to $\SI{2.5}{m/s}$. According to NASA documentation, Apollo~11 touched down at $\SI{0.7}{m/s}$. The NASA target value was $\SI{1}{m/s}$, and the acceptable range was between $0.5$ and $\SI{2.5}{m/s}$. The structural limit was reached at $\SI{3}{m/s}$. A value below $\SI{0.5}{m/s}$ would have resulted in unnecessary fuel consumption.

\lstsource{SRC/02 Examples/03 Moonlander/v09/config.py}{15}{16}{python}{Moonlander (\reqref{reqAutopilot}) -- some constants}{srcMoon09a} 

Now back to \reqref{reqAutopilot}: In \texttt{watch\_for\_events()}, the key press \keys{h} is detected and forwarded to the \texttt{Lander}.

\lstsource{SRC/02 Examples/03 Moonlander/v09/moonlander.py}{367}{370}{python}{Moonlander (\reqref{reqAutopilot}) -- extension of \texttt{watch\_for\_events()}}{srcMoon09b} 

In the constructor of \texttt{Lander}, the flag \texttt{ai} is introduced and initialized with \false{} -- although the term \emph{ai} is admittedly a bit ambitious here ;-)

\lstsource{SRC/02 Examples/03 Moonlander/v09/moonlander.py}{136}{138}{python}{Moonlander (\reqref{reqAutopilot}) -- extension of \texttt{Lander.\_\_init\_\_()}}{srcMoon09c} 

The \texttt{update()} method is extended as well. Starting at \zeiref{moonlander0901}, the \texttt{ai} flag is toggled on or off. When it is switched off, any thrust that may have been triggered by the autopilot must be stopped. In \zeiref{moonlander0902}, it is then checked whether the autopilot is active; if so, control is handed over to the autopilot.

\lstsource{SRC/02 Examples/03 Moonlander/v09/moonlander.py}{224}{232}{python}{Moonlander (\reqref{reqAutopilot}) -- extension of \texttt{Lander.update()}}{srcMoon09d} 

Before diving into the actual implementation of the control logic, we first need to play around a bit with some physical formulas.

The formula for the final velocity in free fall is:

\begin{align}
	v = \sqrt{2 \cdot g \cdot h} \label{v_eins}
\end{align}

This equation gives us the final velocity~$v$ for a given gravitational acceleration~$g$ and a fall height~$h$, assuming the initial velocity was $\SI{0}{m/s}$. However, we are not actually interested in the final velocity. What we really care about is the height~$h$: from which height do we have to start applying counter-thrust in order to reach our target velocity?

So we rearrange \gleiref{v_eins} to solve for~$h$:

\begin{align}
	v             &=& \sqrt{2 \cdot g \cdot h} &\hspace{0.5cm}\| x^2\nonumber\\
	v^2           &=& 2 \cdot g \cdot h        &\hspace{0.5cm}\| :(2 \cdot g)\nonumber\\
	\frac{v^2}{2 \cdot g} &=& h &  \label{h_eins}
\end{align}

However, we are no longer dealing with lunar gravity alone; the counter-thrust of the lander also comes into play. In this case, the following applies:

\begin{align}
	acc = g_{Moon} + acc_{Lander} \label{a_eins}
\end{align}

Note that the sign of $acc_{Lander}$ is opposite to that of the Moon’s gravity $g_{Moon}$ -- that is, it is negative. We now substitute \gleiref{a_eins} into \gleiref{h_eins}:

\begin{align}
	h &=& \frac{v^2}{2 \cdot acc}&\hspace{0.5cm}\| \leftarrow \ref{a_eins}\nonumber \\
	h &=&\frac{v^2}{2 \cdot (g_{Moon} + acc_{Lander})} \label{h_zwei}
\end{align}

And \gleiref{h_zwei} can already serve as the basis for our implementation. First, starting at \zeiref{moonlander0903}, we check whether the lander is still on its final approach. If not, all thrust is turned off and we are done, because there is nothing left to do.

In \zeiref{moonlander0904}, the net acceleration from \gleiref[vref]{a_eins} is computed, and then the target velocity is defined. This value is chosen to be far enough away from the maximum structural limit by using $\SI{50}{\%}$. In \zeiref{moonlander0905}, we then check whether the current velocity is already below this safe velocity. If it is, there is nothing to do—except that thrust must be switched off.

Next, following \gleiref[vref]{h_zwei}, the distance to the ground at which counter-thrust must begin is calculated. In \zeiref{moonlander0906}, counter-thrust is then activated or deactivated accordingly. All clear?

With this, the lander is fully implemented for the purposes of this script. If the autopilot performs a safe landing, it should look like the scene shown in \abbref[vref]{picMoonlander11}.

\lstsource{SRC/02 Examples/03 Moonlander/v09/moonlander.py}{238}{251}{python}{Moonlander (\reqref{reqAutopilot}) -- \texttt{Lander.controller()}}{srcMoon09e} 

\myebild{moonlander11}{0.6}{Moonlander (13) -- autopilot}{picMoonlander11}





\listoffigures
\listoftables
%\lstlistoflistings

\printglossaries
\printindex
\printindex[pyg][Index of the \texttt{pygame} Namespace]

\end{document}
