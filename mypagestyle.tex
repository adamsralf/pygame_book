\clearpairofpagestyles
\clearplainofpairofpagestyles
\automark[section]{chapter}
\setkomafont{pagehead}{\small\sffamily\bfseries}
\setkomafont{pagefoot}{\small\sffamily}
\renewcommand*{\chapterpagestyle}{scrheadings}


\tcbset{
	page/header/.style={
		enhanced,
		frame hidden,
		colback=headblue,
		coltext=white,
		boxrule=0pt,
		arc=0mm,
		left=6mm,
		right=6mm,
		top=1.5mm,
		bottom=1.5mm,
%		width=\paperwidth,
		enlarge left by=-\hoffset,
		enlarge right by=\hoffset,
	},
	page/footer/.style={
		enhanced,
		frame hidden,
		colback=headblue,
		coltext=white,
		boxrule=0pt,
		arc=0mm,
		left=6mm,
		right=6mm,
		top=1.2mm,
		bottom=1.2mm,
%		width=\paperwidth,
		enlarge left by=-\hoffset,
		enlarge right by=\hoffset,
	},
}



% ---------------- HEADER ----------------
\newcommand{\tcbheader}[1]{%
	\begin{tcolorbox}[page/header]%
		\hspace{0.1em}#1\hspace{0.1em}%
	\end{tcolorbox}%
}
\ihead[\tcbheader{\rightmark}]{\tcbheader{\rightmark}}
\ohead[\tcbheader{\leftmark}]{\tcbheader{\leftmark}}


\newcommand{\tcbfooter}[2]{%
	\begin{tcolorbox}[page/footer]%
		\NoLinkColor
		#1\hfill #2%
	\end{tcolorbox}%
}
% ---------------- FOOTER ----------------
\ifoot{\tcbfooter{2D Game Programming with Pygame-ce}{Page \pagemark\ of \pageref{LastPage}\hspace{0.1em}}}
\ofoot{\tcbfooter{Page \pagemark\ of \pageref{LastPage}}{2D Game Programming with Pygame-ce\hspace{0.1em}}}

\pagestyle{scrheadings}

% -------------------------------------------------
% Header
% -------------------------------------------------
% Chapter title on even pages (left pages)
%\ihead{\leftmark\hspace{0.1em}}
% Section title on odd pages (right pages)
%\ohead{\rightmark\hspace{0.1em}}

% -------------------------------------------------
% Footer
% -------------------------------------------------
%\ifoot[2D Game Programming with Pygame-ce\hspace{0.1em}]{2D Game Programming with Pygame-ce\hspace{0.1em}}
%\ofoot[Page \pagemark\ of \pageref{LastPage}\hspace{0.1em}]{Page \pagemark\ of \pageref{LastPage}\hspace{0.1em}}


\definecolor{headblue}{RGB}{90,135,200}
\definecolor{chapblue}{RGB}{120,165,220}
\definecolor{secblue}{RGB}{232,242,255}

% Styles für Kapitel/Abschnitt
\tcbset{
	heading/chapter/.style={
		enhanced,
		colback=chapblue,
		coltext=white,
		boxrule=0.6pt,
		arc=2mm,
		left=2.5mm, right=2.5mm, top=2.3mm, bottom=2.0mm, % kleine Innenabstände
		width=\textwidth,
		boxsep=0pt,
		halign title=center,
		frame hidden,        % <-- Rahmen AUS
	},
	heading/section/.style={
		enhanced,
		colback=secblue,
%		coltext=secorangetext,
		boxrule=0.5pt,
		arc=1.6mm,
		left=2.2mm, right=2.2mm, top=1.8mm, bottom=1.5mm, % noch etwas kompakter
		width=\textwidth,
		boxsep=0pt,
		frame hidden,        % <-- Rahmen AUS
	},
}
% Kapitel/Section-Abstände (kompakter als Default)
\renewcommand*{\chapterpagestyle}{scrheadings}
\renewcommand*\chapterheadstartvskip{\vspace*{-\topskip}} % bündig oben
\renewcommand*\chapterheadendvskip{\vspace*{0pt}}         % kein Extra-Abstand danach
\RedeclareSectionCommand[
beforeskip= 0ex,%-1.2ex plus -0.4ex minus -0.2ex,
afterskip =  1.0ex plus  0.2ex
]{chapter}

\RedeclareSectionCommand[
beforeskip=  0.9ex plus 0.2ex,
afterskip =  0.6ex plus 0.1ex
]{section}

% Optional auch für subsections:
% \RedeclareSectionCommand[beforeskip=0.7ex, afterskip=0.4ex]{subsection}

\makeatletter

% Kapitel: #1 = "Chapter", #2 = Nummer (kann leer sein), #3 = Titel
\renewcommand*\chapterlinesformat[3]{%
	\begin{tcolorbox}[heading/chapter]%
		%\raggedright%
		{\Large\bfseries%
			\ifstrempty{#2}{#3}{#2\enspace #3}%
		}%
	\end{tcolorbox}%
}

% Section: #1 = Name, #3 = Nummer (kann leer), #2 = Einzug, #4 = Titel
\renewcommand*\sectionlinesformat[4]{%
	\begin{tcolorbox}[heading/section]%
		\raggedright%
		{\large\bfseries%
			\ifstrempty{#3}{#4}{#3\enspace #4}%
		}%
	\end{tcolorbox}%
}

\makeatother

